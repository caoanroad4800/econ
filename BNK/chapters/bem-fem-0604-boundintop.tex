%!TEX root = ../DSGEnotes.tex
\subsection{边界积分算子的映射特性}
\label{sec:bvp-bie-mapping-properties}
至今我们分析的一系列边界积分算子的特性,都是从牛顿位势的映射$\widetilde{N}_{0}:\widetilde{H}^{-1}(\Omega) \mapsto H^{1}(\Omega)$,以及逆定理和对偶配对而展开的。对于利普希茨域,我们可以得到更为通用的结论。

\begin{theorem}[牛顿位势在利普希茨域中的映射特性]
  \label{theorem:bie-newton-mapping}
  对于$s \in [-2,0]$,牛顿位势$\widetilde{N}_{0}:\widetilde{H}^{s}(\Omega) \mapsto H^{s+1}(\Omega)$是一个连续映射,即
  \begin{equation*}
    \left\|
    \widetilde{N}_{0} f
    \right\|_{H^{s+2}(\Omega)}
    \le c \, \left\| f \right\|_{\widetilde{H}^{s}(\Omega)}, \quad \forall \, f \in \widetilde{H}^{s}(\Omega).
  \end{equation*}
\end{theorem}
\begin{proof}
  \begin{enumerate}
    \item 设$s \in [-1,0]$。牛顿位势$f \in H^{s}(\Omega)$的定义如\eqref{eq:bvp-newton-potential-def}。$\widetilde{f} \in H^{s}(\mathbb{R}^{d})$是$f$的延拓,我们有
    \begin{equation*}
      \begin{split}
        \left\| \widetilde{f} \right\|_{H^{s}(\mathbb{R}^{d})}
        & = \sup_{0 \neq \nu \in H^{-s}(\mathbb{R}^{d})}
        \frac{
        \langle \widetilde{f}, \nu \rangle_{\mathbb{R}^{d}}
        }{
        \left\| \nu \right\|_{H^{-s}(\mathbb{R}^{d})}
        } \\
        & \le \sup_{0 \neq \nu \in H^{-s}(\Omega)}
        \frac{
        \langle f, \nu \rangle_{\Omega}
        }{
        \left\| \nu \right\|_{H^{-s}(\Omega)}
        } \\
        & = \left\| f \right\|_{H^{s}(\Omega)},
      \end{split}
    \end{equation*}
    由牛顿位势算子的映射定理\ref{theorem:bvp-newton-potential-mappting-property}得
    \begin{equation*}
      \left\| \widetilde{N}_{0} f \right\|_{H^{s+2}(\Omega)}
      \le c \, \left\| f \right\|_{\widetilde{H}^{s}(\Omega)}, \quad \forall \, f \in \widetilde{H}^{s}(\Omega).
    \end{equation*}

    \item 设$s \in [-2,-1)$。由于牛顿位势$\widetilde{N}_{0}$是自伴随算子,根据配偶配对我们有
    \begin{equation*}
      \begin{split}
        \left\| \widetilde{N}_{0} f \right\|_{H^{s+2}(\Omega)}
        & = \sup_{0 \neq g \in \widetilde{H}^{-2 - s }(\Omega)}
        \frac{
        \langle \widetilde{N}_{0}f, g \rangle_{\Omega}
        }{
        \left\| g \right\|_{\widetilde{H}^{-2 - s }(\Omega)}
        }\\
        & = \sup_{0 \neq g \in \widetilde{H}^{-2 - s }(\Omega)}
        \frac{
        \langle f, \widetilde{N}_{0} g \rangle_{\Gamma}
        }{
        \left\| g \right\|_{\widetilde{H}^{-2 - s}(\Omega)}
        }\\
        & \le
        \left\| f \right\|_{\widetilde{H}^{s}(\Omega)} \,
        \sup_{0 \neq g \in \widetilde{H}^{-2 - s }(\Omega)}
        \frac{
        \left\| \widetilde{N}_{0} g \right\|_{H^{-s}(\Omega)}
        }{
        \left\| g \right\|_{\widetilde{H}^{-2 - s}(\Omega)}
        } \\
        & \le c \, \left\| f \right\|_{\widetilde{H}^{s}(\Omega)}
      \end{split}
    \end{equation*}
  \end{enumerate}
\end{proof}

利用牛顿位势在利普希茨域中的映射特性(定理\ref{theorem:bie-newton-mapping}),我们可以得到单层位势$\widetilde{V}$\eqref{eq:bvp-single-layer-potential-operator}的映射特性。进而,利用$\widetilde{V}$的迹,可得边界积分算子$V \coloneqq \gamma_{0}^{\text{int}} \widetilde{V}$的映射特性。
\begin{theorem}[单层位势算子的映射特性]
  \label{theorem:bie-single-layer-mapping}
  对于$\left| s \right| < \frac{1}{2}$,单层位势$V: H^{-\frac{1}{2} + s} \mapsto H^{\frac{1}{2} + s}$有界,即
  \begin{equation*}
    \left\| V w \right\|_{H^{\frac{1}{2}+ s}(\Gamma)}
    \le c \, \left\| w \right\|_{H^{-\frac{1}{2} + s}(\Gamma)}, \quad \forall \, w \in H^{-\frac{1}{2} + s}(\Gamma).
  \end{equation*}
\end{theorem}
\begin{proof}
  \begin{enumerate}
    \item 对于$\varphi \in C^{\infty}(\Omega)$,有
    \begin{equation*}
      \begin{split}
        \langle \widetilde{V} w, \varphi \rangle_{\Omega}
        & = \int_{\Omega} \varphi(x)
        \int_{\Gamma} U^{*}(x,y) w(y) \, d s_{y}
        d x \\
        & = \int_{\Gamma} w(y)
        \int_{\Gamma} U^{*}(x,y) \varphi(x) \, dx
        d s_{y}\\
        & = \langle w, \gamma_{0}^{\text{int}} \widetilde{N}_{0} \varphi \rangle_{\Gamma} \\
        & \le
        \left\| w \right\|_{H^{-\frac{1}{2} + s}(\Gamma)}
        \, \left\| \gamma_{0}^{\text{int}} \widetilde{N}_{0} \varphi \right\|_{H^{\frac{1}{2} - s}(\Gamma)}.
      \end{split}
    \end{equation*}
    根据迹定理\ref{theorem:sobolev-manifold-trace-theorem}可得
    \begin{equation*}
      \begin{split}
        \langle \widetilde{V} w, \varphi \rangle_{\Omega}
        & \le
        \left\| w \right\|_{H^{-\frac{1}{2} + s}(\Gamma)}
        \, \left\| \gamma_{0}^{\text{int}} \widetilde{N}_{0} \varphi \right\|_{H^{\frac{1}{2} - s}(\Gamma)} \\
        & \le
        c_{T} \, \left\| w \right\|_{H^{-\frac{1}{2} + s}(\Gamma)} \,
         \left\| \widetilde{N}_{0} \varphi \right\|_{H^{1 - s}(\Gamma)}.
      \end{split}
    \end{equation*}

  \item 利用Theorem \ref{theorem:bie-newton-mapping},可得
  \begin{equation*}
    \begin{split}
      \langle \widetilde{V} w, \varphi \rangle_{\Omega}
      & \le
      c_{T} \, \left\| w \right\|_{H^{-\frac{1}{2} + s}(\Gamma)} \,
      \left\| \widetilde{N}_{0} \varphi \right\|_{H^{1 - s}(\Gamma)} \\
      & \le
      c \,  \left\| w \right\|_{H^{-\frac{1}{2} + s}(\Gamma)} \,
      \left\| \varphi \right\|_{\widetilde{H}^{-1-s}(\Omega)}, \forall \, \varphi \in C^{\infty}(\Omega).
    \end{split}
  \end{equation*}

  \item 将$w$视为密度方程,根据定义可得$\widetilde{V} w \in H^{1+s}(\Omega)$。对$\widetilde{V} w$取迹,在满足$\frac{1}{2} + s >0$的情况下,我们有$V w : \gamma_{0}^{\text{int}} \widetilde{V} w \in H^{\frac{1}{2}} (\Gamma)$。
  \end{enumerate}
\end{proof}

对于利普希茨域$\Omega$的情况,可由Theorem \ref{theorem:bie-single-layer-mapping}求得全部边界积分算子的映射特性。
\begin{theorem}[边界积分算子的映射特性]
  \label{theorem:bie-mapping-operators}
  设利普希茨域$\Omega$,对应边界$\Gamma \coloneqq \partial \Omega$。则对于$S \in \left[ - \frac{1}{2}, \frac{1}{2} \right]$,以下边界积分算子都有界,且满足映射关系\citep{Costabel:1988dw}
  \begin{equation*}
    \begin{split}
      V&: H^{-\frac{1}{2}+s} (\Gamma) \mapsto H^{\frac{1}{2}+s}(\Gamma), \\
      K&: H^{\frac{1}{2}+s} (\Gamma) \mapsto H^{\frac{1}{2}+s}(\Gamma), \\
      K'&:H^{-\frac{1}{2}+s} (\Gamma) \mapsto H^{-\frac{1}{2}+s}(\Gamma),\\
      D &: H^{\frac{1}{2}+s} (\Gamma) \mapsto H^{-\frac{1}{2}+s}(\Gamma).
    \end{split}
  \end{equation*}
\end{theorem}
\begin{proof}
  \begin{enumerate}
    \item $\left| s \right| < \frac{1}{2}$时单层位势$V$的映射特性可由Theorem \ref{theorem:bie-single-layer-mapping}证得。$\left| s \right| = \frac{1}{2}$时$V$的映射特性,见\cite{Verchota:1984fp, McLean:2000ta}。

    \item 在此基础上,$K,K',D$的映射特性可由共法导数算子推得。已知单层位势$u(x) = \widetilde{V} w (x), x \in \Omega$构成齐次偏微分方程的解,对应狄利克雷数$\gamma_{0}^{\text{int}} u(x) = \left( V w \right)(x), x \in \Gamma$。

    已知单层位势$V:L^{2}(\Gamma) \mapsto L^{1}(\Gamma)$是连续的,根据Theorem \ref{theorem:var-bvp-strong-solution}我们有
    \begin{equation*}
      \left\| \gamma_{1}^{\text{int}} u \right\|_{L^{2}(\Gamma)}
      \le c \, \left\| V w \right\|_{H^{1}(\Gamma)}
      \le \widetilde{c} \, \left\| w \right\|_{L^{2}(\Gamma)},
    \end{equation*}
    进而可证得伴随双层位势算子$\gamma_{1}^{\text{int}} \widetilde{V} = \sigma I + K':L^{2}(\Gamma) \mapsto L^{2}(\Gamma)$的连续性。另一方面,由对偶配对可得
    \begin{equation*}
      \left\| \gamma_{1}^{\text{int}} u \right\|_{L^{2}(\Gamma)}
      = \sup_{0 \neq \varphi \in H^{1}(\Gamma)}
      \frac{
      \langle \gamma_{1}^{\text{int}}, \varphi \rangle_{\Gamma}
      }{
      \left\| \varphi \right\|_{H^{1}(\Gamma)}
      }.
    \end{equation*}

  对于任一$\varphi \in H^{1}(\Gamma)$,狄利克雷边界值问题
  \begin{equation*}
    \begin{cases}
      L \nu (x) = 0 & x \in \Gamma,\\
      \gamma_{0}^{\text{int}} \nu(x) = \varphi(x) & x \in \Gamma
    \end{cases}
  \end{equation*}
  有唯一解$\nu \in H^{\frac{3}{2}}(\Omega)$。在此基础上,由Theorem \ref{theorem:var-bvp-strong-solution}可得
  \begin{equation*}
    \left\| \gamma_{1}^{\text{int}} \nu \right\|_{L^{2}(\Gamma)}
    \le c \, \left\| \gamma_{0}^{\text{int}} \nu \right\|_{H^{1}(\Gamma)}
    \le c \, \left\| \varphi \right\|_{H^{1}(\Gamma)}.
  \end{equation*}

  既然$u = \widetilde{V} w$和$\nu$都是齐次偏微分方程的解,连续两次使用格林第一恒等式\eqref{eq:bvp-a-u-nu-inner-prod}有
  \begin{equation*}
    \begin{split}
    & \langle \gamma_{1}^{\text{int}} u, \varphi \rangle_{\Gamma}
    = a \left( u,\nu \right) \\
    & = a \left( \nu, u \right) =
    \langle \gamma_{1}^{\text{int}} \nu, \gamma_{0}^{\text{int}} u \rangle_{\Gamma}\\
    & \le \left\| \gamma_{1}^{\text{int}} \nu \right\|_{L^{2}(\Gamma)} \,
    \left\| \gamma_{0}^{\text{int}} u \right\|_{L^{2}(\Gamma)} \\
    & \le c \, \left\| \varphi \right\|_{H^{1}(\Gamma)} \, \left\| V w \right\|_{L^{2}(\Gamma)}.
  \end{split}
\end{equation*}

可见,由单层位势$V:H^{-1}(\Gamma) \mapsto L^{2}(\Gamma)$的连续性,我们有
\begin{equation*}
  \left\| \gamma_{1}^{\text{int}} u \right\|_{H^{-1}(\Gamma)}
  \le c \, \left\| V w \right\|_{L^{2}(\Gamma)}
  \le \widetilde{c} \left\| w \right\|_{H^{-1}(\Gamma)} ,
\end{equation*}

进而可得共法导数算子$\gamma_{1}^{\text{int}} \widetilde{V} : \sigma I + K': H^{-1}(\Gamma) \mapsto H^{-1}(\Gamma)$的连续性。在此基础上,若满足$s \le \frac{1}{2}$,那么利用插值定理可得$K' : H^{-\frac{1}{2} + s}(\Gamma) \mapsto H^{-\frac{1}{2} + s}(\Gamma)$。

\item 由于
\begin{equation*}
  \begin{split}
    \left\| K \nu \right\|_{H^{\frac{1}{2}+s}(\Gamma)}
    & = \sup_{0 \neq w \in H^{-\frac{1}{2}-s}(\Gamma)}
    \frac{
    \langle K \nu, w \rangle_{\Gamma}
    }{
    \left\| w \right\|_{H^{-\frac{1}{2}-s}(\Gamma)}
    } \\
    & = \sup_{0 \neq w \in H^{-\frac{1}{2}-s}(\Gamma)}
    \frac{
    \langle \nu, K' w \rangle_{\Gamma}
    }{
    \left\| w \right\|_{H^{-\frac{1}{2}-s}(\Gamma)}
    } \\
    & = \left\| \nu \right\|_{H^{\frac{1}{2} + s}(\Gamma)}
    \sup_{0 \neq w \in H^{-\frac{1}{2}-s}(\Gamma)}
    \frac{
    \left\| K' w \right\|_{H^{-\frac{1}{2}-s}(\Gamma)}
    }{
    \left\| w \right\|_{H^{-\frac{1}{2}-s}(\Gamma)}
    }\\
    & \le c \, \left\| \nu \right\|_{H^{\frac{1}{2} + s}(\Gamma)},
  \end{split}
\end{equation*}
  进而可得对于$\left| s \right| \le \frac{1}{2}$,有$K: H^{\frac{1}{2} + s}(\Gamma) \mapsto H^{\frac{1}{2} + s}(\Gamma)$。

  \item 由双层位势的定义$u(x) = \left( W u \right)(x)$可得,利用Theorem \ref{theorem:var-bvp-strong-solution},有
  \begin{equation*}
    \begin{split}
      \left\| D \nu \right\|_{L^{2}(\Gamma)}
      & = \left\| \gamma_{1}^{\text{int}} u \right\|_{L^{2}(\Gamma)} \\
      & \le c \, \left\| \gamma_{0}^{text{int}} u \right\|_{H^{1}(\Gamma)} \\
      & = c \, \left\|
      \left[
      \left( \sigma - 1 \right) I + K
      \right] \nu
      \right\|_{H^{1}(\Gamma)},
    \end{split}
  \end{equation*}
  进而可得$D:H^{1}(\Gamma) \mapsto L^{2}(\Gamma)$。进一步使用配偶配对以及插值法,得$D:H^{\frac{1}{2}}(\Gamma) \mapsto H^{-\frac{1}{2}}(\Gamma)$。
\end{enumerate}
\end{proof}

如果有界域$\Omega \subset \mathbb{R}^{d}$的边界$\gamma = \partial \Omega$是分段平滑的,那么即使$\left| s \right|$延展到$\frac{1}{2}$之外,Theorem \ref{theorem:bie-mapping-operators}也可能成立。举例来说,若$\Omega \subset \mathbb{R}^{2}$是个有界的多边形,有$J$个角,内夹角$\alpha_{j}$,可定义
\begin{equation*}
  \sigma_{0} \coloneqq \min_{j=1,\ldots,J}
  \left\{
  \min
  \left[
  \frac{\pi}{\alpha_{j}},
  \frac{\pi}{2 \pi - \alpha_{j}}
  \right]
  \right\},
\end{equation*}
此时Theorem \ref{theorem:bie-mapping-operators}对于所有$\left| s \right| < \sigma_{0}$均成立\citep{Costabel:1985dl};如果边界$\Gamma$是$C^{\infty}$的,则Theorem \ref{theorem:bie-mapping-operators}对于所有$s \in \mathbb{R}$都成立。
