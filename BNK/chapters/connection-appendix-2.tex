%!TEX root = ../DSGEnotes.tex
\section{大数法则}
\label{sec:lln}
介绍弱大数法则、强大数法则和(强)均匀大数法则。

\subsection{弱大数法则}
\label{sec:lln-wlln}
大数法则(law of large numbers, LLN)\index{law of large numbers (LLN)! \dotfill 大数法则}可以追溯至概率论的萌芽时期:
\begin{itemize}
  \item 伯努利(James Bernoulli)在1700年前后的随机抛硬币实验中发现。但直到死后,才在1713年的遗著中出版,为后人所知。
  \item 1800年泊松(Poisson)将扩展为彼此独立事件下的抛硬币问题。
  \item 1866年切比雪夫(Chebyshev)提出切比雪夫不等式,进一步扩展到含有独立随机变量的任意过程及其二阶矩。
  \item 切比雪夫的学生马尔科夫(Markov)进一步扩展到相互关联的随机变量的情况。
\end{itemize}

我们先来介绍马尔科夫不等式。
\begin{theorem}[马尔科夫不等式]
  \label{theorem:markov-inequality}
  对一组连续的非负随机变量$X$有
  \begin{equation}
    \label{eq:markov-inequality}
    P \left( X \ge t \right) \le \frac{E X}{t}, \quad \forall \, t > 0.
  \end{equation}
\end{theorem}
\begin{proof}
  连续版的证明:
  \begin{equation*}
    \begin{split}
      E X & = \int_{-\infty}^{\infty} x f_{X} \left( x \right) \, \mathrm{d} x \\
      & = \int_{0}^{\infty} x f_{X} \left( x \right) \, \mathrm{d} x  \qquad \text{(由于$X$非负)}\\
      & \ge \int_{t}^{\infty} x f_{X} \left( x \right) \, \mathrm{d} x \\
      & \ge \int_{t}^{\infty} t f_{X} \left( x \right) \, \mathrm{d} x \qquad \text{( 在求积域中$x > t$ )} \\
      & t \int_{t}^{\infty} f_{X} \left( x \right) \, \mathrm{d} x \\
      & = t P \left( X \ge t \right), \quad \forall \, t>0
    \end{split}
  \end{equation*}

  \begin{equation*}
    \hookrightarrow P \left( X \ge t \right) \le \frac{E X}{t}, \quad \forall \, t > 0.
  \end{equation*}

  离散版的证明。对于$t>0$,有
  \begin{equation*}
    X \ge X1_{\left[X \ge t \right]} \ge t1_{\left[ X \ge t\right]},
  \end{equation*}
  进而有期望值的单调性有
  \begin{equation*}
    EX \ge t P \left( X \ge t \right).
  \end{equation*}
\end{proof}

由马尔科夫不等式可以得到两个重要结论,分别是切比雪夫不等式和切尔诺夫边界。

\begin{theorem}[切比雪夫不等式]
  \label{theorem:chebyshev-inequality}
  如果$X$是非负随机变量,那么有
  \begin{equation}
    \label{eq:chebyshev-inequality}
      P \left( \left| X - E X \right| \ge t \right) \le \frac{Var(X)}{t^{2}}, \quad \forall \, t > 0.
  \end{equation}
\end{theorem}
\begin{proof}
  简化表述,定义$Y = \left( X - EX \right)^{2}$。由$X$的假设条件可得$Y$也是一个非负随机变量。那么可将马尔科夫不等式\eqref{eq:markov-inequality}应用到$Y$上
  \begin{equation*}
    P \left( Y \ge t^{2} \right) \le \frac{E Y}{t^{2}}, \quad \forall \, t > 0.
  \end{equation*}

  此外我们有
  \begin{align*}
    & EY = E \left( X - EX \right)^{2} = Var(X), \\
    & P  \left( Y \ge t^{2} \right) = P \left( \left( X - EX \right)^{2}  \ge t^{2} \right) = P \left( \left| X - E X \right| \ge t \right),
  \end{align*}

  代回上式中,有
  \begin{equation*}
    P \left( \left| X - E X \right| \ge t \right) \le \frac{Var(X)}{t^{2}}, \quad \forall \, t > 0.
  \end{equation*}
\end{proof}

由切比雪夫不等式可得,随机变量$X$和它期望值$E X$之间的距离不可能无限大,其上界由方差$Var(X)$予以限制。这就引出切尔诺夫边界的概念。

\begin{theorem}[切尔诺夫法]
  \label{theorem:chernoff-method}
  切尔诺夫法(Chernoff method)\index{Chernoff method \dotfill 切尔诺夫法}将马尔科夫不等式\eqref{eq:markov-inequality}中的非负随机变量$X$替换为$\exp \left( s X \right)$,可得
  \begin{equation}
    \label{eq:chernoff-method-rv}
    P \left( \exp \left( s X \right) \ge \exp \left( s t \right) \right)
    \le \frac{E \exp \left( s X \right)}{\exp \left( s t \right)}.
  \end{equation}

  如果$X$是由$n$个i.i.d.随机变量$X_{1},\ldots,X_{n}$组成的向量,那么样本均值$\overline{X}_{n} = \frac{1}{n} \sum_{i=1}^{n} X_{i}$,总体均值$E X$,方差$\frac{Var(X)}{n}$。由切比雪夫不等式\eqref{eq:chebyshev-inequality}得
  \begin{equation}
    \label{eq:chernoff-method-vector}
    P \left( \left| \overline{X}_{n} - E X \right| \ge \epsilon \right)
    = \frac{Var(X)}{n \epsilon^{2}}, \quad \forall \, \epsilon > 0.
  \end{equation}
\end{theorem}

进而我们得到弱大数法则(weak law of large numbers)\index{law of large numbers (LLN)!weak(WLLN) \dotfill 弱大数法则}:\eqref{eq:chernoff-method-vector}在$n \rightarrow \infty$的情况下变为
\begin{equation}[弱大数法则]
  \label{eq:lln-wlln}
  \lim_{n \rightarrow \infty} P \left( \left| \overline{X}_{n} - E X \right| \ge \epsilon  \right) = 0, \quad \forall \, \epsilon > 0,
\end{equation}
随着样本数不断增大直至正无穷,样本的期望值以概率1逐渐收敛至全体的期望值。

回到掷硬币的问题,即二进制随机变量$Bin(p)$,满足
\begin{equation}
  P \left( X = 1 \right) = 1 - P \left( X = 0 \right) \equiv p,
\end{equation}
其对应的切比雪夫边界(Chebyshev bound)\index{Chebyshev bound \dotfill 切比雪夫边界}为
\begin{equation*}
  P \left(
  \left| \overline{X}_{n} - p \right| \ge \epsilon
  \right) \le \frac{p \left( 1 - p \right)}{n \epsilon^{2}}, \quad \forall \, \epsilon > 0,
\end{equation*}
也就是说在观察到的样本中,事件$X=1$(头像朝上)发生的概率,逐渐收敛到全体的真实概率$EX=1$(头像朝上)。

\subsection{强大数法则}
\label{sec:lln-slln}
弱大数法则用切比雪夫不等式和切尔诺夫方法来确定概率的上下界,其本质上是利用观测到随机变量$X$和全体期望值$EX$之间的方差$Var(x)$来描述边界的。这种方法并不理想:方差相对于$X$的变化是有限的。

我们用矩生成方程(moment generating function)\index{moment generating function (MGF) \dotfill 矩生成方程}来进行扩展,定义为$M_{X} \left( \lambda \right)$
\begin{equation}
  \label{eq:lln-slln-mgf}
  M_{X} \left( \lambda \right) = E \left[ \exp \left( \lambda X \right) \right],
\end{equation}
不难看出,随着$\lambda$值的变化,$M_{X} \left( \lambda \right)$甚至可以无限大。

利用MGF可以计算切尔诺夫边界。
\begin{theorem}[切尔诺夫边界]
  \label{theorem:chernoff-bounds}
  非负的随机变量$X$,对于任意$t \ge 0$,其分布的上、下界分别为
  \begin{align}
    \label{eq:chernoff-bound-upper}
    P \left( X \ge E X + t \right) \le \min_{ \left\{ \lambda \ge 0 \right\}}
    E \left\{ \exp \left[ \lambda \left( X - E X \right)\right]\right\} \exp \left( - \lambda t \right)
    = \min_{\left\{ \lambda \ge 0 \right\}}
    M_{X - E X} \left( \lambda \right) \exp \left( - \lambda t \right)
    , \\
    \label{eq:chernoff-bound-bottom}
    P \left( X \le E X + t \right) \le \min_{ \left\{ \lambda \ge 0 \right\}}
    E \left\{ \exp \left[ \lambda \left( E X - X \right)\right]\right\} \exp \left( - \lambda t \right)
    = \min_{\left\{ \lambda \ge 0 \right\}}
    M_{E X - X} \left( \lambda \right) \exp \left( - \lambda t \right).
  \end{align}
\end{theorem}
\begin{proof}
  对切尔诺夫上界\eqref{eq:chernoff-bound-upper}的证明。对于任意$\lambda > 0$,已知
  \begin{equation*}
    \left\{ X \ge E X + t \right\}
    \Longleftrightarrow
    \left\{
    \exp \left( \lambda X \right) \ge \exp \left[ \lambda \left( E X + t \right) \right]
    \right\}
    \Longleftrightarrow
    \left\{
    \exp \left[ \lambda \left( X - E X \right) \right]
    \ge \exp \left( \lambda X \right)
    \right\},
  \end{equation*}
  其对应的马尔科夫不等式由\eqref{eq:markov-inequality}写为
  \begin{equation*}
    \begin{split}
      P \left(
      X - E X \ge t
      \right)
      & = P \left(
      \exp \left[ \lambda \left( X - E X \right) \right]
      \ge
      \exp \left( \lambda t \right)\right).
    \end{split}
  \end{equation*}

切尔诺夫下界\eqref{eq:chernoff-bound-bottom}也可用类似的方法证得。
\end{proof}

这样,在给定MGF具体形式的基础上,我们可以根据切尔诺夫边界(Theorem \ref{theorem:chernoff-bounds})计算出概率分布的上、下界。最常见、也是最适合的形式为
\begin{equation}
  \label{eq:lln-slln-mgf-explicit}
  M_{X} \left( \lambda \right) = E \left[ \exp \left( \lambda X \right) \right]
  \le \exp \left( \frac{C^{2} \lambda^{2}}{2} \right), \quad \forall \, \lambda \in \mathbb{R},
\end{equation}
其中常数$C \in \mathbb{R}$是一个与$X$的分布有关的常数。举两个例子。一个例子是正态分布$X \sim \mathbb{N} \left( 0, \sigma^{2} \right)$,对应MGF
\begin{equation*}
  M_{X} \left( \lambda \right)
  = E \left[ \exp \left( \lambda X \right)\right] \le \exp \left( \frac{\lambda^{2} \sigma^{2}}{2}\right).
\end{equation*}

另一个例子是拉德马赫随机变量(Radmacher random variables)\index{Radmacher random variable \dotfill 拉德马赫随机变量},又称随机符号变量(random sign variable)\index{random sign variable \dotfill 随机符号变量}。
\begin{proposition}[拉德马赫随机变量的切尔诺夫边界]
  \label{proposition:radmacher-sign-variable}
  设拉德马赫随机变量,或称符号随机变量$S$满足关系
 \begin{equation*}
   S =
   \begin{cases}
   1 & \text{概率}\frac{1}{2}, \\
   -1 &   \text{概率}\frac{1}{2},
   \end{cases}
 \end{equation*}
 那么它的MGF可以表示为
  \begin{equation}
    \label{eq:lln-slln-radmacher-mgf}
    M_{S} \left( \lambda \right)
    = E \left[ \exp \left( \lambda S \right) \right]
    \le \exp \left( \frac{\lambda^{2}}{2} \right), \quad \forall \, \lambda \in \mathbb{R},
  \end{equation}
  在此基础上,其切尔诺夫边界为
  \begin{equation}
    \label{eq:lln-slln-radmacher-chernoff}
    P \left( X \ge t \right) \le \exp \left( - \frac{t^{2}}{2 n} \right).
  \end{equation}
\end{proposition}

\begin{proof}
首先来看$S$的MGF计算过程。对$\exp (x)$作泰勒级数展开
\begin{equation*}
  \exp (x) = \sum_{k=0}^{\infty} \frac{x^{k}}{k!}.
\end{equation*}

$S$的取值定义,有
\begin{equation*}
  E S^{k} = \begin{cases}
  0 & \text{k是奇数}, \\
  1 & \text{k是偶数}.
  \end{cases}
\end{equation*}

对应地有S的MGF
\begin{equation*}
  \begin{split}
    M_{s} \left( \lambda \right)
    & = E \exp \left( \lambda S \right) \\
    & = \sum_{k=0}^{\infty} \frac{
    \lambda^{k} E \left[ S^{k} \right]
    }{
    k!
    } \\
    & = \sum_{k=0,2,4,\ldots}^{\infty}
    \frac{\lambda^{k} }{k!} \\
    & = \sum_{k=0}^{\infty} \frac{\lambda^{2k}}{\left(2k\right)!}.
  \end{split}
\end{equation*}

由于
\begin{equation*}
  \left( 2 k \right)! \ge  \left( 2^{k} k! \right), \quad \forall \, k = 0,1,2,\ldots
\end{equation*}

我们进一步有
\begin{equation*}
  \begin{split}
    M_{s} \left( \lambda \right) & = E \exp \left( \lambda S \right) = \sum_{k=0}^{\infty} \frac{\lambda^{2k}}{\left(2k\right)!}\\
    & \le \sum_{k=0}^{\infty}
    \frac{
    \left( \lambda^{2} \right)^{k}
    }{
    2^{k} k!
    } \\
    & \sum_{k=0}^{\infty} \left( \frac{\lambda^{2}}{2} \right)^{k} \, \frac{1}{k!} \\
    & = \exp \left( \frac{\lambda^{2}}{2} \right),
  \end{split}
\end{equation*}
即\eqref{eq:lln-slln-radmacher-mgf}成立。

在此基础上,假定$S$是一组随机符号变量的和
\begin{equation*}
  S = \sum_{i=1}^{n} S_{i}, \quad S_{i} \in \left\{ \pm 1 \right\},
\end{equation*}

由此我们可得其期望值$E S =0$,代入切尔诺夫边界条件\eqref{eq:chernoff-bound-upper}可得
\begin{equation*}
  \begin{split}
    P \left( S \ge t \right)
    & \le
    \min_{\left\{ \lambda \ge 0 \right\}}
    \exp \left( - \lambda t \right) E
    \left[
    \exp
    \left( \lambda S \right)
    \right] \\
    & = \min_{\left\{ \lambda \ge 0 \right\}}
    \exp \left( - \lambda t \right) Eshi
    \left[
    \exp
    \left( \lambda S_{1} \right)^{n}
    \right] \\
    & =
    \min_{\left\{ \lambda \ge 0 \right\}}
    \exp \left( - \lambda t \right) \,
    \exp \left( \frac{n \lambda^{2}}{2} \right),
  \end{split}
\end{equation*}

最小化求解,对应最优系数值$\lambda^{*}$
\begin{equation*}
  \lambda^{*} = \argmin_{\left\{ \lambda \ge 0 \right\}}
  \left(
  \frac{n \lambda^{2}}{2} - \lambda t
  \right) = \frac{t}{n},
\end{equation*}

代回上式可得
\begin{equation*}
  P \left( S \ge t \right) \le \exp \left( - \frac{t^{2}}{2 n}\right),
\end{equation*}
即\eqref{eq:lln-slln-radmacher-chernoff}成立。
\end{proof}

若进一步假设$t = \left( 2 n \log \frac{1}{\delta} \right)^{\frac{1}{2}}$,有
\begin{equation*}
  P \left( S \ge \left( 2 n \log \frac{1}{\delta} \right)^{\frac{1}{2}} \right)
  \le \delta,
\end{equation*}
换句话说存在相当高的概率满足$S = \sum_{i=1}^{n} S_{i} = \mathcal{O} \left( \sqrt{n} \right)$:根据切尔诺夫边界,$n$个独立随机符号之和的分布基本上绝不大于$\mathcal{O} \left( \sqrt{n} \right)$。

现在将变量形式从随机符号变量扩展到一般情况:当一组有界的随机变量之和太大或太小时,其和的概率的边界可用霍夫丁不等式(Hoeffding inequality)计算得到。在介绍霍夫丁不等式之前,首先需要介绍霍夫丁引理(Hoeffding lemma)。
\begin{lemma}[霍夫丁引理]
  \label{lemma:hoeffding-lemma}
  设$X \in [a,b]$是一个有界的随机变量,那么$ \left( X - E X \right)$的MGF的期望值有界,边界为
  \begin{equation}
    \label{eq:hoeffding-lemma}
    E M_{X - E X} \left( \lambda \right)
    = E \left\{
    \exp \left[ \lambda \left( X - E X \right) \right]
    \right\}
    \le \exp
    \left[
    \frac{\lambda^{2} \left( b - a \right)^{2}}{8}
    \right], \quad \forall \, \lambda \in \mathbb{R},
  \end{equation}
  称为霍夫丁引理(Hoeffding lemma)\index{Hoeffding lemma \dotfill 霍夫丁引理}。
\end{lemma}
\begin{proof}
  我们作一个弱形式的证明(在弱形式证明中,RHS分母由8变为2)。证明需要两处背景知识,一是随机符号的MGF边界(Proposition \ref{proposition:radmacher-sign-variable}),二是延森不等式。

  延森不等式(Jensen inequality)\index{Jensen inequality \dotfill 延森不等式}是指,如果$f:\mathbb{R} \mapsto \mathbb{R}$是一个凸方程(convex function),那么有
  \begin{equation}
    \label{eq:lln-slln-jensen-inequality}
    f \left[ E X \right] \le E f(x).
  \end{equation}
  (延森不等式的证明略)\footnote{一个最简单的记忆延森不等式的方法是考虑$f(t)=t^{2}$,此外结合
  \begin{equation*}
    E[X] = 0 \Longrightarrow f \left[ E X \right] = 0,
  \end{equation*}
  ,通常来说$E x^{2} > 0$。$f(t) = \exp \left( t \right)$和$f(t) = \exp \left( - t \right)$都是凸方程。 }。

  在此基础上,我们用对称化(symmetrization)\index{symmetrization (probaility theory) \dotfill 对称化(概率论)}这一概率论中的常用技术来给出证明。首先设$X'$是一个与$X$有相同分布的独立拷贝,满足
  \begin{equation*}
    X' \in \left[a,b \right], \quad E_{X'} X' = E_{X} X, \quad \text{并且X和X'彼此独立}.
  \end{equation*}

  那么我们有
  \begin{equation*}
    \begin{split}
      E_{X} \left\{
      \exp
      \left[
      \lambda
      \left(
      X - E_{X} X
      \right)
      \right]
      \right\}
      & =
      E_{X}
      \left\{
      \exp
      \left[
      \lambda
      \left(
      X - E_{X'} X'
      \right)
      \right]
      \right\} \\
      & \le E_{X} E_{X'}
      \left\{
      \exp \left[
      \lambda
      \left(
      X - X'
      \right)
      \right]
      \right\},
    \end{split}
  \end{equation*}
最后一行由严森不等式推得。上式简化为

\begin{equation}
  \label{eq:lln-slln-jensen-inequality-mid1}
  E_{X} \left\{
  \exp \left[ \lambda \left( X - E X \right)\right]
  \right\}
  \le
  \underbrace{
  E_{X,X'} \left\{
  \exp \left[
  \lambda \left( X - X' \right)
  \right]
  \right\}
  }_{\eqqcolon \mathcal{A}}.
\end{equation}

由$X$和$X'$的性质,根据对称化,$\left( X - X' \right)$围绕着$0$对称。如果$S \in \left\{ -1,1\right\}$是一个随机符号变量,那么$S \left( X - X' \right)$的分布与$\left( X - X' \right)$相同。则RHS进一步改写为
\begin{equation*}
  \begin{split}
    \mathcal{A} & \coloneqq E_{X,X'} \left\{
    \exp \left[
    \lambda \left( X - X' \right)
    \right]
    \right\}
    = E_{X,X',S} \exp \left[ \lambda S \left( X - X' \right) \right] \\
    & = E_{X,X'} \left\{
    \underbrace{
    E_{S}
    \left[
    \exp \left[ \lambda S \left( X - X' \right) \right]
    | X,X'
    \right]
    }_{\eqqcolon \mathcal{B}}
    \right\},
  \end{split}
\end{equation*}

根据随机符号变量的MGF不等式\eqref{eq:lln-slln-radmacher-mgf},上式RHS中$\mathcal{B}$满足
\begin{equation*}
  \mathcal{B} \coloneqq     E_{S}
      \left[
      \exp \left[ \lambda S \left( X - X' \right) \right]
      | X,X'
      \right]
      \le \exp
      \left[
      \frac{\lambda^{2}}{2}
      \left( X - X' \right)^{2}
      \right],
\end{equation*}

即\eqref{eq:lln-slln-jensen-inequality-mid1}改写为
\begin{equation}
  \label{eq:lln-slln-jensen-inequality-mid2}
  E_{X} \left\{
  \exp \left[ \lambda \left( X - E X \right)\right]
  \right\}
  \le E_{X,X'}
  \exp
  \left[
  \frac{\lambda^{2}}{2}
  \left( X - X' \right)^{2}
  \right].
\end{equation}

此外根据假定条件有
\begin{equation*}
  \left\{ \left| X - X' \right| \le \left( b - a \right) \right\}
  \Longleftrightarrow
  \left\{
  \left( X - X' \right)^{2} \le \left( b - a \right)^{2}
  \right\},
\end{equation*}
代回\eqref{eq:lln-slln-jensen-inequality-mid2}得
\begin{equation*}
  \begin{split}
    E_{X} \left\{
    \exp \left[ \lambda \left( X - E X \right)\right]
    \right\}
    & \le E_{X,X'}
    \exp
    \left[
    \frac{\lambda^{2}}{2}
    \left( X - X' \right)^{2}
    \right] \\
    = \exp \left[
    \frac{\lambda^{2} \left( b -a \right)^{2}}{2}
    \right],
  \end{split}
\end{equation*}
证得霍夫丁引理\eqref{eq:hoeffding-lemma}成立。
\end{proof}

在此基础上,来看霍夫丁不等式。
\begin{theorem}[霍夫丁不等式]
  \label{theorem:hoeffding-inequality}
  设一组独立有界随机变量$Z_{1},\ldots,Z_{n}$,边界$Z_{i} \in \left[ a, b\right], \quad \forall \, i =1,\ldots,n, \quad - \infty < a < b < \infty$。那么其分布的边界满足
  \begin{align}
    \label{eq:hoeffding-inequality-upper}
    P \left(
    \frac{1}{n} \sum_{i=1}^{n}
    \left( X_{i} - E X_{i} \right)
    \ge t
    \right)
    & \le \exp \left[
    - \frac{
    2 n t^{2}
    }{
    \left( b - a \right)^{2}
    }
    \right], \\
    \label{eq:hoeffding-inequality-lower}
    P \left(
    \frac{1}{n} \sum_{i=1}^{n}
    \left( X_{i} - E X_{i} \right)
    \le -t
    \right)
    & \le \exp \left[
    - \frac{
    2 n t^{2}
    }{
    \left( b - a \right)^{2}
    }
    \right], \quad \forall \, t > 0,
  \end{align}
称霍夫丁不等式(hoeffding inequality)\index{Hoeffding inequality \dotfill 霍夫丁不等式}。
\end{theorem}
\begin{proof}
  来证明上界不等式条件\eqref{eq:hoeffding-inequality-upper}。结合切尔诺夫上边界条件\eqref{eq:chernoff-bound-upper}和霍夫丁引理式\eqref{eq:hoeffding-lemma}有
  \begin{equation*}
    \begin{split}
      & P \left(
      \frac{1}{n} \sum_{i=1}^{n}
      \left( X_{i} - E X_{i} \right)
      \ge t
      \right)
      = P \left(
      \sum_{i=1}^{n}
      \left( X_{i} - E X_{i} \right)
      \ge n t
      \right)\\
      & \le \min_{\left\{ \lambda \ge 0\right\}}\,
      \exp \left( - \lambda n t \right) \,
      E \left[
      \exp \sum_{i=1}^{n} \left(
      X_{i} - E X_{i}
      \right)
      \right] \\
      & = \min_{\left\{ \lambda \ge 0\right\}} \,
      \exp \left( - \lambda n t \right) \,
      \prod_{i=1}^{n}
      E \left[
      \exp \left(
      X_{i} - E X_{i}
      \right)
      \right] \\
      & \le
      \min_{\left\{ \lambda \ge 0\right\}} \,
      \exp \left( - \lambda n t \right) \,
      \left\{
      \prod_{i=1}^{n}
      \exp
      \left[
      \frac{
      \lambda^{2} \left( b - a \right)^{2}
      }{8}
      \right]
      \right\} \\
      & \le
      \underbrace{
      \min_{\left\{ \lambda \ge 0\right\}} \,
      \exp
      \left\{
      \frac{
      n \lambda^{2} \left( b - a \right)^{2}
      }{
      8
      }
      - \lambda n t
      \right\}
      }_{\eqqcolon \mathcal{A}}.
    \end{split}
  \end{equation*}

对$\mathcal{A}$,求最小化时的系数$\lambda^{*}$
\begin{equation*}
  \begin{split}
    & \lambda^{*} = \argmin \left\{
    \frac{
    n \lambda^{2} \left( b - a \right)^{2}
    }{
    8
    }
    - \lambda n t
    \right\}
    = \frac{4 t }{\left( b - a \right)^{2}}, \\
    & \hookrightarrow \mathcal{A} = \exp \left[
    \frac{- 2 n t^{2}}{
    \left( b - a \right)^{2}
    }
    \right],
  \end{split}
\end{equation*}

因此我们有
\begin{equation*}
  P \left(
  \frac{1}{n} \sum_{i=1}^{n}
  \left( X_{i} - E X_{i} \right)
  \ge t
  \right)
  \le \exp \left[
  \frac{- 2 n t^{2}}{
  \left( b - a \right)^{2}
  }
  \right],
\end{equation*}
上边界条件\eqref{eq:hoeffding-inequality-upper}成立。采用类似的思路,也可证得下边界条件\eqref{eq:hoeffding-inequality-lower}成立。
\end{proof}

\subsection{强均匀大数法则}
\label{sec:lln-sulln}

大数法则的严谨脉络,基本与概率论的范式变化相一致。
\begin{itemize}
\item 在概率论的早期阶段,弱大数法则主要有伯努利、泊松及其相关改进如切比雪夫、马尔科夫等,基本反映早期概率论的核心思想。

\item 随后发展起来的强大数法则如Borel, Cantelli, Kolmogorov等,反映了概率论进一步公理化、正规化,在20世纪上半叶逐渐成为测度论的一部分。

\item 概率论的更新进展,以强一致大数法则为代表,它是1970年代开始,概率论与统计学习相交融的产物,如万普尼克(Vapnik)、泽凡尼杰斯(Chervonekis)等。
\end{itemize}

本节对强一致大数法则作介绍。先来对霍夫丁不等式(Theorem \ref{theorem:hoeffding-inequality})做更为一般的表述,把它写成一个关于独立随机变量的方程,满足差分有界(bounded diffrences):

\begin{definition}[差分有界]
  \label{definition:bounded-diff}
  设一个$n$维
  标准向量空间$\mathbb{R}^{n}$中有一个标准向量(canonical vector)的集合$S \subset \mathbb{R}^{n}$,集合中$e_{i} \in \mathbb{R}^{n}$表示第i个标准向量,向量$e_{i}$中第$i$个元素的值为$1$,其余均为$0$。对于一个方程$h:S \mapsto \mathbb{R}$,如果满足
  \begin{equation}
    \label{eq:lln-sulln-bounded-diff}
    \left| h(x) - h(x + t e_{i}) \right| \le c_{i}, \quad i=1,\ldots,n, \, \forall \, x \in S, \, \forall t \in \mathbb{R}, \, \left( x + t e_{i} \right) \in S,
  \end{equation}
  那么我们称方程$h$的差分有界(bounded differences)\index{bounded differences \dotfill 差分有界},即是说,随着$x$的变化,$h$值的变化不会超过$c_{i}$.
\end{definition}

由差分有界可进一步引出McDiarmid不等式
\begin{theorem}[McDiarmid不等式]
  \label{theorem:mcdiamid-inequality}
  设方程$h$差分有界,则有
  \begin{equation}
    \label{eq:mcdiamid-inequality}
    P \left(
    \left|
    h \left( X_{1}, X_{2}, \ldots, X_{n} \right)
    - E h
    \right| \ge t
     \right)
     \le \exp
     \left(
     - \frac{2 t^{2}}{\sum_{i=1}^{n} c_{i}^{2}}
     \right).
  \end{equation}
\end{theorem}
\begin{proof}
  设
  \begin{equation*}
    h - E h = \sum_{i=1}^{n} Z_{i},
  \end{equation*}
  其中
  \begin{equation*}
    Z_{i} \left( X_{1}, \ldots, X_{i} \right) = E
    \left[
    h | X_{1},\ldots,X_{i}
    \right]
    - E
    \left[
    h | X_{1},\ldots,X_{i-1}
    \right].
  \end{equation*}

  这一组$\left\{ Z_{i} \right\}$的均值为$0$,(几乎总是)有界,对应区间$\left[ L_{i}, U_{i} \right]$。由此可见,上下边界$L_{i}, U_{i}$都之和$X_{1},\ldots,X_{i-1}$有关,并且根据$h$差分有界的假定我们有$U_{i} - L_{i} \le c_{i}$。那么利用切尔诺夫法Theorem \ref{theorem:chernoff-method}和霍夫丁引理Lemma \ref{lemma:hoeffding-lemma}可得,
  \begin{equation}
    \label{eq:mcdiamid-inequality-mid1}
    \begin{split}
      P \left( h - Eh \ge t \right)
      & = P
      \left(
      \exp \left[
      s \left( h - Eh \right)
      \right]
      \ge \exp \left( s t \right)
      \right) \\
      & \min_{\left\{ s \ge 0 \right\}}
      \le \frac{
      E \exp \left[ s \left( h - E h \right) \right]
      }{
      \exp \left( st \right)
      } \\
      & = \min_{\left\{ s \ge 0 \right\}}
      \exp \left( - s t \right) E
      \left[
      \exp \left(
      s \sum_{i=1}^{n} Z_{i}
      \right)
      \right] \\
      & = \min_{\left\{ s \ge 0 \right\}}
      \exp \left( - s t \right)
      E \left\{
      s \sum_{i=1}^{n-1} Z_{i} \,
      E \left[ \exp \left( s Z_{n} | X_{1}, \ldots, X_{n-1} \right) \right]
      \right\} \\
      & = \min_{\left\{ s \ge 0 \right\}}
      \exp \left( - s t \right)
      \exp \left[
      s^{2} \sum_{i=1}^{n} \frac{c_{i}^{2}}{8}
      \right]
    \end{split}
  \end{equation}

求解最小化
\begin{equation*}
  s^{*} = \argmin_{\left\{s \ge 0\right\}}
  \left[
  s^{2} \sum_{i=1}^{n} \frac{c_{i}^{2}}{8} - st
  \right]
  = \frac{4 t}{\sum_{i=1}^{n} c_{i}},
\end{equation*}

代回\eqref{eq:mcdiamid-inequality-mid1}
\begin{equation*}
  P \left( h - Eh \ge t \right)
  \le \exp \left( - s^{*} t \right)
   \exp \left[
   s^{*,2} \sum_{i=1}^{n} \frac{c_{i}^{2}}{8}
   \right]
   = \exp \left\{
   - \frac{2 t^{2}}{\sum_{i=1}^{n} c_{i}}
   \right\},
\end{equation*}
即McDiarmid不等式\eqref{eq:mcdiamid-inequality}成立。
\end{proof}


\begin{corollary}
  当$h = \sum_{i=1}^{n} X_{i}$时,McDiarmid不等式\eqref{eq:mcdiamid-inequality}变为一种特殊情况即霍夫丁不等式\eqref{eq:hoeffding-inequality-upper}-\eqref{eq:hoeffding-inequality-lower}。
\end{corollary}
































































\end{subappendices}
