%!TEX root = ../DSGEnotes.tex
\chapter{Medium-Sized DSGE模型}
\section{产品的生产部门}
\label{sec:production-sector}


\subsection{最终产品生产部门}
\label{sec:final-production-sector}
最终产品生产部门以中间产品$Y_t(i), i \in (0,1)$的组合为投入要素,产出$Y_t$,符合完全竞争假定。\cite{Dixit:1977tv}形式的生产规模报酬不变生产函数为
\begin{equation}
  \label{eq:fin-prod-prod-func-me}
  Y_t = \left[ \int_{0}^{1} Y_t(i)^{\frac{\varepsilon_{f} - 1}{\varepsilon_{f}}} \, di \right]^{\frac{\varepsilon_{f}}{\varepsilon_{f} - 1}},
\end{equation}
其中$\varepsilon_f$表示$i^{th}$中间产品的替代弹性。

最终产品厂商问题:在给定中间产品价格$P_t(i)$的情况下,通过选择$Y_t(i)$的投入追求利润最大化

\begin{equation*}
%  \label{eq:fin-prod-problem-max}
  \max_{Y_t(i)} P_t \cdot Y_t - \int_{0}^{1} P_t(i) \cdot Y_t(i) di.
\end{equation*}

引入式\eqref{eq:fin-prod-prod-func},FOC整理可得对$i$中间产品的需求函数
\begin{equation}
  \label{eq:demand-for-intm-i}
  Y_t(i) = \left( \frac{P_t(i)}{P_t}
\right)^{-\varepsilon_f} \cdot Y_t,
\end{equation}

进而根据完全竞争市场假定
\begin{equation*}
  \begin{split}
    P_t \cdot Y_t & \equiv \int_0^1 P_t(i) \cdot Y_t(i) di \\
    &=\left[\int_0^1 \left(\frac{P_t(i)}{P_t}\right)^{-\varepsilon_f} \cdot Y_t \cdot P_t(i) di \right] \\
    &=\left[\int_0^1 P_t(i)^{1-\varepsilon_f} di \right] \cdot P_t^{\varepsilon_f} \cdot Y_t,\\
  \end{split}
\end{equation*}
整理得最终产品价格的决定(aggregate price index):
\begin{equation}
  \label{eq:agg-price-index-me}
  P_t = \left[
    \int_0^1 P_t(i)^{1-\varepsilon_f} di
  \right]^{\frac{1}{1-\varepsilon_f}}
\end{equation}


\subsection{中间产品生产部门}
\label{sec:intm-production-sector}
每个$i^{th}$中间产品有且只有一个生产者,处于垄断竞争状态,视式\eqref{eq:demand-for-intm-i}-\eqref{eq:agg-price-index}为需求函数和价格;最终产品的总产出$Y_t$和总价格$P_t$是外生的。$i^{th}$中间品生产者的生产函数为
\begin{equation}
  \label{eq:intm-prodc-func-me}
  Y_t(i) = K_t(i)^{\alpha} \cdot \left[ z_t \cdot H_{t}(i) \right]^{1-\alpha}  - z_t^+ \cdot \varphi,
\end{equation}
其中投入要素$H_t(i)$和$K_{t}(i)$分别表示同质的劳动力和实物资本服务,产出系数$0<\alpha<1$。$z_t$和$\Psi_t$分别表示neutral和investment-specific类型的随机技术冲击,见第\ref{sec:total-output-allocation}节。$\varphi$为生产的固定成本;$z_t^+$定义如下
\begin{equation}
  \label{eq:ztplus-zt-Psi}
  z_t^+ = \Psi_t^{\frac{\alpha}{1-\alpha}} \cdot z_t,
\end{equation}
可见沿着非随机的稳态增长路径,$Y_{t}(i)/z_t^+$收敛至常数\footnote{膏按:模型引入固定成本的设定,以及采取一个变量乘以一个常量形式设定的考虑。模型假定中间产品生产者处于垄断地位,且假定没有进出(no entry and exit),以确保垄断利润得以长期持续,因此在中间产品生产函数中引入常数固定生产成本$\varphi$,使得稳态利润是零,被固定成本抵消,维持no entry的假设成立。但这需要fixed cost的增速与产出增速相等,由此对常数$\varphi$额外乘以一个系数$z_t^+$。此外经验研究的发现也肯定了这种设定的意义:在引入规模报酬递增的情况下,正的货币政策冲击产生后,劳动的生产率随着提高,这与对现实的观测基本接近。}。

$i^{th}$中间产品生产者雇佣同质劳动服务$H_{t}(i)$,假定生产者必须100\%借款来支付工资,根据\cite{Christiano:2005ib}的working capital channel设定加以简化$(v_t = 0, \psi = 1)$,每一单位劳动服务的成本等于
\begin{equation*}
  cost = (1-v_t) \cdot (1-\psi + \psi \cdot R_t) \cdot W_t = R_t \cdot W_t,
\end{equation*}
其中$W_t$和$R_t$分别表示总量水平上的名义工资和working capital借贷的名义利率\footnote{Working Capital Channel可追溯至\cite{Basu:1995vl}。\cite{Christiano:2005ib}对资本和劳动服务的working capital做了建模。近期研究可见\cite{Phaneuf:2015vp}。根据不含working capital channel的DSGE模型模拟出来的结果,对经济体注入正的货币政策冲击会导致通货膨胀率有大幅度的提升,而现实世界中通货膨胀并没有上升的如此剧烈。\cite{Christiano:2005ib}因此引入了working capital channel设定以缓解该问题,其思路如下:正的货币政策冲击降低了名义利率,进而降低企业进行外部融资(支付员工工资)的边际成本,而边际成本对企业价格决策的制定至关重要。一系列对经济现实的VAR-based观察发现,扩张性货币政策冲击出现后的初期,通货膨胀率出现小幅的下降。这从侧面印证了working capiutal channel假设的重要性,除了可能存在计量意义上的设定失误之外,恐怕只有working capital channel的假设可以解释它了。计量意义上的设定失误,\cite{Sims:1992bw}最早提出这种可能;\cite{Christiano:1999uw}进一步探讨了该可能性;另外可见\cite{Bernanke:2005vn}。}。



经济体受到的外部技术冲击有二,分别为$z_t$和$\Psi_t$,均假定为对数形式的单位根过程\footnote{膏按:式\eqref{eq:tech-shock-zt}直接将中性技术进步定义为有drift的随机游走过程,依据如下:
  \begin{enumerate}
  \item estimation部分,\cite{Smets:2003ik}估计$\log z_t$发现它高度自相关,
  \item \cite{Prescott:1986ej}的经验研究 ,
  \item \cite{Fernald2009}对美国经济部门的经验研究发现,1947年第2季度到2009年第三季度TFP增速的一阶自回归系数为0.0034。
  \end{enumerate}
}:
\begin{equation}
  \label{eq:tech-shock-zt}
  \Delta \log z_t = \mu_z + \varepsilon_{z,t}, E(\varepsilon_{z,t}) = \sigma^2_{z},
\end{equation}
\begin{equation}
  \label{eq:tech-shock-Psi}
  \Delta \log \Psi_t = (1-\rho_{\Psi}) \cdot \mu_{\Psi} + \rho_{\Psi} \cdot \Delta \log \Psi_{t-1} + \varepsilon_{\Psi, t}, E(\varepsilon_{\Psi,t}) = \sigma^2_{\Psi}.
\end{equation}


\subsection{中间产品生产者的边际成本}
\label{seq:marg-cost-intm-producer}
$i^{th}$中间产品生产者的最优行为表现为两阶段优化。第一阶段是成本最小化,选择合适的投入要素$\{K_{t}(i), H_{t}(i)\}$数量,生产出满足最终产品生产部门需求的$Y_t(i)$。投入要素价格分别为市场给定的$\bar{R_t}$及$\bar{W}_t$,
\begin{align*}
  \min_{K_{t}(i), H_{t}(i)} &\bar{W}_t \cdot H_t(i) + \bar{R}_t \cdot K_{t}(i), \\
  & st. \quad  Y_t(i) = K_t(i)^{\alpha} \cdot \left[ z_t \cdot H_{t}(i) \right]^{1-\alpha}  -  z_t^+ \cdot \varphi.
\end{align*}

建Lagrangian
\begin{equation*}
  \mathcal{L} = \bar{W}_t \cdot H_t(i) + \bar{R}_t \cdot K_{t}(i) +\lambda_t \cdot \left\{Y_t(i) - K_t(i)^{\alpha} \cdot \left[ z_t \cdot H_{t}(i) \right]^{1-\alpha}  +  z_t^+ \cdot \varphi \right\}.
\end{equation*}

FOCs:
\begin{align}
  \label{eq:intm-marg-min-cost-FOC-H}
  \frac{\partial \mathcal{L}}{\partial H_{t}(i)} = 0 &\Rightarrow \bar{W}_t = \lambda_t \cdot (1-\alpha) \cdot z_t^{1-\alpha} \cdot H_{t}(i)^{-\alpha} \cdot K_{t}(i)^{\alpha},\\
  \label{eq:intm-marg-min-cost-FOC-K}
\frac{\partial \mathcal{L}}{\partial K_{t}(i)} = 0 &\Rightarrow \bar{R}_t = \lambda_t \cdot \alpha \cdot z_t^{1-\alpha} \cdot H_{t}(i)^{1-\alpha} \cdot K_{t}(i)^{\alpha-1}.
\end{align}

式\eqref{eq:intm-marg-min-cost-FOC-H}-  \eqref{eq:intm-marg-min-cost-FOC-K}可得
\begin{equation}
  \label{eq:intm-marg-min-cost-FOC-W-R}
  \frac{\bar{W}_t}{\bar{R}_t} = \left(\frac{\alpha}{1-\alpha}\right)^{-1} \cdot \left(\frac{K_{t}(i)}{H_{t}(i)}\right).
\end{equation}

由式\eqref{eq:intm-marg-min-cost-FOC-W-R}可得 \begin{equation}
\label{eq:intm-homo-K-H-no-i} \frac{K_t(i)}{H_{t}(i)} \equiv \frac{K_t}{H_t},
\quad \forall i, \end{equation} 即所有中间产品生产者以及总量层面上的投入要素之比相等。

将式\eqref{eq:intm-marg-min-cost-FOC-W-R}带回生产函数式\eqref{eq:intm-prodc-func},得
\begin{equation*}
  Y_t(i) + z_t^+ \cdot \varphi
= z_t^{1-\alpha} \cdot \left(\frac{\bar{W}_t}{\bar{R}_t}\right)^{-(1-\alpha)} \cdot \left( \frac{\alpha}{1-\alpha} \right)^{-(1-\alpha)} \cdot K_{t}(i)
= z_t^{1-\alpha} \cdot \left(\frac{\bar{W}_t}{\bar{R}_t}\right)^{\alpha} \cdot \left( \frac{\alpha}{1-\alpha} \right)^{\alpha} \cdot H_{t}(i).
\end{equation*}

进而最优投入要素的数量
\begin{align*}
%  \label{eq:intm-marg-min-cost-K}
  K_t(i) &= \left[ Y_t(i) + z_t^+ \cdot \varphi \right] \cdot z_t^{-(1-\alpha)} \cdot \left(\frac{\bar{W}_t}{\bar{R}_t}\right)^{1-\alpha} \cdot \left( \frac{\alpha}{1-\alpha} \right)^{1-\alpha}, \\
%  \label{eq:intm-marg-min-cost-H}
  H_t(i) &= \left[ Y_t(i) + z_t^+ \cdot \varphi \right] \cdot z_t^{-(1-\alpha)} \cdot \left(\frac{\bar{W}_t}{\bar{R}_t}\right)^{-\alpha} \cdot \left( \frac{\alpha}{1-\alpha} \right)^{-\alpha}.
\end{align*}

$i^{th}$中间产品生产者的成本函数
\begin{equation}
  \label{eq:intm-prodc-cost}
  C(Y_t(i),K_t(i),H_t(i)) =\bar{W}_t \cdot H_t(i) + \bar{R}_t \cdot K_{t}(i) = \left(Y_t(i) + z_t^+ \cdot \varphi \right) \cdot z_t^{-(1-\alpha)} \cdot \left(\frac{\bar{W}_t}{1-\alpha}\right)^{1-\alpha} \cdot \left(\frac{\bar{R}_t}{\alpha}\right)^{\alpha}.
\end{equation}

因此名义边际成本$S_t(i)$等于
\begin{equation}
  \label{eq:intm-nominal-marg-cost}
  S_t(i) = \frac{\partial C(Y_t(i),K_t(i),H_t(i))}{\partial Y_t(i)} = z_t^{-(1-\alpha)} \cdot \left(\frac{\bar{W}_t}{1-\alpha}\right)^{1-\alpha} \cdot \left(\frac{\bar{R}_t}{\alpha}\right)^{\alpha},
\end{equation}
类似地,$S_t(i) \equiv S_t, \forall i$。

根据定义,投入要素的市场价格
\begin{align}
  \label{intm-mkt-price-real-price-W}
  \bar{W}_t &= (1-v_t) \cdot (1-\psi + \psi \cdot R_t) \cdot W_t,\\
  \label{intm-mkt-price-real-price-R}
  \bar{R}_t &= (1-v_t) \cdot (1-\psi + \psi \cdot r_t^k) \cdot P_t
\end{align}

引入$v_t=0,\psi = 1$的设定后,实际边际成本$s_t$等于
\begin{align}
  \label{intm-mkt-real-mrg-cost}
s_t \equiv \frac{S_t}{P_t} = \frac{\left(\frac{R_t \cdot W_t}{1-\alpha}\right)^{1-\alpha} \cdot \left(\frac{r_t^k \cdot P_t}{\alpha}\right)^{\alpha}}{z_t^{1-\alpha} \cdot P_t}.
\end{align}

引入式  \eqref{eq:real-rental-rate-capital}、  \eqref{eq:scaled-real-wage}、\eqref{eq:ztplus-zt-Psi}后,式\eqref{intm-mkt-real-mrg-cost}改写为scaled形式
\begin{equation}
  \label{eq:intm-mkt-real-mrg-cost-scaled}
  s_t = \frac{\left(\frac{R_t \cdot \left(\bar{w}_t \cdot z_t^+ \cdot P_t \right)}{1-\alpha}\right)^{1-\alpha} \cdot \left(\frac{\left(\frac{\bar{r}_t^k}{\Psi_t}\right) \cdot P_t}{\alpha}\right)^{\alpha}}{z_t^{1-\alpha} \cdot P_t} = \left( \frac{\bar{w}_t \cdot R_t} {1-\alpha}\right)^{1-\alpha} \cdot \left( \frac{\bar{r}_{t}^k}{\alpha}\right)^{\alpha}.
\end{equation}

此外,根据productive efficiency,生产额外1单位$Y_t(i)$的边际成本$s_t$,等于雇佣额外1单位劳动的成本$W_t \cdot R_t$,除以劳动的名义边际产出$P_t \cdot \frac{\partial Y_t(i)}{\partial H_{t}(i)}$,即
\begin{equation}
  \label{eq:intm-mkt-real-mrg-cost-efficiency}
\begin{split}
  s_t
  &= \frac{W_t \cdot R_t}{P_t \cdot \frac{\partial Y_t(i)}{\partial H_t(i)}} \\
  &= \frac{\left(\bar{w}_t \cdot z_t^+ \cdot P_t\right) \cdot R_t}{(1-\alpha) \cdot \left[\frac{K_t(i)}{H_{t}(i)}\right]^{\alpha} \cdot z_t^{1-\alpha} \cdot P_t} \\
  &= \frac{\bar{w}_t \cdot R_t \cdot z_t^+ \cdot P_t}{(1-\alpha) \cdot \left[\frac{\frac{k_t(i)}{z_{t-1}^+ \cdot \Psi_{t-1}}}{H_{t}(i)}\right]^{\alpha} \cdot \left(z_t^+ \cdot \Psi^{-\frac{\alpha}{1-\alpha}}\right)^{1-\alpha} \cdot P_t} \\
      &=\frac{\bar{w}_t \cdot R_t}{(1-\alpha) \cdot \left[\frac{k_{t}(i)}{H_t(i) \cdot \mu_{z^{+},t} \cdot \mu_{\Psi, t}}\right]^{\alpha}}\\
        &= \frac{\bar{w}_t \cdot R_t \cdot z_t^+ \cdot P_t}{(1-\alpha) \cdot \left[\frac{\frac{k_t(i)}{z_{t-1}^+ \cdot \Psi_{t-1}}}{H_{t}(i)}\right]^{\alpha} \cdot \left(z_t^+ \cdot \Psi^{-\frac{\alpha}{1-\alpha}}\right)^{1-\alpha} \cdot P_t} \\
      &=\frac{\bar{w}_t \cdot R_t}{(1-\alpha) \cdot \left[\frac{k_{t}}{H_t \cdot \mu_{z^{+},t} \cdot \mu_{\Psi, t}}\right]^{\alpha}}.
\end{split}
\end{equation}
其中最后一个等号利用式\eqref{eq:output-scaled}消除异质性。



\subsection{中间产品生产者的定价}
\label{sec:intm-pricing-calvo}

\subsubsection{updating生产者的定价策略}
\label{seq:intm-updating-pricing}

价格刚性。假定在$t$时期,有$0 \le \phi_f \le 1$比例的中间产品生产者$i \in [0,1]$无法调整价格;有$1-\phi_f$比例的生产者不可以调整价格。可以调整价格的中间产品垄断生产者,其forward-looking问题可以描述为,制定合理的垄断价格$P_t^{\#}(i)$以满足利润最大化
\begin{equation}
  \label{eq:intm-pricing-problem-1}
  \max_{P_{t}^{\#}(i)} E_t \sum_{j=0}^{\infty}\left( \phi_f \cdot \beta \right)^j \cdot v_{t+j} \cdot \left[P_{t}^{\#}(i) \cdot Y_{t+j}(i) - P_{t+j} \cdot s_{t+j} \cdot Y_{t+j}(i)\right],
\end{equation}
st. 对$i^{th}$中间产品的需求式\eqref{eq:demand-for-intm-i}。$s_t$的定义见式  \eqref{eq:intm-mkt-real-mrg-cost-scaled}。中括号中的内容为$t+j$时刻$i^{th}$生产者的利润;$\beta^j \cdot v_{t+j}$为家庭部门跨期预算约束条件的乘数\footnote{中间产品生产部门的垄断利润,假设全部返回家庭部门;家庭部门基于自己的消费偏好作跨期消费决策;因此对于$i^{th}$来说,$\beta^j \cdot v_{t+j}$是外生给定的。},满足$v_{t}=\frac{\partial U_t(\cdot) / \partial C_t}{P_t}$。\footnote{膏按:补一个eqref,嵌到hh部门的max problem上去。}

引入式\eqref{eq:demand-for-intm-i}替代$Y_{t+j}(i)$,整理得
\begin{align*}
%  \label{eq:intm-pricing-problem-2}
  &\max_{P_{t}^{\#}(i)} E_t \sum_{j=0}^{\infty}\left( \phi_f \cdot \beta \right)^j \cdot
\left(\frac{U_{C,t+j}}{P_{t+j}}\right) \cdot
\left(
  \frac{P_{t+j}^{\#}(i)}{P_{t+j}}
\right)^{-\varepsilon_f} \cdot Y_{t+j} \cdot
\left[P_{t}^{\#}(i)  - P_{t+j} \cdot s_{t+j} \right]\\
=&\max_{P_{t}^{\#}(i)} E_t \sum_{j=0}^{\infty}\left( \phi_f \cdot \beta \right)^j \cdot
\left(
U_{C,t+j} \cdot Y_{t+j}
\right) \cdot
\left[P_{t}^{\#}(i)^{-(\varepsilon_f - 1)}  \cdot P_{t+j}^{\varepsilon_f -1}- P_{t}^{\#}(i)^{-\varepsilon_f} \cdot P_{t+j}^{\varepsilon_f} \cdot s_{t+j} \right].
\end{align*}

FOC wrt $P_{t}^{\#}(i)$,
\begin{align*}
&  \left[-(\varepsilon_f - 1)\right] \cdot
\max_{P_{t}^{\#}(i)} E_t \sum_{j=0}^{\infty}\left( \phi_f \cdot \beta \right)^j \cdot
\left(
U_{C,t+j} \cdot Y_{t+j}
\right) \cdot
\left[
  P_{t}^{\#}(i)^{-\varepsilon_f} \cdot P_{t+j}^{\varepsilon_f - 1}
\right] \\
=& -\varepsilon_f \cdot
\max_{P_{t}^{\#}(i)} E_t \sum_{j=0}^{\infty}\left( \phi_f \cdot \beta \right)^j \cdot
\left(
U_{C,t+j} \cdot Y_{t+j}
\right) \cdot
\left[
  P_{t}^{\#}(i)^{-\varepsilon_f - 1} \cdot P_{t+j}^{\varepsilon_f} \cdot s_{t+j}
\right]
\end{align*}

整理得
\begin{equation}
  \label{eq:intm-calvo-pricing-1}
  P_{t}(i)^{\#} = \frac{\varepsilon_f}{\varepsilon_f -1} \cdot \frac
{
  E_t \sum_{j=0}^{\infty}\left( \phi_f \cdot \beta \right)^j \cdot
\left(
U_{C,t+j} \cdot Y_{t+j}
\right) \cdot
\left(
P_{t+j}^{\varepsilon_f} \cdot s_{t+j}
\right)
}
{
  E_t \sum_{j=0}^{\infty}\left( \phi_f \cdot \beta \right)^j \cdot
\left(
U_{C,t+j} \cdot Y_{t+j}
\right) \cdot
\left(
P_{t+j}^{\varepsilon_f - 1}
\right)
}.
\end{equation}
式\eqref{eq:intm-calvo-pricing-1}的RHS与个体企业$i^{th}$无关;因此$P_t(i)^{\#} \equiv P_t^{\#}, \forall i$。

定义两个辅助变量$X_{1,t}^f$,$X_{2,t}^f$
\begin{align}
\label{eq:intm-prod-auxiliary-X-1}
  X_{1,t}^f &\equiv   E_t \sum_{j=0}^{\infty}\left( \phi_f \cdot \beta \right)^j \cdot
\left(
U_{C,t+j} \cdot Y_{t+j}
\right) \cdot
\left(
P_{t+j}^{\varepsilon_f} \cdot s_{t+j}
\right)
=U_{C,t} \cdot Y_{t} \cdot P_{t}^{\varepsilon_f} \cdot s_t + \phi_f \cdot \beta \cdot E_t X_{1,t+1}^f,\\
\label{eq:intm-prod-auxiliary-X-2}
  X_{2,t}^f &\equiv   E_t \sum_{j=0}^{\infty}\left( \phi_f \cdot \beta \right)^j \cdot
\left(
U_{C,t+j} \cdot Y_{t+j}
\right) \cdot
\left(
P_{t+j}^{\varepsilon_f-1}
\right)=U_{C,t} \cdot Y_{t} \cdot P_{t}^{\varepsilon_f-1} + \phi_f \cdot \beta \cdot E_t X_{2,t+1}^f.
\end{align}


进一步对辅助变量作scaling
\begin{equation}
\label{eq:intm-prod-auxiliary-x-1}
\begin{split}
  x_{1,t}^f &\equiv   \frac{X_{1,t}^f}{P_t^{\varepsilon_f}} \\
&=U_{C,t} \cdot Y_{t} \cdot s_t + \phi_f \cdot \beta \cdot E_t (1+\pi_{t+1})^{\varepsilon_f} \cdot x_{1,t+1}^f,\\
&= \psi_{z^+,t} \cdot y_t \cdot s_t + \phi_f \cdot \beta \cdot E_t (1+\pi_{t+1})^{\varepsilon_f} \cdot x_{1,t+1}^f.
\end{split}
\end{equation}

\begin{equation}
\label{eq:intm-prod-auxiliary-x-2}
\begin{split}
  x_{2,t}^f &\equiv   \frac{X_{2,t}}{P_t^{\varepsilon_f -1}} \\
  &=U_{C,t} \cdot Y_{t} + \phi_f \cdot \beta \cdot E_t (1+\pi_{t+1})^{\varepsilon_f -1} \cdot x_{2,t+1}^f \\
  &=\psi_{z^+,t} \cdot y_t  + \phi_f \cdot \beta \cdot E_t (1+\pi_{t+1})^{\varepsilon_f -1} \cdot x_{2,t+1}^f.
\end{split}
\end{equation}
其中上两式的最后一个等号根据式\eqref{eq:HH-max-FOC-C-intm}、\eqref{eq:scaled-product}和\eqref{eq:scaled-produc-cost-adj-coef}求得
\begin{equation}
\label{eq:U-C-t-Y-t}
U_{C,t} \cdot Y_t = \left(v_t \cdot P_t \right) \cdot \left(z^+_t \cdot y_t \right) = \psi_{z^+,t} \cdot y_t,
\end{equation}

式\eqref{eq:intm-calvo-pricing-1}因此变为
\begin{equation}
  \label{eq:intm-calvo-pricing-2}
  P_t^{\#} = \frac{\varepsilon_f}{\varepsilon_f - 1} \cdot \frac{X_{1,t}^f}{X_{2,t}^f} = \frac{\varepsilon_f}{\varepsilon_f - 1} \cdot \frac{x_{1,t}^f}{x_{2,t}^f} \cdot P_t.
\end{equation}

定义reset price inflation $1+\pi_t^{\#} \equiv \frac{P_t^{\#}}{P_{t-1}}$,则  式\eqref{eq:intm-calvo-pricing-2}两侧同时除以$P_{t-1}$得
  \begin{equation}
    \label{eq:intm-calvo-pricing-3-scaling}
    (1+\pi_t^{\#}) = \frac{\varepsilon_f}{\varepsilon_f -1} \cdot (1 + \pi_{t}) \cdot \frac{x_{1,t}^f}{x_{2,t}^f}.
  \end{equation}


\subsubsection{Aggregate Price Index}
\label{intm-aggregate-price-index}

利用Calvo pricing方法\citep{Calvo:1983uqa},aggregate price index式\eqref{eq:agg-price-index}可以改写为
\begin{equation*}
  P_t^{1-\varepsilon_f} = \int_{0}^1 P_t(i)^{1-\varepsilon_f} di = \int_{0}^{1-\phi_f} P_t^{\#,1-\varepsilon_f} di + \int_{1-\phi_f}^{1}p_{t-1}(i)^{1-\varepsilon_f} di = (1-\phi_f) \cdot P_t^{\#,1-\varepsilon_f}  + \phi_f \cdot P_{t-1}^{1-\varepsilon_f}.
\end{equation*}

等式两侧同时除以$P_{t-1}^{1-\varepsilon_f}$,整理得
\begin{equation}
  \label{eq:intm-prod-agg-price-idx-calvo}
  (1+\pi_t^{\#}) = \left[\frac
{
  (1+\pi_t)^{1-\varepsilon_f} - \phi_f
}
{
  1-\phi_f
}\right]^{\frac{1}{1-\varepsilon_f}}.
\end{equation}

式\eqref{eq:intm-calvo-pricing-3-scaling}与式\eqref{eq:intm-prod-agg-price-idx-calvo}联立,整理可得
\begin{equation}
    \label{eq:intm-prod-agg-price-idx-calvo-aux}
    \left[
      \frac{
        1-\phi_f \cdot (1+\pi_t)^{\varepsilon_f -1}
      }{
        1-\phi_f
      }
    \right]^{\frac{1}{1-\varepsilon_f}} = \frac{\varepsilon_f}{\varepsilon_f -1} \cdot \frac{x_{1,t}^f}{x_{2,t}^f}.
\end{equation}

\subsection{price dispersion index}
\label{seq:intm-price-dispersion}
Medium-sized DSGE模型中,同质总产出$Y^{sum}_t$是$i^{th}$种中间产品产出$Y_t(i)$之和,结合式\eqref{eq:intm-prodc-func-me}可得
\begin{equation}
\label{eq:sum-output}
\begin{split}
Y^{sum}_t &= \int^1_0 Y_t(i) di \\
&=\int^1_0 \left[\left(z_t \cdot H_{t}(i)\right)^{1-\alpha} \cdot K_{t}(i)^{\alpha} - z^+_t \cdot \varphi \right] di \\
&=\int^1_0 \left[z_t^{1-\alpha} \cdot \left(\frac{K_{t}(i)}{H_{t}(i)}\right)^{\alpha} \cdot H_{t}(i) - z^+_t \cdot \varphi \right] di \\
&=z_t^{1-\alpha} \cdot \left(\frac{K_{t}}{H_{t}}\right)^{\alpha} \cdot \int_0^1 H_t(i) di - z^+_t \cdot \varphi \\
&= z_t^{1-\alpha} \cdot K_t^{\alpha} \cdot H_t^{1-\alpha} - z^+_t \cdot \varphi,
\end{split}
\end{equation}
其中$K_t$和$H_t$分别表示经济体中同质资本服务品和同质劳动力的数量。成本最小化的中间产品企业面临相同的要素价格,因此致力于雇佣同等比例的资本-劳动投入品进行生产,最后一个等号因此消除了企业$i$的异质性。市场出清情况下对中间产品的总需求式\eqref{eq:demand-for-intm-i}等于总供应式\eqref{eq:sum-output},整理得
\begin{equation}
  \label{eq:intm-price-idx-nu}
  Y_t = \frac{
    z_t^{1-\alpha} \cdot K_t^{\alpha} \cdot H_t^{1-\alpha} -z_t^+ \cdot \varphi
  }{
    \int_0^1 \left(\frac{p_t(i)}{p_t}\right)^{-\varepsilon_f} di
  }=\frac{
    z_t^{1-\alpha} \cdot K_t^{\alpha} \cdot H_t^{1-\alpha} -z_t^+ \cdot \varphi
  }{
    \nu_t^f
  }.
\end{equation}
其中定义了产品价格的分布指标$\nu_t^f \ge 1$。当$\nu_t^f = 1$时,不存在price dispersion,实际总产出最大。利用calvo pricing可得
\begin{align}
  \label{eq:prod-price-dispersion-calvo}
  \nu_t^f & \equiv \int_0^1 \left( \frac{P_t(i)}{P_t} \right)^{-\varepsilon_f} di \nonumber \\
           &=\int_{0}^{1-\phi_f} \left( \frac{P_t^{\#}}{P_t}\right)^{-\varepsilon_f} di + \int_{1-\phi_f}^{1} \left( \frac{P_{t-1}(i)}{P_t} \right)^{-\varepsilon_f} di \nonumber \\
&=(1-\phi_f) \cdot \left[
  \frac{
  1+\pi_t
  }{
  1+\pi_t^{\#}
  }
\right]^{\varepsilon_f}
+ \phi_f \cdot (1+\pi_t)^{\varepsilon_f}.
\end{align}

式\eqref{eq:intm-prod-agg-price-idx-calvo}与式\eqref{eq:prod-price-dispersion-calvo}联立得
\begin{equation}
  \label{eq:price-dispersion-index-iter}
  \nu_t^f = (1-\phi_f) \cdot
\left[\frac
{
  1-\phi_f \cdot (1+\pi_t)^{\varepsilon_f -1}
}
{
  1-\phi_f
}\right]^{\frac{\varepsilon_f}{\varepsilon_f-1}} + \phi_f \cdot (1+\pi_t)^{\varepsilon_f} \cdot \nu_{t-1}^f.
\end{equation}

此外,联立式\eqref{eq:scaled-product}、\eqref{eq:intm-price-idx-nu}和\eqref{eq:K-bar-K-ut-bar-k}可得
\begin{equation}
  \label{eq:output-scaled}
\begin{split}
  y_t &\equiv \frac{Y_t}{z_t^+} = \frac{1}{\nu_t^f} \cdot \left[ \frac{z_t^{1-\alpha} \cdot K_t^{\alpha} \cdot H_t^{1-\alpha}}{z_t^+} - \varphi \right] \\
  &=
\end{split}
\end{equation}

结合式\eqref{eq:K-bar-K-ut-bar-k}、式\eqref{eq:H-h-relationship-wage-dispersion},可将式\eqref{eq:output-scaled}变为如下scale形式
\begin{equation}
y_t = \frac{1}{\nu^f_t} \cdot \left[
  \left(
  \frac{u_t \cdot \bar{k}_t}{\mu_{z^+,t} \cdot \mu_{\Psi,t}}
  \right)^{\alpha}
  \cdot
  \left(
  h_t \cdot \nu^w_t
  \right)^{1-\alpha}
  -\varphi
\right]
\end{equation}




\subsection{投资的调节成本}
\label{sec:adjustment-cost}
$F(I_t,I_{t-1})$表示投资的调节成本,定义为
\begin{equation}
  \label{eq:adjustment-cost-func-CEE}
  F(I_t,I_{t-1}) = \left[1-S\left( \frac{I_t}{I_{t-1}} \right) \right] \cdot I_t,
\end{equation}
$S(I_t,I_{t-1})$表示为投资的调节成本,也有用实物资本的调节形式表现的。adjustment cost的详细讨论见第\ref{sec:adjustment-cost-types-compar}节。隐函数$S(\cdot)$满足$S(1)=S'(1)=0$,$S''=\kappa$是个常数,即DSGE模型的稳定状态与$\kappa$系数值无关。


%\subsection{资源约束条件}
%\label{sec:resource-constraint}

\subsection{总产出的分配}
\label{sec:total-output-allocation}
市场出清条件下总产出的分配
\begin{equation}
  \label{eq:output-expenditures}
  Y_t = C_t + \tilde{I}_t + G_t,
\end{equation}
其中$G_t$表示外生的政府支出。$C_t$表示家庭消费支出。同质的投资品$\tilde{I}_t$定义式如下
\begin{equation}
  \label{eq:investment-goods-homo}
  \tilde{I}_t \equiv \frac{I_t + a(u_t) \cdot \bar{K}_t}{\Psi_t},
\end{equation}
同质投资品$\tilde{I}_t$用于形成投资品$I_t$。$I_t$被家庭部门用于增加下一个时间期的实物资本存量$\bar{K}_{t+1}$。$a(u_t) \cdot \bar{K}_t$是实物资本的维护成本,$0 \le u_t \le 1$是可变的资本利用率,反映家庭的variable capital utilization决策,递增且convex的成本函数$a(u_t)$表示对应于$u_t$,利用实物资本存量从事生产的成本;在稳定状态下,家庭会设$u=1$即实物资本$\bar{K}_t$全部用于生产活动,且成本函数$a(u) = 0$。

unit root的investment specific technology shock $\Psi_t$ 程度越大,同等单位$\tilde{I}_t$所能形成的投资品$I_t$越多。$\Psi_t$的定义式见  \eqref{eq:tech-shock-Psi}。

实物资本服务品$K_t$与实物资本存量$\bar{K}_t$和利用率$u_t$有关:
\begin{equation}
  \label{eq:physical-capital-services}
  K_t \equiv u_t \cdot \bar{K}_t.
\end{equation}

对式\eqref{eq:physical-capital-services}作scale调整。根据式\eqref{eq:physical-capital-services}和\eqref{eq:scaled-physical-capital}得
\begin{equation}
\label{eq:K-bar-K-ut-bar-k}
\begin{split}
K_t &= u_t \cdot \bar{K}_t \\
&=u_t \cdot \left(\bar{k}_t \cdot z^+_{t-1} \cdot \Psi_{t-1}\right) \\
&= \frac{u_t \cdot \bar{k}_t \cdot z^+_t \cdot \Psi_t}{\mu_{z^+,t} \cdot \mu_{\Psi,t}}.
\end{split}
\end{equation}

进而总产出分配式\eqref{eq:output-expenditures}的scale形式如下
\begin{equation}
\label{eq:output-expenditures-scaled}
%\begin{split}
y_t = g_t + c_t + i_t + \frac{
  a(u_t) \cdot \bar{k}_t
  }{\mu_{z^+,t} \cdot \mu_{\Psi,t}}.
%\end{split}
\end{equation}

\subsubsection{可变的资本利用率}
\label{sec:variable-capital-utili}
如果$u_t=u$即资本利用率是个常数,模型模拟出来的结果显示,通货膨胀率会随着外生货币政策冲击而发生很大波动;然而在实际经济运行过程中观测,往往通货膨胀的波动并没有那么大。在模型中引入可变的资本利用率$u_t$,有助于解释通货膨胀对货币政策冲击的响应速度为何缓慢:价格很大程度上由生产成本所决定;生产成本又受到生产要素弹性的影响;如果弹性很大,则一个较小的成本上升即可导致较大的投入要素量变化,从而使得通货膨胀率对货币冲击的响应缓慢且温和。建模时使投入要素弹性变大的方法有很多,其一便是让实物资本利用率是可变的,可以使实物资本服务品更有弹性——如果$a()$函数的曲率很低(very little curvature inthe $a$ function),那么家庭部门可以在确保成本不大幅度提高的情况下,增加资本服务品的供应。

\section{家庭部门}
\label{sec:household-sector}
家庭部门拥有全部生产资料(劳动力和实物资本),向生产部门供应生产资料以获得收入(工资和垄断利润)。假定劳动力市场上存在具有不同特征的异质劳动力$h_t(j)$,$j \in (0,1)$,且对于$j' \neq j$,$h_t(j)$和$h_t(j')$(至少部分地)可以相互替代。引入工资粘性的设定\citep{Erceg:2000dm},存在一个垄断者为每一种$h_t(j)$分别定价,并且由于可替代性的存在,其垄断定价的能力受到市场竞争的限制\citep{Christiano:2010wla}。

\subsection{劳动力投入:同质化假定还是异质化假定?}
\label{sec:interpretation-H}
家庭的效用函数中引入劳动力投入$h_t^{1+\eta}$,反映休闲带来的效用(或者劳动带来的负效用)。幂指数的倒数$1/\eta$反映在保持消费水平不变的情况下,劳动力供应相对于真实工资水平的弹性。宏观经济学研究中$h_t$进而$\eta$反映何种含义,引起广泛争论。

一种观点是假定经济体中的家庭部门都是同质的,$h_t$反映了用就业市场上一个典型劳动者工作时间(小时数),体现了家庭的labor effort。此时$1/\eta$系数又称Frisch elasticity of labor supply\footnote{见附录\ref{sec:Frish-elasticity}。},描述典型劳动者随着工资的变化,愿意增加或减少的工作时间。

另一种做法是仍然假定家庭部门的同质性,$h_t$反映了劳动力市场上的就业人数。$1/\eta$描述了额外一个边际劳动者,随着工资的变化,愿意进入或是离开劳动力市场;它不反映某一个特定个人的劳动力供应的弹性。

已有大量基于家庭调查数据的微观层面经验研究发现,Frisch弹性尽管跨国差异较大,但总的来说值比较小,这往往意味着$\eta \ge 1$。早期宏观经济的经验研究往往采用这一设定,但存在不足:引入外生冲击后,RBC模型的结果往往显示,就业的波动要远远大于工资的波动,这与实际观测到的数据不符。并且从(宏观经济学的)道理上讲,理性经济行为者应该对工资的波动做出较有弹性的应对,使得就业率波动小于工资,劳动力供应的弹性较大,$\eta <1$。

宏观和微观研究中的分歧似乎可以从这个角度来理解:微观层面上的Frisch elasiticty,和宏观层面上的labor supply elasticity,所描述的对象并不一致。\cite{Rogerson:1988js,Hansen:1985ku}等人论证了,在Frisch elasticity of labor supply 等于0的情况下,总就业仍然可能随着实际工资的小幅度变化而出现大幅度震荡\citep{Rogerson:2009eza}。

有鉴于此,\cite{Gali:2005gp}提出了新的模型设定思路,随后被\cite{Christiano:2010wla,Mulligan:2001wf,Krusell:2008bw,Krusell:2011bv}等所采用:
\begin{itemize}
\item 经济体中每个典型家庭都有大量成员$j \in (0,1)$,
\item 任何工资水平下的Frisch labor supply elasticity都为$0$,
\item 每个家庭成员只有两个状态,对应两种效用函数:
  \begin{itemize}
  \item 被雇佣,$\log C^{employed}_t - l^{\eta}$,
  \item 失业,$\log C^{unemployed}_t$,

其中$l$代表对工作厌恶程度:$l$越高的家庭成员(老幼病人)越厌恶工作。
  \end{itemize}
\item 家庭效用最大化的目标是追求内部所有成员整体效用的最大化,即所有成员无论工作与否,消费水平相等,$C_t = C^{employed}_t = C^{unemployed}_t$。
\end{itemize}

如果家庭需要提供$H_t$单位劳动力,那么家庭全部成员中,$0\le l \le H_t$的去工作,$l > H_t$的不工作。那么对所有$l \in [0,1]$求积分后,消费水平为$C_t$,就业水平为$H_t$的典型家庭的效用函数为$\log C_t - \frac{H_t^{1+\eta}}{1+\eta}$。

在这样的设定下,$H_t$重新表示劳动者的数量,$\eta$不再作为衡量Frisch弹性(设为0)的指标,它表示在受到外部经济环境冲击的情况下,进入或离开劳动力市场的家庭成员变化的弹性。如果$\eta$比较大,反映出家庭内部各个成员之间厌恶工作的程度差异较大,分布较平均。此时工资的变化只会导致就业量发生较小的变化。如果$\eta$比较小,反映出家庭内部各个成员中,厌恶工作的差异程度较小,且集中在对于工作不工作无所谓的水平附近,此时工资的变化会导致就业量发生较大的变化。

\subsection{劳动力承包商}
\label{sec:hh-labor-contractors}
设产品生产部门所需要的同质劳动服务$H_t$由劳动力承包商提供。承包商负责向家庭部门征集一系列具有不同特性$j\in(0,1)$的劳动力投入$h_{t}(j)$,满足生产部门的需求,投入产出关系符合Dixit-Stiglitz形式\citep{Dixit:1977tv}
\begin{equation}
  \label{eq:hh-H-output-htj}
  H_t = \left[ \int_{0}^{1} h_t(j)^{\frac{\varepsilon_w -1}{\varepsilon_w}} dj\right]^{\frac{\varepsilon_w }{\varepsilon_w -1}}.
\end{equation}

承包商是完全竞争的,视$W_t$和$H_t$为生产部门所给定,视$W_t(j)$为家庭部门所给定。劳动承包商的最大化问题:
\begin{equation*}
  \max_{h_t(j)} W_t \cdot H_t - \int_{0}^{1} W_t(j) \cdot h_t(j) dj.
\end{equation*}

引入式\eqref{eq:hh-H-output-htj},FOC整理得对$h_t(j)$的需求函数
\begin{equation}
  \label{eq:hh-demand-htj}
  h_t(j) = \left( \frac{W_t(j)}{W_t}\right)^{-\varepsilon_w} \cdot H_t.
\end{equation}

\begin{equation*}
  \begin{split}
    W_t \cdot H_t & \equiv \int_0^1 W_t(j) \cdot h_t(j) dj \\
    &=\left[\int_0^1 \left(\frac{W_t(j)}{W_t}\right)^{-\varepsilon_w} \cdot H_t \cdot W_t(j) dj \right] \\
    &=\left[\int_0^1 W_t(j)^{1-\varepsilon_w} dj \right] \cdot W_t^{\varepsilon_w} \cdot H_t,\\
  \end{split}
\end{equation*}
整理得最终工资的决定(aggregate wage index):
\begin{equation}
  \label{eq:agg-wage-index}
  W_t = \left[
    \int_0^1 W_t(j)^{1-\varepsilon_w} dj
  \right]^{\frac{1}{1-\varepsilon_w}}
\end{equation}

\subsection{家庭行为}
\label{sec:hh-behav}
经济体中存在一系列同质化家庭,相关假定及评述第\ref{sec:interpretation-H}节。一个典型家庭中存在许多成员,对应异质化的劳动力特征$j \in [0,1]$。
%$j^{th}$类型的劳动者集合具有同样的厌恶工作程度$l\in[0,1]$。
诸多家庭劳动承包商供应$h_t(j)$异质劳动力,满足式  \eqref{eq:hh-demand-htj}。诸多家庭的$j^{th}$类型劳动力汇总到垄断竞争的$j^{th}$劳工联盟,由联盟制定工资$W_t(j)$,见下节\footnote{膏按:补一个reference。}。

假定总量层面上,家庭的效用函数表现为
\begin{equation}
  \label{eq:hh-utility-C-t-h-int}
  U(C_t,h_{t}) = \log(C_t - b \cdot C_{t-1}) - A \cdot \int_{0}^{1} \frac{h_{t}(j)^{1+\eta}}{1+\eta} dj.
\end{equation}
效用来自消费和休闲\footnote{休闲的效应以劳动的负效果(disutility)形式体现,因此用负号。}两方面,二者对效用的正效果相加得到总效用函数,是内部可分的(intratemporal separability)\footnote{内部不可分的效用函数的例子,可见\cite{King:1988bk,King:1988kf,Greenwood:1988jn,GuerronQuintana:2008jo}。}。

\subsubsection{(消费)习惯的形成}
\label{sec:hh-consumption-habit-formation}
消费的效应是跨期可分(intertemporal separability)的\footnote{效用函数中,关于intertemporal- 和intratemporal (non)separability的介绍,可见Eric Sims(2015)讲义。}。$b>0$表示habit persistence parameter\footnote{经验研究中常常限定$b<1$,这是出于计算方便的需要:如果$b=1$,则稳定状态下消费的边际效用就变成无穷大了。},$b=0$时当期效用仅与当期消费有关,habit formation不存在,模型回到经典的PIH模式(permanent income hypothesis)。$0<b<1$时,当期效用不仅取决于当期消费水平,还与当期消费相对于过去消费水平的变化程度有关,因此又称消费偏好的习惯形成。

宏观经济模型中引入消费习惯的形成设定,主要是为了减小传统PIH-based模型的模拟结果与实际情况间的背离,如:
\begin{enumerate}
\item Excess smoothness puzzle。消费习惯越是强($b$越接近于1),在permanent income出现波动时,消费水平随之变化的幅度就越小\citep{Carroll:1997ba,Carroll:2000em}。
\item Asset pricing方面的Equity primium puzzle。消费习惯越是强($b$越接近于1),消费者的行为决策看起来就更加具有风险规避的特征,从而使我们在构建模型时,不必为相对风险规避系数定义一个大的离谱的值,如$C_t^{1-\sigma}/(1-\sigma)$中的$\sigma$:一方面可以设$\sigma = 1$($\sigma$越大,消费者越厌恶风险);另一方面使得消费者行为仍然表现出厌恶风险的特性\citep{Constantinides:1990cu,Boldrin:2001hb}。
\end{enumerate}

引入消费habit formation设定后的DSGE模型,其模拟的冲击-响应结果更符合现实(基于VAR对现实数据的观察):货币政策等外生冲击可以造成消费响应的hump-shape。而传统PIH based DSGE模型中消费的响应往往做不到这一点\citep{Fuhrer:2000ez}。除了消费之外,HF-based DSGE模型还可以生成实际利率持续下降的结果。

\subsection{劳工联盟的工资策略}
\label{sec:wage-setting-monopoly-union}
$j^{th}$劳工联盟居于垄断竞争地位,满足劳动力承包商对$h_t(j)$的需求式\eqref{eq:hh-demand-htj}。假定价格刚性存在,$t$时期有$0 \le \phi_{w} \le 1$比例的劳工联盟$j \in [0,1]$无法调整价格。

对于剩下的可以调整价格的$1-\phi_{w}$联盟而言,名义工资的决定式
\begin{equation}
  \label{eq:union-wage-no-adj}
  W_{t+1}(j) \equiv \left(1+\tilde{\pi}_{w,t+1}\right) \cdot W_t(j),
\end{equation}
其中
\begin{equation}
  \label{eq:union-wage-no-adj-inflation}
  1+\tilde{\pi}_{w,t+1} \equiv (1+\pi_t)^{\kappa_w} \cdot (1+\pi)^{1-\kappa_w} \cdot \mu_{z^+}, \quad 0<\kappa_w<1.
\end{equation}

可以调整工资的垄断劳工联盟,其forward-looking问题可以描述为,制定合理的垄断工资$W_t^{\#}(j)$以追求效用最大化
\begin{equation}
  \label{eq:union-wage-adj-max-prob}
  \max E_t \sum_{m=0}^{\infty} \left(\phi_w \cdot \beta \right)^{m}  \cdot \left[v_{t+m} \cdot W_{t+m}^{\#}(j) \cdot h_{t+m}(j) - A_L \cdot \frac{h_{t+m}(j)^{1+\eta}}{1+\eta}\right],
\end{equation}
中括号中的内容为$t+j$时刻$i^{th}$类型劳动力供应为家庭部门带来的效用;与第\ref{seq:intm-updating-pricing}节类似,$\beta^m \cdot v_{t+m}$为家庭部门跨期预算约束条件的乘数,$v_{t}=\frac{\partial U_t(\cdot) / \partial C_t}{P_t}$。

$t$时刻制定的垄断工资直到$t+m$时段都无法再调整,因此由式\eqref{eq:union-wage-no-adj}得
\begin{align}
  \label{eq:W-sharp-iteration}
  W_{t+m}^{\#}(j) &= W_{t+m-1}^{\#}(j) \cdot (1+\tilde{\pi}_{w,t+m}) \nonumber \\
                  &= W_{t+m-2}^{\#}(j) \cdot (1+\tilde{\pi}_{w,t+m}) \cdot (1+\tilde{\pi}_{w,t+m-1}) \nonumber \\
                  &= \ldots  \nonumber \\
                  &= W_t^{\#}(j) \cdot \left[(1+\tilde{\pi}_{w,t+m}) \cdot \ldots \cdot (1+\tilde{\pi}_{w,t+1})\right].
\end{align}

并且
\begin{equation}
  \label{eq:zplus-t-tplusm}
  \frac{z_{t+m}^+}{z_t^+} = \frac{z_{t+m}^+}{z_{t+m-1}^+} \cdot \frac{z_{t+m-1}^+}{z_{t+m-2}^+} \cdot \ldots \cdot \frac{z_{t+1}^+}{z_{t}^+} = \mu_{z^+,t+m} \cdot \ldots \cdot \mu_{z^+,t+1},
\end{equation}

\begin{equation}
  \label{eq:Price-t-tplusm}
  \frac{P_{t+m}}{P_t} = \frac{P_{t+m}}{P_{t+m-1}} \cdot \frac{P_{t+m-1}}{P_{t+m-2}} \cdot \ldots \cdot \frac{P_{t+1}}{P_{t}} = (1+\pi_{t+m}) \cdot \ldots \cdot (1+\pi_{t+1}),
\end{equation}

式\eqref{eq:W-sharp-iteration}两侧同时除以$W_{t+m}$,引入scaling式  \eqref{eq:scaled-real-wage}、\eqref{eq:zplus-t-tplusm}和  \eqref{eq:Price-t-tplusm}得
\begin{align}
\label{Wage-sharp-tplusm-W-tplusm}
  \frac{W_{t+m}^{\#}(j)}{W_{t+m}} &= \frac
{
  W_t^{\#}(j) \cdot \left(\tilde{\pi}_{w,t+m} \cdot \ldots \cdot \tilde{\pi}_{w,t+1}\right)
}{
W_{t+m}
} \\
&=\frac
{
  \left(\frac{W_t^{\#}(j)}{W_t}\right) \cdot \left(\bar{w}_{t} \cdot z_{t}^+ \cdot P_{t}\right)  \cdot \left(\tilde{\pi}_{w,t+m} \cdot \ldots \cdot \tilde{\pi}_{w,t+1}\right)
}{
\bar{w}_{t+m} \cdot z_{t+m}^+ \cdot P_{t+m}
} \\
&= \left(\frac{W_t^{\#}(j)}{W_t}\right) \cdot \left(\frac{\bar{w}_t}{\bar{w}_{t+m}}\right) \cdot \frac{
  \left[(1+\tilde{\pi}_{w,t+m}) \cdot \ldots \cdot (1+\tilde{\pi}_{w,t+1})\right]
  }{
  \left[(1+\pi_{t+m} \cdot \ldots \cdot (1+\pi_{t+1})\right] \cdot \left[\mu_{z^+,t+m} \cdot \ldots \cdot \mu_{z^+,t+1}\right]
  } \\
&=\left(\frac{W_t^{\#}(j)}{W_t}\right) \cdot \left(\frac{\bar{w}_t}{\bar{w}_{t+m}}\right) \cdot \mathcal{X}_{t,m},
\end{align}
其中定义辅助变量
\begin{equation}
  \label{eq:mathcal-X-auxiliary-definition}
  \mathcal{X}_{t,m}=
\begin{cases}
\frac{
  \left[(1+\tilde{\pi}_{w,t+m}) \cdot \ldots \cdot (1+\tilde{\pi}_{w,t+1})\right]
  }{
  \left[(1+\pi_{t+m} \cdot \ldots \cdot (1+\pi_{t+1})\right] \cdot \left[\mu_{z^+,t+m} \cdot \ldots \cdot \mu_{z^+,t+1}\right]
  } &\mbox{if } m \ge 0, \\
1 &\mbox{if } m=0.
\end{cases}
\end{equation}

利用式\eqref{Wage-sharp-tplusm-W-tplusm}可以将式\eqref{eq:hh-demand-htj}改写为
\begin{equation}
  \label{eq:h-H-t-m-j-interm}
  h_{t+m}(j) =\left[\left(\frac{W_t^{\#}(j)}{W_t}\right) \cdot \left(\frac{\bar{w}_t}{\bar{w}_{t+m}}\right) \cdot \mathcal{X}_{t,m} \right]^{-\varepsilon_w} \cdot H_{t+m}.
\end{equation}

效用最大化问题式\eqref{eq:union-wage-adj-max-prob}改写为
\begin{equation*}
  \max E_t \sum_{m=0}^{\infty} \left(\phi_w \cdot \beta \right)^{m}  \cdot \left[v_{t+m} \cdot W_{t+m} \cdot \left(\frac{W_{t+m}^{\#}(j)}{W_{t+m}}\right) \cdot h_{t+m}(j) - A_L \cdot \frac{h_{t+m}(j)^{1+\eta}}{1+\eta}\right] ,
\end{equation*}
中括号中的内容进一步调整为
\begin{align*}
&\left(\frac{v_{t+m} \cdot W_{t+m}}{\bar{w}_{t+m}}\right) \cdot \left[\bar{w}_t \cdot \left(\frac{W_t^{\#}(j)}{W_{t}}\right) \cdot \mathcal{X}_{t,m}\right] \cdot \left[ \left(\frac{W_t^{\#}(j)}{W_t}\right) \cdot \left(\frac{\bar{w}_t}{\bar{w}_{t+m}}\right) \cdot \mathcal{X}_{t,m} \right]^{-\varepsilon_w} \cdot H_{t+m} \\
&- A_L \cdot \frac{1}{1+\eta}  \cdot \left[ \left(\frac{W_t^{\#}(j)}{W_t}\right) \cdot \left(\frac{\bar{w}_t}{\bar{w}_{t+m}}\right) \cdot \mathcal{X}_{t,m} \right]^{-\varepsilon_w \cdot (1+\eta)} \cdot H_{t+m}^{1+\eta}.
\end{align*}

并且由式\eqref{eq:scaled-real-wage}、\eqref{eq:scaled-produc-cost-adj-coef}得
\begin{equation*}
  \frac{v_{t+m} \cdot W_{t+m}}{\bar{w}_{t+m}} = v_{t+m} \cdot p_{t+m} \cdot z_{t+m}^+ = \psi_{z^+,t+m}.
\end{equation*}

可得调整后的最大化问题式
\begin{align}
  \label{eq:union-wage-adj-max-prob-2}
    \max E_t \sum_{m=0}^{\infty} \left(\phi_w \cdot \beta \right)^{m}  \cdot \{
      &\psi_{z^+,t+m} \cdot \bar{w}_{t}^{1-\varepsilon_w} \cdot \bar{w}_{t+m}^{\varepsilon_m} \cdot \mathcal{X}_{t,m}^{1-\varepsilon_w} \cdot \left(\frac{W_t^{\#}(j)}{W_t}\right)^{1-\varepsilon_w} \cdot H_{i,t} \nonumber \\
& - \frac{A_L}{1+\eta} \cdot \bar{w}_t^{-\varepsilon_w \cdot (1+\eta)} \cdot \bar{w}_{t+m} \cdot \mathcal{X}_{t,m}^{-\varepsilon_w \cdot (1+\eta)} \cdot \left(\frac{W_t^{\#}(j)}{W_t}\right)^{-\varepsilon_w \cdot (1+\eta)} \cdot H_{t+m}^{1+\eta} \}.
\end{align}
设$w_t^{\#}(j) \equiv W_t^{\#}(j)/W_t$,劳工联盟FOCs wrt $w_t^{\#}$ ,整理得
\begin{equation}
  \label{eq:union-max-prob-FOC-wsharp}
  w_t^{\#}(j)^{1+\varepsilon_w \cdot \eta} = \frac{A_L}{\bar{w}_t} \cdot \frac{\varepsilon_w}{\varepsilon_w -1} \cdot \frac{
  E_t \sum_{m=0}^{\infty} \left(\beta \cdot \phi_w\right)^m  \left[
\left( \frac{\bar{w}_t}{\bar{w}_{t+m}} \cdot \mathcal{X}_{t,m} \right)^{-\varepsilon_w} \cdot H_{t+m} \right]^{1+\eta}
  }
{
  E_t \sum_{m=0}^{\infty} \left(\beta \cdot \phi_w\right)^m  \cdot \psi_{z^+,t+m} \cdot \left( \frac{\bar{w}_t}{\bar{w}_{t+m}} \cdot \mathcal{X}_{t,m} \right)^{-\varepsilon_w} \cdot H_{t+m} \cdot \mathcal{X}_{t,m}
}
\end{equation}

可见$W_t^{\#}(j) = W_t^{\#} = W_t,\forall j$,劳工联盟工资策略的异质性特征得以消除。为了进一步简化,定义两个辅助变量\footnote{也可以在此基础上,作工资的philips曲线,见第\ref{sec:wage-Philips-Curve-me}节。}
\begin{align}
  \label{eq:union-auxiliary-xw1}
  x_{1,t}^{w}\equiv   E_t \sum_{m=0}^{\infty} \left(\beta \cdot \phi_w\right)^m  \left[
\left( \frac{\bar{w}_t}{\bar{w}_{t+m}} \cdot \mathcal{X}_{t,m} \right)^{-\varepsilon_w} \cdot H_{t+m} \right]^{1+\eta}, \\
  \label{eq:union-auxiliary-xw2}
    x_{2,t}^{w}\equiv     E_t \sum_{m=0}^{\infty} \left(\beta \cdot \phi_w\right)^m  \cdot \psi_{z^+,t+m} \cdot \left( \frac{\bar{w}_t}{\bar{w}_{t+m}} \cdot \mathcal{X}_{t,m} \right)^{-\varepsilon_w} \cdot H_{t+m} \cdot \mathcal{X}_{t,m}.
\end{align}

对$x_{1,t}^{w}$的迭代简化
\begin{align}
\label{eq:union-auxiliary-xw1-iter}
  x_{1,t}^{w} &= H_t^{1+\eta} + \left(\beta \cdot \phi_w\right) \cdot \left[\left(\frac{\bar{w}_t}{\bar{w}_{t+1}} \cdot \mathcal{X}_{t,1}\right)^{-\varepsilon_w} \cdot H_{t+1}\right]^{1+\eta} + \left(\beta \cdot \phi_w\right)^2 \cdot \left[\left(\frac{\bar{w}_t}{\bar{w}_{t+2}} \cdot \mathcal{X}_{t,2}\right)^{-\varepsilon_w} \cdot H_{t+2}\right]^{1+\eta} + \ldots \nonumber \\
              &= H_t^{1+\eta} + E_t \left(\beta \cdot \phi_w\right) \cdot \left[\frac{\bar{w}_{t}}{\bar{w}_{t+1}} \cdot \frac{(1+\pi)^{1-\kappa_w} \cdot (1+\pi_t)^{\kappa_w} \cdot \mu_{z^+}}{(1+\pi_{t+1}) \cdot \mu_{z^+,t+1}}\right]^{-\varepsilon_w \cdot (1+\eta)} \cdot \nonumber \\
              & \left\{
  H_{t+1}^{1+\eta} + \left(\beta \cdot \phi_w\right) \cdot
  \left[
  \left(
  \frac{\bar{w}_{t+1}}{\bar{w}_{t+2}} \cdot
  \frac{(1+\pi)^{1-\kappa_w} \cdot (1+\pi_{t+1})^{\kappa_w} \cdot \mu_{z^+}}{(1+\pi_{t+2}) \cdot \mu_{z^+,t+2}}
  \right)^{-\varepsilon_w \cdot (1+\eta)} \cdot H_{t+2}^{1+\eta}
  \right] + \ldots
  \right\} \nonumber \\
              &=H_t^{1+\eta} + (\beta \cdot \phi_w) \cdot E_t \left(
                \frac{\bar{w}_{t}}{\bar{w}_{t+1}} \cdot
                \frac{(1+\pi)^{1-\kappa_w} \cdot (1+\pi_{t})^{\kappa_w} \cdot \mu_{z^+}}{(1+\pi_{t+1}) \cdot \mu_{z^+,t+1}}
                \right)^{-\varepsilon_w \cdot (1+\eta)} \cdot x_{1,t+1}^{w}\nonumber \\
              &=H_t^{1+\eta} + (\beta \cdot \phi_w) \cdot E_t \left(\frac{
  1+\tilde{\pi}_{w,t+1}
                }{
                1+\pi_{w,t+1}
                }\right)^{-\varepsilon_w \cdot (1+\eta)} \cdot x_{1,t+1}^{w},
\end{align}
其中最后一个等号用到了由式\eqref{eq:scaled-wage-inflation}而得的衍生:
\begin{equation}
  \label{eq:inflation-w-tplus-in-pi}
  1+\pi_{w,t+1} = \frac{W_{t+1}}{W_{t}} = \frac{\bar{w}_{t+1} \cdot P_{t+1} \cdot z_{t+1}^+}{\bar{w}_{t} \cdot P_{t} \cdot z_{t}^+} = \frac{\bar{w}_{t+1}}{\bar{w}_{t}} \cdot \left(1+\pi_{t+1}\right) \cdot \mu_{z^+,t+1}.
\end{equation}

对$x_{2,t}^{w}$的迭代简化
\begin{align}
\label{eq:union-auxiliary-xw2-iter}
x_{2,t}^{w} &= \psi_{z^+,t} \cdot H_t + \left(\beta \cdot \phi_w\right) \psi_{z^+,t+1} \cdot \left(\frac{\bar{w}_{t}}{\bar{w}_{t+1}}\right)^{-\varepsilon_w} \cdot \mathcal{X}_{t,1}^{1-\varepsilon_w} \cdot H_{t+1} \nonumber \\
&+ \left(\beta \cdot \phi_w\right)^2 \psi_{z^+,t+2} \cdot \left(\frac{\bar{w}_{t}}{\bar{w}_{t+2}}\right)^{-\varepsilon_w} \cdot \mathcal{X}_{t,2}^{1-\varepsilon_w} \cdot H_{t+2} + \ldots \nonumber \\
&=\psi_{z^+,t}\cdot H_t + \left(\beta \cdot \phi_w\right) \cdot  \left(\frac{\bar{w}_{t}}{\bar{w}_{t+1}}\right)^{-\varepsilon_w}
\cdot \left[\frac{(1+\pi)^{1-\kappa_w} \cdot (1+\pi_t)^{\kappa_w} \cdot \mu_{z^+}}{(1+\pi_{t+1}) \cdot \mu_{z^+,t+1}}\right]^{1-\varepsilon_w} \cdot \nonumber \\
             &\left\{
               \psi_{z^+,t+1} \cdot H_{t+1} + \left(\beta \cdot \phi_w\right) \cdot \psi_{z^+,t+2} \cdot \left(\frac{\bar{w}_{t+1}}{\bar{w}_{t+2}}\right)^{-\varepsilon_w} \cdot
\left(\frac{(1+\pi)^{1-\kappa_w} \cdot (1+\pi_{t+1})^{\kappa_w} \cdot \mu_{z^+}}{(1+\pi_{t+2}) \cdot \mu_{z^+,t+2}}\right)^{1-\varepsilon_w} \cdot H_{t+2} + \ldots
 \right\} \nonumber \\
&= \psi_{z^+,t} \cdot H_t + \left(\beta \cdot \phi_w\right) \cdot E_t \left(\frac{\bar{w}_{t+1}}{\bar{w}_{t}}\right) \cdot \left(
  \frac{
  1+\tilde{\pi}_{w,t+1}
  }{
  1+ \pi_{w,t+1}
  }
  \right)^{1-\varepsilon_w} \cdot x_{2,t+1}^{w}
\end{align}
其中最后一个等号用到了式\eqref{eq:inflation-w-tplus-in-pi}。

结合辅助变量式\eqref{eq:union-auxiliary-xw1-iter}、\eqref{eq:union-auxiliary-xw2-iter}, 可得repotimizing劳工联盟的工资
\begin{equation}
  \label{eq:union-repotimizing-wage}
  w_t^{\#} = \left[
    \frac{A_L}{\bar{w}_{t}} \cdot \frac{\varepsilon_w}{\varepsilon_w -1} \cdot \frac{x_{1,t}^{w}}{x_{2,t}^w}
  \right]^{\frac{1}{1+\varepsilon_w \cdot \eta}}
\end{equation}

另一方面,对Aggregate wage index式\eqref{eq:agg-wage-index}作calvo pricing
\begin{align*}
\label{eq:agg-wage-index-calvo}
  W_t^{1-\varepsilon_w} &= \int_{0}^{1} W_t(j)^{1-\varepsilon_w} dj \nonumber \\
                        &= \int_{0}^{1-\phi_w} \left[W_t(j)^{\#}\right]^{1-\varepsilon_w} dj+ \int_{1-\phi_w}^{1} \left[(1+\tilde{\pi}_{w,t}) \cdot W_{t-1}(j)\right]^{1-\varepsilon_w} dj \nonumber \\
&= (1-\phi_w) \cdot \left(W_t^{\#}\right)^{1-\varepsilon_w} + \phi_w \cdot \left[\left(1+\tilde{\pi}_{w,t}\right) \cdot W_{t-1}\right]^{1-\varepsilon_w},
\end{align*}
其中最后一个等号消除$j^{th}$劳工联盟工资定价的异质性特征,见式\eqref{eq:union-max-prob-FOC-wsharp}。式两侧同时除以$W_t^{1-\varepsilon_w}$,整理得
\begin{equation}
  \label{eq:agg-wage-index-calvo-inflation}
  w_t^{\#} = \left[\frac{
      1-\phi_w \cdot \left(\frac{
          1+\tilde{\pi}_{w,t}
        }{
          1+\pi_{w,t}
        }\right)^{1-\varepsilon_w}
    }{
      1-\phi_w
    }\right]^{\frac{1}{1-\varepsilon_w}}.
\end{equation}


联立式\eqref{eq:union-repotimizing-wage}-\eqref{eq:agg-wage-index-calvo-inflation}可得工资的决定
\begin{equation}
  \label{eq:wage-rate-auxiliary}
  \frac{A_L}{\bar{w}_{t}} \cdot \frac{\varepsilon_w}{\varepsilon_w -1} \frac{x_{1,t}^w}{x_{2,t}^w} =\left[\frac{1-\phi_w \cdot \left(\frac{1+\tilde{\pi}_{w,t}}{1+\pi_{w,t}}\right)^{1-\varepsilon_w}}{1-\phi_w} \right]^{\frac{1+\varepsilon_w \cdot \phi_w}{1-\varepsilon_w}}.
\end{equation}

\subsection{wage dispersion index}
\label{sec:wage-dispersion-index}
在市场出清情况下,劳动力承包商所提供的全部同质劳动$H_t$和家庭部门提供的全部异质劳动时间$h_t$的关系为:
\begin{equation*}
  h_t \equiv \int_{0}^{1}h_t(j) dj = H_t \cdot \int_{0}^{1} \left(\frac{W_t(j)}{W_t} \right)^{-\varepsilon_w} \cdot dj,
\end{equation*}
其中用到了式\eqref{eq:hh-demand-htj}。整理可得
\begin{equation}
  \label{eq:H-h-relationship-wage-dispersion}
  H_t = \frac{h_t}{\nu_t^w},
\end{equation}
其中wage dispersion index $\nu_t^w \ge 1$
\begin{equation}
  \label{eq:wage-dispersion-index}
  \nu_t^w = \int_{0}^{1} \left(\frac{W_t(j)}{W_t} \right)^{-\varepsilon_w} \cdot dj
\end{equation}
反映劳动力市场上各类型$j$劳动力工资的差异程度:差异越大,$\nu^w_t$值越高,同质的总劳动力产出越小;lower bound $\nu_t^p = 1$表明在没有wage setting friction时,所有$j$劳工联盟都会设定同样的工资。利用Calvo pricing可得
\begin{align}
  \label{eq:wage-dispersion-index-calvo}
  v_t^w &= \int_{0}^1 \left[\frac{W_t(j)}{W_t}\right]^{-\varepsilon_w} dj \nonumber \\
  &= \int_{0}^{1-\phi_w} \cdot \left[\frac{W_t^{\#}(j)}{W_t}\right]^{-\varepsilon_w} dj + \int_{1-\phi_w}^{1} \cdot \left[(1+\tilde{\pi}_{w,t}) \cdot \frac{W_{t-1}(j)}{W_t}\right]^{-\varepsilon_w} dj\nonumber \\
  &=(1-\phi_w) \cdot  \left[\frac{W_t^{\#}}{W_t}\right]^{-\varepsilon_w} + \phi_w \cdot \left(1+\tilde{\pi}_{w,t}\right)^{-\varepsilon_w} \cdot \int_{0}^{1} \cdot \left[ \frac{W_{t-1}(j)}{W_{t-1}} \cdot \frac{W_{t-1}}{W_t}\right]^{-\varepsilon_w} dj \nonumber \\
&=(1-\phi_w) \cdot (w_t^{\#})^{-\varepsilon_w} + \phi_w \cdot \left(\frac{1+\tilde{\pi}_{w,t}}{1+\pi_{w,t}}\right)^{-\varepsilon_w} \cdot v_{t-1}^{w} \nonumber \\
&=(1-\phi_w) \cdot \left[\frac{
      1-\phi_w \cdot \left(\frac{
          1+\tilde{\pi}_{w,t}
        }{
          1+\pi_{w,t}
        }\right)^{1-\varepsilon_w}
    }{
      1-\phi_w
    }\right]^{\frac{\varepsilon_w}{\varepsilon_w -1}} + \phi_w \cdot \left(\frac{1+\tilde{\pi}_{w,t}}{1+\pi_{w,t}}\right)^{-\varepsilon_w} \cdot v_{t-1}^{w}
\end{align}

\subsection{家庭预算约束条件}
\label{sec:HH-opt-prob-budget-constraint}
家庭面临的预算约束条件为\footnote{膏按:上文所需要的等式。这一章节还需要补充相关文字。}
\begin{equation}
\label{eq:HH-opt-prob-budget-constraint}
P_t \cdot \left(C_t + \frac{1}{\Psi_t} \cdot I_t \right) + B_{t+1} + P_t \cdot P_{k',t} \cdot \Delta_t \le \int_{0}^{1} W_{t}(j) \cdot h_{t}(j) dj + X^k_t \cdot \bar{K}_t
 + R_{t-1} \cdot B_t,
\end{equation}
其中$B_{t}$为家庭部门购买的无风险债券。$R_t$表示$t_1$时刻购买的债券,在$t$时刻兑现,所对应的名义利率。

\subsection{资本积累}
\label{capital-accumulation}
假定家庭部门
\begin{itemize}
\item 是实物资本的所有者,
\item 设定实物资本的利用率(utilization rate),
\item 并进一步在竞争市场上将其租给(中间产品)生产者使用,收益为资本租金,成本为维护成本(fast depreciation)。
\end{itemize}

当期资本存量尽管是由前期投资形成的,但家庭部门的当期经济决策仍可更集中/更少地使用已经形成的资本,投入到生产活动中去。决策依据取决于对当前经济形势的观察,和/或对未来的预期。实物资本存量积累式
\begin{equation}
  \label{eq:physical-capital-accumulation}
  \bar{K}_{t+1} = (1-\delta) \cdot \bar{K}_t + F(I_t,I_{t-1}) + \Delta_t,
\end{equation}
其中折旧率为常数$\delta$\footnote{膏按:也有将折旧率视为与$u_t$有关的变量,$\delta(u_t)$,补充reference。}。

将式\eqref{eq:adjustment-cost-func-CEE}带回式\eqref{eq:physical-capital-accumulation},资本积累式可改写为
\begin{equation}
\label{eq:phy-cap-accumu-cap-adj-cost}
\bar{K}_{t+1} = (1-\delta) \cdot \bar{K}_t + \left[ 1 - S \cdot \left(\frac{I_t}{I_{t-1}} \right) \right] \cdot I_t + \Delta_t.
\end{equation}

$\Delta_t$表示该家庭从其他家庭购买的“净”实物资本,均衡条件下$\Delta_t = 0$,模型设定中保留$\Delta_t$以便于测算实物资本存量的价值。$\Delta_t$的市场价格为$P_t \cdot P_{k',t}$。

利用均衡条件下$\Delta_t=0$,以及式,将\eqref{eq:scaled-physical-capital}、\eqref{eq:scaled-investment-goods},对上式作scale
\begin{equation}
\bar{k}_{t+1} =
\frac{
  \left( 1 - \delta \right) \cdot \bar{k}_t
}{
  \mu_{z^+,t} \cdot \mu_{\Psi,t}
}
+ \left[
  1-s \cdot \left(
  \frac{i_t}{i_{t-1}} \cdot \mu_{z^+,t} \cdot \mu_{\Psi,t}
  \right)
\right] \cdot i_t
\end{equation}

\subsubsection{资本净收益}
$t$时期的投资带来$t+1$时期的实物资本积累增加;$\bar{K}_{t+1}$每增加1单位,给家庭带来的净收益(net cash payment) $X_{t+1}^k$为
\begin{equation}
  \label{eq:capital-net-payment}
  X_{t+1}^k = u_{t+1} \cdot P_{t+1} \cdot r_{t+1}^k - \frac{P_{t+1}}{\psi_{t+1}} \cdot a(u_{t+1}),
\end{equation}
RHS前半部分为考虑到资本利用率之后的名义净rental收入(扣除掉折旧);后半部分为使用成本(capital utilization cost)。$P_{t+1}/\Psi_{t+1}$为同质投资品$I_t$的均衡市场价格(名义量),由式\eqref{eq:output-expenditures}-\eqref{eq:investment-goods-homo}给出。

\subsection{家庭部门最大化问题}
\label{sec:mHH-opti}
家庭部门问题可以表示为:选择投入组合$\{ C_t, I_t, \Delta_t, B_{t+1}, \bar{K}_{t+1}, u_t\}$,基于给定的预算约束式\eqref{eq:HH-opt-prob-budget-constraint}、资本积累式\eqref{eq:phy-cap-accumu-cap-adj-cost}和资本净收益式\eqref{eq:capital-net-payment},来追求式\eqref{eq:hh-utility-C-t-h-int}效用函数最大化。
\begin{equation*}
\begin{split}
&\max_{\left\{C_t,\Delta_t, I_t, \bar{K}_{t+1}, B_{t+1}, u_t\right\}} \mathcal{L} = E_0 \sum_{t=0}^{\infty} \beta \cdot \{ \ln \left(C_t - b \cdot C_{t-1}\right) - A_L \cdot \frac{\int^1_{0} h_t(j)^{1+\eta}}{1+\eta} dj \\
&+v_t \cdot \left[
  \int^1_0 W_t(j) \cdot h_t(j) dj + X_t^k \cdot \bar{K}_t + R_{t-1} \cdot B_t - P_t \cdot C_t - P_t \cdot \frac{I_t}{\Psi_t} - B_{t+1} -P_t \cdot P_{k',t} \cdot \Delta_t
\right] \\
&+\omega_t \cdot \left[
  \Delta_t + \left( 1 - \delta \right) \cdot \bar{K}_t + \left[1-S\left(\frac{I_t}{I_{t-1}}\right)\right] \cdot I_t - \bar{K}_{t+1}
\right] \\
&+\tau_{t} \cdot \left[
  u_{t+1} \cdot P_{t+1} \cdot r_{t+1}^k - \frac{P_{t+1}}{\psi_{t+1}} \cdot a(u_{t+1}) - X_{t+1}^k
\right]\}.
\end{split}
\end{equation*}
取消异质性$j\in(0,1)$特征。依次求解FOCs,为表述简便,在不产生歧义的情况下省略式中的期望符号。


\subsubsection{FOC wrt C}
\begin{equation}
\label{eq:HH-max-FOC-C-intm}
 \frac{\partial \mathcal{L}}{\partial C_t} = 0 \Rightarrow \frac{1}{C_t - b \cdot C_{t-1}} -  \frac{\beta \cdot b}{C_{t+1} - b \cdot C_t} = v_t \cdot P_t,
\end{equation}
代入式\eqref{eq:ztplus-zt-Psi}、\eqref{eq:scaled-consumption-goods}、\eqref{eq:scaled-growth-fixed-cost-shock}、\eqref{eq:scaled-produc-cost-adj-coef},进一步整理得
\begin{equation}
\label{eq:HH-max-FOC-C}
\frac{1}{
  c_t - \frac{b}{\mu_{z^+,t}} \cdot c_{t-1}
}
- \frac{\beta \cdot b}{
  \mu_{z^+,t+1} \cdot c_{t+1}  - b \cdot c_t
}
=\psi_{z^+,t}.
\end{equation}

\subsubsection{FOC wrt $\Delta_t$}
\label{sec:FOC-wrt-Delta-t}
\begin{equation}
\label{eq:HH-max-FOC-Delta-t}
\frac{\partial \mathcal{L}}{\partial \Delta_t} = 0 \Rightarrow P_t \cdot P_{k',t} = \frac{\omega_t}{v_t},
\end{equation}
其中RHS两个影子价格之比反映了Tobin's q,见第\ref{sec:adjustment-cost-types-compar}节。

\subsubsection{FOC wrt I}
\label{sec:FOC-wrt-I}
\begin{equation}
\label{eq:S-I-t-1-partial-I}
\begin{split}
&\frac{\partial \omega_t \cdot \left[1-S \left(\frac{I_t}{I_{t-1}}\right)\right] \cdot I_t}{\partial I_t}  =\frac{\partial \omega_t \cdot I_t}{\partial I_t} - \frac{
 \left[ \partial \omega_t \cdot I_t \cdot S \left(\frac{I_t}{I_{t-1}}\right)\right]
}{I_t} \\
&=\omega_t
- \omega_t \cdot \frac{I_t}{I_{t-1}} \cdot S'\left(\frac{I_t}{I_{t-1}}\right)
- \omega_t \cdot S\left(\frac{I_t}{I_{t-1}}\right)
- \frac{
  \partial \beta \cdot E_t \left\{
  \omega_{t+1} \cdot I_{t+1} \cdot S\left(\frac{I_{t+1}}{I_{t}}\right)
  \right\}
}{\partial I_t} \\
&= \omega_t
- \omega_t \cdot \frac{I_t}{I_{t-1}} \cdot S'\left(\frac{I_t}{I_{t-1}}\right)
- \omega_t \cdot S\left(\frac{I_t}{I_{t-1}}\right) - \beta \cdot E_t \left\{
  \omega_{t+1} \cdot I_{t+1} \cdot S'\left(\frac{I_{t+1}}{I_{t}}\right) \cdot \left(-1\right) \cdot \frac{I_{t+1}}{I_t^2}
\right\} \\
&= \omega_t \cdot \left[
  1- S\left(\frac{I_t}{I_{t-1}}\right) - \frac{I_t}{I_{t-1}} \cdot S'\left(\frac{I_t}{I_{t-1}}\right)
\right]
  + \beta \cdot E_t \left\{\omega_{t+1} \cdot \left(\frac{I_{t+1}}{I_{t}}\right)^2 \cdot S'\left(\frac{I_{t+1}}{I_{t}}\right)\right\}.
  \end{split}
\end{equation}

式\eqref{eq:scaled-produc-cost-adj-coef}、\eqref{eq:scaled-physical-capital-price}代入式\eqref{eq:S-I-t-1-partial-I}得
\begin{equation}
\label{eq:omega-t-v-t-P-t}
\omega_t = \left(v_t \cdot P_t \right) \cdot \left(P_{k',t}\right) = \left(\frac{\psi_{z^+,t}}{z^+_t}\right) \cdot \left(\frac{p_{k',t}}{\Psi_t}\right)
\end{equation}

此外由式\eqref{eq:scaled-investment-goods}得
\begin{equation}
\label{eq:I-t-1-i-t-i}
\frac{i_t}{i_{t-1}} = \frac{
  i_t \cdot z^+_t \cdot \psi_t
}{
  i_{t-1} \cdot z^+_{t-1} \cdot \psi_{t-1}
}
=\frac{i_t}{i_{t_1}} \cdot \mu_{z^+,t} \cdot \mu_{\Psi,t}.
\end{equation}

FOC wrt $I_t$
\begin{equation}
\label{eq:eq:HH-max-FOC-I-intm}
\frac{\partial \mathcal{L}}{\partial I_t} = 0 \Rightarrow \frac{v_t \cdot P_t}{\psi_t} =\frac{\partial \omega_t \cdot \left[1-S \left(\frac{I_t}{I_{t-1}}\right)\right] \cdot I_t}{\partial I_t},
\end{equation}
LHS代入式\eqref{eq:scaled-produc-cost-adj-coef},RHS代入式\eqref{eq:S-I-t-1-partial-I}、\eqref{eq:omega-t-v-t-P-t}得
\begin{equation}
\begin{split}
&\frac{\psi_{z^+,t}}{z^+_t \cdot \psi_t} = \frac{\psi_{z^+,t} \cdot p_{k',t}}{z^+_t \cdot \psi_t} \cdot \mathcal{A} + \beta \cdot E_t \frac{\psi_{z^+,t+1} \cdot p_{k',t+1}}{z^+_{t+1} \cdot \psi_{t+1}} \cdot \mathcal{B},\\
& \mathcal{A} \equiv 1-S\left(
\mu_{z^+,{t}} \cdot \mu_{\psi,t} \cdot \frac{i_{t}}{i_{t-1}}
\right)
-S'\left(
\mu_{z^+,{t}} \cdot \mu_{\psi,t} \cdot \frac{i_{t}}{i_{t-1}}
\right) \cdot
\left(
\mu_{z^+,{t}} \cdot \mu_{\psi,t} \cdot \frac{i_{t}}{i_{t-1}}
\right), \\
& \mathcal{B} \equiv S'\left(
\mu_{z^+,{t+1}} \cdot \mu_{\psi,t+1} \cdot \frac{i_{t+1}}{i_{t}}
\right) \cdot
\left(
\mu_{z^+,{t+1}} \cdot \mu_{\psi,t+1} \right)
\cdot \left(
  \frac{i_{t+1}}{i_{t}}
\right)^2,
\end{split}
\end{equation}
等式两侧同时乘以$z^+_t \cdot \psi_t$,整理得
\begin{equation}
\label{eq:HH-max-FOC-I}
\begin{split}
\psi_{z^+,t} = &\psi_{z^+,t} \cdot p_{k',t} \cdot \left[
1-S\left(
\mu_{z^+,{t}} \cdot \mu_{\psi,t} \cdot \frac{i_{t}}{i_{t-1}}
\right)
-S'\left(
\mu_{z^+,{t}} \cdot \mu_{\psi,t} \cdot \frac{i_{t}}{i_{t-1}}
\right) \cdot
\left(
\mu_{z^+,{t}} \cdot \mu_{\psi,t} \cdot \frac{i_{t}}{i_{t-1}}
\right)
\right] \\
&+ \beta \cdot E_t \left[
\left(\psi_{z^+,t+1} \cdot p_{k',t+1} \right) \cdot
\left(
\mu_{z^+,{t+1}} \cdot \mu_{\psi,t+1} \right)
\cdot
S'\left(
\mu_{z^+,{t+1}} \cdot \mu_{\psi,t+1} \cdot \frac{i_{t+1}}{i_{t}}
\right) \cdot
\left(
  \frac{i_{t+1}}{i_{t}}
\right)^2
\right].
\end{split}
\end{equation}

\subsubsection{FOC wrt K}
\label{sec:FOC-wrt-K}
\begin{equation}
\label{eq:HH-max-FOC-K-intm}
\begin{split}
\frac{\partial \mathcal{L}}{\partial \bar{K}_{t+1}} = 0 \Rightarrow \omega_t &=\frac{
  \partial v_t \cdot X^k_t \cdot \bar{K}_t
}{\partial \bar{K}_{t+1}} +
\frac{
  \partial \omega_t \cdot \left(1-\delta \right) \cdot \bar{K}_t
}{\partial \bar{K}_{t+1}} \\
&=\beta \cdot E_t X^k_{t+1} \cdot v_{t+1} + \beta \cdot (1-\delta) \cdot E_t \omega_{t+1},
\end{split}
\end{equation}


代入式\eqref{eq:omega-t-v-t-P-t}以消去$\omega_t$,整理得
\begin{equation}
\label{eq:HH-max-FOC-K}
\begin{split}
v_t  &= \beta \cdot E_t v_{t+1} \cdot \frac{
  X^k_{t+1} + (1-\delta) \cdot P_{t+1} \cdot P_{k',t+1}
}{P_t \cdot P_{k',t}} \\
&= \beta \cdot E_t \cdot v_{t+1} \cdot R^k_{t+1}, \\
&\text{其中定义 } \quad R^k_{t+1} \equiv \frac{
  X^k_{t+1} + (1-\delta) \cdot P_{t+1} \cdot P_{k',t+1}
}{P_t \cdot P_{k',t}}
\end{split}
\end{equation}
$R^k_t$表示资本的回报率(rate of return on capiutal)。

对式\eqref{eq:HH-max-FOC-K}两侧同时乘以$P_t \cdot z^+_t$
\begin{equation}
\label{eq:HH-max-FOC-K-scaled}
\begin{split}
\psi_{z^+,t} &= \beta \cdot E_t \frac{
  v_{t+1} \cdot P_{t+1} \cdot z^+_{t+1}
}{
  \frac{P_{t+1}}{P_t} \cdot \frac{z^+_{t+1}}{z^+_{t}}
} \cdot R^k_{t+1}\\
&=
\beta \cdot E_t \cdot \frac{
  \psi_{z^+, t+1}
}{\left( 1+\pi_{t+1} \right) \cdot \mu_{z^+,t+1}} \cdot R^k_{t+1}
\end{split}
\end{equation}

将net cash payment $X^k_{t+1}$ 式\eqref{eq:capital-net-payment}代入资本回报率$R^k_{t+1}$定义式\eqref{eq:HH-max-FOC-K},并利用\eqref{eq:real-rental-rate-capital}、\eqref{eq:scaled-physical-capital-price}改写为scaled variables形式
\begin{equation}
\label{eq:HH-max-FOC-K-R-scaled}
\begin{split}
R^k_{t+1} &=
\frac{
  u_{t+1} \cdot \frac{P_{t+1}}{P_t} \cdot r^k_{t+1}
}{P_{k',t}} -
\frac{
  \frac{P_{t+1}}{P_t} \cdot a(u_{t+1})
}{P_{k',t} \cdot \psi_{t+1}} + \frac{P_{t+1}}{P_t} \cdot \frac{P_{k',t+1}}{P_{k',t}} \cdot (1-\delta) \\
&= \frac{1+\pi_{t+1}}{\mu_{\psi,t+1}} \cdot \frac{
  u_{t+1} \cdot \bar{r}^k_{t+1} - \left( 1+\pi_{t+1} \right) \cdot a(u_{t+1}) + p_{k',t+1} \cdot (1-\delta)
  }{p_{k',t}}
\end{split}
\end{equation}

\subsubsection{FOC wrt u}
\label{sec:FOC-wrt-u}
家庭选择capital utilization rate $u_t$,使得同时段的net cash payments of physical capital $X^k_t$最大化,这表现为家庭问题中的静态比较。
\begin{equation}
\label{eq:HH-max-FOC-u-scaled}
\begin{split}
\frac{\partial \mathcal{L}}{\partial u_t} = 0 \Rightarrow &P_t \cdot r_t^k = \frac{P_t}{\Psi_t} \cdot a'(u_t),\\
&a'(u_t) = r^k_t \cdot \Psi_t = \bar{r}^k_t.
\end{split}
\end{equation}

\subsubsection{FOC wrt B}
\label{sec:FOC-wrt-B}
\begin{equation}
\label{eq:HH-max-FOC-B-intm}
\begin{split}
\frac{\partial \mathcal{L}}{\partial B_{t+1}} = 0 \Rightarrow
&\frac{
  \partial \left[\beta \cdot E_t v_{t+1} \cdot R_t \cdot B_{t+1} \right]
}{
  \partial B_{t+1}
}
- \frac{
  \partial v_t \cdot B_{t+1}
}{
  \partial B_{t+1}
}
=0, \\
& v_t = \beta \cdot E_t v_{t+1} \cdot R_t,
\end{split}
\end{equation}
代入scale式\eqref{eq:scaled-produc-cost-adj-coef}
\begin{equation*}
\frac{\psi_{z^+,t}}{P_t \cdot z^+_t}
= \beta \cdot \frac{\psi_{z^+,t+1}}{P_{t+1} \cdot z^+_{t+1}} \cdot R_{t},
\end{equation*}
整理得
\begin{equation}
\label{eq:HH-max-FOC-B}
\psi_{z^+,t} = \beta \cdot E_t \frac{\psi_{z^+,t+1}}{\mu_{z^+,t+1} \cdot \left(1+\pi_{t+1}\right)} \cdot R_t.
\end{equation}














































\section{Price Philips Curve}
\label{sec:price-PC}

\subsection{price dispersion index的线性近似}
\label{sec:price-dispersion-index-lin}

式\eqref{eq:price-dispersion-index-iter} $\Rightarrow$
\begin{equation*}
\ln \nu^f_t = \ln
\left[
\left(1-\phi_f\right) \cdot \left(1+\pi_t\right)^{\varepsilon_f} \cdot \left(1+\pi^{\#}_t\right)^{-\varepsilon_f}
+\phi_f \cdot \left(1+\pi_t\right)^{\varepsilon_f} \cdot \nu^f_{t-1}
\right]
\end{equation*}
\begin{equation*}
\begin{split}
\frac{\nu^f_{t} - \nu^f}{\nu^f} &=\varepsilon_f \cdot \left(1-\phi_f\right) \cdot
\left(1+\pi\right)^{\varepsilon_f-1 } \cdot \left(1+\pi^{\#}\right)^{-\varepsilon_f} \cdot \left[\pi_t-\pi\right] \\
&+\left(-\varepsilon_f\right) \cdot \left(1-\phi_f\right) \cdot \left(1+\pi\right)^{\varepsilon_f} \cdot \left(1+\pi^{\#}\right)^{-\varepsilon_f -1 } \cdot \left[\pi^{\#}_t-\pi^{\#}\right] \\
&+ \varepsilon_f \cdot \phi_f \cdot \left(1+\pi\right)^{\varepsilon_f-1} \cdot \nu^f \cdot \left[\pi_t - \pi\right],
\end{split}
\end{equation*}
沿着$\pi=0$,$\pi^{\#}=0$,$\nu^f=1$作线性近似
\begin{equation}
\hat{\nu}^f_t = \varepsilon_f \cdot \hat{\pi}_t - \varepsilon_f \cdot \left(1-\phi_f\right) \cdot \hat{\pi}^{\#}_t + \phi_f \cdot \hat{\nu}{f}_{t-1}.
\end{equation}

\subsection{两个辅助变量的线性近似}
对辅助变量$x_{1,t}^f$作线性近似。已知$U_{c,t}=1/C_t$\footnote{膏按:在写完家庭部门优化条件FOC之后,把$U_{c,t}$对应的FOC的连接写进去。},式\eqref{eq:intm-prod-auxiliary-x-1} $\Rightarrow$

\begin{equation}
\label{eq:intm-produ-x1-lin-mid}
\begin{split}
\frac{x^f_{1,t} - x^f_1}{x_1} &= \frac{- \frac{Y}{C^2} \cdot s \cdot \left(C_t - C\right)}{x^f_1} + \frac{\frac{s}{C} \cdot \left(Y_t - Y\right)}{x^f_1} + \frac{\frac{Y}{C} \cdot \left(s_t - s\right)}{x^f_1} \\
&+\beta \cdot \phi_f \cdot (1+\pi)^{\varepsilon_f -1} \cdot \varepsilon_f \cdot \left( \pi_{t+1} - \pi \right) \\
&+\beta \cdot \phi_f \cdot \left(1+\pi\right)^{\varepsilon_f} \cdot \frac{x^f_1 - x^f_1}{x^f_{1}}
\end{split}
\end{equation}
稳定状态下我们有$\pi = \pi^{\#} = 0$,因而稳态$x^f_1$的值由式\eqref{eq:intm-prod-auxiliary-x-1}给出
\begin{equation}
\label{eq:intm-prod-auxiliary-x-1-ss}
x^f_1 = \frac{1}{1-\beta \cdot \phi_f} \cdot \frac{Y \cdot s}{C}.
\end{equation}

式\eqref{eq:intm-prod-auxiliary-x-1-ss}带回式\eqref{eq:intm-produ-x1-lin-mid},进一步整理得
\begin{equation}
\label{eq:intm-prod-auxiliary-x-1-lin-final}
\hat{x}^f_{1,t} = \left(1-\beta \cdot \phi_f \right) \cdot \left(\hat{Y}_t + \hat{s}_t - \hat{C}_t\right) + \beta \cdot \phi_f \cdot E_t \left(\varepsilon_f \cdot \hat{\pi}_{t+1} + \hat{x}^f_{1,t+1}\right).
\end{equation}

对辅助变量$x^f_{2,t}$作线性近似。式\eqref{eq:intm-prod-auxiliary-x-2} $\Rightarrow$
\begin{equation}
\label{eq:intm-produ-x2-lin-mind}
\begin{split}
\frac{x^f_{2,t} - x^f_2}{x_2} &= \frac{- \frac{Y}{C^2} \cdot \left(C_t - C\right)}{x^f_2} + \frac{\frac{1}{C} \cdot \left(Y_t - Y\right)}{x^f_1}
+ \beta \cdot \phi_f \cdot \left(\varepsilon_f -1 \right) \cdot \left(1+\pi\right)^{\varepsilon - 1 - 1} \cdot \left(\pi_{t+1}-\pi\right) \\
&+\beta \cdot \phi_f \cdot (1+\pi)^{\varepsilon_f -1} \cdot \frac{x^f_{2,t+1}-x^f_2}{x^f_2},
\end{split}
\end{equation}
进一步整理得
\begin{equation}
\label{eq:intm-prod-auxiliary-x-2-lin-final}
\hat{x}^f_{2,t} = \left(1-\beta \cdot \phi_f \right) \cdot \left(\hat{Y}_t  - \hat{C}_t\right) + \beta \cdot \phi_f \cdot E_t \left[\left(\varepsilon_f -1\right) \cdot \hat{\pi}_{t+1} + \hat{x}^f_{2,t+1}\right].
\end{equation}

\subsection{reset price inflation的线性近似}
\label{sec:reset-price-inflation-lin}
式\eqref{eq:intm-prod-agg-price-idx-calvo} $\Rightarrow$
\begin{equation*}
\left( 1-\varepsilon_f \right) \cdot \ln \left(1+\pi_t\right) = \ln \left[\left(1-\phi_f\right) \cdot \left(1 - \pi^{\#}_t\right)^{1-\varepsilon_f} + \phi_f\right],
\end{equation*}
\begin{equation*}
\left( 1-\varepsilon_f \right) \cdot \left(\pi_t - \pi \right) =
\frac{\left(1-\phi_f\right) \cdot \left(1-\varepsilon_w\right) \cdot \left( 1+\pi^{\#} \right)^{-\varepsilon_w} \cdot \left(\pi^{\#}_t - \pi^{\#}\right)}{\left(1-\phi_f\right) \cdot \left(1 - \pi^{\#}_t\right)^{1-\varepsilon_f} + \phi_f},
\end{equation*}
由此可得reset price inflation的线性近似式
\begin{equation}
\label{eq:reset-price-inflation-inflation-lin}
\hat{\pi}_t = \left(1-\phi_f \right) \cdot \frac{\left(1+\pi^{\#}\right)^{-\varepsilon}}{\left(1+\pi\right)^{1-\varepsilon}} \cdot \hat{\pi}^{\#}_t \approx \left(1-\phi_f \right) \cdot \hat{\pi}^{\#}_t
\end{equation}


\subsection{Price Philips Curve}
式\eqref{eq:intm-calvo-pricing-3-scaling} $\Rightarrow$
\begin{equation*}
\ln \left( 1+\pi^{\#}_t \right) - \ln \left(1+\pi_t\right) = \ln \left(\frac{\varepsilon_f}{\varepsilon_f -1}\right) + \ln x^f_{1,t} - \ln x^f_{2,t},
\end{equation*}

\begin{equation}
\begin{split}
\hat{\pi}^{\#}_t - \hat{\pi}_t &= \hat{x}^f_{1,t} - \hat{x}^f_{2,t} \\
&= \left(1 - \beta \cdot \phi_f \right) \cdot \hat{s}_t
+ \beta \cdot \phi_f \cdot E_t \hat{\pi}_{t+1}
+ \beta \cdot \phi_f \cdot E_t \left(\hat{x}^f_{1,t+1} - \hat{x}^f_{2,t+1}\right),
\end{split}
\end{equation}
将式\eqref{eq:reset-price-inflation-inflation-lin}的代入上式替代$\hat{\pi}^{\#}_t$,得到price philips curve
\begin{equation*}
\begin{split}
\hat{\pi}_t &= \left(\frac{1- \phi_f}{\phi_f}\right) \cdot \left(1-\beta \cdot \phi_f \right) \cdot \hat{s}_t +  \beta \cdot E_t \hat{\pi}_{t+1} \\
&=\left(\frac{1- \phi_f}{\phi_f}\right) \cdot \left(1-\beta \cdot \phi_f \right) \cdot \left[\alpha \cdot \left(1+\phi_f\right) \cdot \hat{X}_t + \hat{R}_t\right]
+ \beta \cdot E_t \hat{\pi}_{t+1}
\end{split}
\end{equation*}
其中$\hat{s}_t$由式\eqref{eq:intm-mkt-real-mrg-cost-efficiency}作近似线性测得;$X_t$表示output gap,$\hat{s}_t$和$\hat{X}_t$的关系参见基准NK模型相关部分\footnote{膏按:这部分还没看明白,需要进一步搞清楚。}。

据此得到price philips curve
\begin{equation}
\label{eq:price-PC-lin}
\begin{split}
\hat{\pi}_t &= \kappa_f \cdot \left[\alpha \cdot \left(1+\eta\right) \cdot \hat{X}_t + \hat{R}_t\right] + \beta \cdot E_t \hat{\pi}_{t+1},\\
&\text{其中定义系数 } \kappa_f \equiv \frac{1-\phi_f}{\phi_f} \cdot \left(1-\beta \cdot \phi_f \right).
\end{split}
\end{equation}

\section{Wage Philips Curve}
\label{sec:wage-Philips-Curve-me}
% \begin{eqnarray*}
%   x_{1,t}^{w} & = & H_t^{1+\eta} + \left(\beta \cdot \phi_w\right) \cdot \left[\left(\frac{\bar{w}_t}{\bar{w}_{t+1}} \cdot \mathcal{X}_{t,1}\right)^{-\varepsilon_w} \cdot H_{t+1}\right]^{1+\eta} \\
%   & & {} + \left(\beta \cdot \phi_w\right)^2 \cdot \left[\left(\frac{\bar{w}_t}{\bar{w}_{t+2}} \cdot \mathcal{X}_{t,2}\right)^{-\varepsilon_w} \cdot H_{t+2}\right]^{1+\eta} \\
%   & & {} + \left(\beta \cdot \phi_w\right)^3 \cdot \left[\left(\frac{\bar{w}_t}{\bar{w}_{t+3}} \cdot \mathcal{X}_{t,3}\right)^{-\varepsilon_w} \cdot H_{t+3}\right]^{1+\eta}\\
%   & & {} + \ldots
% \end{eqnarray*}

% \begin{eqnarray*}
%   x_{1,t}^{w} & = & H_t^{1+\eta} + \left(\beta \cdot \phi_w\right) \cdot \left[\left(\frac{\bar{w}_t}{\bar{w}_{t+1}} \cdot \mathcal{X}_{t,1}\right)^{-\varepsilon_w} \cdot H_{t+1}\right]^{1+\eta} \\
%   & & {} + \left(\beta \cdot \phi_w\right)^2 \cdot \left[\left(\frac{\bar{w}_t}{\bar{w}_{t+2}} \cdot \mathcal{X}_{t,2}\right)^{-\varepsilon_w} \cdot H_{t+2}\right]^{1+\eta} \\
%   & & {} + \left(\beta \cdot \phi_w\right)^3 \cdot \left[\left(\frac{\bar{w}_t}{\bar{w}_{t+3}} \cdot \mathcal{X}_{t,3}\right)^{-\varepsilon_w} \cdot H_{t+3}\right]^{1+\eta}\\
%   & & {} + \ldots
% \end{eqnarray*}

式\eqref{eq:union-max-prob-FOC-wsharp}的最优化问题可以调整为
\begin{equation}
\left[\frac{1}{1-\varepsilon_w}\cdot \frac{1}{w^{\#}_t}\right] \cdot
E_t \sum_{m=0}^{\infty} \cdot \left( \beta \cdot \phi_w\right)^{m} \cdot \psi_{z^+,t+m} \cdot h_{t+m}^{t} \cdot \left[
w_t^{\#} \cdot \bar{w}_t \cdot \mathcal{X}_{t,m} - \frac{\varepsilon_w}{\varepsilon_w -1} \cdot \left(
\frac{A_L \cdot \left( h^{t}_{t+m} \right)^{\eta}}{\psi_{z^+,t+m}}
\right)
\right],
\end{equation}
或者进一步变为
\begin{equation}
\label{eq:union-max-wage-PC-1}
E_t \sum_{m=0}^{\infty} \cdot \left( \beta \cdot \phi_w\right)^{m} \cdot \psi_{z^+,t+m} \cdot h_{t+m}^{t} \cdot \left[
w_t^{\#} \cdot \bar{w}_t \cdot \mathcal{X}_{t,m} - \frac{\varepsilon_w}{\varepsilon_w -1} \cdot MRS_{t+m}^t
\right],
\end{equation}

其中
\begin{equation}
\label{eq:union-max-wage-PC-MRS}
MRS_{t+m}^t \equiv \frac{A_L \cdot \left( h^{t}_{t+m} \right)^{\eta}}{\psi_{z^+,t+m}}
\end{equation}
表示边际劳动成本,上角标和下角标表示在$t$时间调整工资,在随后的$t+1,\ldots t+m$时间均未调整。稳定状态下forward-looking的劳工联盟最优策略为,将工资设为等于提供额外1单位劳动力(劳动时间)的边际成本乘以price markup,即使式\eqref{eq:union-max-wage-PC-1}中中括号内的部分等于零;线性展开见下文式\eqref{eq:union-max-wage-philips-curve}。

\subsection{稳定状态的描述}
\label{sec:union-wage-PC-ss-description}
稳定状态下,$w^{\#}=\mathcal{X}=1$,$\bar{w}=\frac{\varepsilon_w}{\varepsilon_w -1} \cdot MRS$,$1+\tilde{\pi} = (1+\pi) \cdot \mu_{z^+}$。

\subsection{wage inflation}
\label{sec:union-wage-PC-wage-inflation}
对$\left( 1+\tilde{\pi}_{w,t+1} \right)$作线性近似。式\eqref{eq:union-wage-no-adj-inflation} $\Rightarrow$
\begin{equation*}
\ln (1+\tilde{\pi}_{w,t+1}) = \kappa_w \cdot \ln (1+\pi_t) + (1-\kappa_w) \cdot \ln (1+\pi_t) + \ln \mu_{z^+},
\end{equation*}
\begin{align}
\label{eq:union-max-wage-PC-lin-wage-inflation}
\hat{\tilde{\pi}}_{w,t+1} &\approx \tilde{\pi}_{w,t+1} - \tilde{\pi}_{w} \approx \kappa_w \cdot (\pi_t - \pi) = \kappa_w \cdot \hat{\pi}_t \\
\label{eq:union-max-wage-PC-ss-wage-inflation-pi-z}
\tilde{\pi}_{w} &= \pi \cdot \mu_{z^+}
\end{align}

\subsection{辅助变量}
\label{sec:union-wage-PC-wage-auxliary}
对$\hat{\mathcal{X}}_{t,m}$作线性近似。式\eqref{eq:mathcal-X-auxiliary-definition} $\Rightarrow$
\begin{align}
\label{eq:union-max-wage-PC-Xtm-n0}
\hat{\mathcal{X}}_{t,m \neq 0} &\approx   \left[\left(
  \hat{\tilde{\pi}}_{w,t+1} + \hat{\tilde{\pi}}_{w,t+2} + \ldots +\hat{\tilde{\pi}}_{w,t+m}
  \right) - \left(
  \hat{\pi}_{t+1} + \hat{\pi}_{t+2} + \ldots \hat{\pi}_{t+m}
  \right)\right] \nonumber \\
  &-\left(
  \hat{\mu}_{z^+,t+1} + \hat{\mu}_{z^+,t+2} + \ldots \hat{\mu}_{z^+,t+m}
  \right) \nonumber \\
  &= -\left[
  \left(\hat{\pi}_{t+1} - \kappa_w \cdot \hat{\pi}_{t}\right) +
  \left(\hat{\pi}_{t+2} - \kappa_w \cdot \hat{\pi}_{t+1}\right) +
  \ldots +
  \left(\hat{\pi}_{t+m} - \kappa_w \cdot \hat{\pi}_{t+m-1}\right)
  \right] \nonumber \\
  &-\left(
  \hat{\mu}_{z^+,t+1} + \hat{\mu}_{z^+,t+2} +\ldots + \hat{\mu}_{z^+,t+m}
  \right) \nonumber \\
  &\equiv - \left(\Delta \kappa_w \pi_{t+1} + \Delta \kappa_w \pi_{t+2} + \ldots \Delta \kappa_w \pi_{t+m}\right)-\left(
  \hat{\mu}_{z^+,t+1} + \hat{\mu}_{z^+,t+2} + \ldots + \hat{\mu}_{z^+,t+m}
  \right)
\end{align}
以及
\begin{align}
\label{eq:union-max-wage-PC-Xtm-0}
\hat{\mathcal{X}}_{t,0} = 0,
\end{align}
其中定义
\begin{align}
\label{def-kappa-Delta-pi-w}
\Delta_{\kappa_w} \pi_{w,t+1} &\equiv \hat{\pi}^{\#}_{w,t+1} - \kappa_w \cdot \hat{\pi}_t, \\
\label{def-kappa-Delta-pi}
\Delta_{\kappa_w} \pi_{t+1} &\equiv \hat{\pi}_{t+1} - \kappa_w \cdot \hat{\pi}_t.
\end{align}

\subsection{劳动力供应}
\label{sec:union-wage-PC-hours-worked}
对$h_{t+m}$ 式\eqref{eq:h-H-t-m-j-interm}作线性近似。 $\Rightarrow$
\begin{align}
\frac{\bar{w}_t}{\bar{w}_{t+m}} &=
\frac{\frac{W_t}{P_t \cdot z_t^+}}{\frac{W_{t+m}}{P_{t+m} \cdot z_{t+m}^{+}}} \nonumber \\
&= \frac{W_t}{W_{t+m}} \cdot \frac{P_{t+m}}{P_{t}} \cdot \frac{z^+_{t+m}}{z^+_t} \nonumber \\
&= \frac{\left(1+\pi_{t+1}\right) \cdot \left(1+\pi_{t+2}\right) \cdot \ldots \left(1+\pi_{t+m}\right) \cdot \left(\mu_{z^+,t+1} \cdot \mu_{z^+,t+2} \cdot \ldots \mu_{z^+,t+m}\right)}{\left(1+\tilde{\pi}_{w,t+1}\right) \cdot \left(1+\tilde{\pi}_{w,t+2}\right) \cdot \ldots \cdot \left(1+\tilde{\pi}_{w,t+m}\right)},
\end{align}
其中最后一个等号用到了式\eqref{eq:union-wage-no-adj-inflation}。结合式\eqref{eq:union-max-wage-PC-ss-wage-inflation-pi-z},对上式线性展开,整理得
\begin{equation}
\label{eq:union-wage-PC-wtm-lin}
\hat{\left(\frac{\bar{w}_t}{\bar{w}_{t+m}}\right)} \approx \hat{\mu}_{z^+,t+1} \cdot \hat{\mu}_{z^+,t+2} \cdot \ldots \cdot \hat{\mu}_{z^+,t+m}.
\end{equation}

将式\eqref{eq:union-wage-PC-wtm-lin}代入式\eqref{eq:h-H-t-m-j-interm} $\Rightarrow$
\begin{equation}
\label{eq:union-wage-PC-htm-lin}
\hat{h}^t_{t+m} - \hat{H}_{t+m} =
\begin{cases}
-\varepsilon_w \cdot \left[\hat{w}^{\#}_t - \left( \Delta \kappa_w \pi_{t+1} + \Delta \kappa_w \pi_{t+2} + \ldots + \Delta \kappa_w \pi_{t+m}\right)\right], & \text{当 }m>0, \\
-\varepsilon_w \cdot \hat{w}^{\#}_t, & \text{当 }m=0.
\end{cases}
\end{equation}



\subsection{边际劳动成本}
\label{sec:union-wage-PC-wage-MRS}
对$RMS^t_{t+m}$作线性近似。式\eqref{eq:union-max-wage-PC-MRS} $\Rightarrow$
\begin{equation*}
\ln MRS^t_{t+m} = \ln A_L + \eta \cdot \ln^t_{t+m} - \ln \Psi_{z^+,t+m},
\end{equation*}
\begin{equation}
\label{eq:union-wage-MRS-lin}
\hat{MRS}^t_{t+m} = -\hat{\Psi}_{z^+,t+m}+ \eta \cdot\left(\hat{h}^t_{t+m} - \hat{H}_{t+m}\right)+ \eta \cdot \hat{H}_{t+m}.
\end{equation}

\subsection{wage price inflation}
\label{sec:union-wage-agg-union-lin}
式\eqref{eq:agg-wage-index-calvo-inflation}反映了,aggreate wage index $W_t$对劳工联盟制定工资指数$W^{\#}_t$的限制。调整得
\begin{equation*}
1 = \left( 1- \phi_w \right) \cdot w_t^{\#,1-\varepsilon_w} + \phi_w \cdot \left( \frac{1+\pi^{\#}_{w,t}}{1+\pi_{w,t}} \right)^{1-\varepsilon_w},
\end{equation*}
两侧取$\ln$后作近似线性展开
\begin{align}
\label{union-wage-inflation-index-lin-interm}
0 &= \frac{\left(1-\phi_w \right) \cdot \left(1-\varepsilon_w \right) \cdot w^{\#,-\varepsilon_w} \cdot \left(w^{\#}_t - w^{\#}\right)}{\left(1-\phi_w\right) \cdot w^{\#,1-\varepsilon_w} + \phi_w \cdot \left( \frac{\pi^{\#}_{w}}{\pi_{w}} \right)^{1-\varepsilon_w}} \nonumber \\
&+ \frac{\phi_w \cdot \left(1-\varepsilon_w \right) \cdot \left( \frac{\pi^{\#}_{w}}{\pi_{w}} \right)^{1-\varepsilon_w} \cdot \pi^{\#,-1}_{w} \cdot \left(\pi^{\#}_{w,t}-\pi^{\#}_{w}\right)}{\left(1-\phi_w\right) \cdot w^{\#,1-\varepsilon_w} + \phi_w \cdot \left( \frac{\pi^{\#}_{w}}{\pi_{w}} \right)^{1-\varepsilon_w}}  \nonumber \\
&+ \frac{\phi_w \cdot \left(\varepsilon_w -1 \right) \cdot \left( \frac{\pi^{\#}_{w}}{\pi_{w}} \right)^{1-\varepsilon_w} \cdot \pi^{-1}_{w} \cdot \left(\pi_{w,t}-\pi_{w}\right)}{\left(1-\phi_w\right) \cdot w^{\#,1-\varepsilon_w} + \phi_w \cdot \left( \frac{\pi^{\#}_{w}}{\pi_{w}} \right)^{1-\varepsilon_w}}  \nonumber \\
&= \frac{  \left(1-\varepsilon_w \right) \cdot \left[ (1-\phi_w) \cdot w^{\#, 1-\varepsilon_w} \cdot \hat{w}^{\#}_t + \phi_w \cdot \left(\frac{\pi^{\#}_{w}}{\pi_w}\right)^{1-\varepsilon_w} \cdot \left(\hat{\pi}^{\#}_{w,t} - \hat{\pi}_{w,t} \right) \right] }{\left(1-\phi_w\right) \cdot w^{\#,1-\varepsilon_w} + \phi_w \cdot \left( \frac{\pi^{\#}_{w}}{\pi_{w}} \right)^{1-\varepsilon_w}}.
\end{align}


稳定状态下,$w^{\#}=1$,$\pi^{\#}_{w}/\pi_{w}=1$,且根据式\eqref{def-kappa-Delta-pi-w}得
\begin{equation*}
\hat{\pi}^{\#}_{w,t}-\hat{\pi}_{w,t} = -\left(\hat{\pi}_{w,t} - \kappa_w \cdot \hat{\pi}_{t-1} \right) = -\Delta_{\kappa_w}\hat{\pi}_{w,t}.
\end{equation*}

因此式\eqref{union-wage-inflation-index-lin-interm}可以进一步改写为如下wage price inflation index
\begin{equation}
\label{union-wage-inflation-index-lin}
\hat{w}^{\#}_t = \frac{\varepsilon_w}{1-\varepsilon_w} \cdot \Delta_{\kappa_w} \hat{\pi}_{w,t}.
\end{equation}

\subsection{其他辅助变量}
\label{sec:union-wage-PC-wage-auxiliary-others}
由\eqref{eq:union-wage-PC-htm-lin}可得
\begin{equation*}
\begin{split}
&\sum_{m=0}^{\infty} \eta \cdot \left(\hat{h}_{t+m}^t - \hat{H}_{t+m}\right) \\
&=-\varepsilon_w \cdot \eta \cdot \hat{w}^{\#}_t \\
&-\left(\beta \cdot \phi_w \right) \cdot \varepsilon_w \cdot \eta \cdot \left[\hat{w}^{\#}_t - \Delta\kappa_w\pi_{t+1}\right] \\
&-\left(\beta \cdot \phi_w \right)^2 \cdot \varepsilon_w \cdot \eta \cdot \left[\hat{w}^{\#}_t - \Delta\kappa_w\pi_{t+1} - \Delta\kappa_w\pi_{t+2}\right] \\
&-\left(\beta \cdot \phi_w \right)^3 \cdot \varepsilon_w \cdot \eta \cdot \left[\hat{w}^{\#}_t - \Delta\kappa_w\pi_{t+1} - \Delta\kappa_w\pi_{t+2} - \Delta\kappa_w\pi_{t+3}\right] \ldots\\
&=-\varepsilon_w \cdot \eta \cdot \left[1 + \left(\beta \cdot \phi_w\right) + \left(\beta \cdot \phi_w\right)^2 + \ldots + \left(\beta \cdot \phi_w\right)^m\right] \cdot \hat{w}^{\#}_t \\
& +\varepsilon_w \cdot \eta \cdot \left[\left(\beta \cdot \phi_w\right) + \left(\beta \cdot \phi_w\right)^2 + \ldots + \left(\beta \cdot \phi_w\right)^m\right] \cdot \Delta\kappa_w\pi_{t+1} \\
& +\varepsilon_w \cdot \eta \cdot \left[\left(\beta \cdot \phi_w\right)^2 + \left(\beta \cdot \phi_w\right)^3 + \ldots + \left(\beta \cdot \phi_w\right)^m\right] \cdot \Delta\kappa_w\pi_{t+2} \\
& +\varepsilon_w \cdot \eta \cdot \left[\left(\beta \cdot \phi_w\right)^3 + \left(\beta \cdot \phi_w\right)^4 + \ldots + \left(\beta \cdot \phi_w\right)^m\right] \cdot \Delta\kappa_w\pi_{t+3} \\
& + \ldots \\
& + \varepsilon_w \cdot \eta \cdot \left[\left(\beta \cdot \phi_w\right)^m\right] \cdot \Delta\kappa_w\pi_{t+m} \\
&\approx -\varepsilon_w \cdot \eta \cdot \left(\beta \cdot \phi_w\right)^{0} \cdot \left[1+\left(\beta \cdot \phi_w\right)+\left(\beta \cdot \phi_w\right)^2 + \ldots + \left(\beta \cdot \phi_w\right)^m\right] \cdot \hat{w}^{\#}_t \\
&+\varepsilon_w \cdot \eta \cdot \left(\beta \cdot \phi_w\right)^{1} \cdot \left[1+\left(\beta \cdot \phi_w\right)+\left(\beta \cdot \phi_w\right)^2 + \ldots + \left(\beta \cdot \phi_w\right)^m\right] \cdot  \Delta\kappa_w\pi_{t+1}\\
&+\varepsilon_w \cdot \eta \cdot \left(\beta \cdot \phi_w\right)^{2} \cdot \left[1+\left(\beta \cdot \phi_w\right)+\left(\beta \cdot \phi_w\right)^2 + \ldots + \left(\beta \cdot \phi_w\right)^m\right] \cdot  \Delta\kappa_w\pi_{t+2}\\
&+\varepsilon_w \cdot \eta \cdot \left(\beta \cdot \phi_w\right)^{3} \cdot \left[1+\left(\beta \cdot \phi_w\right)+\left(\beta \cdot \phi_w\right)^2 + \ldots + \left(\beta \cdot \phi_w\right)^m\right] \cdot  \Delta\kappa_w\pi_{t+3} + \ldots \\
\end{split}
\end{equation*}

由于$\left[1+\left(\beta \cdot \phi_w\right)+\left(\beta \cdot \phi_w\right)^2 + \ldots + \left(\beta \cdot \phi_w\right)^m\right] \approx 1/(1-\beta \cdot \phi_w)$,随着$m \rightarrow \infty$ 上式化简为
\begin{equation}
\label{eq:union-wage-PC-htm-rev}
\begin{split}
&\sum_{m=0}^{\infty} \eta \cdot \left(\hat{h}_{t+m}^t - \hat{H}_{t+m}\right) \\
&\approx - \frac{\varepsilon_w \cdot \eta}{1-\beta \cdot \phi_w} \cdot
\left\{\hat{w}^{\#}_t - \left[\left(\beta \cdot \phi_w \right) \cdot \Delta\kappa_w \pi_{t+1}+\left(\beta \cdot \phi_w \right)^2 \cdot \Delta\kappa_w \pi_{t+2} + \ldots + \left(\beta \cdot \phi_w \right)^m \cdot \Delta\kappa_w \pi_{t+m}\right]\right\},
\end{split}
\end{equation}

将上式中定义中括号中的内容定义为$S_{w,t}$,进而改写为递归形式
\begin{equation}
\begin{split}
\label{eq:union-wage-PC-Swt}
S_{w,t} &\equiv \left(\beta \cdot \phi_w \right) \cdot \Delta\kappa_w \pi_{t+1}+\left(\beta \cdot \phi_w \right)^2 \cdot \Delta\kappa_w \pi_{t+2} + \ldots + \left(\beta \cdot \phi_w \right)^m \cdot \Delta\kappa_w \pi_{t+m}\\
&\approx \left(\beta \cdot \phi_w \right) \cdot \left(\Delta \kappa_w \pi_{t+1} + S_{w,t+1}\right),
\end{split}
\end{equation}

式\eqref{eq:union-wage-PC-Swt}代回式\eqref{eq:union-wage-PC-htm-rev}可得
\begin{equation}
\label{eq:union-wage-PC-h-H-tm-Swt-lin}
\sum_{m=0}^{\infty} \eta \cdot \left(\hat{h}_{t+m}^t - \hat{H}_{t+m}\right) \approx - \frac{\varepsilon_w \cdot \eta}{1-\beta \cdot \phi_w} \cdot \left[ \hat{w}^{\#}_t -S_{w,t}\right],
\end{equation}

结合式\eqref{eq:union-wage-PC-h-H-tm-Swt-lin}和式\eqref{eq:union-wage-MRS-lin},作时间$m$的加总得
\begin{equation}
\begin{split}
\label{eq:union-wage-PC-SMRSt-interm}
S_{MRS,t} &\equiv \sum_{m=0}^{\infty} \left(\beta \cdot \phi_w\right)^m \cdot \hat{MRS}^{t}_{t+m} \\
&=\sum_{m=0}^{\infty} \left(\beta \cdot \phi_w\right)^m \cdot \left[-\hat{\Psi}_{z^+,t+m} + \eta \cdot \hat{H}_{t+m} \right] + \sum_{m=0}^{\infty} \left(\beta \cdot \phi_w\right)^m \cdot \left[\eta \cdot \left( \hat{h}^t_{t+m} - \hat{H}_{t+m} \right) \right],
\end{split}
\end{equation}
第二个等式中后半部分由式\eqref{eq:union-wage-PC-h-H-tm-Swt-lin}给出。将前半部分定义为$S_{o,t}$,进而改写为递归形式
\begin{equation}
\begin{split}
\label{eq:union-wage-PC-Sot}
S_{o,t} &\equiv \sum_{m=0}^{\infty} \left(\beta \cdot \phi_w\right)^m \cdot \left[-\hat{\Psi}_{z^+,t+m} + \eta \cdot \hat{H}_{t+m} \right]\\
&\approx - \hat{\Psi}_{z^+,t} + \eta \cdot \hat{H}_t + \left(\beta \cdot \phi_w \right) \cdot S_{o,t+1},
\end{split}
\end{equation}

由此,式\eqref{eq:union-wage-PC-h-H-tm-Swt-lin}、式\eqref{eq:union-wage-PC-Sot}代回式\eqref{eq:union-wage-PC-SMRSt-interm},进一步调整得
\begin{equation}
\label{eq:union-wage-PC-SMRSt}
S_{MRS,t} = S_{o,t} - \frac{\varepsilon_w \cdot \eta}{1 - \beta \cdot \phi_w} \cdot \left[ \hat{w}^{\#}_t - S_{w,t}\right].
\end{equation}


结合式\eqref{eq:union-max-wage-PC-Xtm-n0}对$\hat{\mathcal{X}}_{t,m}$作沿时间$m$的加总,定义为$S_{x,t}$,进而改写为递归形式,得
\begin{equation}
\label{eq:union-wage-PC-Sxt}
\begin{split}
S_{x,t} &\equiv \sum_{m=0}^{\infty} \left( \beta \cdot \phi_w \right)^m \cdot \hat{\mathcal{X}}_{t,m} \\
&= \frac{1}{1-\beta \cdot \phi_w} \cdot \sum_{m=0}^{\infty} \cdot \left( \beta \cdot \phi_w \right)^m \cdot \left[ - \Delta \kappa_w \hat{\pi}_{t+1} - \hat{\mu}_{z^+,t+1} \right] \\
&\approx \frac{\beta \cdot \phi_w }{1-\beta \cdot \phi_w} \cdot \left[ - \Delta \kappa_w \hat{\pi}_{t+1} - \hat{\mu}_{z^+,t+1} \right] + \left( \beta \cdot \phi_w \right) \cdot S_{x,t+1}.
\end{split}
\end{equation}


稳定状态下的最优决策,式\eqref{eq:union-max-wage-PC-1}的中括号中内容应当为零,线性近似,整理得
\begin{align}
&\ln w^{\#}_t + \ln \bar{w}_t  = \ln \left(\frac{\bar{w}_t}{\bar{w}_{t+m}}\right) + \ln MRS_{t+m}^{t} -\ln \mathcal{X}_{t,m}, \nonumber \\
\label{eq:union-wage-PC-curve-intm-2}
&\frac{\hat{w}^{\#}_t + \hat{\bar{w}}_t}{1-\beta \cdot \phi_w}  + S_{x,t} - S_{MRS,t} = 0,
\end{align}
将式\eqref{union-wage-inflation-index-lin},\eqref{eq:union-wage-PC-Sxt},\eqref{eq:union-wage-PC-SMRSt}的$\hat{w}^{\#}_t, S_{x,t}, S_{MRS,t}$代入上式,得
\begin{equation}
\label{eq:union-wage-PC-curve-intm-3}
\begin{split}
&\frac{1}{1-\beta \cdot \phi_w} \cdot \hat{\bar{w}}_t + \frac{1}{1-\beta \cdot \phi_w} \cdot  \hat{w}^{\#}_t  + S_{x,t} \\
&= S_{o,t} - \frac{1}{1-\beta \cdot \phi_w} \cdot \varepsilon_w \cdot \eta \cdot \left[
  \frac{\phi_w}{1-\phi_w} \cdot \Delta_{\kappa_w} \hat{\pi}_{w,t} - S_{w,t}
\right].
\end{split}
\end{equation}

将式\eqref{eq:union-wage-PC-curve-intm-3}改为$t$时刻对$t+1$期的期望
\begin{equation}
\label{eq:union-wage-PC-curve-intm-3-1}
\begin{split}
&\frac{1}{1-\beta \cdot \phi_w} \cdot \left(\beta \cdot \phi_w \cdot \hat{\bar{w}}_{t+1}\right) + \frac{1}{1-\beta \cdot \phi_w} \cdot  \left(\beta \cdot \phi_w \cdot \hat{w}^{\#}_{t+1} \right) + \left( \beta \cdot \phi_w \cdot S_{x,t+1}\right) \\
&= \left( \beta \cdot \phi_w \cdot S_{o,t+1} \right) - \frac{1}{1-\beta \cdot \phi_w} \cdot \varepsilon_w \cdot \eta \cdot \left[
  \frac{\phi_w}{1-\phi_w} \cdot \left(\beta \cdot \phi_w \cdot \Delta_{\kappa_w} \hat{\pi}_{w,t+1}\right) - \left( \beta \cdot \phi_w \cdot S_{w,t+1} \right)
\right],
\end{split}
\end{equation}

式\eqref{eq:union-wage-PC-curve-intm-3}减去式\eqref{eq:union-wage-PC-curve-intm-3-1},整理得
\begin{equation}
\label{eq:union-wage-PC-curve-intm-4}
\begin{split}
&\frac{1}{1-\beta \cdot \phi_w} \cdot \left(\hat{\bar{w}}_{t} - \beta \cdot \phi_w \cdot \hat{\bar{w}}_{t+1}\right)
+ \frac{1}{1-\beta \cdot \phi_w} \cdot \left( \hat{w}^{\#}_{t} - \beta \cdot \phi_w \cdot \hat{w}^{\#}_{t+1} \right)
+ \frac{\beta \cdot \varepsilon_w }{1-\beta \cdot \phi_w} \cdot  \left( \Delta_{\kappa_w} \hat{\pi}_{t+1} + \hat{\mu}_{z^+,t+1} \right) \\
&= \left( -\Psi_{z^+,t} + \eta \cdot \hat{H}_t \right)
- \frac{\varepsilon_w \cdot \eta}{1-\beta \cdot \phi_w} \cdot
  \frac{\phi_w}{1-\phi_w} \cdot \left(\Delta_{\kappa_w} \hat{\pi}_{w,t} - \beta \cdot \phi_w \cdot \Delta_{\kappa_w} \hat{\pi}_{w,t+1}\right) \\
&+ \frac{\varepsilon_w \cdot \eta \cdot \beta \cdot \phi_w }{1-\beta \cdot \phi_w} \cdot \Delta_{\kappa_w}\hat{\pi}_{w,t+1},
\end{split}
\end{equation}

根据式\eqref{eq:scaled-real-wage}得
\begin{equation*}
%\label{eq:}
  \frac{\bar{w}_t}{\bar{w}_{t-1}} = \frac{W_t}{W_{t-1}} \cdot \frac{P_{t-1}}{P_t}\cdot \frac{z^+_{t-1}}{z^+_{t}},
\end{equation*}
线性近似展开
\begin{equation*}
\ln \bar{w}_t = \ln \bar{w}_{t-1} + \ln \pi_{w,t} - \ln \pi_t - \ln \mu_{z^+,t},
\end{equation*}
\begin{equation}
\label{eq:union-wage-PC-curve-intm-5}
\hat{\bar{w}}_t = \hat{\bar{w}}_{t-1} + \hat{\pi}_{w,t} - \hat{\pi}_t - \hat{\mu}_{z^+,t},
\end{equation}

对式\eqref{def-kappa-Delta-pi}作近似线性展开,并利用式\eqref{eq:union-wage-PC-curve-intm-5}替代$\hat{\pi}_t$得
\begin{align}
\label{eq:union-wage-PC-curve-intm-6}
\Delta_{\kappa_w} \hat{\pi}_t &= \hat{\pi}_{t} - \kappa_w \cdot \hat{\pi}_{t-1} \nonumber \\
&= -\left(\hat{\bar{w}}_t - \hat{\bar{w}}_{t-1}\right) + \left(\hat{\pi}_{w,t} - \kappa_w \cdot \hat{\pi}_{t-1}\right) - \hat{\mu}_{z^+,t} \nonumber \\
&=\left(\hat{\bar{w}}_{t-1} - \hat{\bar{w}}_{t}\right) + \Delta_{\kappa_w} \hat{\pi}_{w,t} - \hat{\mu}_{z^+,t}
\end{align}

式\eqref{eq:union-wage-PC-curve-intm-5}-\eqref{eq:union-wage-PC-curve-intm-6}代回式\eqref{eq:union-wage-PC-curve-intm-4},重写为
\begin{align*}
&\frac{1}{1-\beta \cdot \phi_w} \cdot \left( \hat{\bar{w}}_t - \beta \cdot \phi_w \cdot \hat{\bar{w}}_{t+1}\right)
+ \frac{1}{1-\beta \cdot \phi_w} \cdot \frac{\phi_w}{1-\phi_w} \cdot \left(\Delta_{\kappa_w}\hat{\pi}_{w,t} -\beta \cdot \phi_w \cdot \Delta_{\kappa_w} \hat{\pi}_{w,t+1}\right) \\
&- \frac{\beta \cdot \phi_w }{1-\beta \cdot \phi_w} \cdot \left[
  \Delta_{\kappa_w} \cdot \hat{\pi}_{w,t+1} + \left(\hat{\bar{w}}_t - \hat{\bar{w}}_{t+1}\right)
\right]  \\
&=\left(-\hat{\Psi}_{z^+,t} + \eta \cdot \hat{H}_t\right)
- \frac{\varepsilon_w \cdot \eta }{1-\beta \cdot \phi_w} \cdot  \frac{\phi_w}{1-\phi_w} \cdot \left(\Delta_{\kappa_w} \hat{\pi}_{w,t} - \beta \cdot \phi_w \cdot \Delta_{\kappa_w} \hat{\pi}_{w,t+1}\right) \\
&+\frac{\varepsilon_w \cdot \eta \cdot \beta \cdot \phi_w }{1-\beta \cdot \phi_w} \cdot \Delta_{\kappa_w} \hat{\pi}_{w,t+1},
\end{align*}
进一步调整为
\begin{align*}
&\frac{1}{1-\beta \cdot \phi_w} \cdot \left(\hat{\bar{w}}_t - \beta \cdot \phi_w \cdot \hat{\bar{w}}_t\right)
- \frac{\beta \cdot \phi_w}{1-\beta \cdot \phi_w} \cdot\left(\hat{\bar{w}}_{t+1} - \hat{\bar{w}}_{t+1} \right)
+ \frac{1 + \varepsilon_w \cdot \eta }{1 - \beta \cdot \phi_w} \cdot \frac{\phi_w}{1-\phi_w} \cdot \left( \Delta_{\kappa_w} \hat{\pi}_{w,t} \right) \\
& - \frac{\beta \cdot \phi_w}{1-\beta \cdot \phi_w} \cdot \frac{1 + \varepsilon_w \cdot \eta }{1 -  \phi_w} \cdot \Delta_{\kappa_w} \hat{\pi}_{w,t+1}
= - \hat{\Psi}_{z^+,t} + \phi \cdot \hat{H}_t,
\end{align*}
再整理得wage philips curve
\begin{equation}
\label{eq:union-max-wage-philips-curve}
\begin{split}
  &\left(1+\varepsilon_w \cdot \eta \right) \cdot \Delta_{\kappa_w} \hat{\pi}_{w,t} \\
  &= \left(1-\beta \cdot \phi_w \right) \cdot \left(\frac{1-\phi_w}{\phi_w}\right) \cdot \left( -\Psi_{z^+,t} + \eta \cdot \hat{H}_t - \bar{w}_t \right) + \beta \cdot \left(1+\eta \cdot \varepsilon_w \right) \cdot \Delta_{\kappa_w} \cdot \hat{\pi}_{w,t+1}  \\
  &= \kappa_w \cdot \left(-\Psi_{z^+,t} + \eta \cdot \hat{H}_t - \bar{w}_t\right) + \beta \cdot \left(1+ \eta \cdot \varepsilon_w \right) \cdot \Delta_{\kappa_w} \hat{\pi}_{w,t+1},\\
&\text{其中定义系数} \quad  \kappa_w \equiv  \left(1-\beta \cdot \phi_w \right) \cdot \left(\frac{1-\phi_w}{\phi_w}\right). \end{split}
\end{equation}
根据工资philips curve式\eqref{eq:union-max-wage-philips-curve},当期名义工资率的增速$\Delta_{\kappa_w}\hat{\pi}_{w,t}$与对未来名义工资率增速的预期正相关,与额外1单位劳动投入的边际成本$\left(-\hat{\Psi}_{z^+,t}+\phi \cdot \hat{H}_t\right)$正相关,与实际工资$\hat{\bar{w}}_t$正相关。



\section{比较两条philips curve}
\label{sec:price-PC-wage-PC-comp}
比较price philips curve式\eqref{eq:price-PC-lin}和wage philips curve式\eqref{eq:union-max-wage-philips-curve}。
\begin{equation*}
\begin{split}
\hat{\pi}_t &= \kappa_f \cdot \left[\alpha \cdot \left(1+\eta\right) \cdot \hat{X}_t + \hat{R}_t\right] + \beta \cdot E_t \hat{\pi}_{t+1},\\
&\text{其中 } \quad \kappa_f \equiv \frac{1-\phi_f}{\phi_f} \cdot \left(1-\beta \cdot \phi_f \right),
\end{split}
\end{equation*}
\begin{equation*}
\begin{split}
  \Delta_{\kappa_w} \hat{\pi}_{w,t}  &= \frac{\kappa_w}{\left(1+\varepsilon_w \cdot \eta \right)} \cdot \left(-\Psi_{z^+,t} + \eta \cdot \hat{H}_t - \bar{w}_t\right) + \beta \cdot \Delta_{\kappa_w} \hat{\pi}_{w,t+1},\\
&\text{其中 } \quad  \kappa_w \equiv  \left(1-\beta \cdot \phi_w \right) \cdot \left(\frac{1-\phi_w}{\phi_w}\right).
\end{split}
\end{equation*}

假定price stickiness和wage stikiness的程度相等,即$\kappa_f = \kappa_w$。那么,如果$1+\varepsilon_w \cdot \eta > 1$的话,则PPC中边际成本$\left(\alpha \cdot (1+\eta) \cdot \hat{X_t}\right)$的斜率$\kappa_f$要高于WPC中边际成本$\left(-\Psi_{z^+,t}+\eta \cdot \hat{H}_t\right)$的斜率$\left(\kappa_w / \left( 1+\varepsilon_w \cdot \eta \right)\right)$\footnote{在弹性劳动力需求曲线和/或陡峭边际成本曲线的假设下,wpc的斜率比ppc斜率平缓,其原理类似于firm-specific capital导致ppc曲线斜率降低,参见\cite{Sveen:2005db, Altig:2011eu, DeWalque:2006wf}。}。

若要满足$1+\varepsilon_w \cdot \eta > 1$,需要
\begin{enumerate}
\item 劳动的需求弹性$\varepsilon_w$较大,和/或
\item 劳动者提供额外1单位劳动的边际成本$\eta$较高,即随着劳动供应的增加,边际劳动成本MRS式\eqref{eq:union-max-wage-PC-MRS}迅速上升。
\end{enumerate}

假定$j^{th}$劳工联盟出于某种目的,考虑提高名义工资。分三种情况来讨论。
\begin{enumerate}
\item 若在市场上对劳动力需求函数的斜率$\varepsilon_w$保持不变,则$j^{th}$种类劳动力工资提高,对$j^{th}$劳动力的需求降低,边际成本MRS随之降低;并且MRS的斜率越陡峭即$\eta$越大,边际成本下降的幅度越大。
\item 若边际成本MRS的斜率$\eta$保持不变,则对$j^{th}$劳动力需求曲线的斜率越平缓,$j^{th}$工资上升带来导致的对$j^{th}$劳动力需求的下降就会越剧烈。
\item 在MRS线斜率向上倾斜即$\eta$变大的情况下,$j^{th}$工资上调也会使得边际成本大幅度下降。
\end{enumerate}

因此,若$j^{th}$劳工联盟想要提高$j^{th}$工资,在弹性需求函数以及陡峭边际成本曲线的条件下,需要做好面临边际成本大幅度下降的准备。但另一方面,ceteris paribus,若当前边际成本较低,也会抑制住$j^{th}$劳工联盟提高工资的冲动。
