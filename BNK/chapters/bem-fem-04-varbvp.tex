%!TEX root = ../DSGEnotes.tex
\section{变分法求解边界值问题}
\label{sec:variational-bvp}

在介绍了常见的变分方法(第\ref{sec:variational-methods}节)之后,本届关注如何利用变分法分析求解二阶椭圆边界值问题,尤其是位势方程(potential equations, 第\ref{sec:bem-fem-potential-bvp}节)的边界值。应用变分法构建的边界值问题的若方程,是有限元分析的重要基础之一。

\subsection{位势方程基本介绍}
\label{sec:var-bvp-potential-equation}

如第\ref{sec:bem-fem-potential-bvp}节所讨论,先介绍三个算子,分别为实值标量的偏微分算子\eqref{eq:bvp-self-adjoint-pde-operator}
\begin{equation}
  \label{eq:var-bvp-self-adjoint-pde-operator}
  \left( L \, u \right)(x) \coloneqq - \sum_{i,j=1}^d \frac{\partial}{\partial x_j} \left[ a_{ji} (x) \frac{\partial}{\partial x_i} u(x)\right], \quad x \in \Omega \in \mathbb{R}^d,
\end{equation}

内界迹算子\index{trace \dotfill 迹}\eqref{eq:bvp-interior-boundary-trace}
\begin{equation*}
  \gamma_0^{int} u(x) \coloneqq \lim_{\Omega \owns \tilde{x} \mapsto x \in \Gamma} u \left( \tilde{x} \right), \quad x \in \Gamma = \partial \Omega,
\end{equation*}

与之相对应的内部共形导数\eqref{eq:bvp-int-conformal-derivative}
\begin{equation}
  \label{eq:var-bvp-int-conformal-derivative}
  \gamma_1^{int}u(x) \coloneqq \lim_{\Omega \owns \tilde{x} \mapsto x \in \Gamma} \left[
\sum_{i,j=1}^{d} n_j(x) a_{ji}\left( \tilde{x} \right) \frac{\partial}{\partial \tilde{x}_i} u \left( \tilde{x} \right)
  \right], \quad x \in \Gamma = \partial \Omega.
\end{equation}

当$u, \nu \in H^1(\Omega)$时,$Lu \in \widetilde{H}^{-1}(\Omega)$,格林第一恒等式\index{Green identities!first 格林第一恒等式} \eqref{eq:bvp-a-u-nu-inner-prod}化简为
\begin{equation}
  \label{eq:var-bvp-green-1st-identity}
\begin{split}
    a\left(u,\nu \right) &=\int_{\Omega} \left( L \, u \right)(x) \, \nu(x) \, dx + \int_{\Gamma} \left[ \gamma_1^{int} u(x) \right]  \left[ \gamma_0^{int} \nu(x) \right] \, d s_x \\
    &= \langle Lu, \nu \rangle_{\Omega} + \langle \gamma_1^{\text{int}} u, \gamma_0^{\text{int}} \nu \rangle_{\Gamma},
\end{split}
\end{equation}
其中对称双线性泛函$a(.,.)$的定义
\begin{equation}
  \label{eq:var-bvp-bilinear-form-a-def}
  a \left(u,\nu \right) \coloneqq \langle u, \nu \rangle = \sum_{i,j=1}^d \int_{\Omega} a_{ji}(x) \frac{\partial}{\partial x_i} u(x) \, \frac{\partial}{\partial x_j} \nu(x) \, dx
\end{equation}

\begin{lemma}
  \label{lemma:var-bvp-aform-inequality}
  假设$a_{ij} \in L^{\infty}(\Omega), \, i,j = 1,\ldots,d$满足
  \begin{equation}
    \label{eq:var-bvp-coeff-norm}
    \big\| a \big\|_{L^{\infty}(\Omega)} \coloneqq \max_{i,j=1,\ldots,d} \, \sup_{x \in \Omega} \big| a_{ij}(x) \big|.
  \end{equation}

  那么我们可得线性形泛函$a(.,.): H^{1}(\Omega) \times H^{1}(\Omega) \mapsto \mathbb{R}$有界,并且满足
  \begin{equation}
    \label{eq:var-bvp-a-c2a-inequality}
    \begin{split}
    &\big| a (u, \nu) \big| \le c_2^A \, \big| u \big|_{H^{1}(\Omega)} \, \big| \nu \big|_{H^1(\Omega)}, \quad \forall u, \nu \in H^{1}(\Omega), \\
    & c_2^A \coloneqq d \, \big\| a \big\|_{L^{\infty}(\Omega)}.
  \end{split}
\end{equation}
\end{lemma}

\begin{proof}
  由\eqref{eq:var-bvp-bilinear-form-a-def},\eqref{eq:var-bvp-coeff-norm}可得
  \begin{equation*}
    \begin{split}
      \big| a \left(u,\nu \right) \big| &= \Big| \sum_{i,j=1}^d \int_{\Omega} a_{ji}(x) \frac{\partial}{\partial x_i} u(x) \, \frac{\partial}{\partial x_j} \nu(x) \, dx \Big| \\
      & \le \big\| a \big\|_{L^{\infty}(\Omega)} \,
      \int_{\Omega} \left\{
      \sum_{i=1}^{d} \Big| \frac{\partial}{\partial x_i} \, u(x) \Big| \,
      \sum_{j=1}^{d} \Big| \frac{\partial}{\partial x_j} \, \nu(x) \Big| \right\} \,
      dx,
      \end{split}
  \end{equation*}
连续两次使用柯西——施瓦茨不等式(Definition \ref{definition:cauchy-schwarz-inequality}),上式变为
\begin{equation*}
  \begin{split}
    \big| a \left(u,\nu \right) \big|
    & \le \big\| a \big\|_{L^{\infty}(\Omega)} \,
    \left\{
    \int_{\Omega}
    \left[
    \sum_{i=1}^{d} \Big| \frac{\partial}{\partial x_i} \, u(x) \Big|
    \right]^2  dx
    \right\}^{\frac{1}{2}} \,
    \left\{
    \int_{\Omega}
    \left[
    \sum_{j=1}^{d} \Big| \frac{\partial}{\partial x_j} \, \nu(x) \Big|
    \right]^2  dx
    \right\}^{\frac{1}{2}} \\
    & \le \big\| a \big\|_{L^{\infty}(\Omega)} \,
    \left\{
    \int_{\Omega} d
    \sum_{i=1}^{d}
    \Big|
    \frac{\partial}{\partial x_i} \, u(x) \Big|^2
    dx
    \right\}^{\frac{1}{2}} \,
    \left\{
    \int_{\Omega} d
    \sum_{j=1}^{d}
    \Big|
    \frac{\partial}{\partial x_j} \, \nu(x) \Big|^2
    dx
    \right\}^{\frac{1}{2}} \, \\
    & = \underbrace{d \, \big\| a \big\|_{L^{\infty}(\Omega)}}_{\eqqcolon c_2^A} \,
    \big\| \triangledown u \big\|_{L^{2}(\Omega)} \,
    \big\| \triangledown \nu \big\|_{L^{2}(\Omega)}. \, %\\
    % & = c_2^A
    % \big| u \big|_{H^{1}(\Omega)} \, \big| \nu \big|_{H^1(\Omega)},
  \end{split}
\end{equation*}
\end{proof}

可见,由Lemma \ref{lemma:var-bvp-aform-inequality}可得,\eqref{eq:var-bvp-a-c2a-inequality}又进一步表示为
\begin{equation}
  \label{eq:var-bvp-aform-inequality-norm}
  \big| a \left(u,\nu \right) \, \big| \le c_2^A \, \big\| u \big\|_{H^{1}(\Omega)} \, \big\| \nu \big\|_{H^{1}(\Omega)}, \quad \forall u,\nu \in H^{1}(\Omega).
\end{equation}

\begin{lemma}[双线性算子的半椭圆特性]
  \label{lemma:var-bvp-operator-ellipticity-property}
  设$L$是一个如\eqref{eq:var-bvp-self-adjoint-pde-operator}所定义的一致椭圆偏微分算子。用双线性形式$(a.,.)$ \eqref{eq:var-bvp-bilinear-form-a-def}表示,我们有
  \begin{equation}
    \label{eq:var-bvp-bilinear-a-nunu}
    a(\nu,\nu) \ge \lambda_0 \, \big| \nu \big|_{H^{1}(\Omega)}^2, \quad \forall \nu \in H^{1}(\Omega),
  \end{equation}
  其中常数$\lambda_0 > 0$,见椭圆算子\eqref{eq:bvp-def-uniformly-elliptic}。
\end{lemma}

\begin{proof}
  设一个$w_i(x)$
  \begin{equation*}
    w_i(x) \coloneqq \frac{\partial}{\partial x_i} \nu(x), \quad i=1,\ldots,d,
  \end{equation*}
  \begin{equation*}
    \begin{split}
      \hookrightarrow a(\nu, \nu) & = \sum_{i,j=1}^{d} \int_{\Omega}
      a_{ji}(x) \frac{\partial}{\partial x_i} \nu(x) \frac{\partial}{\partial x_i} \nu(x)  \, dx \\
      &= \int_{\Omega} \left[ A \underline{w}(x), \underline{w}(x) \right] \, dx \\
      & \ge \lambda_0 \int_{\Omega} \left[ \underline{w}(x), \underline{w}(x) \right] \, dx \\
      & = \lambda_0 \, \big\| \triangledown \nu \big\|_{L^{2}(\Omega)}^2
      = \lambda_0 \big\| \nu \big\|_{W^{1,2}(\Omega)}^2
      = \lambda_0 \big\| \nu \big\|_{H^1(\Omega)}^2.
    \end{split}
  \end{equation*}
\end{proof}

\subsection{狄利克雷边界值问题1}
\label{sec:var-bvp-dirichlet}

\subsubsection{将狄利克雷边界值问题改写为变分问题}

回顾一下第\ref{sec:bem-fem-potential-bvp}节的狄利克雷边界值问题\index{Dirichlet boundary value condition \dotfill 狄利克雷边界值条件}
\eqref{eq:bvp-extension-omega-cond}-\eqref{eq:bvp-extension-gamma-dirichlet}:基于给定的$f$和$g$,求解
\begin{equation}
  \label{eq:var-dbvp-problem}
  \begin{split}
    &(L u)(x) = f(x), \quad x \in \Omega, \\
    &\gamma_0^{\text{int}} u(x) = g(x), \quad x \in \Gamma,
  \end{split}
\end{equation}
对应弱形式下的流形空间
\begin{equation}
  \label{eq:var-dbvp-problem-manifold}
\begin{split}
  &V_g \coloneqq
  \left\{
  \nu \in H^{1}(\Omega): \gamma_0^{\text{int}} \nu(x) = g(x), \quad x \in \Gamma
  \right\},\\
  &V_0 = H_{0}^{1}(\Omega).
\end{split}
\end{equation}

根据第\ref{sec:variational-methods}介绍的知识,我们可以将狄利克雷边界值问题\eqref{eq:var-dbvp-problem}-\eqref{eq:var-dbvp-problem-manifold}改写为变分问题:
\begin{equation}
  \label{eq:var-dbvp-variational-problem}
  a(u,\nu) = \langle f, \nu \rangle_{\Omega}, \quad \forall \, \nu \in V_0,
\end{equation}
其中双线性泛函$a(u,\nu)$由格林第一恒等式\eqref{eq:var-bvp-green-1st-identity}定义。研究目标是寻找变分问题的解$u \in V_g$。由此狄利克雷边界问题表现为流形$V_g$中的一个附属条件,因此又称为基本边界条件(essential boundary conditions)。

\subsubsection{解的存在性与唯一性}
由上所述可以看出,狄利克雷边界值问题对应的变分问题,可以归入带限制条件的算子方程问题类型,可用相应的变分方法求解(第\eqref{sec:var-constraints}节)。借助Theorem \ref{theorem:var-constraint-solution-exist-uniq}和Corollary \ref{corollary:var-constraint-norm-equivalence},我们可以证明$u \in V_g$存在且唯一。

\begin{theorem}[狄利克雷边界值问题的弱形式解]
  \label{theorem:var-dvbp-uniq-exist-solution}
  给定$f \in H^{-1}(\Omega), g \in H^{\frac{1}{2}}(\Gamma)$,变分问题\eqref{eq:var-dbvp-problem}存在唯一的解$u \in H^{1}(\Omega)$,满足
  \begin{equation}
    \label{eq:var-dvbp-uniq-exist-solution}
    \big\| u \big\|_{H^{1}(\Omega)} \le \frac{1}{c_1^A} \, \big\| f \big\|_{H^{-1}(\Omega)} + \left( 1 + \frac{c_2^A}{c_1^A} \right) c_{\text{IT}} \, \big\| g \big\|_{H^{\frac{1}{2}}(\Gamma)}.
  \end{equation}
\end{theorem}

\begin{proof}
  与Theorem \ref{theorem:var-constraint-solution-exist-uniq}类似,假设解$u$由两部分相加而得,$u \coloneqq u_0 + u_g$,其中$u_g$是$\gamma_0^{\text{int}} u_g = g$的解,$u_0$是$\gamma_1^{\text{int}} (u_0 + u_g) = f$的解。显然,$u$的唯一存在,可由$u_g$和$u_0$的唯一存在所分别证得。
\begin{enumerate}
  \item 证明$u_g$的唯一存在性。根据给定条件$g \in H^{\frac{1}{2}}(\Gamma)$,应用逆迹定理Theorem \ref{theorem:sobolev-manifold-inverse-trace-theorem},可得存在唯一一个有界延拓$u_g \in H^{\Omega}$,满足
  \begin{equation*}
    \begin{split}
      & \gamma_{0}^{\text{int}} u_g = g, \\
      & \big\| u_g \big\|_{H^{1}(\Omega)} \le c_{\text{IT}} \, \big\| g \big\|_{H^{\frac{1}{2}}(\Gamma)}.
    \end{split}
  \end{equation*}

  \item 在求得唯一解$u_g$的基础上,继续求唯一解$u_0$。变分问题\eqref{eq:var-dbvp-variational-problem}变为,寻找解$u_0 \in V_0 = H_{0}^{1}(\Omega)$,使满足新的变分问题
  \begin{equation}
    \label{eq:var-dbvp-variational-problem-u0}
    a(u_0, \nu) = \langle f, \nu \rangle_{\Omega} - a (u_g, \nu), \quad \forall \, \nu \in V_0.
  \end{equation}
  \begin{enumerate}
    \item 由范数等价定理(Theorem \ref{theorem:equivalence-norm-theorem}-\ref{theorem:sobolev-equivalence-norm-theorem}, \eqref{eq:sobolev-equivalence-norm-w12omega})得,$H^{1}(\Omega)$中的等价范为
    \begin{equation*}
%      \begin{split}
        \big\| u_0 \big\|_{W^{1,2}(\Omega), \Gamma} = \left\{
      \left[
      \int_{\Gamma} \gamma_{0}^{\text{int}} u_0(x) \, ds_x
      \right]^2
      + \big\| \triangledown u_0 \big\|^2_{L^2(\Omega)}
       \right\}^{\frac{1}{2}},
%     \end{split}
    \end{equation*}
    即$a(.,.)$有界。

    \item 由Lemma \ref{lemma:var-bvp-operator-ellipticity-property} \eqref{eq:var-bvp-bilinear-a-nunu}可得
    \begin{equation}
      \label{eq:var-bvp-operator-a-ellipticity}
      \begin{split}
        a(u_0,u_0) &\ge \lambda_0 \, \big| u_0 \big\|_{H^{1}(\Omega)} \\
        &= \lambda_0 \, \big\| \triangledown u_0 \big\|_{W^{1,2}(\Omega), \Gamma}^2 \\
        &\ge c_1^A \big\| u_0 \big\|_{H^1(\Omega)}^2,
      \end{split}
    \end{equation}
    即$a(.,.)$是个$V_0$-椭圆算子。

    \item $a(.,.)$有界且$V_0$-椭圆,满足拉克斯一密格拉蒙定理(Theorem \ref{theorem:lax-milgram-lemma})所需的前提条件,根据该定理,变分问题\eqref{eq:var-dbvp-variational-problem-u0}有唯一解$u_0 \in V_0$,满足
    \begin{equation*}
      \begin{split}
        c_1^A \, \big\| u_0 \big\|_{H^1(\Omega)}^2 & \le a(u_0, u_0) \\
        & = \langle f, u_0 \rangle_{\Omega} - a(u_g, u_0),
      \end{split}
    \end{equation*}
    代入Lemma \ref{lemma:var-bilinear-form-to-A},上式进一步变为
    \begin{equation*}
      \begin{split}
        &c_1^A \, \big\| u_0 \big\|_{H^1(\Omega)}^2  \le
        \left(
        \big\| f \big\|_{H^{-1}(\Omega)} + c_2^A \, \big\| u_g \big\|_{H^{1}(\Omega)}
        \right) \, \big\| u_0 \big\|_{H^{1}(\Omega)},\\
        \hookrightarrow &
        \big\| u_0 \big\|_{H^1(\Omega)} \le \frac{1}{c_1^A} \, \big\|f \big\|_{H^{-1}(\Omega)} + \frac{c_2^A}{c_1^A} \, \big\|u_g \big\|_{H^{1}(\Omega)}.
      \end{split}
    \end{equation*}
  \end{enumerate}

  \item 在此基础上我们有
  \begin{equation}
    \begin{split}
      &\big\| u_0 \big\|_{H^1(\Omega)} + \big\| u_g \big\|_{H^1(\Omega)} \le
      \frac{1}{c_1^A} \, \big\|f \big\|_{H^{-1}(\Omega)} + \left( 1+  \frac{c_2^A}{c_1^A} \right) \, \big\|u_g \big\|_{H^{1}(\Omega)}, \\
      \hookrightarrow &\big\| u \big\|_{H^{1}(\Omega)} \le \frac{1}{c_1^A} \, \big\| f \big\|_{H^{-1}(\Omega)} + \left( 1 + \frac{c_2^A}{c_1^A} \right) c_{\text{IT}} \, \big\| g \big\|_{H^{\frac{1}{2}}(\Gamma)}.
    \end{split}
  \end{equation}
\end{enumerate}
\end{proof}

\subsubsection{共形导数}
变分问题\eqref{eq:var-dbvp-variational-problem}的唯一解$u \in V_g$,常常又称为狄利克雷边界值问题\eqref{eq:var-dbvp-problem}的弱形式解。在此基础上,对于$f \in \widetilde{H}^{-1}(\Omega)$,可通过构建变分问题进一步求解共形导数$\gamma_{1}^{\text{int}} u \in H^{-\frac{1}{2}(\Gamma)}$
\begin{equation}
  \label{eq:var-dbvp-conormal-problem}
  \langle \gamma_{1}^{\text{int}} u, z \rangle _{\Gamma} = a(u,\varepsilon_z) - \langle f, \varepsilon_z \rangle_{\Omega}, \quad \forall \, z \in H^{\frac{1}{2}}(\Gamma),
\end{equation}
其中$\varepsilon_z$是逆迹定理Theorem \ref{theorem:sobolev-manifold-inverse-trace-theorem}所定义的有界延拓算子。\eqref{eq:var-dbvp-conormal-problem}的唯一可解条件由Theorem \ref{theorem:var-solution-exist-uniq}给出,由此我们可以假定如下稳定性条件
\begin{equation}
  \label{eq:var-dbvp-conormal-stability-condition}
  \big\| w \big\|_{H^{-\frac{1}{2}}(\Gamma)} = \sup_{0 \neq z \in H^{\frac{1}{2}}(\Gamma)} \frac{
  \langle w, z \rangle_{\Gamma}
  }{
  \big\| z \big\|_{H^{\frac{1}{2}}(\Gamma)}
  }, \quad \forall \, w \in H^{-\frac{1}{2}(\Gamma)}.
\end{equation}

在此基础上我们有共形导数的解
\begin{lemma}[共形导数的解]
  \label{lemma:var-dbvp-conormal-solution}
  设给定$f \in \widetilde{H}^{-1}(\Omega), g \in H^{\frac{1}{2}}(\Gamma)$,我们有$u \in H^{1}(\Gamma)$是狄利克雷边界值问题\eqref{eq:var-dbvp-variational-problem}的唯一解。

  那么相应的共形导数$\gamma_{1}^{\text{int}} u \in H^{-\frac{1}{2}}(\Gamma)$满足
  \begin{equation}
    \label{eq:var-dbvp-conormal-solution}
    \big\| \gamma_1^{\text{int}} u \big\|_{H^{-\frac{1}{2}}(\Gamma)}
    \le c_{\text{IT}} \,
    \left\{
    \big\| f \big\|_{\widetilde{H}^{-1}(\Omega)} +
    c_2^A \, \big| u \big|_{H^{1}(\Omega)}
    \right\}.
  \end{equation}
\end{lemma}

\begin{proof}
  稳定性条件\eqref{eq:var-dbvp-conormal-stability-condition}$\Rightarrow$
  \begin{equation*}
    \big\| \gamma_1^{\text{int}} u \big\|_{H^{-\frac{1}{2}}(\Gamma)} =
    \sup_{0 \neq z \in H^{\frac{1}{2}}(\Gamma)} \frac{
    \langle \gamma_1^{\text{int}} u , z \rangle_{\Gamma}
    }{
    \big\| z \big\|_{H^{\frac{1}{2}}(\Gamma)}
    },
  \end{equation*}
变分问题\eqref{eq:var-dbvp-conormal-problem}、Lemma \eqref{lemma:var-bilinear-form-to-A}$\Rightarrow$
\begin{equation*}
\begin{split}
  \big\| \gamma_1^{\text{int}} u \big\|_{H^{-\frac{1}{2}}(\Gamma)} &=
  \sup_{0 \neq z \in H^{\frac{1}{2}}(\Gamma)} \frac{
  \big|
  a(u,\varepsilon_z) - \langle f, \varepsilon_z \rangle_{\Omega}
  \big|_{\Gamma}
  }{
  \big\| z \big\|_{H^{\frac{1}{2}}(\Gamma)}
  },\\
  &\le
  \left\{
  c_2^A \, \big| u \big|_{H^{1}(\Omega)} + \big\| f \big\|_{\widetilde{H}^{-1}(\Omega)}
  \right\} \,
  \underbrace{\sup_{0 \neq z \in H^{\frac{1}{2}}(\Gamma)}
  \frac{
  \big\| \varepsilon_z \big\|_{H^{1}(\Omega)}
  }{
  \big\| z \big\|_{H^{\frac{1}{2}}(\Gamma)}
  }}_{\eqqcolon \mathcal{A}}
\end{split}
\end{equation*}

由逆迹定理Theorem \ref{theorem:sobolev-manifold-inverse-trace-theorem}可得
\begin{equation*}
  \mathcal{A} \le c_{\text{IT}},
\end{equation*}

因此
\begin{equation*}
  \big\| \gamma_1^{\text{int}} u \big\|_{H^{-\frac{1}{2}}(\Gamma)}
  \le c_{\text{IT}} \,
    \left\{
    \big\| f \big\|_{\widetilde{H}^{-1}(\Omega)} +
    c_2^A \, \big| u \big|_{H^{1}(\Omega)}
    \right\}.
\end{equation*}
\end{proof}

\subsubsection{带有齐次偏微分方程的狄利克雷边界值问题弱解}
\label{sec:var-dbvp-f0-solution}
来考虑一类特殊的狄利克雷边界值问题,即含有齐次偏微分方程$f \equiv 0$的情况,其解$u$对于边界积分算子的分析具有重要意义。
\begin{corollary}[带有齐次偏微分方程的狄利克雷边界值问题弱解]
  \label{corollary:var-dbvp-f0-solution}
  设$u \in H^{1}(\Omega)$是以下带有齐次偏微分方程的狄利克雷边界值问题的弱形式解
  \begin{equation*}
    \begin{split}
      &\left( L u \right) (x) = 0, \quad x \in \Omega, \\
      &\gamma_0^{\text{int}} u(x) = g(x), \quad x \in \Gamma,
    \end{split}
  \end{equation*}
  其中$L$是一个一致椭圆的二阶偏微分算子。

  那么我们有
  \begin{equation}
    \label{eq:var-dbvp-f0-solution}
    a(u,u) \ge c \big\| \gamma_{1}^{\text{int}} u \big\|_{H^{-\frac{1}{2}}(\Gamma)}^2.
  \end{equation}
\end{corollary}
\begin{proof}
  \begin{enumerate}
    \item 当$f \equiv 0$时,共形导数算子和逆迹的关系,由Lemma \ref{lemma:var-dbvp-conormal-solution}的  \eqref{eq:var-dbvp-conormal-solution}变为
  \begin{equation*}
    \big\| \gamma_1^{\text{int}} u \big\|_{H^{-\frac{1}{2}}(\Gamma)}^2
    \le \left[ c_{\text{IT}} \,
    c_2^A \right]^2
    \, \big| u \big|_{H^{1}(\Omega)}^2.
  \end{equation*}

\item 由双线性算子的半椭圆特性Lemma \ref{lemma:var-bvp-operator-ellipticity-property} \eqref{eq:var-bvp-bilinear-a-nunu}可得
\begin{equation*}
  \lambda_0
  \, \big| u \big|_{H^{1}(\Omega)}^2 \le a(u,u).
\end{equation*}
\end{enumerate}
\end{proof}

\subsubsection{狄利克雷边界值问题的强解简述}
\label{sec:var-bvp-strong-solution}
当$\Omega$是利普希茨域时,基于给定的$f$和$g$,我们可以引入更严格的假设条件,构建强形式的正则解如$u$、$\gamma_{1}^{\text{int}}u$等。

\begin{theorem}
  \label{theorem:var-bvp-strong-solution}
  设$\Omega \in \mathbb{R}^d$是个有界的利普希茨域,边界为$\Gamma = \partial \Omega$。设$u \in H^{1}(\Omega)$是狄利克雷边界值问题的弱形式解
  \begin{equation*}
    \begin{split}
      &(L u)(x) = f(x), \quad x \in \Omega,\\
      &\gamma_{0}^{\text{int}} u(x) = g(x), \quad x \in \Gamma.
    \end{split}
  \end{equation*}

  如果给定的$f$和$g$满足$f \in L^{2}(\Omega), g \in H^{1}(\Gamma)$,那么我们有$u \in H^{\frac{3}{2}}(\Omega), \gamma_{1}^{\text{int}} u \in L^{2}(\Gamma)$,并且
  \begin{equation*}
  \begin{split}
  &\big\| u \big\|_{H^{\frac{3}{2}}(\Omega)} \le c_1
  \left\{
  \big\| f \big\|_{L^{2}(\Omega)} + H^{1}(\Gamma)
  \right\},\\
  & \big\| \gamma_{1}^{\text{int}} u \big\|_{L^{2}(\Gamma)} \le c_2 \left\{
  \big\| f \big\|_{L^{2}(\Omega)} + H^{1}(\Gamma)
  \right\}.
  \end{split}
  \end{equation*}
\end{theorem}

我们甚至可以使假定条件更加严格,如$\Gamma = \partial \Omega$是平滑或分段平滑的边界,$\Omega$是凹的,$f \in L^{2}(\Omega)$等。若$g = \gamma_{0}^{\text{int}} u $是方程解$u_g \in H^2(\Omega)$的迹,则我们有$u \in H^2(\Omega)$。更多强形式解的讨论,可见\cite{Demkowicz:2006ww,Demkowicz:2007ur}。

\subsection{狄利克雷边界值问题2}
\label{sec:var-bvp-dirichlet-lagrange}

如第\ref{sec:var-mixed-formulations}节所述,狄利克雷边界值问题\eqref{eq:var-dbvp-problem}也可以改写为鞍点变分问题,即混合算子方程,共形导数对应拉格朗日乘子\citep{Babuska:1973gu, Bramble:1981vv}。

从格林第一恒等式\eqref{eq:var-bvp-green-1st-identity}入手,设拉格朗日乘子
\begin{equation*}
  \lambda \coloneqq \gamma_1^{\text{int}} u \in H^{-\frac{1}{2}}(\Gamma),
\end{equation*}

进而鞍点变分问题表示为,寻找解$(u,\lambda) \in H^{1}(\Omega) \times H^{-\frac{1}{2}}(\Gamma)$,使得满足
\begin{equation}
  \label{eq:var-bvp-saddle-problem}
  \begin{split}
    a(u,\nu) - b(\nu,\lambda) &= \langle f,\nu \rangle_{\Omega} \quad \forall \nu \in H^{1}(\Omega),\\
    b(u,\mu) &= \langle g, \mu \rangle_{\Gamma} \quad \forall \, \mu \in H^{-\frac{1}{2}}(\Gamma),
  \end{split}
\end{equation}
其中定义了一个新的双线性泛函算子
\begin{equation*}
  b(\nu,\mu) \coloneqq \langle \gamma_{0}^{\text{int}} \nu, \mu \rangle_{\Gamma}, \quad (\nu,\mu) \in H^{1}(\Gamma) \times H^{-\frac{1}{2}}(\Gamma).
\end{equation*}

\subsubsection{解的唯一存在性}
\label{sec:var-bvp-saddle-solution-uniq}
鞍点变分形式的狄利克雷边界值问题\eqref{eq:var-bvp-saddle-problem},解的存在性和唯一性,可由Theorem \ref{theorem:mixed-saddle-point-variational-problem}证得。使用该定理之前,需要确保两个前提条件得到满足。一是双线性泛函$a(.,.)$的椭圆特性,二是解的稳定性条件。
\begin{enumerate}
\item 椭圆性。类似于式\eqref{eq:var-bvp-operator-a-ellipticity},由Lemma \ref{lemma:var-bvp-operator-ellipticity-property} \eqref{eq:var-bvp-bilinear-a-nunu}可得
\begin{equation*}
  a(.,.) \ge c_1^A \, \big\| \cdot \big\|_{H^{1}(\Omega)}^2,
\end{equation*}
此外由于
\begin{equation*}
  \ker B \coloneqq \left\{
  \nu \in H^{1}(\Omega) : \langle \gamma_{0}^{\text{int}} \nu, \mu \rangle_{\Gamma} = 0, \quad \forall \, \mu \in H^{-\frac{1}{2}}(\Gamma)
  \right\} = H_{0}^{1}(\Omega),
\end{equation*}
我们因此有,$a(.,.)$是一个$\ker B$-椭圆(或$H_{0}^{1}$-椭圆)的双线性形。(通常来说,我们需要求得一个扩展双线性形$\tilde{a}(.,.)$,使得满足$H^1(\Omega)$-椭圆性质,相关讨论见第\ref{sec:var-bvp-saddle-modified}节。)

\item 解的稳定性条件可以表示为
\begin{equation}
  \label{sec:var-bvp-saddle-solution-stability}
  c_S \big\| \mu \big\|_{H^{-\frac{1}{2}}(\Gamma)} \le
  \sup_{0 \neq \nu \in H^{1}(\Omega)} \frac{
  \langle \gamma_{0}^{\text{int}} \nu, \mu \rangle_{\Gamma}
  }{
  \big\| \nu \big\|_{H^{1}(\Omega)}
  }, \quad \forall \, \mu \in H^{-\frac{1}{2}}(\Gamma),
\end{equation}
其证明见可见Lemma \ref{lemma:var-bvp-saddle-solution-stability}。

\item 应用定理Theorem \ref{theorem:mixed-saddle-point-variational-problem},求得鞍点变分问题的唯一解。
\end{enumerate}

\begin{lemma}[鞍点变分形式狄利克雷边界值问题解的稳定条件]
  \label{lemma:var-bvp-saddle-solution-stability}
  稳定条件\eqref{sec:var-bvp-saddle-solution-stability}成立。
\end{lemma}
\begin{proof}
  已知给定的任一$\mu \in H^{-\frac{1}{2}}(\Gamma)$。由里兹表现定理(Theorem \ref{theorem:var-riesz-representation-theorem})可得,存在唯一的一个$u_{\mu} \in H^{\frac{1}{2}}(\Gamma)$,满足
  \begin{equation*}
    \begin{split}
      & \langle u_{\mu}, \nu \rangle_{H^{\frac{1}{2}}(\Gamma)} = \langle \mu, \nu \rangle_{\Gamma} \quad \forall \, \nu \in H^{\frac{1}{2}}(\Gamma), \\
      & \big\| u_{\mu} \big\|_{H^{\frac{1}{2}}(\Gamma)} = \big\| \mu \big\|_{H^{\frac{1}{2}}(\Gamma)}.
    \end{split}
  \end{equation*}

  由逆迹定理Theorem \ref{theorem:sobolev-manifold-inverse-trace-theorem}可得,存在一个延拓算子$\varepsilon u_{\mu} \in H^{1}(\Omega)$,满足
  \begin{equation*}
    \big\| \varepsilon u_{\mu} \big\|_{H^{1}(\Omega)} \le c_{\text{IT}} \, \big\| u_{\mu} \big\|_{H^{\frac{1}{2}}(\Gamma)}.
  \end{equation*}

  那么,对于$\nu = \varepsilon u_{\mu} \in H^{1}(\Omega)$我们有
  \begin{equation*}
    \begin{split}
      & \frac{
      \langle \nu, \mu \rangle_{\Gamma}
      }{
      \big\| \nu \big\|_{H^{1}(\Omega)}
      }
      =
      \frac{
      \langle u_{\mu}, \mu \rangle_{\Gamma}
      }{
      \big\| \varepsilon u_{\mu} \big\|_{H^{1}(\Omega)}
      }
      =
      \frac{
      \langle u_{\mu}, u_{\mu} \rangle_{H^{\frac{1}{2}}(\Gamma)}
      }{\big\| \varepsilon u_{\mu} \big\|_{H^{1}(\Omega)}}\\
      & \ge \frac{1}{c_{\text{IT}}} \,
      \big\| u_{\mu} \big\|_{H^{\frac{1}{2}}(\Gamma)}
      = \frac{1}{c_{\text{IT}}} \,
      \big\| u \big\|_{H^{- \frac{1}{2}}(\Gamma)}
    \end{split}
  \end{equation*}

  $\therefore$稳定性条件\eqref{sec:var-bvp-saddle-solution-stability}成立。
\end{proof}

\subsubsection{调整鞍点变分问题}
\label{sec:var-bvp-saddle-modified}
需要注意的是,在鞍点变分问题\eqref{eq:var-dbvp-problem}中的双线性形式算子$a(.,.)$是$H_{0}^{1}(\Omega)$-椭圆的。我们常常需要将它扩展为一个$H^1(\Omega)$-椭圆的算子$\tilde{a}(.,.)$,对应新的调整鞍点变分问题。调整思路如下:

已知,用拉格朗日乘子$\lambda \coloneqq \gamma_{1}^{\text{int}}u \in H^{-\frac{1}{2}}(\Gamma)$来描述问题解$u$的共形导数,那么利用格林第二恒等式\eqref{eq:bvp-a-nu-u-green-2nd-identity},可得狄利克雷边界值问题的正交条件\eqref{eq:bvp-neumann-green-2}
\begin{equation}
\begin{split}
  \label{eq:var-dbvp-modified-orthogonality-condition}
  &\int_{\Omega} f(x) \, dx + \int_{\Gamma} \lambda(x) d s_x = 0, \\
  \hookrightarrow &\int_{\Gamma} \lambda(x) d s_x \, \int_{\Gamma} \mu(x) \, d s_x = - \int_{\Omega} f(x) \, dx \, \int_{\Gamma} \mu(x) \, d s_x , \quad \forall \, \mu \in H^{-\frac{1}{2}}(\Gamma).
\end{split}
\end{equation}

另一方面,根据狄利克雷边界条件有
\begin{equation}
  \label{eq:var-dbvp-modified-dirichlet-condition}
  \begin{split}
    &\gamma_{0}^{\text{int}} u = g, \\
    \hookrightarrow & \int_{\Gamma} \gamma_{0}^{\text{int}} u(x) \, d s_x
    = \int_{\Gamma} g(x) \, d s_x, \\
    \hookrightarrow & \int_{\Gamma} \gamma_{0}^{\text{int}} u(x) \, d s_x \,
    \int_{\Gamma} \gamma_{0}^{\text{int}} \nu(x) \, d s_x
    = \int_{\Gamma} g(x) \, d s_x \,
    \int_{\Gamma} \gamma_{0}^{\text{int}} \nu(x) \, d s_x, \quad \forall \, \nu \in H^{1}(\Omega).
  \end{split}
\end{equation}

  将\eqref{eq:var-dbvp-modified-orthogonality-condition}、\eqref{eq:var-dbvp-modified-dirichlet-condition}代入\eqref{eq:var-bvp-saddle-problem},得到调整鞍点变分问题:寻找解$(u,\lambda) \in H^{1}(\Omega) \times H^{-\frac{1}{2}}(\Gamma)$,使得$\forall \, (\nu, \mu) \in H^{1}(\Omega) \times H^{-\frac{1}{2}}(\Gamma)$均满足
  \begin{equation}
    \label{eq:var-bvp-saddle-modified-problem}
    \begin{split}
      \underbrace{
      \int_{\Gamma} \gamma_{0}^{\text{int}} u(x) \, d s_x \,
    \int_{\Gamma} \gamma_{0}^{\text{int}} \nu(x) \, d s_x + a(u,\nu)
    }_{\eqqcolon \tilde{a}(u,\nu)}
    - b(\nu,\lambda) &= \langle f,\nu \rangle_{\Omega}  + \int_{\Gamma} g(x) \, d s_x \,
    \int_{\Gamma} \gamma_{0}^{\text{int}} \nu(x) \, d s_x,\\
      b(u,\mu) + \int_{\Gamma} \lambda(x) d s_x \, \int_{\Gamma} \mu(x) \, d s_x  &= \langle g, \mu \rangle_{\Gamma} - \int_{\Omega} f(x) \, dx \, \int_{\Gamma} \mu(x) \, d s_x .
    \end{split}
  \end{equation}

下面的问题就是,调整鞍点变分问题\eqref{eq:var-bvp-saddle-modified-problem}的解是否存在,是否唯一,以及是否与原鞍点变分问题\eqref{eq:var-bvp-saddle-problem}的解一致。换句话说,两个鞍点变分问题是否等价。
\begin{theorem}[变分问题等价]
  \label{theorem:var-dbvp-saddle-equivalance}
  调整鞍点变分问题\eqref{eq:var-bvp-saddle-modified-problem}有唯一的解$(u,\lambda) \in H^{1}(\Omega) \times H^{-\frac{1}{2}}(\Gamma)$,并且与鞍点变分问题\eqref{eq:var-bvp-saddle-problem}的解一致。即,两个变分问题等价。
\end{theorem}
\begin{proof}
  \begin{enumerate}
  \item 证明扩展双线性形$\tilde{a}(u,\nu)$有界。

  $a(u,\nu)$有界$\Rightarrow$
  \begin{equation*}
    \tilde{a}(u,\nu) \coloneqq \int_{\gamma} \gamma_{0}^{\text{int}} u(x) d s_x \, \int_{\gamma} \gamma_{0}^{\text{int}} \nu(x) d s_x +  a(u,\nu), \quad \forall \, u,\nu \in H^{1}(\Omega)
  \end{equation*}
  有界。

  \item 由$a(.,.)$的半椭圆性质(Lemma \ref{lemma:var-bvp-operator-ellipticity-property})和\eqref{eq:sobolev-equivalence-norm-w12omega}有
  \begin{equation*}
\begin{split}
  \tilde{a}(\nu,\nu) &=
  \left[
  \int_{\Gamma} \gamma_{0}^{\text{int}} \nu(x) \, d s_x
  \right]^2 + a(\nu,\nu) \\
  & \ge \min\{1,\lambda_0\} \, \big\| \nu \big\|_{W^{1,2}(\Omega), \Gamma}^2 \\
  & \ge c_1^{\tilde{A}} \, \big\| \nu \big\|_{H^1(\Omega)}^2, \quad \forall \, \nu \in H^{1}(\Omega),
\end{split}
  \end{equation*}
  由此可得$\tilde{a}(u,\nu)$是$H^{1}(\Omega)$-椭圆。

  \item 前提条件得到满足,可通过Theorem \ref{theorem:mixed-saddle-point-variational-problem}、Theorem \ref{theorem:mixed-saddle-point-variational-problem-solution}证得,调整鞍点变分问题\eqref{eq:var-bvp-saddle-modified-problem}有唯一解。

  \item 对于$(\nu,\mu) \equiv (1,1)$的特殊情况,\eqref{eq:var-bvp-saddle-modified-problem}变为
  \begin{equation*}
    \begin{split}
      \big| \Gamma \big| \, \int_{\Gamma} \gamma_{0}^{\text{int}} u(x) \, d s_x - \int_{\Gamma} \lambda(x) \, d s_x &= \int_{\Omega} f(x) \, d x + \big| \Gamma \big| \, \int_{\Gamma} g(x) \, d s_x,\\
      \int_{\Gamma} \gamma_{0}^{\text{int}} u(x) \, d s_x
      + \big| \Gamma \big| \, \int_{\Gamma} \lambda(x) \, d s_x &=
      \int_{\Gamma} g(x) \, d s_x, - \big| \Gamma \big| \, \int_{\Omega} f(x) \, d x.
    \end{split}
  \end{equation*}
进而
  \begin{equation*}
    \left( 1 + \big| \Gamma \big|^2 \right) \, \int_{\Gamma} \gamma_{0}^{\text{int}} u(x) d s_x = \left( 1 + \big| \Gamma \big|^2 \right) \, \int_{\Gamma} g(x) \, d s_x
  \end{equation*}
即\eqref{eq:var-dbvp-modified-dirichlet-condition}。从而有

\begin{equation*}
  \big| \Gamma \big\| \int_{\Gamma} \lambda(x) \, d s_x = - \big| \Gamma \big\| \int_{\Gamma} f(x) \, d s_x ,
\end{equation*}
即\eqref{eq:var-dbvp-modified-orthogonality-condition}。

因此可见,$(u,\lambda)$也是鞍点变分问题\eqref{eq:var-bvp-saddle-problem}的解的解;两个问题等价。
\end{enumerate}
\end{proof}

\subsection{诺依曼边界值问题}
\label{sec:var-nbvp-problem}


\subsubsection{将诺依曼边界值问题改写为变分问题}
回顾一下第\ref{sec:bem-fem-potential-bvp}节的诺依曼边界值问题\index{Neumann boundary value condition \dotfill 诺依曼边界值条件}\eqref{eq:bvp-extension-omega-cond}-\eqref{eq:bvp-extension-gamma-neumann}:基于给定的$f$和$g$,求解
\begin{equation}
  \label{eq:var-nbvp-problem}
  \begin{split}
    & \left( L u\right)(x) = f(x), \quad x\in \Omega,\\
    & \gamma_{1}^{\text{int}} u(x) = g(x), \quad x \in \Gamma.
  \end{split}
\end{equation}

假定$f$和$g$满足可求解性条件 \eqref{eq:bvp-neumann-green-2-new}
\begin{equation}
  \label{eq:var-nbvp-solvability-cond}
  \int_{\Omega} f(x) \, dx + \int_{\Gamma} g(x) \, d s_x = 0.
\end{equation}

基于前文的分析可见,诺依曼边界值问题\eqref{eq:var-nbvp-problem}的解$u \in H^{1}(\Omega)$将与某一个常数有关。为了将此常数予以确定,可以在$H^{1}(\Omega)$中定义一个测试空间$H_{*}^{1}(\Omega)$,测试方程$\nu(x)$用于规模调节。
\begin{equation}
  \label{eq:var-nbvp-trial-space}
  H_{*}^{1} (\Omega) \coloneqq \left\{ \nu \in H^{1}(\Omega): \int_{\Omega} \nu(x) \, dx = 0 \right\}.
\end{equation}

从而构建变分问题,求解$u \in H_{*}^{1}(\Omega)$使得满足
\begin{equation}
  \label{eq:var-nbvp-variational-problem}
  a(u,\nu) = \langle f, \nu \rangle_{\Omega} + \langle g, \gamma_{0}^{\text{int}} \nu \rangle_{\Gamma}, \quad \forall \, \nu \in H_{*}^{1}(\Omega).
\end{equation}

\subsubsection{解的存在性与唯一性}
\begin{theorem}[诺依曼边界值问题的变分法求解]
  给定$f \in \tilde{H}^{-1}(\Omega), g \in H^{-\frac{1}{2}}(\Gamma)$,满足可求解性条件\eqref{eq:var-nbvp-solvability-cond}。

  则诺依曼边界值的变分问题\eqref{eq:var-nbvp-variational-problem}存在唯一的解$u \in H_{*}^{1}(\Omega)$,满足
  \begin{equation*}
    \big\| u \big\|_{H^{1}(\Omega)} \le \frac{1}{\tilde{c}_1^A}
    \left\{
    \big\| f \big\|_{\widetilde{H}^{-1}(\Omega)} +
    c_{T} \, \big\| g \big\|_{H^{-\frac{1}{2}}(\Gamma)}.
    \right\}
  \end{equation*}
\end{theorem}

\begin{proof}
\begin{enumerate}
\item 证$a(.,.)$有界且椭圆。

对于$\nu \in H_{*}^{1}(\Omega)$,它在$H^{1}(\Omega)$中的等价范,可由\eqref{eq:sobolev-equivalence-norm-w12omega}求得
\begin{equation*}
  \big\| \nu \big\|_{W^{1,2}(\Omega), \Omega} = \left\{
  \left[
  \int_{\Omega} \nu(x) \, dx
  \right]^2
  + \big\| \triangledown \nu \big\|^2_{L^2(\Omega)}
   \right\}^{\frac{1}{2}},
\end{equation*}
进而利用Lemma \ref{lemma:var-bvp-operator-ellipticity-property}可得,
\begin{equation}
  \label{eq:var-nbvp-ellipticity}
  a(\nu,\nu) \ge \lambda_0 \, \big\| \triangledown \nu \big\|_{W^{1,2}(\Omega), \Omega}^2 \ge \tilde{c}_{1}^{A} \, \big\| \nu \big\|_{H^{1}(\Omega)}^2, \quad \forall \nu \in H_{*}^{1}(\Omega),
\end{equation}
因此可得双线性形式$a(.,.)$的$H_{*}^{1}$-椭圆特性。

\item 根据拉克斯一密格拉蒙定理(Theorem \ref{theorem:lax-milgram-lemma}),
证明变分问题\eqref{eq:var-nbvp-variational-problem}存在唯一解。

\item 将求得的变分问题唯一解$u \in H_{*}^{1}(\Omega)$代回椭圆条件
\eqref{eq:var-nbvp-ellipticity}
\begin{equation*}
  \begin{split}
    \widetilde{c}_{1}^{A} \, \big\| u \big\|_{H^1(\Omega)}^2 & \le a(u,u) = \langle f, u \rangle_{\Omega} + \langle g, \gamma_{0}^{\text{int}} u \rangle_{\Gamma} \\
    & \le \big\| f \big\|_{\widetilde{H}^{-1}(\Omega)} \,
    \big\| u \big\|_{H^{1}(\Omega)}
    + \big\| g \big\|_{H^{- \frac{1}{2}}(\Gamma)} \,
    \big\| \gamma_{0}^{\text{int}} u \big\|_{H^{\frac{1}{2}}(\Gamma)}.
  \end{split}
\end{equation*}

由迹定理 Theorem \ref{theorem:sobolev-manifold-trace-theorem}得,上式变为
\begin{equation*}
  \begin{split}
    \widetilde{c}_{1}^{A} \, \big\| u \big\|_{H^1(\Omega)}
    \le \big\| f \big\|_{\widetilde{H}^{-1}(\Omega)} + \big\| g \big\|_{H^{-\frac{1}{2}}(\Gamma)} \,
    c_{T} \big\| u \big\|_{H^{1}(\Gamma)}.
  \end{split}
\end{equation*}
\end{enumerate}
\end{proof}




\subsubsection{鞍点变分问题}
类似地,我们也可以构建一个与\eqref{eq:var-nbvp-variational-problem}等价的鞍点变分问题。此时,用于规模调节的测试空间$H_{*}^{1}(\Omega)$以副条件(side condition)的情况出现。使用一个拉格朗日乘子,寻找解$(u,\lambda) \in H^{1}(\Omega) \times \mathbb{R}$,满足
\begin{equation}
  \label{eq:var-nbvp-saddle-var}
  \begin{split}
    & a(u,\nu) + \lambda \int_{\Omega} \nu(x) \, d x =
    \langle f, \nu \rangle_{\Omega} +
    \langle g, \gamma_{0}^{\text{int}} \nu \rangle_{\Gamma}, \\
    & \int_{\Omega} u(x) \, dx = 0, \quad \forall \, \nu \in H^{1}(\Omega).
  \end{split}
\end{equation}

\begin{theorem}[诺依曼边界值问题的鞍点变分法求解]
  \label{theorem:var-nbvp-saddle-var-solution}
  诺依曼边界值的鞍点变分问题\eqref{eq:var-nbvp-saddle-var}有唯一解$(u,\lambda) \times H^{1}{\Omega} \times \mathbb{R}$。
\end{theorem}
\begin{proof}
  \begin{enumerate}
    \item 双线性形式$b(.,.)$有界
    \begin{equation*}
      b(\nu,\mu) \coloneqq \mu \, \int_{\Omega} \nu(x) \, dx, \quad \forall \, \nu \in H^{1}(\Omega), \mu \in \mathbb{R},
    \end{equation*}
并且有$\ker B = H_{*}^{1}(\Omega)$。

    \item 进而由椭圆性\eqref{eq:var-nbvp-ellipticity}得,双线性形$a(.,.)$是$\ker B$-椭圆的。

\item 证明满足稳定性条件
\begin{equation}
  \label{eq:var-nbvp-saddle-var-stability}
  c_{S} \, \big| \mu \big| \le \sup_{0 \neq \nu \in H^{1}(\Omega)}
  \frac{
  b(\nu,\mu)
  }{
  \big\| \nu \big\|_{H^{1}(\Omega)}
  }, \quad \forall \, \mu \in \mathbb{R}.
\end{equation}

对于任一给定的$\mu \in \mathbb{R}$, 定义$\nu^{*} \coloneqq \nu \in H^{1}(\Omega)$,可以证得\eqref{eq:var-nbvp-saddle-var-stability},其中
$ c_S = | \Omega |^{-\frac{1}{2}}$。

\item 根据定理Theorem \ref{theorem:mixed-saddle-point-variational-problem}可求得,诺依曼边界值的鞍点变分问题存在唯一解$(u,\lambda) \in H^{1}(\Omega) \times \mathbb{R}$。

\item 对于测试方程$\nu \equiv 1$的特殊情况,根据诺依曼边界值问题的可求解条件\eqref{eq:var-nbvp-solvability-cond}我们有拉格朗日乘子的值
\begin{equation*}
  \lambda = 0.
\end{equation*}
  \end{enumerate}
\end{proof}



\subsubsection{调整鞍点变分问题}
如前所述,鞍点变分问题\eqref{eq:var-nbvp-saddle-var}也可以改写为一个新的鞍点变分问题:基于给定的任一$f \in \widetilde{H}^{-1}(\Omega), g \in H^{-\frac{1}{2}}(\Gamma)$,寻找$(u,\lambda) \in H^{1}(\Omega) \times \mathbb{R}$,使满足
\begin{equation}
\begin{split}
  \label{eq:var-nbvp-saddle-modified-var}
  &a(u,\nu) + \lambda \int_{\Omega} \nu(x) \, dx = \langle f, \nu \rangle_{\Omega} + \langle g, \gamma_{0}^{\text{int}} \nu \rangle_{\Gamma}, \\
  & \int_{\Gamma} u(x) \, dx - \lambda = 0, \quad \forall \, \nu \in H^{1}(\Omega).
\end{split}
\end{equation}

利用第二行等式求得拉格朗日乘子$\lambda \in \mathbb{R}$代入第一行,我们得到一个新的调整变分问题,寻找$u \in H^{1}(\Omega)$使满足
\begin{equation}
  \label{eq:var-nbvp-saddle-modified}
  a(u,\nu) + \int_{\Omega}  u(x)  dx \, \int_{\Omega} \nu(x) dx =
  \langle f,\nu \rangle_{\Omega} +
  \langle g, \gamma_{0}^{\text{int}} \nu \rangle_{\Gamma}, \quad \forall \, \nu \in H^{1}(\Omega).
\end{equation}

\begin{theorem}[诺依曼边界值的调整变分问题解]
  \label{theorem:var-nbvp-saddle-modified-solution}
  基于给定的任一$f \in \widetilde{H}^{-1}(\Omega), g \in H^{-\frac{1}{2}}(\Gamma)$,调整变分问题\eqref{eq:var-nbvp-saddle-modified}有唯一的解$u \in H^{1}(\Omega)$。

  如果给定的$f \in \widetilde{H}^{-1}(\Omega), g \in H^{-\frac{1}{2}}(\Gamma)$满足可求解条件\eqref{eq:var-nbvp-solvability-cond},那么调整变分问题\eqref{eq:var-nbvp-saddle-modified}的解$u \in H_{*}^{1}(\Omega)$;换句话说,调整变分问题\eqref{eq:var-nbvp-saddle-modified}和变分问题\eqref{eq:var-nbvp-saddle-var}等价。
\end{theorem}

\begin{proof}
  \begin{enumerate}
  \item 由\eqref{eq:var-nbvp-saddle-modified}可见,调整双线性形$\widetilde{a}(.,.)$写为
  \begin{equation*}
    \widetilde{a}(u,\nu) \coloneqq a(u,\nu) + \int_{\Omega} u(x) \, dx \, \int_{\Omega} \nu(x) \, dx.
  \end{equation*}

  由$a(.,.)$的半椭圆属性Lemma \ref{lemma:var-bvp-operator-ellipticity-property}我们有
  \begin{equation}
    \label{eq:var-nbvp-modified-ellipticity}
  \begin{split}
      \widetilde{a}(\nu,\nu) &\ge \lambda_0 \, \big\| \triangledown \nu \big\|_{L^{2}(\Omega)}^2 + \left[ \int_{\Omega} \nu(x) dx \right]^2\\
      &\ge \min\{\lambda_0, 1\} \, \big\| \nu \big\|_{W^{1,2}(\Omega), \Omega}^2 \\
      & \ge \hat{c}_1^A \, \big\| \nu \big\|_{W^{1,2}(\Omega), \Omega}^2, \quad \forall \, \nu \in H^{1}(\Omega),
  \end{split}
  \end{equation}
  可得$\widetilde{a}(.,.)$是$H^{1}(\Omega)$-椭圆且有界的。

  \item 满足前提条件后,可由拉克斯一密格拉蒙定理(Theorem \ref{theorem:lax-milgram-lemma})证得,调整变分问题\eqref{eq:var-nbvp-saddle-modified}有唯一解$u \in H^{1}(\Omega)$。

  \item 设测试方程$\nu(x) \equiv 1$。调整变分问题\eqref{eq:var-nbvp-saddle-modified}变为
  \begin{equation*}
    \begin{split}
      \big| \Omega \big| \, \int_{\Omega} u(x) \, dx &=
      \langle f, 1 \rangle_{\Omega}
      + \langle g, 1 \rangle_{\Gamma}, \\
      & = \int_{\Omega} f(x) \, d x + \int_{\Gamma} g(x) \, d s_x = 0,
    \end{split}
  \end{equation*}
  其中最后一个等式用到可求解性条件\eqref{eq:var-nbvp-solvability-cond}。由此我们有$u \in H_{*}^{1}(\Omega)$,也是变分问题\eqref{eq:var-nbvp-saddle-var}的解;换句话说,两个问题等价。
\end{enumerate}
\end{proof}

\subsubsection{诺依曼边界值问题的通解}
利用变分法求解诺依曼边界值问题\eqref{eq:var-nbvp-problem},所得到的$u \in H_{*}^{1}(\Omega)$是弱形式解。更一般意义上的通解$\widetilde{u} \in H^{1}(\Omega)$可写为
\begin{equation*}
  \widetilde{u} \coloneqq u + \alpha,
\end{equation*}
其中$\alpha \in \mathbb{R}$是任意常数。


\subsection{混合边界值问题}
\label{sec:var-mbvp-problem}

回顾一下第\ref{sec:bem-fem-potential-bvp}节的混合边界值问题\eqref{eq:bvp-extension-omega-cond},\eqref{eq:bvp-extension-gamma-dirichlet}-\eqref{eq:bvp-extension-gamma-neumann}
\begin{equation}
  \label{eq:var-mbvp-problem}
  \begin{split}
    &(L u)(x) = f(x), \quad x \in \Omega, \\
    &\gamma_{0}^{\text{int}} u(x) = g_{D} (x), \quad x \in \Gamma_D,\\
    &\gamma_{1}^{\text{int}} u(x) = g_{N} (x), \quad x \in \Gamma_N,
  \end{split}
\end{equation}

假定$\Gamma = \overline{\Gamma}_{D} \cup \overline{\Gamma}_{N}$ 。对应地,可由格林第一恒等式\eqref{eq:bvp-a-u-nu-inner-prod}建立变分问题:求解$u \in H^{1}(\Omega), \gamma_{0}^{\text{int}} u(x) = g_{D}(x), x \in \Gamma_D$,使满足
\begin{equation}
  \label{eq:var-mbvp-var-problem}
  a(u,\nu) = \langle f, \nu \rangle_{\Omega} +
  \langle g_N, \gamma_{0}^{\text{int}} \nu \rangle_{\Gamma_N},
  \quad \forall \, \nu \in H_{0}^{1}(\Omega, \Gamma_D),
\end{equation}
其中
\begin{equation*}
  H_{0}^{1} (\Omega, \Gamma_D) \coloneqq
  \left\{
  \nu \in H^{1}(\Omega): \gamma_{0}^{\text{int}} \nu(x) =0, \quad x \in \Gamma_D
  \right\}.
\end{equation*}

混合边界值变分问题\eqref{eq:var-mbvp-var-problem}的唯一解,可由以下定理证明
\begin{theorem}[混合边界值变分问题的唯一解]
  \label{theorem:var-mbvp-var-problem-solution}
  给定$f \in \widetilde{H}^{-1}(\Omega), g_{D} \in H^{\frac{1}{2}}(\Gamma_{D}), g_{N} \in H^{-\frac{1}{2}}(\Gamma_{N})$,那么混合边界值变分问题\eqref{eq:var-mbvp-var-problem}有唯一解$u \in H^{1}(\Omega)$,满足
  \begin{equation}
    \label{eq:var-mbvp-var-problem-solution}
    \big\| u \big\|_{H^{1}(\Omega)} \le c \,
    \left[
    \big\| f \big\|_{\widetilde{H}^{-1}(\Omega)}
    + \big\| g_{D} \big\|_{H^{\frac{1}{2}}(\Gamma_{D})}
    + \big\| g_{N} \big\|_{H^{- \frac{1}{2}}(\Gamma_{N})}
    \right].
  \end{equation}
\end{theorem}

\begin{proof}
\begin{enumerate}
\item 定义两个延拓算子。
  \begin{enumerate}
    \item 给定$g_{D} \in H^{\frac{1}{2}}(\Gamma_{D})$,定义一个有界的延拓$\widetilde{g}_{D} \in H^{\frac{1}{2}}(\Gamma)$,满足
    \begin{equation*}
      \big\| \widetilde{g}_{D} \big\|_{H^{\frac{1}{2}}(\Gamma)}
      \le c \, \big\| g_{D} \big\|_{H^{\frac{1}{2}}(\Gamma_{D})}.
    \end{equation*}
    \item 定义第二个延拓$u_{\widetilde{g}_D} \in H^{\frac{1}{2}}(\Omega)$,使得$\gamma_{0}^{\text{int}} u_{\widetilde{g}_D} = \widetilde{g}_D$,并利用逆迹定理Theorem \ref{theorem:sobolev-manifold-inverse-trace-theorem}得
    \begin{equation*}
      \big\| u_{\widetilde{g}_D} \big\|_{H^{1}(\Omega)}
      \le c_{\text{IT}} \,
      \big\| \widetilde{g}_D \big\|_{H^{\frac{1}{2}}(\Gamma)}.
    \end{equation*}
  \end{enumerate}

  \item 基于\eqref{eq:var-mbvp-var-problem},构建新的变分问题。
  \begin{equation*}
    a(u_0, \nu) =
    \langle f, \nu \rangle_{\Omega}
    + \langle g_{N}, \gamma_{0}^{\text{int}} \nu \rangle_{\Gamma_{N}}
    - a(u_{\widetilde{g}_D}, \nu), \quad \forall \, \nu \in H_{0}^{1}(\Omega, \Gamma_D),
  \end{equation*}
  研究目标是求得唯一解$u_0 \in H_{0}^{1}(\Omega, \Gamma_{D})$。

  根据\eqref{eq:sobolev-equivalence-norm-w12omega}可定义$H^{1}(\Omega)$中的等价范
  \begin{equation*}
    \big\| \nu \big\|_{W^{1,2}(\Omega), \Gamma_{D}} \coloneqq
    \left\{
    \left[
    \int_{\Gamma_{D}} \gamma_{0}^{\text{int}} \nu(x) \, d s_x
    \right]^2
    + \big\| \triangledown \nu \big\|_{L^{2}(\Omega)}^2
    \right\}^{\frac{1}{2}}.
  \end{equation*}

由双线性算子$a(.,.)$的半椭圆特性(Lemma \ref{lemma:var-bvp-operator-ellipticity-property})可得
\begin{equation*}
  \begin{split}
    a(\nu,\nu) &\ge \lambda_{0} \, \big\| \triangledown \nu \big\|_{L^2(\Omega)}^2 \\
    &= \lambda_{0} \, \big| \nu \big|_{W^{1,2}(\Omega), \Gamma_{D}}^2 \\
    & \ge c_1^A \, \big\| \nu \big\|_{W^{1,2}(\Omega)}^2.
  \end{split}
\end{equation*}


有界,椭圆,假设条件满足。由拉克斯一密格拉蒙定理(Theorem \ref{theorem:lax-milgram-lemma})可得,变分问题\eqref{eq:var-mbvp-var-problem}有唯一解$u_0 \in H_{0}^{1} (\Omega, \Gamma_{D})$。

\item 将求得的唯一解$u_0$分别代回半椭圆(Lemma \ref{lemma:var-bvp-operator-ellipticity-property})和变分问题,可得
\begin{equation*}
  \begin{split}
    c_{1}^{A} \, \big\| u_0 \big\|_{H^{1}(\Omega)}^{2} &\le a(u_0, u_0) \\
    & = \langle f, u_0 \rangle_{\Omega}
    + \langle g_{N}, \gamma_{0}^{\text{int}} u_0 \rangle_{\Gamma_{N}}
    - a(u_{\widetilde{g}_{D}}, u_0) \\
    & =
    \left[
    \big\| f \big\|_{\widetilde{H}^{-1} (\Omega)}
    + c_{2}^{A} \, \big\| u_{\widetilde{g}_{D}} \big\|_{H^{1}(\Omega)}
    \right] \, \big\| u_0 \big\|_{H^{1}(\Omega)}
    + \big\| g_{N} \big\|_{H^{-\frac{1}{2}}(\Gamma_{N})} \,
    \big\| \gamma_{0}^{\text{int}} u_0 \big\|_{\widetilde{H}^{\frac{1}{2}}(\Gamma_N)},
  \end{split}
\end{equation*}
证得\eqref{eq:var-mbvp-var-problem-solution}。
\end{enumerate}
\end{proof}

\subsection{罗宾边界值问题}
回顾一下第\ref{sec:bem-fem-potential-bvp}节的罗宾边界值问题\index{Robin boundary value condition \dotfill 罗宾边界值条件}:基于给定的$f$和$g$,求解
\begin{equation}
  \label{eq:var-rbvp-var}
  \begin{split}
    &(L u)(x) = f(x), \quad x \in \Omega, \\
    & \gamma_{1}^{\text{int}} u(x) + \kappa(x) \gamma_{0}^{\text{int}} u(x) = g(x), \quad x \in \Gamma.
  \end{split}
\end{equation}

由格林第一恒等式\eqref{eq:bvp-a-u-nu-inner-prod}建立变分问题:求解$u \in H^{1}(\Omega)$,使满足
\begin{equation}
  \label{eq:var-rbvp-var-problem}
  a(u,\nu) + \int_{\Gamma} \kappa(x) \gamma_{0}^{\text{int}} u(x) \gamma_{0}^{\text{int}} \nu(x) \, d s_{x}
  = \langle f, \nu \rangle_{\Omega}
  + \langle g, \gamma_{0}^{\text{int}} \nu \rangle_{\Gamma}, \quad \forall \, \nu \in H^{1}(\Omega).
\end{equation}

\begin{theorem}[罗宾边界值变分问题的唯一解]
  给定$f \in \widetilde{H}^{-1}(\Omega), g\in H^{-\frac{1}{2}}(\Gamma)$,假定$\kappa(x) \ge \kappa_{0} > 0, \forall \, x \in \Gamma$。那么罗宾边界值变分问题\eqref{eq:var-rbvp-var-problem}存在唯一解
  $u \in H^{1}(\Omega)$,满足
  \begin{equation}
    \label{eq:var-rbvp-var-problem-solution}
    \big\| u \big\|_{H^{1}(\Omega)} \le c \,
    \left[
    \big\| f \big\|_{\widetilde{H}^{-1} (\Omega)}
    + \big\| g \big\|_{H^{-\frac{1}{2}}(\Gamma)}
    \right].
  \end{equation}
\end{theorem}
\begin{proof}
  根据\eqref{eq:sobolev-equivalence-norm-w12omega},定义$H^{1}(\Omega)$中的等价范
  \begin{equation*}
    \big\| \nu \big\|_{W^{1,2}(\Omega), \Gamma} \coloneqq
    \left\{
    \big\| \gamma_{0}^{\text{int}} \nu \big\|_{L^2(\Gamma)}^2
    + \big\| \triangledown \nu \big\|_{L^{2}(\Omega)}^2
    \right\}^{\frac{1}{2}}
  \end{equation*}

  结合$\kappa(x) \ge \kappa_{0} > 0, \forall \, x \in \Gamma$,以及半椭圆特性Lemma \ref{lemma:var-bvp-operator-ellipticity-property},有
  \begin{equation*}
    \begin{split}
      a(\nu,\nu) + \int_{\Gamma} \kappa(x)
      \left[
      \gamma_{0}^{\text{int}} \nu(x)
      \right]^2 \, d s_x
      & \ge
      \lambda_{0} \,
      \big\| \triangledown \nu \big\|_{L^{2}(\Omega)}^{2}
      + \kappa_{0} \, \big\| \gamma_{0}^{\text{int}} \nu \big\|_{L^{2}(\Gamma)}^2 \\
      & \ge
      \min \{ \lambda_0, \kappa_0 \} \,
      \big\| \nu \big\|_{H^{1}(\Omega), \Gamma}^2 \\
      & \ge c_{1}^{A} \, \big\| \nu \big\|_{H^{1}(\Omega)}^2.
    \end{split}
  \end{equation*}

进而,根据拉克斯一密格拉蒙定理(Theorem \ref{theorem:lax-milgram-lemma}),可得罗宾边界值变分问题\eqref{eq:var-rbvp-var-problem}的唯一解。由此可得\eqref{eq:var-rbvp-var-problem-solution}。
\end{proof}
