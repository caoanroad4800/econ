%!TEX root = ../DSGEnotes.tex

\chapter{最大似然估计}
\label{sec:mle}

\epigraph{The maximum-likelihood procedure in any problem is what you are most likely to do if you don’t know any statistics.}{\textit{Harrison H. Barrett \\ Foundations of Image Science}}

\section{简介}
\label{sec:mle-intro}
设一个$M \times 1$的向量$g$,描述一组随机数据。对应地,设一个$K \times 1$的参数向量$\theta$,用来描述这组随机数据$g$。那么,概率密度方程(probability density function, PDF)\index{probability density function(PDF) \dotfill 概率密度方程} $pr(g|\theta)$用于表示,在给定$\theta$的情况下,随机数据$g$的抽样分布(sampling distribution)。

反之,若在现实中已经观测到了$g$,则我们可将$pr(g|\theta)$看成是一个关于参数$\theta$的方程,称为似然方程(likelihood function)\index{likelihood function \dotfill 似然方程},定义为$L(\theta | g)$
\begin{equation*}
  L(\theta | g) = pr(g | \theta),
\end{equation*}
需要注意的是,$L(\theta | g)$并不是$\theta$的概率密度方程。

我们将对$\theta$的估计命名为$\hat{\theta}$。$\hat{\theta}$是个关于$g$的方程:由于$g$是随机数据向量,导致$\hat{\theta}(g)$也是个关于$g$的随机变量向量。本质上来说,参数估计反映的是一个映射过程
\begin{equation*}
  \begin{cases}
    \theta \mapsto g \mapsto \hat{\theta}, \\
    pr(\theta) \mapsto pr(g | \theta) \mapsto pr(\hat{\theta} | \theta).
  \end{cases}
\end{equation*}
从这个角度来说,$\hat{\theta}$是一个关于$\theta$的随机变量。

\section{近似的测度}
\label{sec:mle-performance-metrics}
估计值$\hat{\theta}$相比于真实值$\theta$往往存在误差。误差常常来自于两个方面:
\begin{itemize}
  \item 随机误差,又称precision:如估计偏差,噪音等。常用方差(viariance)与以量化描述。
  \item 系统误差,又称accuracy:如估计偏误,校准误差,模型误设定等。常用偏误(bias)或均方误差(MSE, EMSE)予以量化描述。
\end{itemize}
两种误差均可由估计值的条件分布$pr(\hat{\theta} | \theta)$予以量化。

\subsection{估计的偏误:标量参数}
\label{sec:mle-bias-scalar}

以标量参数$\theta$及其(随机)估计值$\hat{\theta}$为例,估计的条件均值$\overline{\hat{\theta}}$可由似然方程计算而得
\begin{equation}
  \label{eq:mle-mean-estimate-theta}
  \overline{\hat{\theta}} =
  \left\langle
  \hat{\theta}(g)
  \right\rangle_{g|\theta}
  = \int d^{M_g} \, pr(g | \theta) \hat{\theta}(g),
\end{equation}
其中$\langle \cdot \rangle_{g|\theta}$表示在给定的$\theta$下,各个随机估计值$\hat{\theta}(g)$的平均值。

如果已知估计$\hat{\theta}$的条件概率密度$pr \left(\hat{\theta} | \theta \right)$,那么\eqref{eq:mle-mean-estimate-theta}可进一步改写为
\begin{equation}
  \label{eq:mle-hat-theta-random}
  \overline{\hat{\theta}} = \int d \hat{\theta} \, pr \left( \hat{\theta} | \theta \right) \, \hat{\theta}.
\end{equation}



\subsubsection{偏误}
\label{sec:mle-bias-def}
我们将$\overline{\hat{\theta}}$相对于$\theta$的偏差称为关于真实值的条件偏误(conditional bias on true parameter)\index{conditional bias \dotfill 条件偏误},定义为$b(\theta)$
\begin{equation}
  \label{eq:mle-conditional-bias}
  b(\theta) = \overline{\hat{\theta}} - \theta.
\end{equation}

如果某个估计$\overline{\hat{\theta}}$,对于任意$\theta$都满足$b(\theta) = 0$,那我们称之为无偏估计(unbiased estimate)\index{unbiased estimate \dotfill 无偏估计}。

\subsubsection{可测度性}
\label{sec:mle-estimability}
可测度性(estimability)或可识别性(identifiability)。如果无论随机数据集$g$的取值如何,都存在关于参数$\theta$的无偏估计$\hat{\theta}$,满足$b(\theta) = 0 \quad \forall \, \theta$,那我们称参数$\theta$是可估计的(estimable)或可识别的(identifiable),表示为$\overline{\overline{\hat{\theta}}}$
\begin{equation}
  \label{eq:mle-estimablility}
  \overline{\overline{\hat{\theta}}}
  = \int d \theta \, pr(\theta) \, \overline{\hat{\theta}}
  = \int d \theta \, pr(\theta) \, \int d^{M_g} \, pr(g | \theta) \, \hat{\theta}(g).
\end{equation}

\subsubsection{测度的方差和均方误差}
\label{sec:mle-estimate-var-mse}
某一测度$\hat{\theta}$相对于测度均值$\overline{\theta}$的波动,我们用方差$var \left( \hat{\theta} \right)$来表示
\begin{equation}
  \label{eq:mle-variance-hat}
  var \left( \hat{\theta} \right)
  = \sigma_{\hat{\theta}}^{2}
  = \left\langle
  \left|
  \hat{\theta}(g) - \overline{\hat{\theta}}
  \right|^{2}
  \right\rangle_{g | \theta}.
\end{equation}

某一测度$\hat{\theta}$相对于真实值$\theta$的波动,我们用均方误差(mean square error, MSE)\index{mean square error (MSE) \dotfill 均方误差}来表示
\begin{equation}
  \label{eq:mle-mse-def}
  MSE(\theta) = \left\langle
  \left| \hat{\theta} - \theta \right|^{2}
  \right\rangle_{g | \theta}.
\end{equation}

考虑到$\theta$随$g$而具有的随机性特征,将均方误差再严$\theta$取平均值,得整体均方误差(ensemble mean square error, EMSE)\index{ensemble mean suqare error (EMSE)\dotfill 整体均方误差}
\begin{equation}
  \label{eq:mle-emse-def}
  EMSE(\theta) =
  \left\langle
  \left\langle
  \left| \hat{\theta} - \theta \right|^{2}
  \right\rangle_{g|\theta}
  \right\rangle_{\theta}.
\end{equation}

\subsection{估计的偏误:向量参数}
\label{sec:mle-bias-vector}
设一个$P$维的参数向量$\theta$,对应估计值$\hat{\theta}$。我们将平均值$\overline{\hat{\theta}}$(有时也写作$\langle \theta \rangle$)定义如下
\begin{equation}
  \label{eq:mle-estimate-vector}
  \overline{\hat{\theta}} \left( g \right) = \int d^{M_g} \, pr \left(g | \theta \right) \, \hat{\theta}\left ( g \right) = \int d^{P_{\hat{\theta}}} \, pr \left( \hat{\theta} | \theta \right) \, \hat{\theta}\left( g \right).
\end{equation}

\subsubsection{平均偏误}
平均偏误$\overline{b}\left( \theta \right)$定义为
\begin{equation}
  \label{eq:mle-estimate-vector-bias-avg}
  \overline{b}(\theta) =
  \langle b \left( \theta \right) \rangle_{\theta}
\end{equation}

\subsubsection{测度的方差和均方误差}
我们将$P$维随机估计向量$\hat{\theta}$中,第$p$个元素的均值表示为$\langle \hat{\theta} \rangle_{p} = \overline{\hat{\theta}}_{p}$。对应的方差为
\begin{equation}
  \begin{split}
  \label{eq:mle-esitmate-vector-variance}
  var \left( \hat{\theta}_{p} \right)
  & = \left\langle
  \left[
  \hat{\theta}_{p} - \langle \hat{\theta} \rangle
  \right] \, \left[
  \hat{\theta}_{p} - \langle \hat{\theta} \rangle
  \right]^{\top}
  \right\rangle_{g | \theta} \\
  & = \int_{\infty} d^{M_{g}} \,
  \left|
  \hat{\theta}_{p} \left( g \right) - \langle \hat{\theta}_{p} \left( g \right) \rangle
  \right|^{2} \, pr \left( g | \theta \right) \\
  & = \int_{\infty} d^{P_{\theta}} \,
  \left|
  \hat{\theta}_{p} - \langle \hat{\theta}_{p} \rangle
  \right|^{2} \,
  pr \left( \hat{\theta} | \theta \right).
\end{split}
\end{equation}

将所有$p \in P$个方差汇总在一起,得到整体的方差协方差矩阵,定义为$K_{\hat{\theta}}$
\begin{equation}
  \label{eq:mle-estimate-vector-varcovar}
  K_{\hat{\theta}} = \left\langle
  \left( \hat{\theta} - \overline{\theta} \right) \,
  \left( \hat{\theta} - \overline{\theta} \right)^{\dagger}
  \right\rangle
  = \left\langle
  \Delta \hat{\theta} \, \Delta \hat{\theta}^{\dagger}
  \right\rangle.
\end{equation}

参数向量的均方误差
\begin{equation}
  \label{eq:mle-estimate-vector-mse}
  \begin{split}
    MSE & = \left\langle
    \left\|
    \hat{\theta} - \theta
    \right\|^{2}
    \right\rangle_{g | \theta} \\
    & = \int_{\infty} d^{M_{g}} \left\| \hat{\theta}(g) - \theta \right\|^{2} \, pr \left( g | \theta \right) \\
    & = \trace \left( K_{\hat{\theta}} \right) +
    \trace \left( b(\theta) b(\theta)^{\dagger} \right)
    \end{split}
\end{equation}

\section{费雪信息矩阵和克拉美罗下界}
\label{sec:mle-fischer-info-cramer-rao-bound}
参数估计的方差协方差矩阵$K$,存在一些限定条件。
\subsection{标量参数}
\label{sec:mle-fischer-info-cramer-rao-bound-scalar}
任意标量参数的无偏估计,其方差都必须满足克拉美罗下界(Cramér-Rao lower bound, CRLB)\index{Cramér-Rao lower Bound \dotfill 克拉美罗下界}
\begin{equation}
  \label{sec:mle-scalar-crlb-unbiased}
  var \left(\hat{\theta} \right) \ge
  \frac{
  1
  }
  {
  \underbrace{
  \left\langle
  \left[
  \frac{\partial}{\partial \theta}
  \log pr \left(g | \theta \right)
  \right]^{2}
  \right\rangle_{g|\theta}
  }_{\eqqcolon F}
  },
\end{equation}
其中我们将分母的部分称为费雪信息矩阵(Fischer information matrix)\index{Fischer information matrix \dotfill 费雪信息矩阵}。

若估计是有偏的,克拉美罗下界为
\begin{equation}
  \label{sec:mle-scalar-crlb-biased}
  var\left( \hat{\theta} \right) \ge
  \frac{
  \left(
  \frac{d b(\theta)}{d \theta} + 1
  \right)^{2}
  }{
  \left\langle
  \left[
  \frac{\partial}{\partial \theta}
  \log pr \left(g | \theta \right)
  \right]^{2}
  \right\rangle_{g|\theta}
  }
\end{equation}

\subsection{向量参数}
\label{sec:mle-fischer-info-cramer-rao-bound-vector}
对于$P$维向量参数$\theta$,费雪信息矩阵$F$成为$P \times P$维的埃米特矩阵(Hermitian matrix),其中第$jk$个元素$F_{jk}$表示为
\begin{equation}
  \label{eq:mle-fischer-info-vector-jk}
  \begin{split}
    F_{jk} & =
    \left\langle
    \left[
    \frac{\partial}{\partial \theta_{j}} \log pr \left(g | \theta \right)
    \right] \,
    \left[
    \frac{\partial}{\partial \theta_{k}} \log pr \left(g | \theta \right)
    \right]
    \right\rangle_{g|\theta} \\
    & = \int_{\infty} d^{M_{g}} \, pr \left( g | \theta \right) \,
    \left[
    \frac{1}{pr \left( g | \theta \right)} \,
    \frac{\partial}{\partial \theta_{j}}
    pr \left( g | \theta \right)
    \right] \,
    \left[
    \frac{1}{pr \left( g | \theta \right)} \,
    \frac{\partial}{\partial \theta_{k}}
    pr \left( g | \theta \right)
    \right].
  \end{split}
\end{equation}

在得到费雪信息矩阵$F$的基础上,来看(参数向量$\theta$的无偏估计向量$\hat{\theta}$的)方差协方差矩阵$K_{\hat{\theta}}$。$n \times n$矩阵$K_{\hat{\theta}}$中的第$nn$个元素,等于$\hat{\theta}$向量中第$nn$个元素的方差,并且也满足克拉美洛下界
\begin{equation}
  \label{eq:mle-fischer-var-vector-nn}
  \left[ K_{\hat{\theta}} \right]_{nn}
  = var \left( \hat{\theta}_{nn} \right)
  \ge \left[ F^{-1} \right]_{nn},
\end{equation}
这就是说,为了计算无偏估计向量$\hat{\theta}$中第$n$个元素方差的下界(或方差协方差矩阵$K_{\hat{\theta}}$中第$nn$元素的下界),我们需要首先计算费雪信息矩阵的逆矩阵,提取其中的第$nn$个元素。

一个满足克拉美罗下界条件的无偏估计是有效估计(efficient)。

\section{最大似然估计}
\label{sec:mle-mle}
前面讨论了一个参数估计$\hat{\theta}$所应当具有的特性。随后的问题就是,如何找到这个估计值?常见的方法之一是最大似然估计(maximum likelihood estimation)
\begin{equation}
  \label{eq:mle-mle-def}
  \hat{\theta}_{ML} \equiv \argmax_{\theta} pr \left( g | \theta \right),
\end{equation}
有时也用对数似然方程来表示
\begin{equation}
  \label{eq:mle-mle-def-log}
  \hat{\theta}_{ML} \equiv \mathop{\argmax}_{\theta} \log pr \left( g | \theta \right),
\end{equation}
即是说,选择合适的参数值$\hat{\theta}_{ML}$,使得在$\hat{\theta}_{ML}$下可能实际观测到数据集合$g$的概率最大。

没有其他额外限定条件的情况下,$\hat{\theta}_{ML}$常取值于(对数)似然方程对$\theta$的梯度(gradient),即一阶导数等于$0$的点
\begin{equation}
  \label{eq:mle-score}
  \underbrace{
  \triangledown_{\theta} \log pr \left( g | \theta \right)
  }_{\eqqcolon s(g)} =0, \quad \leftrightarrow \theta = \hat{\theta}_{ML},
\end{equation}
我们将这个表示斜率的随机向量$s(g)$称为score\index{score (maximum likelihood) \dotfill 评分(最大似然估计)},用于描述(对数)似然方程对参数变化的敏感程度,因此有时也称敏感方程(sensitivity function)\index{sensitivity function (maximum likelihood) \dotfill 敏感方程(最大似然估计)}。

$s(g)$的方差协方差矩阵就等于费雪信息矩阵的逆。

\subsection{最大似然估计:to be or not to be?}

\begin{enumerate}
  \item 选择最大似然估计法的理由
  \begin{enumerate}
    \item 有效。如果的确存在一个有效的最大似然估计值。
    \item 渐进有效。随着观测数据的数量增加,估计效果逐渐提升。
    \item 渐进无偏。
    \item 渐进一致。
    \item 便于计算。相对来说。
    \item 有助于严谨地强制使得估计结果与实际观测数据相一致。
    \item 无需先验信息$pr (\theta)$设定。与之相反地,参考贝叶斯估计。
  \end{enumerate}
  \item 不选择最大似然估计法的理由
  \begin{enumerate}
    \item (严谨地强制使得估计结果与实际观测数据相一致)。

    现实中观测到的数据常常是充满噪音的。强行严谨一致容易导致计算出的最大似然估计值,哪怕是无偏的,也随之充满噪音。

    \item (无需先验信息$pr (\theta)$设定)。

    现实中总是存在着一些先验信息$pr (\theta)$,我们应当充分利用这些信息,哪怕他们有时并非完全无偏\footnote{一种将先验$pr (\theta)$与最大似然估计相结合的方法称为加权似然法(weighted likeihood),如
    \begin{equation*}
      \hat{\theta}_{WL} \equiv \mathop{\argmax}_{\theta} pr \left( g | \theta \right) pr \left( \theta \right) = \mathop{\argmax}_{\theta} pr\left( \theta | g \right).
    \end{equation*}
    }。
  \end{enumerate}
\end{enumerate}

\section{伯努利实验:最大似然估计法}
\label{sec:mle-bernoulli}

现在我们从伯努利实验(Bernoulli trials)\index{Bernoulli trials \dotfill 伯努利实验}入手,介绍如何应用最大似然估计法求解相关问题。基于本节的求解思路,我们随后在python环境中编写程序求解,见第\ref{sec:mle-bernoulli-python}节。

手头一枚一元硬币,现在我们连续掷10次,有7次头朝上。我们想要利用最大似然法,估计每一次投掷得到头朝上的概率。

用$p\in (0,1)$来表示每次投掷得到头朝上的概率(参数的实际值)。那么连续投掷$N$次,在这$N$次中得到$n \in \mathbb{N}$次头朝上的概率可以表示为
\begin{equation}
  \label{eq:mle-bernoulli-head-prob}
  pr \left( n | p \right) = \begin{pmatrix}
  N \\ n
\end{pmatrix} \, p^{n} \, \left( 1 - p \right)^{N-n},
\end{equation}
其中二项式系数满足
\begin{equation}
  \label{eq:mle-bernoulli-head-prob-binomial}
  \begin{pmatrix}
  N \\ n
\end{pmatrix} = \frac{N!}{n! \, \left(N - n \right)!}.
\end{equation}

随机变量$n$的均值和方差分别为
\begin{align}
  \label{eq:mle-random-n-mean}
  \langle n \rangle & = N p, \\
  \label{eq:mle-random-n-variation}
  \sigma^{2} &= N p (1-p).
\end{align}

在上一组实验中我们观测到的数据是投掷N次得到n次头(如10次得到7次头)。我们想要知道如果反复重复这组实验,出现N/n的最大概率p是多少?可用最大似然法进行估计:改变$p$的值,使得$pr(n | p)$达到最大。

对$pr(n | p)$取对数,然后对$p$求导,得score方程$s(n)$
\begin{equation}
  \label{eq:mle-bernoulli-score}
  \begin{split}
      \log pr(n | p) & = constant + n \log p + (N-n) \log (1-p), \\
      s(n) & = \frac{d}{d p} \log pr(n | p) = \frac{n}{p} + \frac{N - n}{1 - p}.
  \end{split}
\end{equation}

设$s(n)=0$,对应的$p=\hat{p}_{ML}$即为$p$的最大似然估计
\begin{equation}
  \label{eq:mle-bernoulli-estimate}
  \hat{p}_{ML} = \frac{n}{N},
\end{equation}
分子表示观测到头的次数,分母表示总的投掷次数。

最大似然估计$\hat{p}_{ML}$的均值和方差,由\eqref{eq:mle-random-n-mean}-\eqref{eq:mle-random-n-variation}得
\begin{align}
  \label{eq:mle-bernoulli-estimate-mean}
  \left\langle \hat{p}_{ML} \right\rangle &= \frac{\langle n \rangle}{N}, \\
  var \left( \hat{p}_{ML} \right) & = \frac{var (n)}{N^{2}} = \frac{p (1-p)}{N }
\end{align}

据此可得估计的费雪信息矩阵
\begin{equation}
  \label{eq:mle-bernoulli-eistimate-information}
  F = \left\langle
  \left[
  \frac{n}{p} - \frac{N-n}{1-p}
  \right]^{2}
  \right\rangle
  = \frac{
  \langle
  \left[ n - Np \right]^{2}
  \rangle
  }{
  p^{2} \left( 1 - p \right)^{2}
  }
  = \frac{var (n)}{p^{2} \left(1-p \right)^{2}}
  = \frac{N}{p \left( 1 - p \right)},
\end{equation}
因此我们有
\begin{equation}
  \label{eq:mle-estimate-fisher-var}
  F^{-1} = var \left( \hat{p}_{ML} \right),
\end{equation}
并且最大似然估计$\hat{p}_{ML}$是有效的。

\section{伯努利实验:最大似然估计的Python程序实现}
\label{sec:mle-bernoulli-python}
伯努利实验(频率法)在Python环境下的实现

伯努利实验\href{https://en.wikipedia.org/wiki/Bernoulli_trial}{(Bernoulli
trials)}
是指重复若干次(设为\(T\)次)某一项实验(或称迭代iteration),每次实验(iteration)的结果可能是二者之一:成功\((0)\)或者失败\((1)\)。假定我们现在来检测成功的概率,可以基于频率学派的思路,使用最大似然估计法来实现(maximum
likelihood, MLB)。

\begin{Verbatim}[commandchars=\\\{\}]
{\color{incolor}In [{\color{incolor}2}]:} \PY{c+c1}{\PYZsh{} 初始设定}
    \PY{o}{\PYZpc{}}\PY{k}{matplotlib} inline
    \PY{k+kn}{from} \PY{n+nn}{\PYZus{}\PYZus{}future\PYZus{}\PYZus{}} \PY{k}{import} \PY{n}{division}
    \PY{k+kn}{import} \PY{n+nn}{numpy} \PY{k}{as} \PY{n+nn}{np}
    \PY{k+kn}{import} \PY{n+nn}{pandas} \PY{k}{as} \PY{n+nn}{pd}
    \PY{k+kn}{import} \PY{n+nn}{statsmodels}\PY{n+nn}{.}\PY{n+nn}{api} \PY{k}{as} \PY{n+nn}{sm}
    \PY{k+kn}{import} \PY{n+nn}{sympy} \PY{k}{as} \PY{n+nn}{sp}
    \PY{k+kn}{import} \PY{n+nn}{pymc3}
    \PY{k+kn}{import} \PY{n+nn}{matplotlib}\PY{n+nn}{.}\PY{n+nn}{pyplot} \PY{k}{as} \PY{n+nn}{plt}
    \PY{k+kn}{import} \PY{n+nn}{matplotlib}\PY{n+nn}{.}\PY{n+nn}{gridspec} \PY{k}{as} \PY{n+nn}{gridspec}
    \PY{k+kn}{from} \PY{n+nn}{mpl\PYZus{}toolkits}\PY{n+nn}{.}\PY{n+nn}{mplot3d} \PY{k}{import} \PY{n}{Axes3D}
    \PY{k+kn}{from} \PY{n+nn}{scipy} \PY{k}{import} \PY{n}{stats}
    \PY{k+kn}{from} \PY{n+nn}{scipy}\PY{n+nn}{.}\PY{n+nn}{special} \PY{k}{import} \PY{n}{gamma}
    \PY{k+kn}{from} \PY{n+nn}{sympy}\PY{n+nn}{.}\PY{n+nn}{interactive} \PY{k}{import} \PY{n}{printing}
    \PY{n}{printing}\PY{o}{.}\PY{n}{init\PYZus{}printing}\PY{p}{(}\PY{p}{)}
\end{Verbatim}

% \begin{Verbatim}[commandchars=\\\{\}]
% /Users/YYZ/anaconda3/lib/python3.6/site-packages/statsmodels/compat/pandas.py:56: FutureWarning: The pandas.core.datetools module is deprecated and will be removed in a future version. Please use the pandas.tseries module instead.
% from pandas.core import datetools
%
% \end{Verbatim}

\subsection{模型构建}
\label{ux6a21ux578bux6784ux5efa}
设一个伯努利实验

\begin{equation}
y \sim Bernoulli(\theta) = Binomial(1,\theta).
\end{equation}

我们将概率密度方程,或称边际似然方程,写为

\begin{equation}
p(y|\theta) = \theta^{y} \, \left(1-\theta \right)^{1-y} = \begin{cases}
\theta & y = 1 \\
1 - \theta & y = 0.
\end{cases}
\end{equation}

设\(\theta = 0.3\),随机进行100次试验。

    \begin{Verbatim}[commandchars=\\\{\}]
{\color{incolor}In [{\color{incolor}3}]:} \PY{c+c1}{\PYZsh{} 模拟数据}
        \PY{n}{np}\PY{o}{.}\PY{n}{random}\PY{o}{.}\PY{n}{seed}\PY{p}{(}\PY{l+m+mi}{123}\PY{p}{)}

        \PY{n}{nobs} \PY{o}{=} \PY{l+m+mi}{100}
        \PY{n}{theta} \PY{o}{=} \PY{l+m+mf}{0.3}
        \PY{c+c1}{\PYZsh{} numpy.random.binomial(n, p, size=None)}
        \PY{c+c1}{\PYZsh{} https://docs.scipy.org/doc/numpy\PYZhy{}1.13.0/reference/generated/numpy.random.binomial.html}
        \PY{n}{Y} \PY{o}{=} \PY{n}{np}\PY{o}{.}\PY{n}{random}\PY{o}{.}\PY{n}{binomial}\PY{p}{(}\PY{l+m+mi}{1}\PY{p}{,} \PY{n}{theta}\PY{p}{,} \PY{n}{nobs}\PY{p}{)}
\end{Verbatim}

    \begin{Verbatim}[commandchars=\\\{\}]
{\color{incolor}In [{\color{incolor}4}]:} \PY{c+c1}{\PYZsh{} 画图}

        \PY{n}{fig} \PY{o}{=} \PY{n}{plt}\PY{o}{.}\PY{n}{figure}\PY{p}{(}\PY{n}{figsize}\PY{o}{=}\PY{p}{(}\PY{l+m+mi}{10}\PY{p}{,}\PY{l+m+mi}{5}\PY{p}{)}\PY{p}{)}
        \PY{n}{gs} \PY{o}{=} \PY{n}{gridspec}\PY{o}{.}\PY{n}{GridSpec}\PY{p}{(}\PY{l+m+mi}{1}\PY{p}{,} \PY{l+m+mi}{2}\PY{p}{,} \PY{n}{width\PYZus{}ratios} \PY{o}{=} \PY{p}{[}\PY{l+m+mi}{5}\PY{p}{,}\PY{l+m+mi}{1}\PY{p}{]}\PY{p}{)}
        \PY{n}{ax1} \PY{o}{=} \PY{n}{fig}\PY{o}{.}\PY{n}{add\PYZus{}subplot}\PY{p}{(}\PY{n}{gs}\PY{p}{[}\PY{l+m+mi}{0}\PY{p}{]}\PY{p}{)}
        \PY{n}{ax2} \PY{o}{=} \PY{n}{fig}\PY{o}{.}\PY{n}{add\PYZus{}subplot}\PY{p}{(}\PY{n}{gs}\PY{p}{[}\PY{l+m+mi}{1}\PY{p}{]}\PY{p}{)}

        \PY{n}{ax1}\PY{o}{.}\PY{n}{plot}\PY{p}{(}\PY{n+nb}{range}\PY{p}{(}\PY{n}{nobs}\PY{p}{)}\PY{p}{,} \PY{n}{Y}\PY{p}{,} \PY{l+s+s2}{\PYZdq{}}\PY{l+s+s2}{x}\PY{l+s+s2}{\PYZdq{}}\PY{p}{)}
        \PY{n}{ax2}\PY{o}{.}\PY{n}{hist}\PY{p}{(}\PY{o}{\PYZhy{}}\PY{n}{Y}\PY{p}{,} \PY{n}{bins}\PY{o}{=}\PY{l+m+mi}{2}\PY{p}{)}

        \PY{n}{ax1}\PY{o}{.}\PY{n}{yaxis}\PY{o}{.}\PY{n}{set}\PY{p}{(}\PY{n}{ticks}\PY{o}{=}\PY{p}{(}\PY{l+m+mi}{0}\PY{p}{,}\PY{l+m+mi}{1}\PY{p}{)}\PY{p}{,} \PY{n}{ticklabels}\PY{o}{=}\PY{p}{(}\PY{l+s+s2}{\PYZdq{}}\PY{l+s+s2}{Failure}\PY{l+s+s2}{\PYZdq{}}\PY{p}{,} \PY{l+s+s2}{\PYZdq{}}\PY{l+s+s2}{Success}\PY{l+s+s2}{\PYZdq{}}\PY{p}{)}\PY{p}{)}
        \PY{n}{ax2}\PY{o}{.}\PY{n}{xaxis}\PY{o}{.}\PY{n}{set}\PY{p}{(}\PY{n}{ticks}\PY{o}{=}\PY{p}{(}\PY{o}{\PYZhy{}}\PY{l+m+mi}{1}\PY{p}{,}\PY{l+m+mi}{0}\PY{p}{)}\PY{p}{,} \PY{n}{ticklabels}\PY{o}{=}\PY{p}{(}\PY{l+s+s2}{\PYZdq{}}\PY{l+s+s2}{Success}\PY{l+s+s2}{\PYZdq{}}\PY{p}{,} \PY{l+s+s2}{\PYZdq{}}\PY{l+s+s2}{Failure}\PY{l+s+s2}{\PYZdq{}}\PY{p}{)}\PY{p}{)}

        \PY{n}{ax1}\PY{o}{.}\PY{n}{set}\PY{p}{(}\PY{n}{title}\PY{o}{=}\PY{l+s+sa}{r}\PY{l+s+s2}{\PYZdq{}}\PY{l+s+s2}{Bernoulli Trial Outcomes \PYZdl{}(}\PY{l+s+s2}{\PYZbs{}}\PY{l+s+s2}{theta = 0.3)\PYZdl{}}\PY{l+s+s2}{\PYZdq{}}\PY{p}{,} \PY{n}{xlabel}\PY{o}{=}\PY{l+s+s2}{\PYZdq{}}\PY{l+s+s2}{Trial}\PY{l+s+s2}{\PYZdq{}}\PY{p}{,} \PY{n}{ylim}\PY{o}{=}\PY{p}{(}\PY{o}{\PYZhy{}}\PY{l+m+mf}{0.2}\PY{p}{,} \PY{l+m+mf}{1.2}\PY{p}{)}\PY{p}{)}
        \PY{n}{ax2}\PY{o}{.}\PY{n}{set}\PY{p}{(}\PY{n}{ylabel}\PY{o}{=}\PY{l+s+s2}{\PYZdq{}}\PY{l+s+s2}{Frequency}\PY{l+s+s2}{\PYZdq{}}\PY{p}{)}

        \PY{n}{fig}\PY{o}{.}\PY{n}{tight\PYZus{}layout}\PY{p}{(}\PY{p}{)}
\end{Verbatim}

    \begin{center}
    \adjustimage{max size={0.9\linewidth}{0.9\paperheight}}{./Figures/output_5_0.png}
    \end{center}
    { \hspace*{\fill} \\}

\subsection{似然方程}\label{ux4f3cux7136ux65b9ux7a0b}
现在假定从随机变量100个\(y\)中,随机提取\(T\)个值,构成一个样本

\begin{equation}
Y = \left(y_1, \ldots y_{T} \right)^{\top},
\end{equation}

对应的联合似然方程为

\begin{equation}
p(Y|\theta) = \prod_{i=1}^{T} \theta^{y_{i}} \, \left( 1 - \theta \right)^{1-y_{i}} = \theta^{s} \, \left( 1-\theta \right)^{T - s}, \quad s = \sum_{i} y_{i},
\end{equation}

\(s\)就是观测到的成功的次数(对应\(y=1\));\(T-s\)是观测到的失败的次数(\(y=0\))。

随后的目标就是,如何将似然方程最大化?

\subsection{最大似然估计:解析法}
\label{ux6700ux5927ux4f3cux7136ux4f30ux8ba1ux89e3ux6790ux6cd5}

一种较为直观的方法是解析法。

\begin{itemize}
\tightlist
\item
  似然方程\(L(\theta ; Y)\)

  \begin{equation}
  L(\theta ; Y) = \theta^{s} \, \left( 1 - \theta \right)^{T-s}.
  \end{equation}
\item
  对数似然方程\(\log (\theta ; Y)\)

  \begin{equation}
  \log L(\theta ; Y) = s \log \theta + \left( T - s \right) \log \left( 1 - \theta \right).
  \end{equation}
\item
  score矩阵 (一阶导数)

  \begin{equation}
  \frac{
  \partial \log L(\theta ; Y)
  }{
  \partial \theta
  }
  = \frac{s}{\theta} - \frac{T-s}{1-\theta} = s \, \theta^{-1} - \left(T - s \right) \, \left( 1 - \theta \right)^{-1}.
  \end{equation}
\item
  黑塞矩阵 (二阶导数)

  \begin{equation}
  H(\theta) = \frac{
  \partial^{2} \log L(\theta ; Y)
  }{
  \partial \theta^{2}
  } = - s \, \theta^{-2} - \left(T - s \right) \, \left( 1 - \theta \right)^{-2}.
  \end{equation}
\item
  信息矩阵

  \begin{equation}\begin{split}
  I[\theta] & = - E [H(\theta)] = - E \left[- s \, \theta^{-2} - \left(T - s \right) \, \left( 1 - \theta \right)^{-2} \right] \\
  & = \left( \theta \, T \right) \, \theta^{-2} + \left(T - \theta \, T \right) \, \left( 1 - \theta \right)^{-2}
  = \frac{T}{\theta (1-\theta) }.
  \end{split}\end{equation}
\item
  方差协方差矩阵(逆信息矩阵)

  \begin{equation}
  I[\theta]^{-1} = \frac{\theta (1-\theta) }{T}.
  \end{equation}
\end{itemize}

    \begin{Verbatim}[commandchars=\\\{\}]
{\color{incolor}In [{\color{incolor}5}]:} \PY{n}{t}\PY{p}{,} \PY{n}{T}\PY{p}{,} \PY{n}{s} \PY{o}{=} \PY{n}{sp}\PY{o}{.}\PY{n}{symbols}\PY{p}{(}\PY{l+s+s2}{\PYZdq{}}\PY{l+s+s2}{theta, T, s}\PY{l+s+s2}{\PYZdq{}}\PY{p}{)}

        \PY{c+c1}{\PYZsh{} 构建符号方程}
        \PY{n}{likelihood} \PY{o}{=} \PY{p}{(}\PY{n}{t} \PY{o}{*}\PY{o}{*} \PY{n}{s}\PY{p}{)} \PY{o}{*} \PY{p}{(}\PY{p}{(}\PY{l+m+mi}{1} \PY{o}{\PYZhy{}} \PY{n}{t}\PY{p}{)}\PY{o}{*}\PY{o}{*}\PY{p}{(}\PY{n}{T} \PY{o}{\PYZhy{}} \PY{n}{s}\PY{p}{)}\PY{p}{)}
        \PY{n}{loglike} \PY{o}{=} \PY{n}{s} \PY{o}{*} \PY{n}{sp}\PY{o}{.}\PY{n}{log}\PY{p}{(}\PY{n}{t}\PY{p}{)} \PY{o}{+} \PY{p}{(}\PY{n}{T} \PY{o}{\PYZhy{}} \PY{n}{s}\PY{p}{)} \PY{o}{*} \PY{n}{sp}\PY{o}{.}\PY{n}{log}\PY{p}{(}\PY{l+m+mi}{1}\PY{o}{\PYZhy{}}\PY{n}{t}\PY{p}{)}
        \PY{n}{score} \PY{o}{=} \PY{p}{(}\PY{n}{s}\PY{o}{/}\PY{n}{t}\PY{p}{)} \PY{o}{\PYZhy{}} \PY{p}{(}\PY{p}{(}\PY{n}{T}\PY{o}{\PYZhy{}}\PY{n}{s}\PY{p}{)}\PY{o}{/}\PY{p}{(}\PY{l+m+mi}{1}\PY{o}{\PYZhy{}}\PY{n}{t}\PY{p}{)}\PY{p}{)}
        \PY{n}{hessian} \PY{o}{=} \PY{o}{\PYZhy{}} \PY{n}{s}\PY{o}{/}\PY{p}{(}\PY{n}{t}\PY{o}{*}\PY{o}{*}\PY{l+m+mi}{2}\PY{p}{)} \PY{o}{\PYZhy{}} \PY{p}{(}\PY{n}{T} \PY{o}{\PYZhy{}} \PY{n}{s}\PY{p}{)}\PY{o}{/}\PY{p}{(}\PY{p}{(}\PY{l+m+mi}{1}\PY{o}{\PYZhy{}}\PY{n}{t}\PY{p}{)}\PY{o}{*}\PY{o}{*}\PY{l+m+mi}{2}\PY{p}{)}
        \PY{n}{information} \PY{o}{=} \PY{n}{T}\PY{o}{/}\PY{p}{(}\PY{n}{t} \PY{o}{*} \PY{p}{(}\PY{l+m+mi}{1}\PY{o}{\PYZhy{}}\PY{n}{t}\PY{p}{)}\PY{p}{)}

        \PY{c+c1}{\PYZsh{} 利用sympy.lambdify 将符号方程转换为numpy.lambda}
        \PY{c+c1}{\PYZsh{} sympy.utilities.lambdify.implemented\PYZus{}function(symfunc, implementation)}
        \PY{c+c1}{\PYZsh{} http://docs.sympy.org/latest/modules/utilities/lambdify.html}
        \PY{n}{\PYZus{}likelihood} \PY{o}{=} \PY{n}{sp}\PY{o}{.}\PY{n}{lambdify}\PY{p}{(}\PY{p}{(}\PY{n}{t}\PY{p}{,}\PY{n}{T}\PY{p}{,}\PY{n}{s}\PY{p}{)}\PY{p}{,} \PY{n}{likelihood}\PY{p}{,} \PY{n}{modules}\PY{o}{=}\PY{l+s+s1}{\PYZsq{}}\PY{l+s+s1}{numpy}\PY{l+s+s1}{\PYZsq{}}\PY{p}{)}
        \PY{n}{\PYZus{}loglike} \PY{o}{=} \PY{n}{sp}\PY{o}{.}\PY{n}{lambdify}\PY{p}{(}\PY{p}{(}\PY{n}{t}\PY{p}{,}\PY{n}{T}\PY{p}{,}\PY{n}{s}\PY{p}{)}\PY{p}{,} \PY{n}{loglike}\PY{p}{,} \PY{n}{modules}\PY{o}{=}\PY{l+s+s1}{\PYZsq{}}\PY{l+s+s1}{numpy}\PY{l+s+s1}{\PYZsq{}}\PY{p}{)}
        \PY{n}{\PYZus{}score} \PY{o}{=} \PY{n}{sp}\PY{o}{.}\PY{n}{lambdify}\PY{p}{(}\PY{p}{(}\PY{n}{t}\PY{p}{,}\PY{n}{T}\PY{p}{,}\PY{n}{s}\PY{p}{)}\PY{p}{,} \PY{n}{score}\PY{p}{,} \PY{n}{modules}\PY{o}{=}\PY{l+s+s1}{\PYZsq{}}\PY{l+s+s1}{numpy}\PY{l+s+s1}{\PYZsq{}}\PY{p}{)}
        \PY{n}{\PYZus{}hessian} \PY{o}{=} \PY{n}{sp}\PY{o}{.}\PY{n}{lambdify}\PY{p}{(}\PY{p}{(}\PY{n}{t}\PY{p}{,}\PY{n}{T}\PY{p}{,}\PY{n}{s}\PY{p}{)}\PY{p}{,} \PY{n}{hessian}\PY{p}{,} \PY{n}{modules}\PY{o}{=}\PY{l+s+s1}{\PYZsq{}}\PY{l+s+s1}{numpy}\PY{l+s+s1}{\PYZsq{}}\PY{p}{)}
\end{Verbatim}

对数似然方程\(\log L(\theta;Y)\)对\(\theta\)求偏导,值为\(0\)处对应的\(\theta\)值定义为\(\widehat{\theta}_{MLE}\)

\begin{equation}
\begin{split}
& \left. \frac{\partial \log L(\theta; Y)}{\partial \theta} \right|_{\widehat{\theta}_{MLE}} = 0, \\
\hookrightarrow & \widehat{\theta}_{MLE} = \frac{s}{T}.
\end{split}
\end{equation}

对应地

\begin{equation}
H \left( \widehat{\theta}_{MLE}\right) = - \frac{T^{2}}{s} - \frac{T^{2}}{T-s} <0, \quad \Leftarrow T > s,
\end{equation}

二阶导数小于\(0\)确保最大似然条件成立。

    \begin{Verbatim}[commandchars=\\\{\}]
{\color{incolor}In [{\color{incolor}6}]:} \PY{c+c1}{\PYZsh{} 计算极大似然估计系数\PYZdl{}\PYZbs{}theta\PYZdl{}的解析值}

        \PY{n}{theta\PYZus{}hat\PYZus{}analytic} \PY{o}{=} \PY{n}{Y}\PY{o}{.}\PY{n}{sum}\PY{p}{(}\PY{p}{)} \PY{o}{/} \PY{n}{nobs}
        \PY{n}{var\PYZus{}analytic} \PY{o}{=} \PY{p}{(}\PY{n}{theta\PYZus{}hat\PYZus{}analytic} \PY{o}{*} \PY{p}{(}\PY{l+m+mi}{1} \PY{o}{\PYZhy{}} \PY{n}{theta\PYZus{}hat\PYZus{}analytic}\PY{p}{)}\PY{p}{)} \PY{o}{/} \PY{n}{nobs}

        \PY{n+nb}{print}\PY{p}{(}\PY{l+s+s1}{\PYZsq{}}\PY{l+s+s1}{Analytic MLE Results: }\PY{l+s+si}{\PYZpc{}.2f}\PY{l+s+s1}{ (}\PY{l+s+si}{\PYZpc{}.10f}\PY{l+s+s1}{)}\PY{l+s+s1}{\PYZsq{}} \PY{o}{\PYZpc{}} \PY{p}{(}\PY{n}{theta\PYZus{}hat\PYZus{}analytic}\PY{p}{,} \PY{n}{var\PYZus{}analytic} \PY{o}{*}\PY{o}{*} \PY{l+m+mf}{0.5}\PY{p}{)}\PY{p}{)}
\end{Verbatim}

    \begin{Verbatim}[commandchars=\\\{\}]
Analytic MLE Results: 0.22 (0.0414246304)

    \end{Verbatim}

\subsection{最大似然估计:数值法}\label{ux6700ux5927ux4f3cux7136ux4f30ux8ba1ux6570ux503cux6cd5}
在很多情况下,我们无法通过解析法求得最大化的解。这时需要数值近似求解,利用\href{https://docs.scipy.org/doc/scipy/reference/optimize.html}{scipy.optimize}最优算法。以下几点需要注意。

\begin{itemize}
\tightlist
\item
  对似然方程的负数求最小化,达到最大化的效果。
\item
  最优算法常常是在不受约束的参数空间\(\mathbb{R}^{k}\)中进行的(\(k\)表示参数数量即维度)。如果实际研究对象的参数值是受限的,如本例\(\theta \in (0,1)\),则需要将受限参数首先转换为无约束空间中的参数\((0,1) \ni \theta \rightarrow \varphi \in \mathbb{R}\),然后使用最优算法求解。对应的转换矩阵

  \begin{equation}
  \theta = g(\varphi) = \frac{\exp(\varphi)}{ 1 + \exp(\varphi)}, \quad \varphi \in \mathbb{R}.
  \end{equation}
\item
  只有得到了黑塞矩阵,根据逆黑塞矩阵的负数才能计算方差协方差矩阵。而黑塞矩阵的解析形往往无法求得------可以利用\href{https://docs.scipy.org/doc/scipy/reference/generated/scipy.optimize.fmin_bfgs.html}{scipy.optimize.fmin\_bfgs}数值近似黑塞矩阵。
\item
  利用数值近似黑塞矩阵算得的方差协方差矩阵\(Cov(\varphi)\),需要转换为\(Cov(\theta)\)

  \begin{equation}
  Cov(\theta) = \left. \frac{\partial g(\varphi)}{\partial \varphi} \right|_{\varphi_{MLE}} \times Cov(\varphi) \times \left. \frac{\partial g(\varphi)}{\partial \varphi}^{\top} \right|_{\varphi_{MLE}}
  \end{equation}
\end{itemize}

转换矩阵的偏导数(斜率)为

\begin{equation}
\frac{\partial g(\varphi)}{\partial \varphi} = \frac{
\partial \left[ \exp(\varphi) \, \left( 1 + \exp(\varphi) \right)^{-1} \right]
}{\partial \varphi} = \frac{\exp(\varphi)}{\left( 1 + \exp(\varphi) \right)^{2}}
\end{equation}

\subsection{\texorpdfstring{无约束参数\(g(\varphi)\)构成的模型系统}{无约束参数g(\textbackslash{}varphi)构成的模型系统}}\label{ux65e0ux7ea6ux675fux53c2ux6570gvarphiux6784ux6210ux7684ux6a21ux578bux7cfbux7edf}

\begin{itemize}
\tightlist
\item
  逆转换方程

  \begin{equation}
  \begin{split}
  \theta & = g(\varphi) = \frac{\exp(\varphi)}{1+\exp(\varphi)}, \\
  1+\exp(\varphi) \, \theta & = \exp(\varphi), \\
  \theta & = \exp(\varphi) \, (1 - \theta), \\
  \varphi & = \log \left( \frac{\theta}{1 - \theta} \right) = g^{-1}(\theta)
  \end{split}
  \end{equation}
\item
  似然方程\(L(\theta ; Y)\)

  \begin{equation}
  L(g(\varphi) ; Y) = g(\varphi)^{s} \, \left( 1 - g(\varphi) \right)^{T-s}
  \end{equation}
\item
  对数似然方程\(\log (g(\varphi) ; Y)\)

  \begin{equation}
  \log L(g(\varphi) ; Y) = s \log g(\varphi) + \left( T - s \right) \log \left( 1 - g(\varphi) \right)
  \end{equation}
\item
  score矩阵 (一阶导数)

  \begin{equation}
  \begin{split}
  \frac{
  \partial \log L(g(\varphi) ; Y)
  }{
  g(\varphi)
  }
  & = \frac{
  \partial \log L(g(\varphi) ; Y)
  }{
  g(\varphi)
  } \,
  \frac{\partial g(\varphi)}{\partial \varphi} \\
  & = \left(\frac{s}{g(\varphi)} - \frac{T-s}{1-g(\varphi)} \right)
  \frac{\exp(\varphi)}{\left( 1 + \exp(\varphi) \right)^{2}}\\
  & = \left(\frac{s}{ \frac{\exp(\varphi)}{1+\exp(\varphi)} } - \frac{T-s}{1-\frac{\exp(\varphi)}{1+\exp(\varphi)}} \right)
  \frac{\exp(\varphi)}{\left( 1 + \exp(\varphi) \right)^{2}} \\
  & = s - T \, \frac{\exp(\varphi)}{1+\exp(\varphi)} = s - T \, g(\varphi).
  \end{split}
  \end{equation}
\item
  黑塞矩阵 (二阶导数)

  \begin{equation}
  H(\varphi) = \frac{
  \partial^{2} \log L(g(\varphi) ; Y)
  }{
  \partial g(\varphi)^{2}
  } = - T.
  \end{equation}
\item
  信息矩阵

  \begin{equation}
  I[\varphi]  = - E [H(\varphi)] = T.
  \end{equation}
\item
  方差协方差矩阵

  \begin{equation}
  I[\varphi]^{-1} = \frac{1}{T}.
  \end{equation}
\end{itemize}

\subsubsection{Delta近似法}\label{deltaux8fd1ux4f3cux6cd5}
进而我们有Delta近似法

\begin{equation}
  \begin{split}
    I_{Delta}[\widehat{\theta}_{MLE}]^{-1}
& = \left. \frac{\partial g(\varphi)}{\partial \varphi} \right|_{\varphi_{MLE}}
\times I[\varphi]^{-1} \times
\left. \frac{\partial g(\varphi)}{\partial \varphi}^{\top} \right|_{\varphi_{MLE}} \\
& = \left\{
\frac{
\exp \left[ \log \left( \frac{\widehat{\theta}_{MLE}}{1 - \widehat{\theta}_{MLE}} \right) \right]
}{
\left(
1 + \exp \left[ \log \left( \frac{\widehat{\theta}_{MLE}}{1 - \widehat{\theta}_{MLE}} \right) \right]
\right)^{2}
}
\right\}^{2} \frac{1}{T}
 = \left[
\left( \frac{\widehat{\theta}_{MLE}}{1 - \widehat{\theta}_{MLE}} \right)
\,
\left( 1 + \frac{\widehat{\theta}_{MLE}}{1 - \widehat{\theta}_{MLE}} \right)^{-2}
\right]^{2} \, \frac{1}{T} \\
& = \left[
\frac{\widehat{\theta}_{MLE}}{1 - \widehat{\theta}_{MLE}}
\,
\left(1 - \widehat{\theta}_{MLE} \right)^{2}
\right]^{2} \, \frac{1}{T}
= \left[ \widehat{\theta}_{MLE} \, \left( 1 - \widehat{\theta}_{MLE} \right) \right]^{2} \, \frac{1}{T} \\
& = I \left[ \widehat{\theta}_{MLE} \right]^{-1} \,
\left[
\widehat{\theta}_{MLE} \, \left( 1 - \widehat{\theta}_{MLE} \right)
\right]
  \end{split}
\end{equation}

Delta近似容易导致方差协方差矩阵被低估。可以写出误差项

\begin{equation}
I \left[ \theta \right]^{-1} - I_{Delta} \left[ \theta \right]^{-1} =
\frac{\theta \left( 1 - \theta \right) - \left[ \theta \left( 1 - \theta \right) \right]^{2}}{T},
\end{equation}

或者误差项的偏移(相对误差)

\begin{equation}
\frac{
\left| I \left[ \theta \right]^{-1} - I_{Delta} \left[ \theta \right]^{-1} \right|
}{
I \left[ \theta \right]^{-1}
} =
\frac{
\frac{\theta \left( 1 - \theta \right) - \left[ \theta \left( 1 - \theta \right) \right]^{2}}{T}
}{
\frac{\theta \, \left( 1 - \theta \right)}{T}
} = 1 - \theta \left( 1 - \theta \right)
\end{equation}

    \begin{Verbatim}[commandchars=\\\{\}]
{\color{incolor}In [{\color{incolor}7}]:} \PY{n}{fig} \PY{o}{=} \PY{n}{plt}\PY{o}{.}\PY{n}{figure}\PY{p}{(}\PY{n}{figsize} \PY{o}{=} \PY{p}{(}\PY{l+m+mi}{12}\PY{p}{,}\PY{l+m+mi}{4}\PY{p}{)}\PY{p}{)}

        \PY{c+c1}{\PYZsh{} 误差项}
        \PY{n}{ax} \PY{o}{=} \PY{n}{fig}\PY{o}{.}\PY{n}{add\PYZus{}subplot}\PY{p}{(}\PY{l+m+mi}{121}\PY{p}{,} \PY{n}{projection}\PY{o}{=}\PY{l+s+s2}{\PYZdq{}}\PY{l+s+s2}{3d}\PY{l+s+s2}{\PYZdq{}}\PY{p}{)}
        \PY{n}{delta} \PY{o}{=} \PY{l+m+mf}{0.025}
        \PY{n}{T\PYZus{}range} \PY{o}{=} \PY{n}{np}\PY{o}{.}\PY{n}{arange}\PY{p}{(}\PY{l+m+mi}{0}\PY{p}{,}\PY{l+m+mi}{100}\PY{p}{,}\PY{l+m+mi}{10}\PY{p}{)}
        \PY{c+c1}{\PYZsh{} numpy.linspace(start, stop, num=50, endpoint=True, retstep=False, dtype=None)[source]}
        \PY{c+c1}{\PYZsh{} https://docs.scipy.org/doc/numpy\PYZhy{}1.13.0/reference/generated/numpy.linspace.html}
        \PY{n}{TH\PYZus{}range} \PY{o}{=} \PY{n}{np}\PY{o}{.}\PY{n}{linspace}\PY{p}{(}\PY{l+m+mi}{0}\PY{p}{,}\PY{l+m+mi}{1}\PY{p}{)}
        \PY{c+c1}{\PYZsh{} numpy.meshgrid(*xi, **kwargs)[source]}
        \PY{c+c1}{\PYZsh{} https://docs.scipy.org/doc/numpy\PYZhy{}1.13.0/reference/generated/numpy.meshgrid.html}
        \PY{c+c1}{\PYZsh{} [X,Y] = meshgrid(x,y) 将向量x和y定义的区域转换成矩阵X和Y,}
        \PY{c+c1}{\PYZsh{} 这两个矩阵可以用来表示mesh和surf的三维空间点以及两个变量的赋值。}
        \PY{c+c1}{\PYZsh{} 其中矩阵X的行向量是向量x的简单复制,而矩阵Y的列向量是向量y的简单复制。}
        \PY{n}{T\PYZus{}grid}\PY{p}{,} \PY{n}{TH\PYZus{}grid} \PY{o}{=} \PY{n}{np}\PY{o}{.}\PY{n}{meshgrid}\PY{p}{(}\PY{n}{T\PYZus{}range}\PY{p}{,} \PY{n}{TH\PYZus{}range}\PY{p}{)}
        \PY{c+c1}{\PYZsh{} 误差项的测算}
        \PY{n}{E\PYZus{}grid} \PY{o}{=} \PY{p}{(}\PY{n}{TH\PYZus{}grid} \PY{o}{*} \PY{p}{(}\PY{l+m+mi}{1} \PY{o}{\PYZhy{}} \PY{n}{TH\PYZus{}grid}\PY{p}{)} \PY{o}{\PYZhy{}} \PY{p}{(}\PY{n}{TH\PYZus{}grid} \PY{o}{*} \PY{p}{(}\PY{l+m+mi}{1}\PY{o}{\PYZhy{}}\PY{n}{TH\PYZus{}grid}\PY{p}{)}\PY{p}{)} \PY{o}{*}\PY{o}{*} \PY{l+m+mi}{2}\PY{p}{)}\PY{o}{/}\PY{n}{T\PYZus{}grid}
        \PY{n}{ax}\PY{o}{.}\PY{n}{plot\PYZus{}wireframe}\PY{p}{(}\PY{n}{T\PYZus{}grid}\PY{p}{,} \PY{n}{TH\PYZus{}grid}\PY{p}{,} \PY{n}{E\PYZus{}grid}\PY{p}{,} \PY{n}{color}\PY{o}{=}\PY{l+s+s2}{\PYZdq{}}\PY{l+s+s2}{\PYZsh{}348ABD}\PY{l+s+s2}{\PYZdq{}}\PY{p}{)}
        \PY{n}{ax}\PY{o}{.}\PY{n}{set}\PY{p}{(}
            \PY{n}{title} \PY{o}{=} \PY{l+s+s2}{\PYZdq{}}\PY{l+s+s2}{Absolute Error from the Delta Method}\PY{l+s+s2}{\PYZdq{}}\PY{p}{,}
            \PY{n}{xlabel}\PY{o}{=} \PY{l+s+s2}{\PYZdq{}}\PY{l+s+s2}{T}\PY{l+s+s2}{\PYZdq{}}\PY{p}{,}
            \PY{n}{ylabel}\PY{o}{=} \PY{l+s+sa}{r}\PY{l+s+s2}{\PYZdq{}}\PY{l+s+s2}{\PYZdl{}}\PY{l+s+s2}{\PYZbs{}}\PY{l+s+s2}{theta\PYZdl{}}\PY{l+s+s2}{\PYZdq{}}
        \PY{p}{)}
        \PY{n}{ax}\PY{o}{.}\PY{n}{xaxis}\PY{o}{.}\PY{n}{set}\PY{p}{(}\PY{n}{pane\PYZus{}color} \PY{o}{=} \PY{p}{(}\PY{l+m+mi}{1}\PY{p}{,}\PY{l+m+mi}{1}\PY{p}{,}\PY{l+m+mi}{1}\PY{p}{)}\PY{p}{)}
        \PY{n}{ax}\PY{o}{.}\PY{n}{yaxis}\PY{o}{.}\PY{n}{set}\PY{p}{(}\PY{n}{pane\PYZus{}color} \PY{o}{=} \PY{p}{(}\PY{l+m+mi}{1}\PY{p}{,}\PY{l+m+mi}{1}\PY{p}{,}\PY{l+m+mi}{1}\PY{p}{)}\PY{p}{)}

        \PY{c+c1}{\PYZsh{} 误差项的偏移(相对误差)}
        \PY{n}{ax} \PY{o}{=} \PY{n}{fig}\PY{o}{.}\PY{n}{add\PYZus{}subplot}\PY{p}{(}\PY{l+m+mi}{122}\PY{p}{)}
        \PY{n}{X} \PY{o}{=} \PY{n}{np}\PY{o}{.}\PY{n}{linspace}\PY{p}{(}\PY{l+m+mi}{0}\PY{p}{,}\PY{l+m+mi}{1}\PY{p}{)}
        \PY{n}{ax}\PY{o}{.}\PY{n}{plot}\PY{p}{(}\PY{n}{X}\PY{p}{,} \PY{l+m+mi}{1} \PY{o}{\PYZhy{}} \PY{n}{X} \PY{o}{*} \PY{p}{(}\PY{l+m+mi}{1}\PY{o}{\PYZhy{}}\PY{n}{X}\PY{p}{)}\PY{p}{)}
        \PY{n}{ax}\PY{o}{.}\PY{n}{set}\PY{p}{(}
            \PY{n}{title} \PY{o}{=} \PY{l+s+s2}{\PYZdq{}}\PY{l+s+s2}{Relative Error from the Delta Method}\PY{l+s+s2}{\PYZdq{}}\PY{p}{,}
            \PY{n}{xlabel} \PY{o}{=} \PY{l+s+sa}{r}\PY{l+s+s2}{\PYZdq{}}\PY{l+s+s2}{\PYZdl{}}\PY{l+s+s2}{\PYZbs{}}\PY{l+s+s2}{theta\PYZdl{}}\PY{l+s+s2}{\PYZdq{}}\PY{p}{,}
            \PY{n}{ylim} \PY{o}{=} \PY{p}{(}\PY{l+m+mf}{0.7}\PY{p}{,} \PY{l+m+mi}{1}\PY{p}{)}
        \PY{p}{)}
        \PY{n}{ax}\PY{o}{.}\PY{n}{vlines}\PY{p}{(}\PY{n}{theta\PYZus{}hat\PYZus{}analytic}\PY{p}{,} \PY{l+m+mf}{0.7}\PY{p}{,} \PY{l+m+mi}{1}\PY{p}{,} \PY{l+s+s2}{\PYZdq{}}\PY{l+s+s2}{r}\PY{l+s+s2}{\PYZdq{}}\PY{p}{)}
\end{Verbatim}

            \begin{Verbatim}[commandchars=\\\{\}]
{\color{outcolor}Out[{\color{outcolor}7}]:} <matplotlib.collections.LineCollection at 0x125ab96a0>
\end{Verbatim}

    \begin{center}
    \adjustimage{max size={0.9\linewidth}{0.9\paperheight}}{./Figures/output_12_1.png}
    \end{center}
    { \hspace*{\fill} \\}

    \begin{itemize}
\tightlist
\item
  左图可以看出,绝对误差项随着样本数量\(T\)的增加而逐渐增大,相对误差的变化则与\(T\)无关。
\item
  右图可以看出,在最大似然估计的解析解(\(\theta = 0.22\))处,相对误差的值约为\(82\%\)。
\end{itemize}

\subsubsection{最大似然估计:算法举例}
\label{ux6700ux5927ux4f3cux7136ux4f30ux8ba1ux7b97ux6cd5ux4e3eux4f8b}
\begin{Verbatim}[commandchars=\\\{\}]
{\color{incolor}In [{\color{incolor}8}]:} \PY{c+c1}{\PYZsh{} 载入}
    \PY{k+kn}{from} \PY{n+nn}{scipy} \PY{k}{import} \PY{n}{optimize}

    \PY{c+c1}{\PYZsh{} 构建转换方程}
    \PY{n}{transform} \PY{o}{=} \PY{k}{lambda} \PY{n}{phi}\PY{p}{:} \PY{n}{np}\PY{o}{.}\PY{n}{exp}\PY{p}{(}\PY{n}{phi}\PY{p}{)} \PY{o}{/} \PY{p}{(}\PY{l+m+mi}{1} \PY{o}{+} \PY{n}{np}\PY{o}{.}\PY{n}{exp}\PY{p}{(}\PY{n}{phi}\PY{p}{)}\PY{p}{)}
    \PY{c+c1}{\PYZsh{}\PYZsh{} 转换方程的斜率(一阶导数)}
    \PY{n}{transform\PYZus{}grad} \PY{o}{=} \PY{k}{lambda} \PY{n}{phi}\PY{p}{:} \PY{n}{np}\PY{o}{.}\PY{n}{exp}\PY{p}{(}\PY{n}{phi}\PY{p}{)} \PY{o}{/} \PY{p}{(}\PY{p}{(}\PY{l+m+mi}{1} \PY{o}{+} \PY{n}{np}\PY{o}{.}\PY{n}{exp}\PY{p}{(}\PY{n}{phi}\PY{p}{)}\PY{p}{)} \PY{o}{*}\PY{o}{*}\PY{l+m+mi}{2}\PY{p}{)}

    \PY{c+c1}{\PYZsh{} 构建优化所调用的方程。为负,是因为我们使用最小化的算法。}
    \PY{n}{f} \PY{o}{=} \PY{k}{lambda} \PY{n}{params}\PY{p}{:} \PY{o}{\PYZhy{}}\PY{n}{\PYZus{}loglike}\PY{p}{(}
        \PY{n}{transform}\PY{p}{(}\PY{n}{params}\PY{p}{[}\PY{l+m+mi}{0}\PY{p}{]}\PY{p}{)}\PY{p}{,} \PY{n}{nobs}\PY{p}{,} \PY{n}{Y}\PY{o}{.}\PY{n}{sum}\PY{p}{(}\PY{p}{)}
    \PY{p}{)}
    \PY{n}{fprime} \PY{o}{=} \PY{k}{lambda} \PY{n}{params}\PY{p}{:} \PY{n}{np}\PY{o}{.}\PY{n}{matrix}\PY{p}{(}
        \PY{o}{\PYZhy{}}\PY{n}{\PYZus{}score}\PY{p}{(}
            \PY{n}{transform}\PY{p}{(}\PY{n}{params}\PY{p}{[}\PY{l+m+mi}{0}\PY{p}{]}\PY{p}{)}\PY{p}{,} \PY{n}{nobs}\PY{p}{,} \PY{n}{Y}\PY{o}{.}\PY{n}{sum}\PY{p}{(}\PY{p}{)}
        \PY{p}{)}
    \PY{p}{)}
    \PY{n}{fhess} \PY{o}{=} \PY{k}{lambda} \PY{n}{params}\PY{p}{:} \PY{n}{np}\PY{o}{.}\PY{n}{matrix}\PY{p}{(}
        \PY{o}{\PYZhy{}}\PY{n}{\PYZus{}hessian}\PY{p}{(}
            \PY{n}{transform}\PY{p}{(}\PY{n}{params}\PY{p}{[}\PY{l+m+mi}{0}\PY{p}{]}\PY{p}{)}\PY{p}{,} \PY{n}{nobs}\PY{p}{,} \PY{n}{Y}\PY{o}{.}\PY{n}{sum}\PY{p}{(}\PY{p}{)}
        \PY{p}{)}
    \PY{p}{)}
\end{Verbatim}

scipy.optimize可以使用一系列算法,这里以两种方法为例作简要介绍。

第一种方法叫做
\href{https://en.wikipedia.org/wiki/Nonlinear_conjugate_gradient_method}{牛顿共轭梯度法(Newton
Conjugate Gradient Method, Newton-CG method)}\index{Newton
Conjugate Gradient Method (Newton-CG method) \dotfill 牛顿共轭梯度法}。

\begin{Verbatim}[commandchars=\\\{\}]
{\color{incolor}In [{\color{incolor}9}]:} \PY{c+c1}{\PYZsh{} 利用牛顿共轭梯度法作最大似然估计}

    \PY{c+c1}{\PYZsh{}\PYZpc{}timeit res\PYZus{}ncg = optimize.fmin\PYZus{}ncg(f, np.array([0.5]), fprime, fhess=fhess, disp=False)}
    \PY{n+nb}{print} \PY{p}{(}\PY{l+s+s2}{\PYZdq{}}\PY{l+s+s2}{Timing: 10 loops, best of 3: 4.89 ms per loop}\PY{l+s+s2}{\PYZdq{}}\PY{p}{)}

    \PY{n}{res\PYZus{}ncg} \PY{o}{=} \PY{n}{optimize}\PY{o}{.}\PY{n}{fmin\PYZus{}ncg} \PY{p}{(}\PY{n}{f}\PY{p}{,} \PY{n}{np}\PY{o}{.}\PY{n}{array}\PY{p}{(}\PY{p}{[}\PY{l+m+mf}{0.5}\PY{p}{]}\PY{p}{)}\PY{p}{,} \PY{n}{fprime}\PY{p}{,} \PY{n}{fhess} \PY{o}{=} \PY{n}{fhess}\PY{p}{)}\PY{p}{[}\PY{l+m+mi}{0}\PY{p}{]}
    \PY{n}{var\PYZus{}ncg} \PY{o}{=} \PY{n}{np}\PY{o}{.}\PY{n}{linalg}\PY{o}{.}\PY{n}{inv}\PY{p}{(}
        \PY{o}{\PYZhy{}}\PY{n}{\PYZus{}hessian}\PY{p}{(}\PY{n}{transform}\PY{p}{(}\PY{n}{res\PYZus{}ncg}\PY{p}{)}\PY{p}{,} \PY{n}{nobs}\PY{p}{,} \PY{n}{Y}\PY{o}{.}\PY{n}{sum}\PY{p}{(}\PY{p}{)}\PY{p}{)}
    \PY{p}{)}\PY{p}{[}\PY{l+m+mi}{0}\PY{p}{,}\PY{l+m+mi}{0}\PY{p}{]}
    \PY{c+c1}{\PYZsh{}var\PYZus{}ncg = np.linalg.inv(\PYZhy{}\PYZus{}hessian(transform(res\PYZus{}ncg), nobs, Y.sum()))[0,0]}
    \PY{n+nb}{print} \PY{p}{(}\PY{l+s+s2}{\PYZdq{}}\PY{l+s+s2}{Newton\PYZhy{}CG MLE estimate: }\PY{l+s+si}{\PYZpc{}.2f}\PY{l+s+s2}{ (}\PY{l+s+si}{\PYZpc{}.10f}\PY{l+s+s2}{)}\PY{l+s+s2}{\PYZdq{}} \PY{o}{\PYZpc{}} \PY{p}{(}\PY{n}{transform}\PY{p}{(}\PY{n}{res\PYZus{}ncg}\PY{p}{)}\PY{p}{,} \PY{n}{var\PYZus{}ncg}\PY{o}{*}\PY{o}{*}\PY{l+m+mf}{0.5}\PY{p}{)}\PY{p}{)}
\end{Verbatim}

\begin{Verbatim}[commandchars=\\\{\}]
Timing: 10 loops, best of 3: 4.89 ms per loop
Optimization terminated successfully.
     Current function value: 52.690796
     Iterations: 35
     Function evaluations: 51
     Gradient evaluations: 85
     Hessian evaluations: 35
Newton-CG MLE estimate: 0.22 (0.0414252425)

\end{Verbatim}

第二种方法叫做BFMG法\href{https://en.wikipedia.org/wiki/Broyden–Fletcher–Goldfarb–Shanno_algorithm}{(Broyden--Fletcher--Goldfarb--Shanno
algorithm, BFMG method)}\index{BFMG algorithm \dotfill BFMG算法}。

\begin{Verbatim}[commandchars=\\\{\}]
{\color{incolor}In [{\color{incolor}10}]:} \PY{c+c1}{\PYZsh{} 利用BFMG法作最大似然估计}
     \PY{c+c1}{\PYZsh{}\PYZpc{}timeit res\PYZus{}bfgs = optimize.fmin\PYZus{}bfgs(f, np.array([0.5]), fprime=fprime, disp=False)}
     \PY{n+nb}{print}\PY{p}{(}\PY{l+s+s2}{\PYZdq{}}\PY{l+s+s2}{Timing: 1000 loops, best of 3: 1.85 ms per loop}\PY{l+s+s2}{\PYZdq{}}\PY{p}{)}

     \PY{n}{res\PYZus{}bfgs\PYZus{}full} \PY{o}{=} \PY{n}{optimize}\PY{o}{.}\PY{n}{fmin\PYZus{}bfgs}\PY{p}{(}\PY{n}{f}\PY{p}{,} \PY{n}{np}\PY{o}{.}\PY{n}{array}\PY{p}{(}\PY{p}{[}\PY{l+m+mf}{0.5}\PY{p}{]}\PY{p}{)}\PY{p}{,} \PY{n}{fprime} \PY{o}{=} \PY{n}{fprime}\PY{p}{,} \PY{n}{full\PYZus{}output} \PY{o}{=} \PY{k+kc}{True}\PY{p}{)}
     \PY{n}{res\PYZus{}bfgs} \PY{o}{=} \PY{n}{res\PYZus{}bfgs\PYZus{}full}\PY{p}{[}\PY{l+m+mi}{0}\PY{p}{]}
     \PY{n}{est\PYZus{}var\PYZus{}bfgs} \PY{o}{=} \PY{n}{res\PYZus{}bfgs\PYZus{}full}\PY{p}{[}\PY{l+m+mi}{3}\PY{p}{]}

     \PY{c+c1}{\PYZsh{} 计算信息矩阵;计算信息矩阵的逆矩阵,得到的方差协方差矩阵}
     \PY{n}{var\PYZus{}bfgs} \PY{o}{=} \PY{n}{np}\PY{o}{.}\PY{n}{linalg}\PY{o}{.}\PY{n}{inv}\PY{p}{(}
         \PY{o}{\PYZhy{}}\PY{n}{\PYZus{}hessian}\PY{p}{(}\PY{n}{transform}\PY{p}{(}\PY{n}{res\PYZus{}bfgs}\PY{p}{)}\PY{p}{,} \PY{n}{nobs}\PY{p}{,} \PY{n}{Y}\PY{o}{.}\PY{n}{sum}\PY{p}{(}\PY{p}{)}\PY{p}{)}
     \PY{p}{)}\PY{p}{[}\PY{l+m+mi}{0}\PY{p}{,}\PY{l+m+mi}{0}\PY{p}{]}
     \PY{c+c1}{\PYZsh{} 根据(解析)delta方法得到的方差协方差矩阵}
     \PY{n}{var\PYZus{}bfgs\PYZus{}delta} \PY{o}{=} \PY{p}{(}\PY{n}{transform\PYZus{}grad}\PY{p}{(}\PY{n}{res\PYZus{}bfgs}\PY{p}{)}\PY{o}{*}\PY{o}{*}\PY{l+m+mi}{2} \PY{o}{*} \PY{n}{est\PYZus{}var\PYZus{}bfgs}\PY{p}{)}\PY{p}{[}\PY{l+m+mi}{0}\PY{p}{,}\PY{l+m+mi}{0}\PY{p}{]}

     \PY{n+nb}{print}\PY{p}{(}
         \PY{l+s+s2}{\PYZdq{}}\PY{l+s+s2}{BFGS MLE estimate: }\PY{l+s+si}{\PYZpc{}.2f}\PY{l+s+s2}{ (}\PY{l+s+si}{\PYZpc{}.10f}\PY{l+s+s2}{)}\PY{l+s+s2}{\PYZdq{}} \PY{o}{\PYZpc{}} \PY{p}{(}\PY{n}{transform}\PY{p}{(}\PY{n}{res\PYZus{}bfgs}\PY{p}{)}\PY{p}{,} \PY{n}{var\PYZus{}bfgs}\PY{o}{*}\PY{o}{*}\PY{l+m+mf}{0.5}\PY{p}{)}
     \PY{p}{)}
     \PY{n+nb}{print}\PY{p}{(}
         \PY{l+s+s2}{\PYZdq{}}\PY{l+s+s2}{BFGS MLE estimate (delta) : }\PY{l+s+si}{\PYZpc{}.2f}\PY{l+s+s2}{ (}\PY{l+s+si}{\PYZpc{}.10f}\PY{l+s+s2}{)}\PY{l+s+s2}{\PYZdq{}} \PY{o}{\PYZpc{}} \PY{p}{(}\PY{n}{transform}\PY{p}{(}\PY{n}{res\PYZus{}bfgs}\PY{p}{)}\PY{p}{,} \PY{n}{var\PYZus{}bfgs\PYZus{}delta} \PY{o}{*}\PY{o}{*} \PY{l+m+mf}{0.5}\PY{p}{)}
     \PY{p}{)}
\end{Verbatim}

\begin{Verbatim}[commandchars=\\\{\}]
Timing: 1000 loops, best of 3: 1.85 ms per loop
Optimization terminated successfully.
     Current function value: 52.690796
     Iterations: 6
     Function evaluations: 7
     Gradient evaluations: 7
BFGS MLE estimate: 0.22 (0.0414246303)
BFGS MLE estimate (delta) : 0.22 (0.0171611895)

\end{Verbatim}

进而我们有误差项和误差项的偏移

\begin{Verbatim}[commandchars=\\\{\}]
{\color{incolor}In [{\color{incolor}11}]:} \PY{n}{absolute\PYZus{}error} \PY{o}{=} \PY{n}{var\PYZus{}bfgs} \PY{o}{\PYZhy{}} \PY{n}{var\PYZus{}bfgs\PYZus{}delta}
     \PY{n}{relative\PYZus{}error} \PY{o}{=} \PY{p}{(}\PY{n}{var\PYZus{}bfgs} \PY{o}{\PYZhy{}} \PY{n}{var\PYZus{}bfgs\PYZus{}delta}\PY{p}{)} \PY{o}{/} \PY{n}{var\PYZus{}bfgs}

     \PY{n+nb}{print}\PY{p}{(}\PY{n}{absolute\PYZus{}error}\PY{p}{)}
     \PY{n+nb}{print}\PY{p}{(}\PY{n}{relative\PYZus{}error}\PY{p}{)}
\end{Verbatim}

\begin{Verbatim}[commandchars=\\\{\}]
0.00142149357495
0.828376209448

\end{Verbatim}

第三种方法是\href{https://en.wikipedia.org/wiki/Newton\%27s_method}{牛顿拉夫森法(Newton-Raphson
method)}\index{Newton-Raphson Algorithm \dotfill 牛顿拉夫森算法}。比起前两种方法来,牛顿拉夫森法的数值计算速度更快,但稳健性较低。python的 \href{http://www.statsmodels.org/stable/index.html}{statsmodels}包提供了这种算法的程序。

\begin{Verbatim}[commandchars=\\\{\}]
{\color{incolor}In [{\color{incolor}20}]:} \PY{c+c1}{\PYZsh{} 利用BFMG法作最大似然估计}
     \PY{c+c1}{\PYZsh{} 导入Statsmodels包}
     \PY{k+kn}{from} \PY{n+nn}{statsmodels}\PY{n+nn}{.}\PY{n+nn}{base}\PY{n+nn}{.}\PY{n+nn}{model} \PY{k}{import} \PY{n}{LikelihoodModel}

     \PY{k}{class} \PY{n+nc}{Bernoulli}\PY{p}{(}\PY{n}{LikelihoodModel}\PY{p}{)}\PY{p}{:}
         \PY{n}{\PYZus{}loglike} \PY{o}{=} \PY{k}{lambda} \PY{n+nb+bp}{self}\PY{p}{,} \PY{n}{theta}\PY{p}{,} \PY{n}{T}\PY{p}{,} \PY{n}{s}\PY{p}{:} \PY{n}{\PYZus{}loglike}\PY{p}{(}\PY{n}{theta}\PY{p}{,} \PY{n}{T}\PY{p}{,} \PY{n}{s}\PY{p}{)}
         \PY{n}{\PYZus{}score} \PY{o}{=} \PY{k}{lambda} \PY{n+nb+bp}{self}\PY{p}{,} \PY{n}{theta}\PY{p}{,} \PY{n}{T}\PY{p}{,} \PY{n}{s}\PY{p}{:} \PY{n}{\PYZus{}score}\PY{p}{(}\PY{n}{theta}\PY{p}{,} \PY{n}{T}\PY{p}{,} \PY{n}{s}\PY{p}{)}
         \PY{n}{\PYZus{}hessian} \PY{o}{=} \PY{k}{lambda} \PY{n+nb+bp}{self}\PY{p}{,} \PY{n}{theta}\PY{p}{,} \PY{n}{T}\PY{p}{,} \PY{n}{s}\PY{p}{:} \PY{n}{\PYZus{}hessian}\PY{p}{(}\PY{n}{theta}\PY{p}{,} \PY{n}{T}\PY{p}{,} \PY{n}{s}\PY{p}{)}

         \PY{k}{def} \PY{n+nf}{\PYZus{}\PYZus{}init\PYZus{}\PYZus{}}\PY{p}{(}\PY{n+nb+bp}{self}\PY{p}{,} \PY{n}{endog}\PY{p}{,} \PY{n}{exog}\PY{o}{=}\PY{k+kc}{None}\PY{p}{,} \PY{o}{*}\PY{o}{*}\PY{n}{kwargs}\PY{p}{)}\PY{p}{:}
             \PY{c+c1}{\PYZsh{} numpy.asarray(a, dtype=None, order=None)[source]}
             \PY{c+c1}{\PYZsh{} https://docs.scipy.org/doc/numpy\PYZhy{}1.13.0/reference/generated/numpy.asarray.html}
             \PY{n}{endog} \PY{o}{=} \PY{n}{np}\PY{o}{.}\PY{n}{asarray}\PY{p}{(}\PY{n}{endog}\PY{p}{)}
             \PY{n+nb}{super}\PY{p}{(}\PY{n}{Bernoulli}\PY{p}{,} \PY{n+nb+bp}{self}\PY{p}{)}\PY{o}{.}\PY{n+nf+fm}{\PYZus{}\PYZus{}init\PYZus{}\PYZus{}}\PY{p}{(}\PY{n}{endog}\PY{p}{,} \PY{n}{exog}\PY{p}{,} \PY{o}{*}\PY{o}{*}\PY{n}{kwargs}\PY{p}{)}
             \PY{n+nb+bp}{self}\PY{o}{.}\PY{n}{T} \PY{o}{=} \PY{n+nb+bp}{self}\PY{o}{.}\PY{n}{endog}\PY{o}{.}\PY{n}{shape}\PY{p}{[}\PY{l+m+mi}{0}\PY{p}{]}
             \PY{n+nb+bp}{self}\PY{o}{.}\PY{n}{s} \PY{o}{=} \PY{n+nb+bp}{self}\PY{o}{.}\PY{n}{endog}\PY{o}{.}\PY{n}{sum}\PY{p}{(}\PY{p}{)}

         \PY{k}{def} \PY{n+nf}{loglike}\PY{p}{(}\PY{n+nb+bp}{self}\PY{p}{,} \PY{n}{params}\PY{p}{)}\PY{p}{:}
             \PY{l+s+sd}{\PYZdq{}\PYZdq{}\PYZdq{}}
     \PY{l+s+sd}{        Joint log\PYZhy{}likelihood for Bernoulli trials}
     \PY{l+s+sd}{        \PYZdq{}\PYZdq{}\PYZdq{}}
             \PY{k}{return} \PY{n+nb+bp}{self}\PY{o}{.}\PY{n}{\PYZus{}loglike}\PY{p}{(}\PY{n}{params}\PY{p}{[}\PY{l+m+mi}{0}\PY{p}{]}\PY{p}{,} \PY{n+nb+bp}{self}\PY{o}{.}\PY{n}{T}\PY{p}{,} \PY{n+nb+bp}{self}\PY{o}{.}\PY{n}{s}\PY{p}{)}

         \PY{k}{def} \PY{n+nf}{score}\PY{p}{(}\PY{n+nb+bp}{self}\PY{p}{,} \PY{n}{params}\PY{p}{)}\PY{p}{:}
             \PY{l+s+sd}{\PYZdq{}\PYZdq{}\PYZdq{}}
     \PY{l+s+sd}{        Gradient of the joint log\PYZhy{}likelihood for Bernoulli trials}
     \PY{l+s+sd}{        \PYZdq{}\PYZdq{}\PYZdq{}}
             \PY{k}{return} \PY{n+nb+bp}{self}\PY{o}{.}\PY{n}{\PYZus{}score}\PY{p}{(}\PY{n}{params}\PY{p}{[}\PY{l+m+mi}{0}\PY{p}{]}\PY{p}{,} \PY{n+nb+bp}{self}\PY{o}{.}\PY{n}{T}\PY{p}{,} \PY{n+nb+bp}{self}\PY{o}{.}\PY{n}{s}\PY{p}{)}

         \PY{k}{def} \PY{n+nf}{hessian}\PY{p}{(}\PY{n+nb+bp}{self}\PY{p}{,} \PY{n}{params}\PY{p}{)}\PY{p}{:}
             \PY{l+s+sd}{\PYZdq{}\PYZdq{}\PYZdq{}}
     \PY{l+s+sd}{        Hessian of the joint log\PYZhy{}likelihood for Bernoulli trials}
     \PY{l+s+sd}{        \PYZdq{}\PYZdq{}\PYZdq{}}
             \PY{k}{return} \PY{n}{np}\PY{o}{.}\PY{n}{matrix}\PY{p}{(}\PY{n+nb+bp}{self}\PY{o}{.}\PY{n}{\PYZus{}hessian}\PY{p}{(}\PY{n}{params}\PY{p}{[}\PY{l+m+mi}{0}\PY{p}{]}\PY{p}{,} \PY{n+nb+bp}{self}\PY{o}{.}\PY{n}{T}\PY{p}{,} \PY{n+nb+bp}{self}\PY{o}{.}\PY{n}{s}\PY{p}{)}\PY{p}{)}
\end{Verbatim}

\begin{Verbatim}[commandchars=\\\{\}]
{\color{incolor}In [{\color{incolor}26}]:} \PY{n}{mod} \PY{o}{=} \PY{n}{Bernoulli}\PY{p}{(}\PY{n}{Y}\PY{p}{)}
     \PY{c+c1}{\PYZsh{}\PYZpc{}timeit res = mod.fit(start\PYZus{}params=[0.5], disp=False)}
     \PY{n+nb}{print}\PY{p}{(}\PY{l+s+s2}{\PYZdq{}}\PY{l+s+s2}{1000 loops, best of 3: 277 µs per loop}\PY{l+s+s2}{\PYZdq{}}\PY{p}{)}
     \PY{c+c1}{\PYZsh{} 最大似然估计}
     \PY{n}{res} \PY{o}{=} \PY{n}{mod}\PY{o}{.}\PY{n}{fit}\PY{p}{(}\PY{n}{start\PYZus{}params}\PY{o}{=}\PY{p}{[}\PY{l+m+mf}{0.5}\PY{p}{]}\PY{p}{)}
     \PY{n+nb}{print} \PY{p}{(}\PY{l+s+s2}{\PYZdq{}}\PY{l+s+s2}{Statsmodels MLE Estimate: }\PY{l+s+si}{\PYZpc{}.2f}\PY{l+s+s2}{ (}\PY{l+s+si}{\PYZpc{}.10f}\PY{l+s+s2}{)}\PY{l+s+s2}{\PYZdq{}} \PY{o}{\PYZpc{}} \PY{p}{(}\PY{n}{res}\PY{o}{.}\PY{n}{params}\PY{p}{[}\PY{l+m+mi}{0}\PY{p}{,}\PY{l+m+mi}{0}\PY{p}{]}\PY{p}{,} \PY{n}{res}\PY{o}{.}\PY{n}{bse}\PY{p}{[}\PY{l+m+mi}{0}\PY{p}{]}\PY{p}{)}\PY{p}{)}
\end{Verbatim}

\begin{Verbatim}[commandchars=\\\{\}]
1000 loops, best of 3: 277 µs per loop
Optimization terminated successfully.
     Current function value: 0.526908
     Iterations 2
Statsmodels MLE Estimate: 0.22 (0.0414246304)

\end{Verbatim}
