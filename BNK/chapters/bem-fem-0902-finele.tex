%!TEX root = ../DSGEnotes.tex
\subsection{形式方程}
\label{sec:finele-form}
基于分解元$\mathcal{T}_{N}$\eqref{eq:finele-ref-decomposition},现在来分析检测空间(trial spaces)\index{trial space \dotfill 检测空间}。检测空间由分段多项式构成,这些多项式组成不同形式的基方程(base functions)\index{base function \dotfill 基方程}。基方程与全局自由度相关,不过更是在局部定义的:是针对有限元$\tau_{\ell}$,通过选取合适的形式方程(form function)\index{form function \dotfill 形式方程}而作局部定义。

考虑这样一个参考元$\tau$,它可以是一个区间($d=1$),三角形($d=2$)或是四面体($d=3$)。

\subsubsection{常数型形式方程}
最简单的形式方程可以是个常数
\begin{equation*}
  \psi_{1}^{0}(\xi) = 1, \quad \xi \in \tau,
\end{equation*}

对应地,如果有限元$\tau_{\ell}$中的某个方程$\nu_{h}(x), \, x \in \tau_{\ell}$是个常数,那么$\nu$的表现形式可写为
\begin{equation*}
  \nu_{h}(x) = \nu_{h} \left( x_{\ell_{1}} + J_{\ell} \xi \right)
  =\nu_{\ell} \, \psi_{1}^{0}(\xi), \quad x \in \tau_{\ell}, \, \xi \in \tau,
\end{equation*}
其中系数$\nu_{\ell}$反映$\nu_{h} \in \tau_{\ell}$在参考元$\tau$中对应的值。由此可得
\begin{equation}
  \label{eq:finele-form-constant-norm}
  \left\| \nu_{h} \right\|_{L^{2}(\tau_{\ell})}^{2} = \Delta_{\ell} \nu_{\ell}^{2}.
\end{equation}

\subsubsection{线性形式方程}
如果$\nu_{h}(x), \, x \in \tau_{\ell}$是个线性方程,那么$\nu_{h}$可由参考元$\tau$中结点的值$\widetilde{\nu}_{k}$所唯一决定
\begin{equation}
  \label{eq:finele-form-lin-nuh-nuk}
  \widetilde{\nu}_{h}\left( \xi \right)
  = \sum_{k=1}^{d+1} \widetilde{\nu}_{k} \psi_{k}^{1}\left( \xi \right), \quad \xi \in \tau,
\end{equation}
其中$\psi_{k}^{1}$的值如下
\begin{itemize}
  \item $d=1$时结点数$k=2$
  \begin{equation*}
    \begin{cases}
      \psi_{1}^{1} & \coloneqq 1 - \xi,\\
      \psi_{2}^{1} & \coloneqq \xi.
    \end{cases}
  \end{equation*}
  \item $d=2$时节点数$k=3$
  \begin{equation*}
    \begin{cases}
      \psi_{1}^{1} & \coloneqq 1 - \xi_{1} - \xi_{2}, \\
      \psi_{2}^{1} & \coloneqq \xi_{1}, \\
      \psi_{3}^{1} & \coloneqq \xi_{2}.
    \end{cases}
  \end{equation*}
  \item $d=3$时节点数$k=4$
  \begin{equation*}
    \begin{cases}
      \psi_{1}^{1} & \coloneqq 1 - \xi_{1} - \xi_{2} - \xi_{3}, \\
      \psi_{2}^{1} & \coloneqq \xi_{1}, \\
      \psi_{3}^{1} & \coloneqq \xi_{2}, \\
      \psi_{4}^{1} & \coloneqq \xi_{3}.
    \end{cases}
  \end{equation*}
\end{itemize}

这样一来我们有:设任意一个有限元$\tau_{\ell}$,对应结点$x_{\ell_{k}}, \, \ell_{k} \in J(\ell)$。$\tau_{\ell}$中的线性方程$\nu_{h}(x), \, x \in \tau_{\ell}$可以写为
\begin{equation}
  \label{eq:finele-form-lin-nuh-reprensentation}
  \nu_{h} \left( x \right)
  = \nu_{h} \left( x_{\ell_{1}} + J_{\ell} \xi \right)
  = \sum_{k=1}^{d+1} \nu_{\ell_{k}} \psi_{k}^{1} \left( \xi \right), \quad x \in \tau_{\ell}, \xi \in \tau.
\end{equation}

与常数形式方程的范数\eqref{eq:finele-form-constant-norm}类似,线性形式方程$\nu_{h}(x), \, x \in \tau_{\ell}$的范数$\left\| \nu_{h} \right\|_{L^{2}} \left(\tau_{\ell} \right)$也可以用结点的值$\nu_{\ell}(\xi), \, \xi \in \tau$来表示,见如下引理。
\begin{lemma}[线性形式方程的范数]
  \label{lemma:finele-form-lin-norm-def}
  设线性方程$\nu_{h}(x), \, x \in \tau_{\ell}$如\eqref{eq:finele-form-lin-nuh-reprensentation}所示。那么我们有范数不等式关系
  \begin{equation}
    \label{eq:finele-form-lin-norm-def}
    \frac{\Delta_{\ell}}{\left( d+1 \right) \left( d+2 \right)}
    \sum_{k=1}^{d+1} \nu_{\ell_{k}}^{2}
    \le \left\| \nu_{h} \right\|_{L^{2}(\tau_{\ell})}^{2}
    \le \frac{\Delta_{\ell}}{\left( d+1 \right)}
    \sum_{k=1}^{d+1} \nu_{\ell_{k}}^{2}
  \end{equation}
\end{lemma}
\begin{proof}
  线性方程$\nu_{h}(x) \in L^{2}(\tau_{\ell})$的范数
  \begin{equation*}
    \begin{split}
      \left\| \nu_{h} \right\|_{L^{2}(\tau_{\ell})}^{2}
      & = \langle \nu_{h}, \nu_{h} \rangle_{L^{2}(\tau_{\ell})} \\
      & = \sum_{i=1}^{d+1} \sum_{j=1}^{d+1} \nu_{i} \nu_{j}
      \int_{\tau} \psi_{i}\left( \xi \right) \psi_{j}\left( \xi \right) \left| \det J_{\ell} \right| \, d \xi \\
      & = \left( G_{\ell} \, \underline{\nu}^{\ell}, \underline{\nu}^{\ell} \right),
    \end{split}
  \end{equation*}
  其中$G_{\ell}$是局部质量矩阵(local mass matrix)\index{mass matrix!local \dotfill 局部质量矩阵}
  \begin{equation*}
    G_{\ell} = \frac{
    \Delta_{\ell}
    }{
    \left( d + 1 \right)\left( d + 2 \right)}
    \underbrace{
    \left(
    I_{d+1} + \underline{e}_{d+1} \underline{e}_{d+1}^{\top}
    \right)
    }_{\eqqcolon \mathcal{A}}
    , \quad \underline{e}_{d+1} = \underline{1} \in \mathbb{R}^{d+1}.
  \end{equation*}

提取$G_{\ell}$矩阵$\mathcal{A}$,计算特征值,Mathematica中代码如下
\begin{verbatim}
  d = 3 (*三维系统d=3。改为2或1,对应二维、一维系统*)
  ee = Table[1, d + 1, d + 1]
  et = Transpose[ee]
  ii = IdentityMatrix[d + 1]
  new = ii + ee*et
  Eigenvalues[new]
\end{verbatim}
可得$\lambda_{1}\left[ \mathcal{A} \right]=1, \, \lambda_{2}\left[ \mathcal{A} \right] = \ldots = \lambda_{d+1} \left[ \mathcal{A} \right] =1$。进而证得\eqref{eq:finele-form-lin-norm-def}。
\end{proof}

许多应用中需要将$\nu_{h}(x)$的斜率与$\nu_{h}(x)$自身关联起来,见下引理。
\begin{lemma}[方程范数与方程斜率的范数]
  \label{lemma:fenele-form-norm-gradient}
  设线性方程$\nu_{h}$如\eqref{eq:finele-form-lin-nuh-reprensentation}所给定。则以下局部逆不等式关系成立
  \begin{equation}
    \label{eq:fenele-form-norm-gradient}
    \left\| \triangledown_{x} \nu_{h} \right\|_{L^{2}(\tau_{\ell})}
    \le c_{I} h_{\ell}^{-1} \left\| \nu_{h} \right\|_{L^{2}(\tau_{\ell})},
  \end{equation}
  其中常数$c_{I} > 0$。
\end{lemma}
\begin{proof}
  分以下几步来证明。

  第一步。由Theorem \ref{theorem:finele-ref-d123-norm-equiv}可得$\nu_{h}(x), \, x \in \tau_{\ell}$方程斜率的范,与形式方程$\widetilde{nu}_{\ell} (\xi), \, \xi \in \tau$方程斜率范的关系

\end{proof}
