%!TEX root = ../DSGEnotes.tex
\section{边界积分算子}
\label{sec:bvp-integral-operators}

回顾一下第\ref{sec:bvp-laplace-fund-solutions}节提出的问题,我们的目标是求解$d=2,3$下的泊松方程(Poisson equation)\index{Poisson equation \dotfill 泊松方程}
\begin{equation*}
  -\Delta u(x) = f(x), \quad x \in \Omega \subset \mathbb{R}^d,
\end{equation*}
$f(x)=0$时的齐次形式泊松方程又称拉普拉斯方程(Laplace function)\index{Laplace function \dotfill 拉普拉斯方程}。

泊松方程的解满足表现形式(representation formula)   \eqref{eq:bvp-fund-var-ux-uy}
\begin{equation}
  \label{eq:bvp-fund-var-ux-uy-repform}
  u(x) = \int_{\Omega} U^{*}(x,y) f(y) \, dy
  + \int_{\Gamma} U^{*}(x,y) \gamma_{1}^{\text{int}} u(y) \, d s_y
  - \int_{\Gamma} U^{*}(x,y) \gamma_{0}^{\text{int}} u(y) \, d s_y,
\end{equation}

为了求解\eqref{eq:bvp-fund-var-ux-uy-repform},首先要求得拉普拉斯算子的基本解$U^{*}(x,y)$,见\eqref{eq:bvp-laplace-fundamental-solution-32d}给出
\begin{equation*}
  U^{*}(x,y) = \begin{cases}
  - \frac{1}{2 \pi} \log | x - y | & d =2, \\
  \frac{1}{4 \pi} \frac{1}{| x - y |} & d = 3.
  \end{cases}
\end{equation*}
在此基础上,需要构建适宜的边界积分方程,以$x \in \Gamma$生成完整的柯西数$\left[ \gamma_{0}^{\text{int}} u(x),  \gamma_{1}^{\text{int}} u(x)\right]$。而这就需要我们探讨表面积位势(surface potential)和体积位势(volume potental)的映射特征。

\subsection{牛顿位势}
\label{sec:bvp-newton-potential}
泊松方程表现式\eqref{eq:bvp-fund-var-ux-uy-repform}中,对于某一给定的方程$f(y), y \in \Omega$,我们可以定义$f(y)$的体积位势,或称牛顿位势(Newton potential)\index{Newton potential \dotfill 牛顿位势}为$\widetilde{N}_{0} f $,满足
\begin{equation}
  \label{eq:bvp-newton-potential-def}
  \left( \widetilde{N}_{0} f \right) (x) \coloneqq \int_{\Omega} U^{*}(x,y) f(y) \, dy, \quad x \in \mathbb{R}^{d}.
\end{equation}

由内积形式可得
\begin{equation*}
  \langle \widetilde{N}_{0} \varphi, \psi \rangle_{\Omega}
  = \int_{\Omega} \psi(x) \int_{\Omega} U^{*} (x,y) \varphi(y) dy dx
  = \langle \varphi, \widetilde{N}_{0}\psi \rangle_{\Omega}, \quad \varphi, \psi \in \mathcal{S}(\mathbb{R}^d),
\end{equation*}
可见$\widetilde{N}_{0} \varphi \in \mathcal{S}(\mathbb{R}^d)$。

进而,牛顿位势算子$\widetilde{N}_{0}: \mathcal{S}'(\mathbb{R}^d) \mapsto \mathcal{S}'(\mathbb{R}^d)$可以定义为
\begin{equation*}
  \langle \widetilde{N}_{0}f, \psi \rangle_{\Omega} \coloneqq
  \langle f, \widetilde{N}_{0} \psi \rangle_{\Omega}, \quad \forall \, \psi \in \mathcal{S}(\mathbb{R}^d).
\end{equation*}

\begin{theorem}[牛顿位势算子的映射]
  \label{theorem:bvp-newton-potential-mappting-property}
  牛顿位势算子$\widetilde{N}_{0}:\widetilde{H}^{-1}(\Omega) \mapsto H^{1}(\Omega)$定义了一个连续映射
  \begin{equation}
    \big\| \widetilde{N}_{0} f \big\|_{H^{1}(\Omega)} \le c \, \big\| f \big\|_{\widetilde{H}^{-1}(\Omega)}.
  \end{equation}
\end{theorem}
\begin{proof}
  对于$\varphi \in C_{0}^{\infty} (\Omega)$我们可得
  \begin{equation*}
    \big\| \varphi \big\|_{H^{-1}(\mathbb{R}^d)}^2
    = \int_{\mathbb{R}^{d}}
    \frac{
    | \widehat{\varphi}(\xi) |^2
    }{
    1 + |\xi|^2
    }
    d \xi,
  \end{equation*}
其中$\widehat{\varphi}(\xi)$表示傅里叶变换
\begin{equation*}
  \widehat{\varphi}(\xi) = \left( 2 \pi \right)^{-\frac{d}{2}}
  \int_{\mathbb{R}^d} \exp \left[ - i \langle x, \xi \rangle \right]
  \varphi(x) \, dx.
\end{equation*}


由$\supp \varphi \subset \Omega$可得(支撑集$\supp$的定义,见\pageref{footnote:support-definition}页脚注)
\begin{equation}
  \label{eq:bvp-newton-potential-norm-inequality}
\begin{split}
  \big\| \varphi \big\|_{H^{-1}(\mathbb{R}^d)}
  & = \sup_{0 \neq \nu \in H^{1}(\mathbb{R}^d)}
  \frac{
  \langle \varphi, \nu \rangle_{L^{2}(\mathbb{R}^d)}
  }{
  \| \nu \|_{H^{1}(\mathbb{R}^d)}
  } \\
  & \le \sup_{0 \neq \nu \in H^{1}(\Omega)}
  \frac{
  \langle \varphi, \nu \rangle_{L^{2}(\Omega)}
  }{
  \| \nu \|_{H^{1}(\Omega)}
  } \\
  & = \big\| \varphi \big\|_{\widetilde{H}^{-1}(\Omega)}.
\end{split}
\end{equation}

由定义\eqref{eq:bvp-newton-potential-def},我们定义一个$u(x)$
\begin{equation}
  \label{eq:bvp-newton-potential-ux-def}
  u(x) \coloneqq \left( \widetilde{N}_{0} \varphi \right)(x)
  = \int_{\Omega} U^{*}(x,y) \varphi(y) \, dy, \quad x \in \mathbb{R}^d.
\end{equation}

设$\Omega \subset B_{R}(0)$,以及一个有紧支撑的非负单调递增cutoff方程$\mu \in C_{0}^{\infty} \left( [0, \infty) \right)$,满足$\mu(r) = 1, r \in [0, 2 R]$。进而定义
\begin{equation}
  \label{eq:bvp-newton-potential-cutoff}
  u_{\mu}(x) \coloneqq
  \int_{\Omega} \mu \left( \big| x - y \big| \right)
  U^{*}(x,y) \varphi(y) \, dy, \quad x \in \mathbb{R}^d.
\end{equation}

\begin{equation*}
\begin{split}
  &x,y \in \Omega, \\
  \hookrightarrow & \big| x - y \big| \ge 0,  \\
  \hookrightarrow & \mu ( | x - y | ) =1, \\
  \hookrightarrow & u_{\mu}(x) = u(x), \quad x \in \Omega, \\
  \hookrightarrow & \big\| u \big\|_{H^{1}(\Omega)} = \big\|u_{\mu}\big\|_{H^{1}(\Omega)} \le \big\| u_{\mu} \big\|_{H^{1}(\mathbb{R}^d)}
\end{split}
\end{equation*}

以及
\begin{equation}
  \label{eq:bvp-newton-potential-umunorm}
  \big\| u_{\mu} \big\|_{H^{1}(\mathbb{R}^d)}^2 = \int_{\mathbb{R}^d}
  \left( 1+ |\xi|^2 \right) \, |\widehat{u}_{\mu} (\xi) |^2 \, d \xi.
\end{equation}

现在来计算$u_{\mu}(x)$的傅里叶变换
\begin{equation}
  \label{eq:bvp-newton-potential-umux-fourier-transform-middle}
\begin{split}
  \widehat{u}_{\mu}(x) &= \left( 2 \pi \right)^{-\frac{d}{2}}
  \int_{\mathbb{R}^d}
  \exp \left[ - i \langle x, \xi \rangle \right] u_{\mu}(x) \, dx \\
  & =  \left( 2 \pi \right)^{-\frac{d}{2}}
  \int_{\mathbb{R}^{d}} \exp \left[ - i \langle x, \xi \rangle \right]
  \int_{\mathbb{R}^{d}} \mu (|x - y |) U^{*}(x,y) \varphi(y) d y d x \\
  &= \left( 2 \pi \right)^{-\frac{d}{2}}
  \int_{\mathbb{R}^d} \int_{\mathbb{R}^d}
  \exp \left[ -i \langle z + y, \xi \rangle \right]
  \mu(|z|)
  U^{*}(z+y,y)
  \varphi(y)
  dy dz \\
  &= \underbrace
  {\left( 2 \pi \right)^{-\frac{d}{2}}
  \int_{\mathbb{R}^{d}} \exp \left[ -i \langle y, \xi \rangle \right]
  \varphi(y) dy
  }_{ = \widehat{\varphi}(\xi)}
  \underbrace{
  \int_{\mathbb{R}^{d}} \exp \left[ -i \langle z, \xi \rangle \right]
  \mu(|z|) U^{*}(z,0) dz
  }_{\eqqcolon I(|\xi|)}.
\end{split}
\end{equation}

求解\eqref{eq:bvp-newton-potential-umux-fourier-transform-middle}需要进一步求$I(|\xi|)$的值。已知$u(|z|)$和$U^*(z,0)$都是只与$|z|$有关的方程,我们可以利用傅里叶变换的旋转对称Lemma \ref{lemma:fourier-transform-rotating-symmetries},在三维坐标\index{spherical coordinate system \dotfill 三维坐标系}$\xi = \left(0,0,|\xi| \right)^{\top}$中测算$I(|\xi|)$。以$d=3$为例,建立坐标系
\begin{equation*}
  z =
  \begin{pmatrix}
    z_1 \\ z_2 \\ z_3
  \end{pmatrix}
   = \begin{pmatrix}
   r \cos \phi \sin \theta \\
   r \sin \phi \sin \theta \\
   r \cos \theta,
   \end{pmatrix} \quad r \in [0, \infty), \phi \in [0, 2 \pi), \theta \in [0, \pi).
\end{equation*}

将\eqref{eq:bvp-laplace-fundamental-solution-2d}代入$I(|\xi|)$
\begin{equation*}
  \begin{split}
    I(|\xi|) &= \frac{1}{4 \pi}
    \int_{\mathbb{R}^d} \exp \left[ -i \langle z, \xi \rangle \right]
    \frac{\mu(|z|)}{|z|} dz \\
    &= \frac{1}{4 \pi}
    \int_{0}^{\infty}
    \int_{0}^{2 \pi}
    \int_{0}^{\pi}
    \exp \left[ -i |\xi| r \cos \theta \right]
    \frac{\mu(r)}{r}
    r^2 \sin \theta
    d \theta d \phi d r \\
    &= \frac{1}{2}
    \int_{0}^{\infty} r \mu(r) \, dr \,
    \underbrace{
    \int_{0}^{\pi}
    \exp \left[ -i r |\xi| \cos \theta \right]
    \sin \theta \,
    d \theta
    }_{}
    d r.
  \end{split}
\end{equation*}

定义$\iota \coloneqq \cos \theta$,我们有
\begin{equation*}
\begin{split}
  &\int_{0}^{\pi}
  \exp \left[ -i r |\xi| \cos \theta \right]
  \sin \theta \,
  d \theta =
  \int_{-1}^{1} \exp \left[ -i r |\xi| \iota \right] d \iota \\
  &=
  \left[
  - \frac{1}{i r |\xi|}
  \exp \left[ - i r |\xi| \iota \right]
  \right]_{-1}^{1} = \frac{2 \sin r |\xi|}{r |\xi|}.
\end{split}
\end{equation*}

由此可得
\begin{equation*}
  I(|\xi|) = \frac{1}{|\xi|^2} \int_{0}^{\infty} \mu(r) \sin r |\xi| \, dr.
\end{equation*}
下面根据$|\xi|$的值,作分部求积。

\begin{enumerate}
\item 来看$|\xi| > 1$的情况。定义$s \coloneqq r |\xi|$可得
\begin{equation*}
  I(|\xi|) = \frac{1}{|\xi|^2} \int_{0}^{\infty} \mu \left( \frac{s}{|\xi|} \right) \sin s \, ds.
\end{equation*}

根据定义可知$0 \le \mu(r) \le 1$且有紧支撑,则
\begin{equation}
  \label{eq:bvp-newton-potential-ixi-ge1}
  I(|\xi|) \le c_{1}(R) \frac{1}{|\xi|^2}, \quad |\xi| \ge 1.
\end{equation}

此外考虑到
\begin{equation*}
  \left( 1+|\xi|^2 \right)^2 \le 4 |\xi|^4, \quad |\xi| \ge 1,
\end{equation*}

结合\eqref{eq:bvp-newton-potential-umunorm},\eqref{eq:bvp-newton-potential-umux-fourier-transform-middle},  \eqref{eq:bvp-newton-potential-ixi-ge1}可得
\begin{equation}
  \label{eq:bvp-newton-potential-xi-ge1}
  \begin{split}
    \big\| u_{\mu} \big\|_{H^{1}(\mathbb{R}^d), |\xi| > 1}^2 &= \int_{|\xi| > 1}
    \left( 1+ |\xi|^2 \right) \, |\widehat{u}_{\mu} (\xi) |^2 \, d \xi \\
    & =
    \int_{|\xi| > 1} \left(1+ |\xi|^2 \right) \, \big|\widehat{\varphi}(\xi) I(|\xi|) \big|^2 \, d \xi \\
    & \le \left[ c_{1}(R) \right]^{2}
    \int_{|\xi| > 1}
    \frac{1+|\xi|^2}{|\xi|^4}
    \big| \widehat{\varphi}(\xi) \big|^2 \,
    d \xi \\
    & \le 4 \left[ c_{1}(R) \right]^{2}
    \int_{|\xi| > 1}
    \frac{1}{ 1 + |\xi|^2 }
    \big| \widehat{\varphi}(\xi) \big|^2 \,
    d \xi.
  \end{split}
\end{equation}

\item 来看$|\xi| \le 1$的情况。
\begin{equation}
  \label{eq:bvp-newton-potential-ixi-le1}
\begin{split}
    I(|\xi|) &= \int_{0}^{\infty} \mu(r) \frac{\sin r |\xi|}{|\xi|} \, dr \\
    & \le c_{2}(R), \quad |\xi| \le 1.
\end{split}
\end{equation}

结合\eqref{eq:bvp-newton-potential-umunorm},\eqref{eq:bvp-newton-potential-umux-fourier-transform-middle},  \eqref{eq:bvp-newton-potential-ixi-le1}可得
\begin{equation}
  \label{eq:bvp-newton-potential-xi-le1}
  \begin{split}
    \big\| u_{\mu} \big\|_{H^{1}(\mathbb{R}^d), |\xi| \le 1}^2 &= \int_{|\xi| \le 1}
    \left( 1+ |\xi|^2 \right) \, |\widehat{u}_{\mu} (\xi) |^2 \, d \xi \\
    & =
    \int_{|\xi| \le 1} \left(1+ |\xi|^2 \right) \, \big|\widehat{\varphi}(\xi) I(|\xi|) \big|^2 \, d \xi \\
    & \le 2 \left[ c_{2}(R) \right]^2
    \int_{|\xi| \le 1} \big| \widehat{\varphi}(\xi) \big|^2 \, d \xi \\
    & \le 4 \left[ c_{2}(R) \right]^2
    \int_{|\xi| \le 1}
    \frac{1}{1+ | \xi |^2}
    \big| \widehat{\varphi}(\xi) \big|^2 \, d \xi.
  \end{split}
\end{equation}
\end{enumerate}

将\eqref{eq:bvp-newton-potential-xi-ge1},\eqref{eq:bvp-newton-potential-xi-le1}汇总,代回\eqref{eq:bvp-newton-potential-umunorm}, \eqref{eq:bvp-newton-potential-umux-fourier-transform-middle}可得

\begin{equation}
  \label{eq:bvp-newton-potential-xi-sum}
  \begin{split}
    \big\| u _{\mu} \big\|_{H^{1}(\mathbb{R}^d)}^{2} &=
    \int_{\xi \in \mathbb{R}^d}
    \left( 1 + |\xi|^2 \right)
    \big| \widehat{u}_{\mu} (\xi) \big|^2
    \, d \xi \\
    & \le c \int_{\xi \in \mathbb{R}^d}
    \frac{1}{1 + | \xi |^2}
    \big| \widehat{\varphi}(\xi) \big|^2 \, d \xi \\
    & = c \big\| \varphi \big\|_{H^{-1}(\mathbb{R}^d)}^2.
  \end{split}
\end{equation}

将\eqref{eq:bvp-newton-potential-cutoff},\eqref{eq:bvp-newton-potential-norm-inequality}代入上式,有
\begin{equation}
  \label{eq:bvp-newton-potential-n0varphi-cvarphi-ineq}
  \begin{split}
    \big\| \widetilde{N}_{0} \varphi \big\|_{H^{1}(\mathbb{R}^d)}^{2} &=
    \big\| u _{\mu} \big\|_{H^{1}(\mathbb{R}^d)}^{2}
    \le c \big\| \varphi \big\|_{H^{-1}(\mathbb{R}^d)}^2 \\
    & \le c \big\| \varphi \big\|_{H^{-1}(\Omega)}^2.
  \end{split}
\end{equation}














\end{proof}
