%!TEX root = ../DSGEnotes.tex
\chapter{DSGE with banking}
\label{sec:BDSGE}

\section{Introduction}


将货币和资产价格纳入到一般均衡模型的分析框架中去,在DSGE模型中加入银行部门,主要基于\cite{Goodfriend:2007fq, Gilchrist:2007iv}的工作。\cite{Chadha:2008uh}对模型做了两点扩展,一是把政府(包括中央银行)部门的预算约束条件也考虑进来,二是对于受到流动性约束的家庭部门,对货币需求与银行部门的贷款供应的比值是变化的。进而\cite{Chadha:2012jl}进一步探讨了当央行对商业银行准备金所支付的利率,也可以作为宏观审慎政策的工具。

\section{模型设定}

\subsection{家庭及生产部门}
家庭受到流动性约束,决定消费的数量以及劳动力的供应量。此外家庭消费不只取决于存款,也取决于他们可能从银行获得的贷款(流动性)。

出于简化模型的考虑,家庭部门也是最终产品的生产者,追求利润最大化,生产活动表现为Cobb-Douglas形式,资本、劳动力投入和生产效率的冲击共同影响产出。产出品遵循标准的\cite{Yun:1996tx}设定,即为Calvo定价形式的垄断竞争\citep{Calvo:1983uq}。

家庭的效用函数
\begin{equation}
\label{eq:hh-utility-fu}
U = E_0 \sum_{t=0}^{\infty} \beta^t \cdot \left[ \phi \cdot \log (c_t) + (1-\phi) \cdot \log(1-n_t^s - m_t^s) \right],
\end{equation}
其中
\begin{itemize}
  \item $c_t$表示实际消费,
  \item $n_t^s$和$m_t^s$分别表示投入到产品生产和银行部门的劳动力供应,$1-n_t^s - m_t^s$进而表示休闲,
  \item $0<\phi<1$区分消费和休闲带来的相对效用满足。
\end{itemize}

家庭效用取决于三个约束条件。第一为预算约束条件,对应拉格朗日乘子$\lambda_t$
\begin{equation}
  \label{eq:hh-bud-cons}
  \begin{split}
  &q_t \cdot (1-\delta) \cdot K_t + \frac{B_t}{P_t^A} + \frac{H_{t-1}}{P_t^A} + w_t \cdot \left( n_t^s + m_t^s \right) + c_t^A \cdot \left( \frac{P_t}{P_t^A} \right)^{1-\theta} \\
  &- w_t \cdot \left( n_t + m_t \right) - \frac{H_t}{P_t^A} - tax_t - q_t \cdot K_{t+1} - \frac{B_{t+1}}{P_t^A \cdot \left(1 + R_t^B\right)}  - c_t = 0,
  \end{split}
  \end{equation}
其中
\begin{itemize}
  \item $K_t$为$t$期初的实物资本存量,$q_t$为实物资本价格,$\delta$为实物资本折旧,
  \item $B_{t+1}$为$t$期末家庭持有的债券量,$P_t^A$表示消费品价格指数,变量加上上角标$A$代表总量层面的因素,个体家庭视为给定条件。
  \item $H_{t}$为$t$期末家庭持有的名义货币。
  \item $w_t$表示实际工资,
  \item $P_t$表示总物价水平,家庭作为垄断产品的生产者,所生产产品的需求替代弹性为$\theta$,生产活动需要雇佣劳动力$n_t$和$m_t$,
  \item $tax_t$表示政府的实际税收,以一揽子税形式表示
  \item $R_t^B$为政府债券的名义利率,由银行部门的贷款生产函数决定,见式\footnote{膏按:补一个reference。}。
\end{itemize}

第二为供应-需求的等价约束条件,家庭部门(最终产品生产部门)市场出清要求总产出等于总供应,对应拉格朗日乘子$\xi_t$
\begin{equation}
\label{eq:mkt-clearing-hh}
K_t^{\eta} \cdot \left( A_{1,t} \cdot n_t \right)^{1-\eta} = c_t^A \cdot \left( \frac{P_t}{P_t^A} \right)^{-\theta}
\end{equation}
等式左侧为生产函数。$\eta$代表资本的产出弹性。$A_{1,t}$为产品生产中的技术冲击,均值增速为$\gamma$。

第三为存款-货币的等价约束条件。

货币出现在效用函数和模型中,充当交易媒介,要求家庭用所持有的货币支付当期消费支出。家庭持有的货币,假定全部以存款的形式保存在商业银行中。则假定家庭消费与存款呈现如下对应关系\footnote{已有大量经验研究,从微观层面上探讨银行的存款和贷款供应间的关系,如\cite{Kashyap2002}}
\begin{equation}
\label{eq:constraint-deposit-money}
c_t = v_t \cdot \frac{D_t}{P_t^A},
\end{equation}
其中$D_t$表示名义存款,一定数量的存款最终被用于消费,以比例$v_t$表示。

\subsection{银行部门}
银行部门吸收家庭部门存款$D_t$,用途一是为家庭提供高能货币$H_t$(high-powered money),满足家庭的消费需求;二是为家庭提供贷款$L_t$,满足家庭生产活动的资金需求,银行的资产负债平衡式
\begin{equation}
\label{eq:bank-deposit-loan-money}
D_t = H_t + L_t.
\end{equation}

根据式\eqref{eq:constraint-deposit-money}和\eqref{eq:bank-deposit-loan-money}可以间接推得经济体对$H_t$的需求\footnote{进而,中央银行通过调整policy rate来引导商业银行的放贷行为,以满足这一需求;而不是通过诸如印钞等直接手段。}。假定
\begin{equation}
\label{eq:fractional-reserve-deposits-ratio}
H_t = rr \cdot D_t,
\end{equation}
其中$0<rr<1$表示商业银行选择将多少比例的存款留用以满足家庭消费需求,设为常数。余下的$1-rr$部分可用于贷款$L_t$。联立式\eqref{eq:bank-deposit-loan-money}-\eqref{eq:bank-deposit-loan-money}可得储蓄和贷款之间的关系
\begin{equation}
\label{eq:bank-deposit-loan}
D_t = \frac{L_t}{1-rr}.
\end{equation}

银行部门的贷款生产函数由此可以设定如下:以家庭部门提供的抵押品和银行部门的贷款监控为投入,以贷款为产出,满足关系
\begin{equation}
\label{eq:bank-loan-production-function}
\frac{L_t}{P_t^A} = F \cdot \left(
\underbrace{b_{t+1} + A_{3,t} \cdot k \cdot q_t \cdot K_{t+1}}_{\text{抵押品}}
\right)^{\alpha} \cdot
\left(
\underbrace{A_{2,t} \cdot m_t}_{\text{贷款监控}}
\right)^{1-\alpha},
\end{equation}
其中$0\le \alpha \le 1$表示抵押品的产出弹性,F表示贷款生产函数的中性技术水平,$b_{t+1}$为家庭的贴现实际债券
\begin{equation}
\label{eq:real-discounted-bonds}
b_{t+1} = \frac{B_{t+1}}{P_t^A \cdot \left(1+R_t^B\right)}.
\end{equation}
式\eqref{eq:bank-loan-production-function}中第一个括号内表示抵押品,银行在发放贷款时,要求借款人(家庭)提供抵押担保,担保来自两部分,一为债券抵押$b_{t+1}$;二为实物资本抵押$q_t \cdot K_{t+1}$。相较于债券抵押品而言,实物资本抵押品需要大量额外的成本支出,包括物理状况检查、市场价格评估等,因此需要做折扣,以参数$0<k<1$来表示。此外实物资本品还受到均值为$0$的技术冲击的干扰,用$A_{3,t}$来表示\footnote{出于简化模型的考虑,我们假定均衡状态下不存在违约的情况。在不完全市场条件下,考虑违约的有限承诺情况可见\citep{Kocherlakota:1996jk}。}。

第二个括号表示贷款监控,银行雇佣额外的劳动力$m_t$,监督借款人对贷款的利用情况,以降低风险。类似地,假定贷款监控也受技术冲击$A_{2,t}$干扰,均值的增速为$\gamma$。

\subsection{政府(中央银行)部门}
政府(包括中央银行)部门的当期预算约束条件为
\begin{equation}
\label{eq:gov-bud-cons-cb}
Transfer_t = g_t - tax_{t} = \frac{H_t-H_{t-1}}{P_t^A} + \frac{\frac{B_{t+1}}{\left(1+R_t^B\right)}-B_t}{P_t^A},
\end{equation}
即政府的转移支付$Transfer_t$为全部支出$g_t$与收入$tax_{t}$之差。

\subsection{技术冲击}
\begin{align*}
%\label{eq:tech-shcoks-a1t}
A_{1,t} &= A_{1,0} \cdot \left(1+\gamma\right)^t, \\
A_{2,t} &= A_{2,0} \cdot \left(1+\gamma\right)^t,
\end{align*}
为简化模型,设$A_{1,0}=A_{2,0}=1$,则我们有

\begin{align}
\label{eq:tech-shcoks-a1t}
A_{1,t} &= \left(1+\gamma\right)^t, \\
\label{eq:tech-shcoks-a2t}
A_{2,t} &= \left(1+\gamma\right)^t.
\end{align}

此外
\begin{equation}
\label{eq:tech-shcoks-a3t}
E(A_{3,t}) = 0.
\end{equation}

\subsection{实际消费}
\eqref{eq:bank-deposit-loan} \eqref{eq:bank-loan-production-function} \eqref{eq:real-discounted-bonds}带回到\eqref{eq:constraint-deposit-money}可得实际消费
\begin{equation}
\label{eq:hh-real-consumption}
\begin{split}
c_t&= v_t \cdot \frac{D_t}{P_t^A} =\frac{v_t}{1-rr} \cdot \frac{L_t}{P_t^A} \\
&=\frac{v_t}{1-rr} \cdot F \cdot
\left(
b_{t+1} + A_{3,t} \cdot k \cdot q_t \cdot K_{t+1}
\right)^{\alpha} \cdot
\left(
A_{2,t} \cdot m_t
\right)^{1-\alpha} \\
&=v_t \cdot \frac{D_t}{P_t^A} =\frac{v_t}{1-rr} \cdot \frac{L_t}{P_t^A} \\
&=\frac{v_t}{1-rr} \cdot F \cdot
\left(
\frac{B_{t+1}}{P_t^A \cdot \left(1+R_t^B\right)} + A_{3,t} \cdot k \cdot q_t \cdot K_{t+1}
\right)^{\alpha} \cdot
\left(
A_{2,t} \cdot m_t
\right)^{1-\alpha}.
\end{split}
\end{equation}


结合式\eqref{eq:hh-real-consumption},由银行部门贷款生产函数式\eqref{eq:bank-loan-production-function}可得抵押品$(b_{t+1} + A_{3,t} \cdot k \cdot q_t \cdot K_{t+1})$的边际产出
\begin{equation}
\label{eq:bank-loan-pf-marginal-collateral}
\begin{split}
&\frac{\partial{\left(L_t/P_t^A\right)}}{\partial \left(b_{t+1} + A_{3,t} \cdot k \cdot q_t \cdot K_{t+1}\right)} = \alpha \cdot \frac{\frac{L_t}{P_t^A}}{\left(b_{t+1} + A_{3,t} \cdot k \cdot q_t \cdot K_{t+1}\right)} =\frac{\Omega_t}{\frac{v_t}{1-rr}},
\end{split}
\end{equation}
其中$\Omega_{t}$定义为
\begin{equation}
\label{eq:bank-def-omega}
\Omega_t = \frac{\alpha \cdot c_t}{\left(b_{t+1} + A_{3,t} \cdot k \cdot q_t \cdot K_{t+1}\right)}.
\end{equation}

进而可得额外1单位(向金融部门)劳动力供应$m_t$,债券$B_{t+1}$,实物资本$K_{t+1}$导致消费$c_t$的边际变化
\begin{equation}
\label{eq:bank-marg-prod-c-m}
\frac{\partial{c_t}}{\partial{m_t}} = \left(1-\alpha \right) \cdot A_{2,t} \cdot \frac{c_t}{m_t}.
\end{equation}

\begin{equation}
\label{eq:bank-marg-prod-c-b}
\frac{\partial{c_t}}{\partial{B_{t+1}}} = \frac{\Omega_t}{P_t^A \cdot \left(1+R_t^B \right)}.
\end{equation}

\begin{equation}
\label{eq:bank-marg-prod-c-k}
\frac{\partial{c_t}}{\partial{K_{t+1}}} = \Omega_t \cdot A_{3,t} \cdot k \cdot q_t.
\end{equation}

\section{一阶条件}
%\subsection{一阶条件}
建拉格朗日
\begin{equation}
\label{eq:hh-optimal-lagrangian}
\begin{split}
\mathcal{L} = E_0 \sum_{t=0}^{\infty} \beta^t \cdot & \left\{
\left[
  \phi \cdot \log (c_t) + (1-\phi) \cdot \log \left(1-n_t^s - m_t^s\right)
\right] \right.\\
&\left.  + \lambda_t \cdot  \left[ q_t \cdot (1-\delta) \cdot K_t + \frac{B_t}{P_t^A} + \frac{H_{t-1}}{P_t^A} + w_t \cdot \left( n_t^s + m_t^s \right) + c_t^A \cdot \left(\frac{P_t}{P_t^A}\right)^{1-\theta} \right. \right.\\
&\left. \qquad \left.  - w_t \cdot \left( n_t + m_t \right) - \frac{H_t}{P_t^A} - tax_t - q_t \cdot K_{t+1} - \frac{B_{t+1}}{P_t^A \cdot \left(1 + R_t^B\right)}  - c_t \right] \right.\\
& \left. +\xi_t \left[ \cdot K_t^{\eta} \cdot \left(A_{1,t} \cdot n_t \right)^{1-\eta} -  c_t^A \cdot \left(\frac{P_t}{P_t^A}\right)^{-\theta} \right] \right\}
\end{split}
\end{equation}

家庭部门优化可以表示为
\begin{equation*}
\max_{\{m_t^s, n_t^s, m_t, n_t, P_{t}, K_{t+1}, B_{t+1}, c_t, H_t\}} \mathcal{L}.
\end{equation*}

FOC wrt $m_t^s$ 或 $n_t^s$,得劳动力的供应
\begin{equation}
\label{hh-opt-wrt-ms}
\frac{\partial{\mathcal{L}}}{\partial{m_t^s}} = 0 \Rightarrow \lambda_t = \frac{1-\phi}{\left(1-n_t^s - m_t^s \right) \cdot w_t}.
\end{equation}

FOC wrt $m_t$,得银行部门对劳动力的需求
\begin{equation}
\label{hh-opt-wrt-m}
\begin{split}
\frac{\partial{\mathcal{L}}}{\partial{m_t}} = 0 \Rightarrow
&\frac{\phi}{c_t} \cdot \frac{\partial c_t}{\partial m_t} - \lambda_t \cdot w_t - \lambda_t \cdot \frac{\partial c_t}{\partial m_t} = 0, \\
&w_t = \left(\frac{\phi}{c_t \cdot \lambda_t} - 1 \right) \cdot \left(1-\alpha\right) \cdot A_{2,t} \cdot \frac{c_t}{m_t}.
\end{split}
\end{equation}

FOC wrt $n_t$,得最终产品生产对劳动力的需求
\begin{equation}
\label{hh-opt-wrt-n}
\frac{\partial{\mathcal{L}}}{\partial{n_t}} = 0 \Rightarrow w_t = \left(\frac{\xi_t}{\lambda_t}\right) \cdot A_{1,t} \cdot \left(1-\eta \right) \cdot \left(\frac{K_{t}}{A_{1,t} \cdot n_t}\right)^{\eta}.
\end{equation}

FOC wrt $P_t$,得影子价格的比值
\begin{equation}
\label{hh-opt-wrt-p}
\frac{\partial{\mathcal{L}}}{\partial{P_t}} = 0 \Rightarrow \frac{\xi_t}{\lambda_t} = - \left(\frac{1- \theta} {\theta}\right) \cdot \left(\frac{P_t}{P_t^A} \right).
\end{equation}

FOC wrt $K_{t+1}$,得
\begin{equation}
\label{hh-opt-wrt-k}
\begin{split}
&\frac{\partial{\mathcal{L}}}{\partial{K_{t+1}}} = 0 \Rightarrow \\
&\frac{\phi}{c_t} \cdot \frac{\partial c_t}{\partial K_{t+1}} + \beta \cdot E_t \lambda_{t+1} \cdot q_{t+1} \cdot (1-\delta) - \lambda_t \cdot q_t - \lambda_t \cdot \frac{\partial c_t}{\partial K_{t+1}} + \beta \cdot E_t \xi_{t+1} \cdot \eta \cdot \left(\frac{K_{t+1}}{A_{1,t+1} \cdot n_{t+1}}\right)^{\eta - 1} = 0, \\
& \left(\frac{\phi}{\lambda_t \cdot c_t} - 1\right) \cdot \Omega_t \cdot A_{3,t} \cdot k \cdot q_t - q_t  + \beta \cdot \left(1-\delta \right) \cdot E_t \frac{\lambda_{t+1}}{\lambda_t} \cdot q_{t+1} + \beta \cdot \eta \cdot E_t \frac{\xi_{t+1}}{\lambda_{t+1}} \cdot \frac{\lambda_{t+1}}{\lambda_t} \cdot \left(\frac{K_{t+1}}{A_{1,t+1} \cdot n_{t+1}}\right)^{\eta - 1} = 0.
\end{split}
\end{equation}

FOC wrt $B_{t+1}$,得
\begin{equation}
\label{hh-opt-wrt-b}
\begin{split}
& \beta \cdot E_t \left(\frac{\lambda_{t+1}}{P_{t+1}^A}\right) - \frac{\lambda_t}{P_t^A \cdot \left(1 + R_t^B \right)} + \phi \cdot \frac{1}{c_t} \cdot \frac{\partial c_t}{\partial B_{t+1}} - \lambda_t \cdot \frac{\partial c_t}{\partial B_{t+1}} = 0, \\
& \left(\frac{\phi}{c_t \cdot \lambda_t} - 1 \right) \cdot \left[ \frac{\Omega_t}{P_t^A \cdot \left(1 + R_t^B \right)}\right] + \beta \cdot E_t \frac{\lambda_{t+1}}{\lambda_t} \cdot \frac{1}{P_{t+1}^A} - \frac{1}{P_t^A \cdot \left(1 + R_t^B \right)} = 0, \\
& \left(\frac{\phi}{c_t \cdot \lambda_t} - 1 \right) \cdot \Omega_t + \beta \cdot E_t \frac{\lambda_{t+1}}{\lambda_t} \cdot \frac{P_t^A}{P_{t+1}^{A}} \cdot \left(1 + R_t^B \right) - 1 = 0.
\end{split}
\end{equation}

FOC wrt $c_{t}$,得
\begin{equation}
\label{hh-opt-wrt-c}
%\begin{split}
\frac{U_{c,t}}{\lambda_t} - 1 = \frac{\phi}{c_t \cdot \lambda_t} - 1 = 0,
%\end{split}
\end{equation}
其中$U_{c,t} = \phi/c_t$表示消费的边际效用,由式\eqref{eq:hh-utility-fu}推得。

\section{均衡条件}
\label{sec:banking-equilibra}
综上,模型的均衡条件由14个等式构成:
\begin{itemize}
  \item 家庭部门的预算约束条件式\eqref{eq:hh-bud-cons},
  \item 家庭部门生产活动的市场出清条件式\eqref{eq:mkt-clearing-hh},
  \item 银行部门的资产负债平衡式\eqref{eq:bank-deposit-loan-money},
  \item 消费存款平衡式\eqref{eq:constraint-deposit-money},
  \item 商业银行的贷款管理(贷款生产函数)式\eqref{eq:bank-loan-production-function},
    \item 反映贴现实际债券的变量$b_{t+1}$定义式\eqref{eq:real-discounted-bonds},
  \item 反映抵押品边际产出(放贷)的变量$\Omega_t$定义式\eqref{eq:bank-def-omega},
  \item 6个一阶条件式\eqref{hh-opt-wrt-ms}, \eqref{hh-opt-wrt-m}, \eqref{hh-opt-wrt-n}, \eqref{hh-opt-wrt-p}, \eqref{hh-opt-wrt-k}, \eqref{hh-opt-wrt-b},
  \item 政府(包括中央银行)部门的预算约束条件式\eqref{eq:gov-bud-cons-cb}。
\end{itemize}

随后的任务就变成了,在给定
\begin{itemize}
\item 3个外生技术冲击$\{A_{i,t}\}, i=1,2,3$,式\eqref{eq:tech-shcoks-a1t}, \eqref{eq:tech-shcoks-a2t}, \eqref{eq:tech-shcoks-a3t},
\item 由政策制定者设定的政策变量$\{H_t, g_t, b_t, \text{Transfer}_{t}\}$,
\item 外生资本存量$K_{t}$\footnote{即是说在模型的均衡条件下,每一期资本存量均等于其稳态值$K_t = K$。需要指出的是,$K$本身是由模型所内生决定的。},
\end{itemize}

的情况下, 探讨一组14个内生变量的变化,包括
\begin{itemize}
  \item 家庭部门的行为决策,包括消费、劳动力供应、购入债券$\{c, n, m, B\}$,
  \item 银行部门的存贷款,以及抵押品对放贷的边际产出$\{D, L, \Omega\}$,
  \item 政府(含中央银行)的税收$tax$,
  \item 市场价格和利率,包括劳动力和资本价格,总物价水平,影子价格和国债利率$\{w, q, P, \lambda, \xi, R^B \}$。
\end{itemize}

\section{利率}
第\ref{sec:banking-equilibra}节的均衡条件中,除了政府债券利率$R_t^B$之外,并不包括其他利率。这是由于根据模型设定,产品生产部门和银行最终由家庭运营,利润也归家庭所有。

\subsection{资产市场基准利率}
首先根据\cite{Goodfriend:2005ts},假想存在一种完全无风险债券。由于无风险,因此借贷者不需要任何抵押品。根据家庭部门的最优消费——储蓄(投资)决策,这种债券的投资回报率反映该项投资的影子价格,我们设为基准利率$R_t^T$。
基于家庭部门优化决策,无风险债券的利率$R_t^T$由类似于式\eqref{hh-opt-wrt-b}的方式决定:
\begin{equation*}
1+R_t^T =\frac{
1 - \left(\frac{U_{c,t}}{\lambda_t} - 1 \right) \cdot \Omega_t
}{
  \beta \cdot E_t \frac{\lambda_{t+1}}{\lambda_{t}} \cdot \frac{P_t^A}{P_{t+1}^A}
},
\end{equation*}
代入家庭的消费最优决策式\eqref{hh-opt-wrt-b},调整得
\begin{equation}
\label{eq:banking-int-rate-t}
1+R_t^T = \frac{1}{\beta} \cdot E_t \frac{\lambda_t}{\lambda_{t+1}} \cdot \frac{P^A_{t+1}}{P_t^A}.
\end{equation}

\subsection{抵押品收益率}
\subsubsection{流动性债券收益率}
无风险利率$R_t^T$定义式\eqref{eq:banking-int-rate-t}带回式\eqref{hh-opt-wrt-b}
\begin{equation*}
\frac{1+R_t^B}{1+R_t^T}=1-\left(
\frac{U_{c,t}}{\lambda_t} - 1
\right) \cdot \Omega_t.
\end{equation*}


$\frac{U_{c,t}}{\lambda_t}$用于比较消费的边际效用与投资的影子价格,可得$\left( \frac{\phi}{c_t \cdot \lambda_t}-1\right) >0$;$\Omega_t$用于近似描述抵押品的边际价值,根据式\eqref{eq:hh-real-consumption}中$0 < \alpha <1$可得$\Omega >0$。因此我们有$R_t^T > R_t^B$,二者之差反映了流动性债券的收益率,定义为$LSY_t^B$
\begin{equation}
\label{eq:banking-lqdt-yield-bond}
LSY_t^B \equiv R_t^T - R_t^B \approx \left(
\frac{\phi }{c_t \cdot \lambda_t} - 1
\right) \cdot \Omega_t.
\end{equation}

\subsubsection{流动性资本收益率}

如贷款生产函数式\eqref{eq:bank-loan-production-function}所定义,债券和实物资本都可以用于贷款抵押,而实物资本比起债券来具有成本劣势,用系数$0<k<1$表示。因此可定义流动性资本收益率$LSY_t^B$,基于式\eqref{eq:banking-lqdt-yield-bond}我们有\footnote{更为详尽的说明,见\cite{Goodfriend:2005ts}。}
\begin{equation}
\label{eq:banking-lqdt-yield-capital}
LSY_t^K = k \cdot LSY_t^B = k \cdot \left(
\frac{\phi }{c_t \cdot \lambda_t} - 1
\right) \cdot \Omega_t.
\end{equation}

\subsection{银行间拆借利率(无抵押)}
根据模型设定,某一商业银行可以在银行间市场以利率$R_t^{IB}$筹集资金,进而贷款给家庭部门,获得回报$R_t^T$。假定不存在套利行为,如果该银行的操作有利可图,即净收益可以弥补全部成本支出的话,上述情况即可能发生。

如式\eqref{eq:bank-loan-production-function}所示,贷款成本包括抵押品成本和监控成本两部分。假定该银行在银行间市场向其他银行的借款无需抵押品,则成本主要指监控成本,即以工资$w_t$雇佣劳动力$m_t$对贷出给家庭部门的款项做监控的劳动成本支出。

银行的最优行为表现为两阶段优化。第一阶段为成本最小化,无抵押贷款生产函数的实际边际成本等于劳动的市场价格(工资),除以劳动投入的边际产出,后者由式\eqref{eq:bank-loan-production-function}求得:
\begin{equation}
\label{eq:banking-real-marginal-cost}
\text{real marginal cost} = \frac{w_t}{\frac{\partial \left(\frac{L_t}{P_t^A}\right)}{\partial m_t}} = \frac{w_t \cdot m_t \cdot v_t}{(1-\alpha) \cdot (1-rr) \cdot c_t}.
\end{equation}

第二阶段,追求利润最大化的银行行为最终会使得下式成立
\begin{equation}
\label{eq:eq:banking-benchmark-interbank-rate}
\left(1+ R_t^{IB}\right) \cdot \left(
%\frac{w_t \cdot m_t \cdot v_t}{(1-\alpha) \cdot (1-rr) \cdot c_t}
1+ \text{real marginal cost}
\right) = 1+R_t^T,
\end{equation}
即银行间拆借利率$R_t^{IB}$乘以贷款的边际生产成本,等于基准利率$R_t^T$。类似地,整理可得
\begin{equation}
\label{eq:banking-interbank-rate}
R_t^T - R_t^{IB} = \frac{w_t \cdot m_t \cdot v_t}{(1-\alpha) \cdot (1-rr) \cdot c_t}
\end{equation}

\subsection{贷款利率(有抵押)}
实际由银行部门向非银行部门贷出的款项,往往是有抵押贷款。有抵押贷款的成本等于实际边际成本乘以监控的要素份额$(1-\alpha)$。利润最大化银行会令抵押贷款利率$R_t^L$的值等于
\begin{equation*}
  1+R_t^{L} = \left( 1+R_t^{IB} \right) \cdot \left[ 1+ (1-\alpha) \cdot \text{real marginal cost} \right],
\end{equation*}
即抵押贷款的利率应该等于无抵押的银行间拆借利率,乘以贷款监控的成本支出。进一步整理得
\begin{equation}
  \label{eq:banking-loan-rate}
  \begin{split}
  & 1+R_t^{L} = \left( 1+R_t^{IB} \right) \cdot \left( 1+ \frac{w_t \cdot m_t \cdot v_t}{\left( 1 - rr \right) \cdot c_t}\right), \\
  & R_t^L - R_t^{IB} = \frac{w_t \cdot m_t \cdot v_t}{\left( 1 - rr \right) \cdot c_t} .
  \end{split}
\end{equation}

\subsection{存款利率}
银行的贷款来自于家庭部门储蓄(存款)$D_t$。全部存款中有$rr$比例作为高能货币$H_t$用于日常消费支出,不产生利息;余下$(1-rr)$部分可被银行作为贷款$L_t$。因此,存款利率$R_t^{D}$的值应为
\begin{equation}
  \label{eq:banking-deposit-rate}
  %\begin{split}
    \frac{R_t^{D}}{1-rr} = R_t^{IB},
  %\end{split}
\end{equation}
即贷存比乘以无抵押银行间拆借利率。

\subsection{外部融资溢价}

\subsubsection{无抵押贷款的外部融资溢价}
家庭部门向银行做无抵押贷款,需要支付的利息包括两部分,一为无抵押银行间拆借利率$R_t^{IB}$,一为银行贷款生产函数的实际边际成本式\eqref{eq:banking-real-marginal-cost},二者之和应等于资产市场的无抵押贷款基准利率$R_t^T$。从这个意义上来说,贷款的实际边际成本即反映了无抵押贷款外部融资的溢价(un-collateralized external financial premium, UEFP)\footnote{关于外部融资溢价,见\cite{Bernanke1999}。},我们用$R_t^T$和$R_t^{IB}$的利差来表示,由式\eqref{eq:banking-interbank-rate}得
\begin{equation}
  \label{eq:un-collateralized-external-financial-premium}
  UEFP_t = \text{real marginal cost} = R_t^T - R_t^{IB} = \frac{w_t \cdot m_t \cdot v_t }{(1-\alpha) \cdot (1-rr) \cdot c_t}.
\end{equation}


\subsubsection{有抵押贷款的外部融资溢价}
类似地,家庭向银行做抵押贷款,需要支付的利息包括两部分,一为银行间拆借利率$R_t^{IB}$,一为有抵押贷款生产函数的边际成本,二者之和应等于有抵押贷款利率$R_t^L$。此时有抵押贷款的边际成本反映了抵押贷款的外部融资溢价(collateralized external financial premium, CEFP),我们用$R_t^L$和$R_t^{IB}$的利差来表示,由式\eqref{eq:banking-loan-rate}得
\begin{equation}
  \label{eq:collateralized-external-financial-premium}
  CEFP_t = (1-\alpha) \cdot \text{real marginal cost} = R_t^L - R_t^{IB} =\frac{w_t \cdot m_t \cdot v_t }{(1-rr) \cdot c_t}
\end{equation}

\subsubsection{比较}
联立~\eqref{eq:un-collateralized-external-financial-premium}-~\eqref{eq:collateralized-external-financial-premium}可得
\begin{equation}
  \label{eq:banking-efp-connection}
  CEFP_t = (1-\alpha) \cdot UEFP_t
\end{equation}
外部融资溢价$CEFP_t< UEFP$,或利差$ (R_t^L-R_t^{IB})<(R_t^{T}-R_t^{IB})$,反映了家庭部门当提供抵押品时所做的借款,只需要支付贷款生产函数边际成本中的一部分$(1-\alpha)$,这一部分被银行用来对贷出款项做监控;余下的$\alpha$部分得以减免。

\subsection{抵押品收益率与外部融资溢价的关联}
由于我们假定银行部门是充分竞争的,并且银行的贷款生产函数是规模报酬不变的,那么均衡状态下银行的利润为$0$,即
\begin{equation*}
\underbrace{(R_t^{T}-R_t^{IB}) \cdot \frac{L_t}{P_t}}_{\text{利息收入}} = \underbrace{\underbrace{b_{t+1} \cdot LSY_t^{B}}_{\text{债券抵押品利息}} + \underbrace{q_t \cdot K_{t+1} \cdot LSY_t^K}_{\text{实物资本抵押品利息}} + \underbrace{w_{t} \cdot w_t}_{\text{监控成本支出}}}_{\text{成本支出}}.
\end{equation*}
从这个意义上来讲,$LSY_{t}^{B,K}$可以看做是银行支付给家庭部门借款人的\textquote{租金}:借款人向银行提供抵押品,抵押品作为投入要素进入银行部门的贷款生产函数中。抵押品的回报率因此低于假想中的无风险债券的基准利率$R_t^T$;两种利率之差体现了抵押品作为贷款生产函数的投入要素的回报,这种回报以利率优惠的形式,反映在银行对家庭借款人收取的贷款利息减免中。

流动性抵押品的收益率,与外部融资溢价之间的关联,可以由式\eqref{eq:banking-lqdt-yield-capital}、\eqref{hh-opt-wrt-m}、\eqref{eq:collateralized-external-financial-premium}联立求得
\begin{equation}
  \label{eq:banking-efp-lsy-connection}
  LSY_t^K = k \cdot LSY_t^B = k \cdot \frac{w_t \cdot m_t}{\left( 1 - \alpha \right) \cdot c_t} = k \cdot \left[ \frac{1-rr}{\left( 1 - \alpha \right) \cdot v_t} \right] \cdot CEFP_t.
\end{equation}

\section{稳定状态}
\subsection{核心稳态系统}
如前述,模型的均衡条件包括22个变量。假定稳定状态下是确定的(deterministic)并且
\begin{enumerate}
  \item 通胀率为零$P^A = P = 1$,
  \item 资产价格$q = 1$,
  \item $\text{Transfer} = g = \text{tax}= 0$。
\end{enumerate}

由式\eqref{eq:tech-shcoks-a1t},\eqref{eq:tech-shcoks-a2t},\eqref{eq:tech-shcoks-a3t}可得
\begin{equation}
  \label{eq:banking-ss-tech-shocks}
  g_{A_1} = g_{A_2} = 1+ \gamma, g_{A_3} = 0,
\end{equation}
由式\eqref{eq:banking-ss-tech-shocks}可得,$\{ c_t,K_t,w_t,\lambda_t \}_{t=0}^{\infty}$的增速均为$1+\gamma$。

根据定义式\eqref{eq:bank-def-omega}可知
\begin{equation}
  \label{eq:banking-ss-bank-def-omega}
  \Omega = \frac{\alpha}{\frac{b}{c}+k \cdot \frac{K}{c}}
\end{equation}

\subsubsection{}
根据一阶条件可知:

式\eqref{hh-opt-wrt-ms} $\Rightarrow$
\begin{equation}
  \label{banking-ss-hh-opt-wrt-ms}
  \lambda = \frac{1-\phi}{\left(1-n-m\right) \cdot w}
\end{equation}

式\eqref{hh-opt-wrt-m} $\Rightarrow$
\begin{equation}
  \label{banking-ss-hh-opt-wrt-m}
  w = (1-\alpha) \cdot \left(\frac{\phi}{\lambda \cdot c} - 1 \right)  \cdot \frac{c}{m}
\end{equation}

式\eqref{hh-opt-wrt-p} $\Rightarrow$
\begin{equation}
  \label{banking-ss-hh-opt-wrt-p}
  \frac{\xi}{\lambda} = - \frac{1-\theta}{\theta}
\end{equation}

式\eqref{hh-opt-wrt-n}结合式\eqref{banking-ss-hh-opt-wrt-p}$\Rightarrow$
\begin{equation}
  \label{banking-eq:ss-hh-opt-wrt-n}
  w = - {\left( 1-\eta \right)} \cdot \frac{1-\theta}{\theta} \cdot   {\left( \frac{K}{n} \right)}^{\eta}
\end{equation}

式\eqref{hh-opt-wrt-k} $\Rightarrow$
\begin{equation}
  \label{banking-ss-hh-opt-wrt-k}
  k \cdot \Omega \cdot {\left(
  \frac{\phi}{\lambda \cdot c} - 1
  \right)} - 1 + \frac{\beta}{1+\gamma} \cdot {\left[
  {\left( 1 - \delta \right)} - \eta \cdot \frac{1-\theta}{\theta} \cdot {\left(\frac{K}{n}\right)}^{\eta - 1}
  \right]} = 0
\end{equation}

\subsubsection{}
家庭部门总预算约束式\eqref{eq:hh-bud-cons}改写为
\begin{equation*}
\begin{split}
&q_t \cdot {\left[
(1-\delta) \cdot K_t - K_{t+1}
\right]}
- {\left[
\frac{H_{t}}{P_t^A} - \frac{H_{t-1}}{P_t^A} + \frac{B_{t+1}}{P_t^A \cdot (1+R_t^B)} - \frac{B_t}{P_t^A} + tax_t
\right]} \\
&+ w_t \cdot {\left[(n_t^s + m_t^s) - (n_t + m_t)\right]}
+ c_t^A \cdot {\left(\frac{P_t}{P_t^A}\right)}^{1-\theta} = c_t,
\end{split}
\end{equation*}
其中等式左侧第二个中括号部分,根据政府部门的预算约束条件式\eqref{eq:gov-bud-cons-cb}等于$g_t$。第三个中括号,根据市场出清条件假定为$0$。第四部分,根据另一个市场出清条件式\eqref{eq:mkt-clearing-hh}予以替代,得
\begin{equation*}
%\begin{split}
q_t \cdot {\left[
(1-\delta) \cdot K_t - K_{t+1}
\right]}
- g_t
+ K_t^{\eta} \cdot (A_{1,t} \cdot n_t)^{1-\eta} \cdot \frac{P_t}{P_t^A} = c_t,
%\end{split}
\end{equation*}
则稳定状态下我们有
\begin{equation}
  \label{eq:banking-ss-hh-opt-wrt-wac}
  {\left(\frac{K}{n}\right)}^{\eta} - \delta \cdot {\left(\frac{K}{n}\right)} = \frac{c}{n}
\end{equation}
\subsubsection{}
式\eqref{eq:hh-real-consumption} $\Rightarrow$
\begin{equation}
  \label{eq:banking-ss-hh-real-consumption}
  1 = \frac{v \cdot F}{1-rr} \cdot {\left(
\frac{b}{c} + k \cdot \frac{K}{c}
  \right)}^{\alpha} \cdot {\left(
\frac{m}{c}
  \right)}^{1-\alpha}
\end{equation}

则22个变量的稳定状态系统,可以改写为含有7个核心变量$\{c,K,w,\lambda,\Omega,m,n\}$的7个方程组:
\eqref{eq:banking-ss-bank-def-omega},\eqref{banking-ss-hh-opt-wrt-ms},\eqref{banking-ss-hh-opt-wrt-m},
\eqref{banking-eq:ss-hh-opt-wrt-n},\eqref{banking-ss-hh-opt-wrt-k},\eqref{eq:banking-ss-hh-opt-wrt-wac}, \eqref{eq:banking-ss-hh-real-consumption}。

\subsection{稳定状态利率}
