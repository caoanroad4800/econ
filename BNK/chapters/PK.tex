%!TEX root = ../DSGEnotes.tex
\chapter{后凯恩斯主义经济学的一个小综述}
\label{sec:PK-user-guide}

主要参考自
\cite{Hart:2015bs}。

\section{Introduction}
\label{sec:PK-intro}

\section{PK的主要方法论}
\label{sec:PK-metho}

\section{PK下的市场结构与定价}
\label{sec:PK-mkt-struc-pricing-decision}

\subsection{PK下的市场结构}
\label{sec:PK-mkt-struc}

\subsection{PK下的定价策略}
\label{sec:PK-pricing-decision}
markup pricing principle的一系列变体,如
\begin{itemize}
\item\cite{Kalecki:1937ks}考虑到企业往往面临内部/外部的融资需求,(其他企业所定的)price markup会影响到企业的现金流,进而使得金融部门,企业融资行为和企业(家庭)投资决策这三者之间存在重大关联,可见货币和金融的确会对实体经济产生影响\citep{Ball:1964vi, Eichner:1973jc,Wood:1975vn, Harcourt:1976kp}。
\item (企业自身所定的)price markup,也会被理解为企业自己在多变、激烈和难以预测的市场竞争中寻求立足、发展的关键决策之一。
\end{itemize}
因此价格更多反映企业的利益决策而非其所在产业或市场的条件;价格是基于企业战略而非基于成本制定的;这一价格反映市场的非均衡状态,对于整个经济体动态发展起到重要作用。

PK视野下,企业不是追求短期利润最大化的行为人,而是在面临巨大不确定性条件下的长期决策制定者。企业的产量和定价决策反映了寡头垄断的市场条件,企业自身及其竞争对手的行为,以及融资情况都是该条件的重要组成部分。


\section{PK下的宏观经济}
\label{sec:PK-macro-economy}
MS更多遵循萨伊定律,即供应会产生它所对应的需求,导致就业和产出都不受需求的限制。PK经济体更多强调有效需求,货币非中性,不确定情况下的决策,以及主观并且变化迅速的期望,这使得PK对萨伊定律提出挑战。值得指出的是,该挑战并不依赖于对市场完全竞争程度或者“粘性”的假定。由此PK坚决反对任何将凯恩斯经济学与一般均衡框架融合在一起的尝试,即反对新古典主义综合。PK也不同于New Keynesian,后者致力于将一系列摩擦(frictions)引入到New Classical体系当中\cite{Akerlof:2007go}。在PK看来,这两种思路忽略了发达资本主义经济体中最本质的不稳定因素,从而仍然使市场运作暴露在重大风险中。

延续Keynes和Kalecki的思路,PK强调真实产出和就业的根本决定因素是有效需求的水平,真实产出和就业的波动主要是由投资支出的变化所导致的,投资支出受到对周边情形的期望(“动物精神”)的影响。对于“储蓄增加导致投资增加”的传统观点,PK持相反意见:只要能够寻找到合适的融资渠道,投资通过乘数效应可以增加收入,进而提高储蓄。

PK在几个方面修正了Keynes的理论体系,尤其是在对金融市场、制度的描述上。在Keynes看来,不确定性是人们持有货币以追求保值的最关键原因,持有货币的行为将人们过去已经发生了的经济行动和不确定性的未来连接在一起,进一步扩大了不确定性的风险。如\cite{Kaldor:1985ve}对货币主义者的批判,货币供应的内生性被过分夸大了:随着货币供应增加,金融机构可以放出更多贷款,导致存款增加和/或对金融财产的购买增加。这一系列借入和借出的决策取决于融资成本以及对未来的期望——金融财产的规模和组成结构直接取决于balance sheet中放贷者和潜在借款人的立场。他们的乐观或悲观心态,会影响到金融市场整体的行为,进而放大、波及对实体经济运转情况的看法。从这个意义上来讲,金融不稳定性假设认为资本主义经济体中,金融部门和实体经济部门的不稳定性是彼此相关和不可避免的\citep{Minsky:2015uq},导致中央银行和财政稳定政策的努力注定无效,“有效市场”及其相关资产定价模型也需要被否定。

PK不再关注于供应-需求互动决定的均衡价格和产品数量,而是关注产业结构和markup定价原则,这使得对产品相对价格的分析不再有意义:在宏观经济层面上,产出就业和总物价均发生波动。excess capacity的存在意味着即便需求增加了,产出也会跟着调整以满足需求,从而不会出现通货膨胀的压力。当逼近full capacity utilisation的情况下,通货膨胀的压力也主要来自mark-up pricing机制,而不是原材料、工资等要素成本。就业水平主要取决于有效需求水平,而不是实际工资。实际工资不取决于市场,而是取决于以下两方面因素,第一如上文所说的企业定价策略,第二是围绕全部国民收入,在劳动收入和资本收入之间的讨价还价,这就涉及到随时间变化的制度性要素。在PK看来,通货膨胀是国民收入中劳动收入和资本收入比重分配不恰当所导致的,也只有重新调整比重分配才能缓解通货膨胀——这呼吁一种permanent income policy的出台。

PK对经济增长和经济发展尤为关注,在PK看来它们是资本主义经济发展动态的最核心要素。然而在PK看来,它们都不是平稳和连续的。

\section{PK视野下的经济政策}
\label{sec:PK-economic-policy}
与主流经济学衍生出的经济政策相比,根据PK所得的经济增侧往往有些反直觉。总的说来,PK认为市场运作在短期和长期都存在失灵和低效率情况,持干预市场运作的立场。
