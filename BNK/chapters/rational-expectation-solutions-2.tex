%!TEX root = ../DSGEnotes.tex
%\subsubsection{伪逆矩阵}
%\label{sec:simple-pseudo-intro}


\begin{subappendices}
%\section{Some title for an appendix-1}

\section{Schur补和对称正(半)定矩阵}
%\lipsum[4]
\label{sec:simple-schur-decomp-matrix}

\subsection{Schur补}
\label{sec:simple-schur-decomp-method}
本节作为\cite[A.5.5.5]{Boyd:2004jr}的详细说明。假定矩阵$M \in \mathbb{R}^{n \times n}$可做如下分解
\begin{equation*}
  \underset{n \times n}{M} = \begin{bmatrix}
  \underset{p \times p}{A} & \underset{p \times q}{B} \\
  \underset{q \times p}{C} & \underset{q \times q}{D}
  \end{bmatrix}, \quad n= p + q, \quad p,q,n > 0.
\end{equation*}

Schur补的思路在于,通过去掉一个变量分块求解线性方程。假定一个线性系统$(x,y)$
\begin{subequations}
  \begin{align}
    \label{eq:simple-schur-equation-sys-ABc}
    &A x + B y = c, \\
    \label{eq:simple-schur-equation-sys-CDd}
    &C x + D y = d,
  \end{align}
\end{subequations}
改写为矩阵形式
\begin{equation}
  \label{eq:simple-schur-equation-matrix-form}
  \begin{bmatrix}
    A & B \\ C & D
  \end{bmatrix} \begin{bmatrix}
  x \\ y
\end{bmatrix} = \begin{bmatrix}
c \\ d
\end{bmatrix}.
\end{equation}
可以利用高斯消元法(Gaussian elimination)求得系统解。

\subsubsection{A的Schur补以及系统解}
假定$D$是可逆的,则\eqref{eq:simple-schur-equation-sys-CDd}改写为
\begin{equation*}
  y= D^{-1} \left( d - C x \right),
\end{equation*}
代回\eqref{eq:simple-schur-equation-sys-ABc}替代$y$得
\begin{equation*}
    \left(A-BD^{-1}C\right) x = c - B D^{-1} d.
\end{equation*}

进一步假定$\left(A-BD^{-1}C\right)$是可逆的,我们有系统解
\begin{subequations}
  \begin{align}
    \label{eq:simple-schur-A-solution-x}
    &x = \left(\underbrace{A-BD^{-1}C}_{\text{A 的 Schur补}}\right)^{-1} \left(c - BD^{-1}d\right), \\
    \label{eq:simple-schur-A-solution-y}
    &y=D^{-1} d - D^{-1} C \left( \left(\underbrace{A-BD^{-1}C}_{\text{A 的 Schur补}}\right)^{-1} \left(c - BD^{-1}d\right) \right).
  \end{align}
\end{subequations}

我们将上式中标记出的$\left(A-BD^{-1}C\right)$定义为$M$中分块矩阵$A$的Schur补。

采用$A$的Schur补,我们有
\begin{equation*}
  \begin{split}
    &x = \left( A - BD^{-1} C \right)^{-1} c - \left( A - BD^{-1} C \right)^{-1} BD^{-1} d, \\
    &y=-D^{-1} C \left( A - BD^{-1} C \right)^{-1} c + \left( D^{-1} + D^{-1} C \left( A - BD^{-1} C \right)^{-1} B D^{-1} \right) d,
  \end{split}
\end{equation*}
改写为矩阵形式
\begin{equation}
  \label{eq:simple-schur-A-rewrite-M}
  \begin{bmatrix}
    x \\ y
  \end{bmatrix} =
  \begin{bmatrix}
    \left( A - BD^{-1} C \right)^{-1} & - \left( A - BD^{-1} C \right)^{-1} BD^{-1} \\
    -D^{-1} C \left( A - BD^{-1} C \right)^{-1} & \left(
      D^{-1} + D^{-1} C \left( A - BD^{-1} C \right)^{-1} B D^{-1}
    \right)
  \end{bmatrix}
  \begin{bmatrix}
    c \\ d
  \end{bmatrix},
\end{equation}
与\eqref{eq:simple-schur-equation-matrix-form}联立不难看出,
\begin{equation}
  \label{eq:simple-schur-equation-matrix-inverse}
  \begin{bmatrix}
    A & B \\ C & D
  \end{bmatrix}^{-1}
  =
  \begin{bmatrix}
    \left( A - BD^{-1} C \right)^{-1} & - \left( A - BD^{-1} C \right)^{-1} BD^{-1} \\
    -D^{-1} C \left( A - BD^{-1} C \right)^{-1} & \left(
      D^{-1} + D^{-1} C \left( A - BD^{-1} C \right)^{-1} B D^{-1}
    \right)
  \end{bmatrix}
\end{equation}
对式右侧做matrix reflection可得
\begin{equation*}
  \begin{split}
    & \begin{bmatrix}
      \left( A - BD^{-1} C \right)^{-1} & - \left( A - BD^{-1} C \right)^{-1} BD^{-1} \\
      -D^{-1} C \left( A - BD^{-1} C \right)^{-1} & \left(
        D^{-1} + D^{-1} C \left( A - BD^{-1} C \right)^{-1} B D^{-1}
      \right)
    \end{bmatrix} \\
    =& \begin{bmatrix}
    \left( A - BD^{-1} C \right)^{-1} & 0 \\
    -D^{-1} C \left( A - BD^{-1} C \right)^{-1} & D^{-1}
    \end{bmatrix}
    \begin{bmatrix}
      I & -BD^{-1} \\
      0 & I
    \end{bmatrix} \\
    =& \begin{bmatrix}
    I & 0 \\
    -D^{-1} C & I
    \end{bmatrix}
    \begin{bmatrix}
      \left( A - BD^{-1} C \right)^{-1} & 0 \\
      0 & D^{-1}
    \end{bmatrix}
    \begin{bmatrix}
      I & -BD^{-1} \\
      0 & I
    \end{bmatrix},
  \end{split}
\end{equation*}
并且第三行三个矩阵中,上下三角矩阵都是可逆的;中间对角矩阵对角元素也是可逆的。

代回\eqref{eq:simple-schur-equation-matrix-inverse}右侧,且等式两侧同时求逆运算,可以得到$M$的解
\begin{equation}
  \label{eq:simple-schur-equation-matrix-M}
  M = \begin{bmatrix}
    A & B \\ C & D
  \end{bmatrix}= \begin{bmatrix}
    I & BD^{-1} \\
    0 & I
  \end{bmatrix}
  \begin{bmatrix}
    A - BD^{-1} C  & 0 \\
    0 & D
  \end{bmatrix}
  \begin{bmatrix}
  I & 0 \\
  D^{-1} C & I
  \end{bmatrix}.
\end{equation}

即,利用A的Schur补求解系统,$(x,y)$的值由\eqref{eq:simple-schur-A-solution-x}-\eqref{eq:simple-schur-A-solution-y}给出,$M$的值由\eqref{eq:simple-schur-equation-matrix-M}给出。这种算法的好处在于,在假定$D$是可逆矩阵的情况下,只需要求得$A$的Schur补,即可求解系统。

\subsubsection{D的Schur补以及系统解}
类似地,若假定$A$是可逆的,
\eqref{eq:simple-schur-equation-sys-ABc}改写为
\begin{equation*}
  x= A^{-1} \left( c - B y \right),
\end{equation*}
代回\eqref{eq:simple-schur-equation-sys-CDd}替代$x$得
\begin{equation*}
    \left( D - CA^{-1} B \right) y = D - C A^{-1} c.
\end{equation*}

进一步假定$\left( D - CA^{-1} B \right)$是可逆的,我们有系统解
\begin{subequations}
  \begin{align}
    \label{eq:simple-schur-D-solution-y}
    &y = \left( \underbrace{D - CA^{-1} B}_{\text{D 的 Schur补}}\right) \left( c - BD^{-1} d \right)\\
    \label{eq:simple-schur-D-solution-x}
    &y=A^{-1} c - A^{-1} B \left( \left(\underbrace{D-CA^{-1}B}_{\text{D 的 Schur补}}\right)^{-1} \left(d - CA^{-1}c\right) \right).
  \end{align}
\end{subequations}

我们将上式中标记出的$\left(D - CA^{-1} B \right)$定义为$M$中分块矩阵$D$的Schur补。

采用$D$的Schur补,我们有
\begin{equation*}
  \begin{split}
    &x = \left(
    A^{-1} + A^{-1} B \left(D - C A^{-1} B\right)^{-1} C A^{-1} \right)
    c -
    \left(
    A^{-1} B  \left(D - C A^{-1} B\right)^{-1}
    \right) d, \\
    &y = - \left(D - C A^{-1} B\right)^{-1} C A^{-1} c +
    \left(D - C A^{-1} B\right)^{-1} d,
  \end{split}
\end{equation*}
改写为矩阵形式
\begin{equation}
  \label{eq:simple-schur-D-rewrite-M}
  \begin{bmatrix}
    x \\ y
  \end{bmatrix} =
  \begin{bmatrix}
    A^{-1} + A^{-1} B \left(D - C A^{-1} B\right)^{-1} C A^{-1} &
    -  A^{-1} B  \left(D - C A^{-1} B\right)^{-1} \\
    - \left(D - C A^{-1} B\right)^{-1} C A^{-1} &
    \left(D - C A^{-1} B\right)^{-1}
  \end{bmatrix}
  \begin{bmatrix}
    c \\ d
  \end{bmatrix},
\end{equation}
与\eqref{eq:simple-schur-equation-matrix-form}联立可见
\begin{equation}
  \label{eq:simple-schur-D-equation-matrix-inverse}
  \begin{bmatrix}
    A & B \\ C & D
  \end{bmatrix}^{-1}
  =
  \begin{bmatrix}
    A^{-1} + A^{-1} B \left(D - C A^{-1} B\right)^{-1} C A^{-1} &
    -  A^{-1} B  \left(D - C A^{-1} B\right)^{-1} \\
    - \left(D - C A^{-1} B\right)^{-1} C A^{-1} &
    \left(D - C A^{-1} B\right)^{-1}
  \end{bmatrix}
\end{equation}
对式右侧做matrix reflection可得
\begin{equation*}
  \begin{split}
    &\begin{bmatrix}
      A^{-1} + A^{-1} B \left(D - C A^{-1} B\right)^{-1} C A^{-1} &
      -  A^{-1} B  \left(D - C A^{-1} B\right)^{-1} \\
      - \left(D - C A^{-1} B\right)^{-1} C A^{-1} &
      \left(D - C A^{-1} B\right)^{-1}
    \end{bmatrix} \\
    =&\begin{bmatrix}
    A^{-1} & - A^{-1} B \left(D - C A^{-1} B\right)^{-1} \\
    0 & \left(D - C A^{-1} B\right)^{-1}
    \end{bmatrix}
    \begin{bmatrix}
      I & 0 \\
      -CA^{-1} & I
    \end{bmatrix} \\
    =& \begin{bmatrix}
    I & - A^{-1} B \\
    0 & I
    \end{bmatrix}
    \begin{bmatrix}
      A^{-1} & 0 \\
      0 & \left(D - C A^{-1} B\right)^{-1}
    \end{bmatrix}
    \begin{bmatrix}
      I & 0 \\
      -C A^{-1} & I
    \end{bmatrix},
  \end{split}
\end{equation*}
同样地,第三行三个矩阵中,上下三角矩阵都是可逆的;中间对角矩阵对角元素也是可逆的。

代回\eqref{eq:simple-schur-D-equation-matrix-inverse}右侧,且等式两侧同时求逆运算,可以得到$M$的解
\begin{equation}
  \label{eq:simple-schur-D-equation-matrix-M}
  M = \begin{bmatrix}
    A & B \\ C & D
  \end{bmatrix}= \begin{bmatrix}
    I & CA^{-1} \\
    0 & I
  \end{bmatrix}
  \begin{bmatrix}
    A  & 0 \\
    0 & D - CA^{-1} B
  \end{bmatrix}
  \begin{bmatrix}
  I & 0 \\
  A^{-1} B & I
  \end{bmatrix}.
\end{equation}

\subsection{利用Schur分解法判断可逆正定矩阵}
对于符合第\ref{sec:simple-schur-decomp}描述的可逆矩阵$M$,假定其分块对角$A,D$都是可逆的,并且$C=B^T$,那么可以将其改写为如下类似于分块——对角矩阵的形式
\begin{equation}
  \label{eq:simple-schulr-howto-M}
  \begin{split}
  M &= \begin{bmatrix}
  A & B \\ B^T & C
  \end{bmatrix} \\
  &= \begin{bmatrix}
  I & BC^{-1} \\
  0 & I
  \end{bmatrix}
  \begin{bmatrix}
    A - BC^{-1}B^T & 0 \\
    0 & C
  \end{bmatrix}
  \begin{bmatrix}
    I & BC^{-1} \\
    0 & I
  \end{bmatrix}^{T}.
\end{split}
\end{equation}
其中第二行利用了$A$的Schur补。我们可以利用这一分块对角形式,查看对称矩阵$M$是否是正定的,即$M \succ 0$。

\begin{proposition}
  对于任一对称矩阵$M = \begin{bmatrix}
  A & B \\ B^{\top} &C \end{bmatrix} $,如果$C$是可逆的,则
  \begin{enumerate}
    \item 当且仅当\begin{equation*}
    \begin{cases}
      &C \succ 0, \quad \text{并且}\\
      &A-BC^{-1}B^{\top} \succ 0
    \end{cases}
    \end{equation*}时,我们有$M \succ 0$。
    \item 如果$C \succ 0$,那么当且仅当$A-BC^{-1}B^{\top} \succeq 0$时,我们有$M \succeq 0$。
  \end{enumerate}
\end{proposition}
\begin{proof}
  略。
\end{proof}

\begin{proposition}
  对于任一对称矩阵$M = \begin{bmatrix}
  A & B \\ B^{\top} & C \end{bmatrix} $,如果$A$是可逆的,则
\begin{enumerate}
  \item 当且仅当\begin{equation*}
  \begin{cases}
    &A \succ 0, \quad \text{并且}\\
    &C-B^{\top}AB \succ 0
  \end{cases}
  \end{equation*}时,我们有$M \succ 0$。
  \item 如果$A \succ 0$,那么当且仅当$C-B^{\top}AB \succeq 0$时,我们有$M \succeq 0$。
\end{enumerate}
\end{proposition}
\begin{proof}
  略。
\end{proof}

\subsection{利用Schur分解法判断可逆正半定矩阵}

当$C$和$A$都是奇异分块矩阵,我们无法直接求得$C^{-1}$和$A^{-1}$。我们首先介绍判断方法,随后用该方法判断$M$是否是正半定的。
\subsubsection{判定方法}

\begin{enumerate}
  \item (通常是nonconvex)的quadratic优化问题:对于满足一定形式的$x$和(非奇异)的$P$,式
  \begin{equation*}
    \min_{\{x\}} f(x) = \frac{1}{2} x^{\top} P x^{\top} + x^{\top} b
  \end{equation*}
  是否存在最小值,以及最小值是多少,见Proposition \ref{prop:simple-schur-quadratic-optim}。
  \item 对于奇异的$P$,我们要计算其对角分块的伪逆矩阵和相应的Schur补,进而判断最小值,见Proposition \ref{prop:simple-schur-singular-optim}。
  %我们要计算其伪逆矩阵$D^{\dagger}$,求得关于$A$的的Schur补$\left( A-BD^{\dagger}B^{\top} \right)$,进而判断$M$。或者,计算伪逆矩阵$A^{\dagger}$,求得关于$D$的Schur补$\left(D - B^{\top} A^{\dagger} B\right)$,进而判断$M$。\todo{refer to proposition 4.2}
\end{enumerate}

\begin{proposition}[非奇异系数矩阵(quadratic优化问题)]
  \label{prop:simple-schur-quadratic-optim}
如果$P$是一个对称的非奇异矩阵,那么只有当$P \succeq 0$时,方程$f(x) = \left(\frac{1}{2}\right) x^{\top} P x + x^{\top} b$才具有唯一的最优(极小)值$x^{\star}=P^{-1} b$,使得$f(x^{\star}) = f(P^{-1} b) = -\frac{1}{2} b^{\top} P^{-1} b$。
\end{proposition}
\begin{proof}
  已知
  \begin{equation*}
    \frac{1}{2}\left( x + P^{-1} b \right)^{\top} P \left( x + P^{-1} b \right) = \frac{1}{2} x^{\top} P x + x ^{\top} b + \frac{1}{2} b^{\top} P^{-1} b,
  \end{equation*}
  因此我们有
  \begin{equation}
    \label{eq:simple-schur-quadratic-Pb}
    f(x) = \frac{1}{2} x^{\top} P x + x^{\top} b = \frac{1}{2}\left( x + P^{-1} b \right)^{\top} P \left( x + P^{-1} b \right) - \frac{1}{2} b^{\top} P^{-1} b.
  \end{equation}

  如果假定$P$中有负的特征根$-\lambda, (\lambda > 0)$,对应特征向量$u$,根据定义我们有$P u = - \lambda u$。另一方面,设$x + P^{-1} b := \alpha u$,其中$\alpha$是任一不为$0$的实数。则我们有
  \begin{equation*}
\begin{split}
&      \left( x + P^{-1} b \right)^{\top} P \left( x + P^{-1} b \right) =\left( \alpha u \right)^{\top} P \left( \alpha u \right) = \alpha^2 u^{\top} P u = \alpha^2 u^{\top} (-\lambda u) = - \alpha^2 \lambda \left\Vert u \right\Vert^{2},
\end{split}
  \end{equation*}
其中$\left\Vert N \right\Vert$表示向量的范数(norm)。则\eqref{eq:simple-schur-quadratic-Pb}进一步改写为
\begin{equation}
  \label{eq:simple-schur-quadratic-uPb}
  f(x) = \underbrace{- \frac{1}{2} \alpha^2 \lambda \left\Vert u \right\Vert^{2}}_{\ge 0} - \frac{1}{2}b^{\top} P^{-1} b,
\end{equation}
由于$\alpha$的取值可以至任意大,在$P$存在负的特征根的情况下$f(x)$不可能有最小值。

因此,$f(x)$最小值存在的先决条件是满秩矩阵$P \succeq 0$。在$P \succeq 0$并且\eqref{eq:simple-schur-quadratic-Pb}右侧第一部分恒大于等于$0$(由\eqref{eq:simple-schur-quadratic-uPb}右侧第一部分恒大于等于$0$推得)的情况下,$f(\cdot)$的最小值只能出现在$x^{\star} = - P^{-1} b$,对应$f(x^{\star}) = - \frac{1}{2} b^{\top} P^{-1} b$。
\end{proof}

\begin{proposition}[奇异矩阵]
  \label{prop:simple-schur-singular-optim}
如果$P$是一个对称的奇异矩阵,那么只有当$P \succeq 0$并且$\left( I-PP^{\dagger}\right) b = 0$时,方程$f(x) = \left(\frac{1}{2}\right) x^{\top} P x + x^T b$才具有唯一的最优(极小)值$x^{\star} = - P^{\dagger} b$,使得$f(x^{\star}) = -\frac{1}{2} b^{\top} P^{\dagger} b$。

进而,如果$P=U^T \Sigma U$是一个关于$P$的SVD,那么系统最优值$f(x^{\star})$对应$x \in \mathbb{R}^{n \times n}$:
\begin{equation*}
  x^{\star} = -P^{\dagger} b + U^{\top} \begin{bmatrix}0 \\ z \end{bmatrix}, \quad \text{对于任一 } \quad z \in \mathbb{R}^{n-r \times n-r}, \quad r:=rank(P).
\end{equation*}
\end{proposition}
\begin{proof}
  对不满秩$(r < n)$的奇异矩阵$P$做分块对角化
  \begin{equation*}
    P = U^{\top} \begin{bmatrix}
    \Sigma_r & 0 \\ 0 & 0
    \end{bmatrix} U,
  \end{equation*}
  其中$U$是正交矩阵,$\Sigma_r \in \mathbb{R}^{r \times r}$是可逆对角矩阵。由此可得
  \begin{equation*}
  \begin{split}
    f(x) &= \frac{1}{2} x^{\top} P x + x^{\top} b \\
    &= \frac{1}{2} x^{\top} U^{\top} \begin{bmatrix}
    \Sigma_r & 0 \\ 0 & 0
  \end{bmatrix} U x + x^{\top} U^{\top} U b \\
  &= \frac{1}{2} \left( U x \right)^{\top} \begin{bmatrix}
  \Sigma_r & 0 \\ 0 & 0
\end{bmatrix} \left(U x\right) + \left( U x \right)^{\top} U b.
  \end{split}
  \end{equation*}

  定义
  \begin{equation*}
    U x :=
    \begin{bmatrix}
    y \\ z
    \end{bmatrix},
    \quad U b :=
    \begin{bmatrix}
    c \\ d
  \end{bmatrix},
  \quad y,c \in \mathbb{R}^{r},
  \quad z,d \in \mathbb{R}^{(n-r) },
  \end{equation*}
  上式可进一步调整为
  \begin{equation}
    \label{eq:simple-schur-fx-y-c-d}
  \begin{split}
    f(x) &= \frac{1}{2} \begin{bmatrix}
    y^{\top} & z^{\top}
    \end{bmatrix} \begin{bmatrix}
  \Sigma_r & 0 \\ 0 & 0
  \end{bmatrix} \begin{bmatrix}
  y \\ z
  \end{bmatrix} + \begin{bmatrix}
  y^{\top} & z^{\top}
  \end{bmatrix} \begin{bmatrix}
  c \\ d
  \end{bmatrix} \\
  &= \frac{1}{2} y^{\top} \Sigma_r y + y^{\top} c + z^{\top} d.
  \end{split}
  \end{equation}

  根据\eqref{eq:simple-schur-fx-y-c-d},我们逐次分析$f(x)$最小值存在的两个条件:
  \begin{enumerate}
    \item 对于$y =0$的情况,我们有$f(x) = z^{\top} d$,此时如果$d \neq 0$取任意值,$f(x)$将没有最小值。因此为了让$f(x)$有最小值,需要使$d = 0$。根据第\ref{sec:simple-pseudo}节可得,$U b= \begin{bmatrix} c \\ 0 \end{bmatrix}$,进而$b = range(P)$\footnote{第\ref{sec:simple-pseudo}节中的$U$对应本节中的$U^{\top}$。};因此$f(x)$存在最小值的条件之一是$b=0$,等价于$\left(I-PP^{\dagger}\right) b = 0$。
    \item  将$b=0$代回原式,我们有$f(x) = \frac{1}{2} y^{\top} \Sigma_r y + y^{\top} c$,由于分块矩阵$\Sigma_{r}$是可逆的,根据Proposition \ref{prop:simple-schur-quadratic-optim}, $f(x)$存在最小值的第二个条件是$\Sigma_{r} \succeq 0$,等价于$P \succeq 0$。
  \end{enumerate}

  假定$f(x)$最小值存在的两个条件都得到了满足。来看$x^{\star}$的取值。
\begin{enumerate}

  \item 对于$d=0$,\begin{equation*}
  \frac{\partial f}{\partial y} = \Sigma_r y + c = 0,
  \end{equation*}
  可得$y^{\star} = -\Sigma_r ^{-1} c$.

  \item 假定$z=0$,\begin{equation*}
  U x^{\star} = \begin{bmatrix}
  y^{\star} \\ z
\end{bmatrix} = \begin{bmatrix}
  -\Sigma_r ^{-1} c \\ 0
\end{bmatrix},
  \end{equation*}
由正交矩阵性质我们有$U^{\top} = U^{-1}$,进而
\begin{equation}
\label{eq:simple-schulr-optimum-x-value}
\begin{split}
    x^{\star} &= - U^{-1}  \Sigma_r ^{-1} c \\
    &= - U^{\top} \begin{bmatrix}
    \Sigma_{r}^{-1} & 0 \\
    0 & 0
    \end{bmatrix}
    \begin{bmatrix}
      c \\ 0
    \end{bmatrix} \\
    &= - U^{\top} \begin{bmatrix}
    \Sigma_{r}^{-1} & 0 \\
    0 & 0
    \end{bmatrix} Ub \\
    &= -P^{\dagger} b,
\end{split}
\end{equation}
并且不难看出,任一$z \in R^{n-r}$对$x^{\star}=-P^{\dagger} b + U^{\top} \begin{bmatrix} 0 \\ z \end{bmatrix}$的取值都不产生影响。
\end{enumerate}

进一步看最优极小值$f(x^{\star})$
\begin{equation}
  \label{eq:simple-schulr-optimum-y-value}
\begin{split}
    f(x^{\star}) &= \frac{1}{2} (x^{\star})^{\top} P x^{\star} + (x^{\star})^{\top} b \\
    &= \frac{1}{2} \left(-P^{\dagger} b\right)^{\top} P \left(-P^{\dagger} b\right) + \left(-P^{\dagger} b\right)^{\top} b \\
    &= \frac{1}{2}  b^{\top} (P^{\dagger})^{\top} P P^{\dagger} b - b^{\top} (P^{\dagger})^{\top} b \\
    &= b^{\top} \left(-\frac{1}{2} (P^{\dagger})^{\top} \right) b \\
    &= -\frac{1}{2} b^{T} P^{\dagger} b,
\end{split}
\end{equation}
并且同样地,任一$z \in R^{n-r}$对$f(x^{\star})$的取值也不产生影响。
\end{proof}

\subsubsection{判断$M$是否正半定}
在此基础上,回到最初的问题上来:如何判断对称矩阵$M = \begin{bmatrix} A & B \\ B^{\top} &C\end{bmatrix}$是正半定矩的。

这个问题等同于:判断方程$f(x,y)$是否有最小值:
\begin{equation*}
  f(x,y) = \begin{bmatrix} x^{\top} & y^{\top} \end{bmatrix}  \begin{bmatrix} A & B \\ B^{\top} & C\end{bmatrix} \begin{bmatrix} x \\ y \end{bmatrix}
\end{equation*}

根据Proposition \eqref{prop:simple-schur-singular-optim}可知:对于给定的常数向量$y$,只有当$A \succeq 0, \left( I - A A^{\dagger} \right) B y=0, C-B^{\top} A^{\dagger} B \succeq 0$时,$f$存在最小值,最小值为
\begin{equation*}
  f(x^{\star}, y ) = - y^{\top} B^{\top} A^{\dagger} B y + y^{\top} C y = y^{\top} \left( C - B^{\top} A^{\top} B\right)y.
\end{equation*}

\begin{theorem}
  \label{theorem:simple-schur-positivedef}
  对于任一对称矩阵$M = \begin{bmatrix} A & B \\ B^{\top} & C\end{bmatrix}$,下述条件之间等价
  \begin{enumerate}
    \item $M \succeq 0$,
    \item $A\succeq 0, (I-AA^{\dagger} B) =0, C- B^{\top} A^{\dagger} B \succeq 0$,
    \item $C \succeq 0, (I-CC^{\dagger} B = 0), A - B^{\top} C^{\dagger} B \succeq 0$,
  \end{enumerate}
  并且可以对M做如下矩阵分解
  \begin{equation*}
    M = \begin{cases}
      &  = \begin{bmatrix}
      I & BC^{\dagger} \\ 0 & I
      \end{bmatrix}
      \begin{bmatrix}
        A - BC^{\dagger} B^{T} & 0 \\ 0 & C
      \end{bmatrix}
      \begin{bmatrix}
        I & 0 \\ C^{\dagger} B^{T} & I
      \end{bmatrix}, \quad \text{或}\\
      & = \begin{bmatrix}
      I & 0 \\ B^{\top} A^{\dagger} & I
      \end{bmatrix}
      \begin{bmatrix}
        A & 0  \\ 0 & C-B^{\top} A^{\dagger} B
      \end{bmatrix}
      \begin{bmatrix}
        I & A^{\dagger} B \\ 0 & I
      \end{bmatrix}.
    \end{cases}
  \end{equation*}
\end{theorem}

\section{伪逆矩阵}
\label{sec:simple-pseudo}
%\lipsum[4]
又称Moore-Penrose Pseudo inverse。可见如\cite[Ch.5]{Bapat:2012ue}。

\subsection{用SVD法做伪逆矩阵分解}
\label{sec:simple-pseudo-svd}
一个方块矩阵$M \in \mathbb{R}^{n \times n}$可以做如下SVD分解(singular value decomposition),以求得伪逆矩阵$M^{\dagger}$
\begin{equation*}
  M = U \Sigma V^{\top},
\end{equation*}
其中
\begin{itemize}
  \item $U, V$是正交矩阵,$U$和$V$的column分别是$MM^{\top}$和$M^{\top}M$的特征向量。需要指出的是,对$M$做SVD,得到的$U$和$V$并不唯一。
  \item 对角矩阵$\Sigma = diag (\sigma_1, \sigma_2, \ldots \sigma_r, 0, 0,\ldots 0)$:
  \begin{itemize}
    \item $\sigma_{1}, \sigma_2, \ldots \sigma_r$表示$M$的秩,并且$\sigma_1 \ge \sigma_2 \ge \ldots \sigma_r$。
    \item $\sigma_i$ 表示$M$的奇异值。 $\sigma_i, i=0,1, \ldots r$表示$MM^{\top}$和$M^{\top}M$的非零特征值的平方根。
  \end{itemize}
\end{itemize}

如果$M = U \Sigma V^{\top}$是某个$M$的SVD,则我们将$M$的伪逆矩阵$M^{\dagger}$定义如下
\begin{equation}
  M^{\dagger} = V \Sigma^{\dagger} U^{\top}, \quad \Sigma^{\dagger} = diag \left(\sigma_1^{-1}, \sigma_2^{-1}, \ldots \sigma_r^{-1}, \ldots 0, 0, \ldots 0 \right).
\end{equation}

不难看出,当$M$是满秩$(r = n)$的(等价于$M$是可逆的)时,$M^{\dagger} = M^{-1}$。因此$M^{\dagger}$是个$M$的``广义逆矩阵''。对$M$做SVD,$U$和$V$有所不同,但同一$M$只有唯一的$M^{\dagger}$与之相对应。此外,根据伪逆矩阵的性质,我们有$MM^{\dagger}M = M$, $M^{\dagger} M M^{\dagger} = M^{\dagger}$,并且$M$和$M^{\dagger}$都是对称矩阵;并且
\begin{equation*}
  \begin{cases}
    MM^{\dagger} &=  U \Sigma V^{\top} V \Sigma^{\top} U^{\top} = U
    \Sigma \Sigma^{\top} U^{\top} = U \begin{bmatrix}
    I_r & 0 \\ 0 & 0_{n-r}
  \end{bmatrix} U^{\top}, \\
  M^{\dagger} M &= V \Sigma^{\dagger} U^{\top} U \Sigma V^{\top} = V \Sigma^{\dagger} \Sigma V^{\top} = V \begin{bmatrix}
  I_r & 0 \\ 0 & 0_{n-r}
\end{bmatrix} V^{\top},
  \end{cases}
\end{equation*}
因此我们有
\begin{equation*}
  \begin{cases}
    \left(M M^{\dagger}\right)^2 &= MM^{\dagger}, \\
    \left(M^{\dagger} M \right)^2 &= M^{\dagger}M,
    \end{cases}
\end{equation*}
即$\left(M M^{\dagger}\right)$和$\left(M^{\dagger} M \right)$是对称的正交投影:
\begin{enumerate}
  \item $MM^{\dagger}$是$range(M)$的正交投影,
  \item $M^{\dagger} M$是$ker(M)^{\bot}$的正交投影,$ker(M)^{\bot}$表示$ker(M)$的正交补(orthogonal complement)。
\end{enumerate}

\begin{proof}
  由于$range (M^{\dagger} M) \subseteq range(M)$,即$M^{\dagger} M$的值域是$M$值域的子集,则在$range(M)$值域中,对于任一$y = M x$,由于$MM^{\dagger} M = M$,我们有
  \begin{equation*}
    MM^{\dagger} y = MM^{\dagger} M x= M x = y,
  \end{equation*}
  因此$range(MM^{\dagger}) \subseteq range(M)$。此外,根据$MM^{\dagger} M = M \Rightarrow ker(M^{\dagger} M) \subseteq ker(M)$。因而有$ker(M^{\top}) = ker(M)$。

  由于$M^{\dagger} M$是埃米特矩阵(Hermitian),$range(M^{\dagger} M) = ker(M^{\dagger} M)^{\bot} = ker(M)^{\bot}$。
\end{proof}

值域$range (M) = range(MM^{\dagger})$中包括所有向量$y \in \mathbb{R}^{n \times n}$,满足$U^T y = \begin{bmatrix} z \\ 0 \end{bmatrix}, z \in \mathbb{R}^{r}$。
\begin{proof}
  对于$y=Mx$,我们有
  \begin{equation*}
      U^{\top} y = U^{\top} M x = U^{\top} U \Sigma V^{\top} x = \Sigma V^{\top} x = \begin{bmatrix}
      \Sigma_{r} & 0 \\ 0 & 0_{n-r}
    \end{bmatrix} V^{\top}  x = \begin{bmatrix} z \\ 0 \end{bmatrix},
  \end{equation*}
  反过来,如果$U^{\top} y = \begin{bmatrix} z \\ 0 \end{bmatrix}$,则我们首先有$y = U \begin{bmatrix} z \\ 0 \end{bmatrix}$,进而
  \begin{equation*}
    \begin{split}
      MM^{\dagger} y &= U \begin{bmatrix}
      I_r & 0 \\ 0 & 0_{n-r}
    \end{bmatrix} U^{\top} y \\
    &= U \begin{bmatrix}
    I_r & 0 \\ 0 & 0_{n-r}
  \end{bmatrix} U^{\top} U
  \begin{bmatrix}
  z \\ 0
  \end{bmatrix} \\
  & = U \begin{bmatrix}
  I_r & 0 \\ 0 & 0_{n-r}
\end{bmatrix}
  \begin{bmatrix}
  z \\ 0
  \end{bmatrix} \\
  & = U \begin{bmatrix}
  z \\ 0
  \end{bmatrix},
    \end{split}
  \end{equation*}
这说明$y \in range(M)$,即$y$在$M$的值域之中。
\end{proof}

类似地,值域$range(M^{\dagger} M) = ker(M)^{\bot}$中包括所有向量$y \in \mathbb{R}^{n \times n}$,满足$V^{\top} y = \begin{bmatrix} z \\ 0 \end{bmatrix}, z \in \mathbb{R}^{r \times r}$。
\begin{proof}
  对于$y = M^{\dagger} M U$,我们有
  \begin{equation*}
    y = M^{\dagger} M U = V \begin{bmatrix}
    I_r & 0 \\ 0 & 0_{n-r}
    \end{bmatrix}
    V^{\top} U = V \begin{bmatrix} z \\ 0 \end{bmatrix}.
  \end{equation*}

  反过来,如果$V^{\top} y = \begin{bmatrix} z \\ 0 \end{bmatrix}$, 则$y = \begin{bmatrix} z \\ 0 \end{bmatrix}$,进而
  \begin{equation*}
    M^{\top} M V \begin{bmatrix} z \\ 0 \end{bmatrix} = V \begin{bmatrix}
    I_r & 0 \\ 0 & 0_{n-r}
  \end{bmatrix} V^{\top} V \begin{bmatrix} z \\ 0 \end{bmatrix} = V \begin{bmatrix}
  I_r & 0 \\ 0 & 0_{n-r}
\end{bmatrix} \begin{bmatrix} z \\ 0 \end{bmatrix} = y.
  \end{equation*}

  这说明$y \in range(M^{\dagger} M)$。
\end{proof}

\subsection{用SVD法做 M 的矩阵分解}
\label{sec:simple-pseudo-svd-M}
对于$M$是一个对称实矩阵的情况。每一个$M=U \Sigma V^{\top}$的SVD分解常常只有不超过1个$\Sigma$,但有多个$U$和$V$,并且通常来说不存在$U=V$的情况。然而,如果对称矩阵 $M \succeq 0$,那么
\begin{enumerate}
  \item $M$的特征值非负,
  \item $M$的非零特征值数量,与奇异值数量相等,
  \item 可以用SVD法分解:$M=U \Sigma V^{\top}$。
\end{enumerate}
如果$M$是一个对称复矩阵,结果与之相类似,除了此时\begin{enumerate}
\item $U$和$V$是埃尔米特矩阵,并且
\item $M^{\top} M$和$M M^{\top}$是埃尔米特正交映射。
\end{enumerate}

对于$M$是一个对称正规矩阵(normal matrix)的情况,即方块矩阵满足$M M^{\top} = M^{\top} M$,此时$M$的SVD分解与分块对角分解之间关系密切,并且$M$的伪逆矩阵可以直接由分块对角分解求得。
\begin{proof}
  对实正规矩阵$M$做分块对角分解:
  \begin{equation*}
  M = U \Lambda U^{\top},
  \end{equation*}
  其中$U$是正交矩阵,$\Lambda = diag\left(B_1, B_2, \ldots B_n \right)$是实系数分块对角矩阵。

  分块$B_j, j=0,1,\ldots n$可以是下属两种形式之一。第一是一个$2 \times 2$的矩阵$ B_j = \begin{bmatrix} \lambda_{j} & - \mu_j \\ \mu_j & \lambda_j \end{bmatrix}$;第二是一个一维分块$B_j = \left( \lambda_j \right)$。

  假定$B_1, B_2, \ldots B_p$是前一种形式,且$\lambda_{2p+1}, \lambda_{n}$是标量。我们已知$\lambda_j \pm i \mu_j$,并且$\lambda_{2p + k}$表示$A$的特征值。

  令$\rho_{2j-1} = \rho_{2j} = \left( \lambda_j^2 + \mu_j^2 \right)^{1/2}\text{ for }j = 1, \ldots p$,以及$\rho _{2p+j} = \lambda_j$,并且假定这些分块已经按照降序排列$\rho_1 \ge \rho_2 \ge \ldots \ge \rho_n$。那么可见
  \begin{equation*}
    UU^{\top} = U^{\top} U = U \Lambda U^{\top} U \Lambda^{\top} U^{\top} = U \Lambda \Lambda^{\top} U^{\top},
  \end{equation*}
  并且
  \begin{equation*}
    \Lambda \Lambda^{\top} = diag(\rho_1^2, \rho_2^2, \ldots \rho_n^2),
  \end{equation*}
  因此可见:$M$的奇异值$\sigma_1 \ge \sigma_2 \ge \ldots \ge \sigma_n$就是矩阵$MM^{\top}$特征值的非负平方根$\rho_1 \ge \rho_2 \ge \ldots \rho_n$,满足$\sigma_j = \rho_j, 1 \le j \le n$。

  接着定义对角矩阵
  \begin{equation*}
    \Sigma = diag(\sigma_1, \sigma_2, \ldots \sigma_r, 0, 0, \ldots ,0),
  \end{equation*}
  其中$r = rank(M)$,$\sigma_1 \ge \sigma_2 \ge \ldots \ge \sigma_r >0$,并且定义
  \begin{equation*}
    \Theta = diag(\sigma_1^{-1} B_1, \sigma_2^{-1} B_2, \ldots \sigma_p^{-1} B_p, 1, 1, \ldots, 1).
  \end{equation*}

  由此可见,$\Theta$是个正交矩阵,并且
  \begin{equation*}
    \Lambda = \Theta \Sigma = (B_1, B_2, \ldots, B_p, \lambda_{2p+1}, \lambda_{2p+2}, \ldots, \lambda_r, 0, 0, \ldots, 0).
  \end{equation*}

  进而
  \begin{equation*}
    \begin{split}
      M &= U \lambda U^{\top} = U \Theta \Sigma U^{\top}.
    \end{split}
  \end{equation*}

  由于$U$和$\Theta$都是正交矩阵,因此$V \equiv U \Theta$也是正交矩阵,并且$M = V \Sigma U^{\top}$是$M$的SVD。

  根据SVD的性质我们有
  \begin{equation*}
    M = V \Sigma U^{\top} = U \Sigma^{\dagger} V^{\top} = U \Sigma^{\dagger} \Theta^{\top} U^{\top},
  \end{equation*}
  其中最后一个等式根据$V=U \Theta \Rightarrow V^{\top} = \Theta{\top} U^{\top}$。

  $\Theta$是正交矩阵$\Rightarrow$ $\Theta^{\top} = \Theta^{-1}$ $\Rightarrow$ $\Sigma^{\dagger} \Theta^{\top} = \Sigma^{\dagger} \Theta^{-1} = \Lambda^{\dagger}$

  因此我们有$M^{\dagger} = U \Lambda ^{\dagger} U^{\top}$。

  来看$\Lambda^{\dagger}$的性质。如果我们将$\Lambda_r$写为$\Lambda_r = \left( B_1, B_2, \ldots, B_p, \lambda_{2p+1}, \ldots, \lambda_r \right)$,则$\Lambda$是可逆矩阵,并且$\Lambda^{\dagger} = \begin{bmatrix} \Lambda_r^{-1} & 0 \\ 0 & 0 \end{bmatrix}$。

  由上可见,一个(实系数)正规矩阵$M$的伪逆矩阵,可以通过对$M$做分块对角分解求得。
\end{proof}


\end{subappendices}
