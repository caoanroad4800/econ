%!TEX root = ../DSGEnotes.tex
\chapter{Keynesian, New Keynesian and New Classical Economics}
\label{sec:K-NK-NCE}

主要参考自\cite{Greenwald:1987ei}。

简要说来。New Classical Economics学派认为,macro-theory缺乏micro基础,需要从一系列微观行为准则(理性、最大化的企业和个人行为)推导出经济总体层面上的动态;承认动态对于了解宏观行为的重要性;在分析动态行为时,期望扮演核心角色,如理性预期的形成。“理性预期学派”的问题并不在于其假定之一,理性的预期是不是可能,而在于另一个古典主义假定,市场出清。根据这样的假定,理性预期模型会得到如下结果:没有失业,政府宏观政策是不必要的。

另一大流派New Keynesian Economics学派,认为现实世界中真实存在的一系列经济现象,如失业、信贷配给、经济周期等,都是不能由标准micro-theory所解决的。因此致力于开发micro理论,来解释macro现象。NK学派的核心工作就是从micro和macro层面了解不充分信息和不完全市场及其后果。市场失灵中的表现之一是失业。与New Classical相比,NK同样追求建立一个single解释框架,但NK的框架致力于解释失业,而不像New Classical学派那样忽视甚至否定失业的存在。

\section{Keynesian的四个核心}
\label{sec:KNK-K-keys}
NK从traditional Keynesian经济学发展而来。Keynes对经济运行有着显著不同于standard neo-classical理论的看法,比如企业行为决策并不是基于理性的计算,而是"动物精神"。他用图形的形式表示他的观点,但当他试图构建理论模型去描述这一观点的时候,他更多回到了传统的neo classical框架中——这很有可能是因为他受过严格的neo-classical训练,束缚了他表达自己全新思想的能力。Hicks等后人将Keynes的观点进一步模型化系统化理论化,
成为理解失业、经济周期等问题的强有力工具。

Keynes的观点中,如下四点对于我们构建一个解释失业和经济周期波动的模型来说至关重要。

\subsection{失业与有效工资理论}
\label{sec:KNK-unemp-efficiency-wages}
第一点。该通用模型必须能够解释失业的持续存在,以及一系列关键经济变量的周期性变化。为了解释失业,就需要发展出一个关于劳动力市场的理论,Keynes从传统经济学思想(如fixed price学派)中借鉴了(名义,实际)工资粘性的假定,但Keynesian从经验和理论两个层面批评了Fixed price学派的工资粘性假定是不完善的。经验层面,大萧条中工资下跌了1/3,经历通货膨胀的国家即使调整了实际工资,仍然出现失业情况。理论层面,Fixed Price学派没能对工资粘性做出解释,没能提供制度上的设计。

根据Keynes的理论,工资并不需要是完全粘性的,而仅仅需要确保工资不会跌落到完全市场出清的水平,如有效工资理论(efficiency wage theories\citep{Stiglitz:1984ud,Stiglitz:1987bz})。有效工资理论的假定如下:
\begin{itemize}
\item 对异质劳动者属性特征的信息掌握是不充分的,
\item 劳动者的工作表现是无法充分监控的,
\item 无法与劳动者签订可以完全反映他劳动成果的工作合同。
\end{itemize}

%\end{enumerate}
根据这样的假定,有效工资理论认为:
\begin{itemize}
\item 劳动力的质量及生产率,企业的利润都可能随着工资的增加而增加,
\item 工资增加可能导致劳动者周转率降低,
\item 由于企业需要承担部分的劳动者周转成本,工资上升在某种程度上可以导致企业利润的上升。
\end{itemize}

然而当失业存在,工资仍可能不会降低,因为企业认识到如果他们降低员工工资,会导致生产率降低,周转率上升,利润降低。这是指充分竞争市场下,企业只在一定范围内是工资制定者的情况。如果瓦尔拉斯工资(Walrasian wage)——劳动力需求等于劳动力供应情况下对应的工资水平——相当低,企业有可能有动机提高自己员工的工资以提高利润,对应的有效工资(efficiency wage)就是企业利润最大化条件下的工资水平。显然外部经济情况的不同会导致有效工资的不同,使得工资并不是完全刚性的;实际工资并不会跌落到市场出清那么低的水平上。这就肯定了政策干预的重要性,如失业救济等会影响到市场上的均衡工资水平。

\subsection{价格变化与经济波动}
\label{sec:KNK-prices-fluctuations}
第二点。该通用模型不仅能够解释失业的存在,还要能够解释失业等核心变量的波动。分为两个问题。
\begin{enumerate}
\item 导致经济体出现大幅度波动的冲击,是怎么产生的?
\begin{itemize}
\item 战争、油价波动等外生原因。
\item 内部机制。
\begin{itemize}
\item 对投资、尤其是存货的需求变动。然而在理论上,对于concave生产函数、低工资、低利率的情况,应当存在生产平滑效果:存货应当降低而不是扩大经济波动。
\item 储蓄的变化。储蓄有助于稳定消费。在储蓄不足的情况下,implicit contracts提供的保障机制起到稳定收入、进而稳定消费的作用。因此implicit contracts应当是降低而不是扩大经济波动。
\end{itemize}
\end{itemize}
Keynes正确地强调了投资变化对理解经济波动的重要性。但他用动物精神(unexplained changes in expectations)来解释投资的变化。这种解释不能让人完全满意:会导致经济周期理论具有无法解释甚至是非理性的特征。
\item 根据typical micro模型,利率、工资等价格的变化有助于缓解需求或供应的波动,进而如存货一样起到稳定经济波动的效果。因此大的外部冲击(影响需求或供应)只可能导致经济体中均衡价格的小幅度变化。然而为什么工资、利率等价格往往是刚性的?

Keynes不仅需要解释为什么对投资的需求曲线shifted,还需要解释为什么利率调整无法有效缓解投资需求变化所产生的波动。Keynes的回答不能让人完全满意:他认为对投资需求的降低会导致利率的降低,然而真实世界中当出现经济衰退,利率有时上升;甚至有时利率的变化不会影响到公司的经济决策。

Keynes只是指出工资可以保持不变,但未能解释工资为什么会保持不变。有效工资理论解释了为什么工资不会跌落到市场出清条件那么低的水平。\cite{Akerlof:1985ku}的near-rational行为模型解释了为什么企业可以调整工资而往往倾向于不这么做。类似地,一个资本市场理论解释了为什么利率不会降低到那么低的水平\citep{Stiglitz:1981ej,Stiglitz:1983jm,Stiglitz:2009gv},甚至常常是变动很小的。同时,有效工资理论也强调了一个企业制定的工资往往和其他企业制定的工资之间存在互动关系。
\end{enumerate}

总之,上述理论对工资,利率,和价格刚性做出了解释,它们互相作用,放大了对经济系统的冲击,导致在波动往往很剧烈而不是很缓和(multiplier effects)。

\subsection{储蓄与投资,信贷配给}
\label{sec:KNK-savings-investment-credit-rationing}
Keynes敏锐注意到了家庭部门的储蓄和生产部门的投资是不同的,要加以区分。然而在构建模型时,Keynes又回到standard neo-classical 模型的老路上,储蓄如何转化为投资,不做进一步分析;投资由利率决定,因此不存在信贷配给(credit rationing)。这使得尽管Keynes强调资本市场的不充分性特征,却也无法做深入的探讨。

\subsection{供应与技术进步}
\label{sec:KNK-supply-tech}
在Marshallian的框架中,经济的波动是由供应波动和需求波动两方面共同作用的结果。Keynes基于对大萧条的分析,将精力投入到分析导致需求波动的诸因素。然而对于供应端的波动,以及技术进步,他并没有做深入解释。

\section{New Keynesian经济学的四个核心}
\label{sec:KNK-NK-keys}
NK是在Keynes思想的基础上发展起来的,但也注意到了Keynes过度依赖于neoclassical理论分析框架,以及虽然认识到了资本市场是不完善的,但没能充分认识到不完善所带来的信息成本损耗及其经济效果。NK的核心有以下几点。

\subsection{有效工资}
\label{sec:KNK-NK-efficiency-wage}
有效工资模型,用于解释非刚性工资为什么不会完全跌落到市场出清条件下的低水平。

\subsection{资产配给}
\label{sec:KNK-NK-equity-rationing}
资本市场是不充分竞争的,表现在企业经理人与潜在投资者之间的信息不对称,信息不对称导致资产配给问题。资产配给问题的重要性在于,如果企业想要获得更多资本用于扩大生产,就需要从市场上借入更多资金,承担更大的投资风险。风险还表现在,在缺乏期货市场的情况下,企业当期筹资用于扩大生产,在未来时间里用增加的产出换取利润抵付,这就存在风险。在分析企业经营决策时,就必须考虑企业的风险承受意愿。

\subsection{信贷配给}
\label{sec:KNK-NK-credit-rationing}
有意愿承担风险扩大生产的企业,还面临借不到钱的问题,即信贷配给问题。与neoclassical的分析框架不同,资本品提供者在面临供不应求的情况下,可能会选择不提高利率,其逻辑类似于失业(劳动力供大于求)情况下企业的工资定价:提高利率可能会使资本品所有者的期望回报率降低,其原因之一是选择效应(selection effects,申请借贷人的集合向不利于借出者的方向变化),其原因之二是激励效应(incentive effects,借入者为了偿还更高的利率,会倾向于采取更激进、风险更高的生产形式)。

\subsection{货币政策}
\label{sec:KNK-NK-monetary-policy}
NK中货币政策的重要性,比起表现在个人持有货币维持账目平衡来,更多体现在信贷的可获得性上。信息不对称导致若商业银行决定缩紧发放贷款,需要资金的生产者很难找到其他合适的放贷人。此外商业银行的放贷决策也类似于企业生产者的生产决策,充满不确定性和风险。中央银行可以通过货币政策影响商业银行的放贷意愿。

\subsection{小结}
\label{sec:KNK-NK-pros}
总之,NK提供一套解释经济运行的通用理论,该理论基于微观经济学原理而建立。与traditional Keynesian理论相比,NK
\begin{enumerate}
\item 弥补了传统K理论的不一致性,如
\begin{enumerate}
\item 内部不一致,如期望的形成机制,
\item 预测和实际观察情况之间的不一致,
\end{enumerate}
\item 填补了传统K理论的空白,如传统K仅仅假定工资是刚性的,NK则至少在一定程度上解释了工资为什么是刚性的。
\end{enumerate}

NK进一步探讨了
\begin{enumerate}
\item 失业问题,如有效工资理论
\item 经济周期波动问题,其基本逻辑是:对经济系统的冲击影响到企业实际利用资本的存量。
\begin{enumerate}
\item 信贷市场问题,企业面对不充分的信贷市场,信息不对称,融资有风险;即使信贷市场是完全的(企业能以合理的利率得到自己所需的全部融资),企业融资数额仍然受自己风险承受能力的限制。
\item 根据借款契约的束缚,企业借贷从事生产活动需要承诺的义务是定死的,随着working capital的降低,借贷风险(破产的概率)升高,生产活动的规模降低,并且需要过了一段时间才能使working capital缓慢回升到正常水平。
\end{enumerate}

这使得NK不仅可以解释总量层面上的冲击(如货币冲击导致的物价暴跌)如何影响总量层面上的经济系统,还可以解释部门层面上的冲击(如未提前预知的demand shift,或操作价格的石油联盟等)如何影响总量层面上的经济系统。从而生产意愿成为关于working capital的concave函数,并且working capital的再分配也会产生总量效果。
\end{enumerate}

\section{凯恩斯的不足}
\label{sec:KNK-K-errors}

NK能够解释分析一些现象,这些现象都是traditional Keynesian理论所很难解释的,如
\begin{enumerate}
\item 为什么经济衰退中的企业不降低产品价格,即markup的周期运动模式,
\item 投资和存货的周期性变化特征,
\item 失业者在愿意降低工资重新求职的情况下,为什么仍然很难再就业,
\item 为什么一个unanticipated 工资-物价下降,通过降低企业的working capital存量,会导致经济衰退恶化而非好转,
\end{enumerate}
下面做详细说明。

Keynes最重要的错误在于他的企业理论,以及对货币如何影响经济活动的解释上。这两点不足都是由于他未能充分理解资本市场的本质。

\subsection{债券和股票的区别}
\label{sec:KNK-K-bond-equity}
Keynes未能正确区分长期债券和长期股票,而将二者统称为长期资产。然而区别的确存在:
\begin{enumerate}
\item 风险性。即便不考虑破产清算问题,二者在风险性上有本质区别:在经济衰退中,债券的价值上升,股票的价值下降。因此在个人理财账户中,二者是互补品而非替代品。
\item 企业承诺。借入债券/借贷的企业,需要承诺在指定日期偿还;而对于股票,企业则不需要受此承诺约束。从这个角度来说,对于企业和投资者而言,二者远非完全替代品。尤其在经济衰退中,需要融资的企业很少借助股票渠道,见\cite{Greenwald:1984uv}的adverse selection model。
\end{enumerate}

\subsection{需求和供应}
\label{sec:KNK-K-dem-supp}
Keynes从需求端对经济波动的解释,有助于回答“为什么企业不改变定价策略,降价促销”,但该解释进一步导致下个问题的出现:为什么一个小的开放经济体会存在Keynesian失业的情况?只需要它改变汇率,就可以产生对该经济体商品近乎无限的需求。

NK不对供应和需求作明确的区分。在有明确需求的情况下,企业倾向于扩大生产(供应),此时需求引导生产;在需求不够明确的情况下,企业未必会扩大生产,而会基于供应曲线从事经济活动。这样,NK可以解释为什么在经济周期中,企业愿意生产的产品数量是波动的。此外NK还可以结识为什么在经济衰退时期,(可以调整自己产品价格的)企业倾向于基于生产成本,做一个更高的price markup。在不充分竞争和不完全信息的市场上,企业在制定产品价格时,面临现在和未来利润的权衡:今天降价导致今天利润下降,以及明天销量上升、利润上升。衰退时期企业产品定价更高的原因在于,生产企业面对更高的资本成本(并不是指市场利率),以及更严重的资产配给问题。

\subsection{投资的决定因素}
\label{KNK-K-investment}
Keynes认为在给定期望水平的情况下,投资的最主要决定因素是利率。不论这里的“利率”应当是指实际利率还是名义利率\footnote{目前经济学界通常认为,实际利率更为恰当。},通过对现实世界的观察能够发现,实际利率相对于其他经济变量,波动相对较小。NK模型中在解释投资时,作为解释变量的(实际)利率不应当是个常数或近似常数。

此外NK认为,投资还受到特定时段信贷可获得程度的影响。所谓特定时段,是指货币政策会对经济活动产生影响的时期。在现实世界中,当面临经济衰退的时候,就算银行仍然愿意以现行利率将款项借出给有良好资质的企业,但很可能愿意借入的企业是不足的,这导致货币政策失效。

Keynesian-neoclassical理论无法解释存货波动的问题,即为什么存货往往激化而不是缓和经济波动。NK可以解释这个问题,认为资本的有效成本增加意味着企业倾向于在经济衰退时期减少库存。资本有效成本的增加是由于equity rationing和/或working capital供应量的减少所导致的。

\subsection{货币政策}

在Keynes的分析框架中,是不可能通过货币政策影响经济运行的。从最简单的思路出发,可行的方案分三步走:
\begin{enumerate}
\item 政府采取改变货币供应量的举措;

NK:至少对完成交易这一目的而言,存在着货币的替代物,可能使得政府改变货币供应的举动无法完成既定目标。相当多的交易可以不依赖于货币完成,仅靠信用足矣,这使得许多基于cash-in-advance的模型无法较好解释现实。此外交易和收入之间的关系较弱:大多数交易体现为财产的交换,而经济周期波动往往伴随着财富的变化,进而财产分配的变化。
\item 在给定个人货币需求函数(假定个人对货币的需求取决于利率和收入)的前提下,利率发生变化;

NK:既然对货币的需求往往是基于财产的,那么显然财富而非收入成为个人经济行为的关注要素。在交易目的考虑之外,短期债券可以充当货币的完全替代品,此时持有货币的机会成本表现为短期货币利率。但影响投资行为的利率必须是实际利率\footnote{此外,Cash Management Accounts方面的最新研究表明,完全可以通过提供interest bearing "money" 来替代直接持有货币的方案,此时经济个体的最优决策就变成了考虑他所拟持有债务的maturity structure。}。

%较后期的Keynes理论,如Tobin,提出了另一种货币政策对经济运行的影响机制:个人资产账户中,不同财产(短期、长期债券)之间是不完全替代品,不同财产的相对供应量的变化影响导致利率的变化以及股票价格的变化。

%NK:Tobin的思路在以下几个方面存在不足。首先,企业绝大多数情况下是不会在证券市场上筹集资金的。股票价格并不会起到直接影响。现实世界中可以观测到财产供应量和股票价格之间的相关性,NK将之解释为,如果对某种产品未来销售呈乐观期望态度,则它的当前股票价格会升高(未来收益高),导致企业当前生产意愿提高。这是相关性而非因果性。其次,既然政府债券的变化不会产生任何再分配效果(现实中这未必成立),政府债券的maturity structure应当不会影响市场均衡状态;它只是改变了个体在当前和未来的(随机)税务负债情况。个体在制定最有资产持有组合的时候,不只要考虑到一系列债券、股票的风险,还要考虑到工资和税收风险;一旦处理得当,这种投资组合决策不会对实际利率产生任何影响。在这种完全的资本市场情况下,Tobin approach似乎就仅取决于非理性行为了。

\item 利率变化导致投资变化。
\end{enumerate}

\section{小结}
\label{sec:KNK-K-concluding-remarks}
方法论。

一系列facts:
市场是很复杂的。经济模型应当致力于描述市场的核心特征,而非完美复制市场中的一切。企业和个人今天的行为决策受到昨天决策的影响,基于对未来的期望。人们对未来的期望是非理性的。市场的确存在但是不完全的。价格的确调整。失业率较高时,工资会下降。

这些facts给经济模型构建者带来挑战。无法建立这样一个动态模型来完美反映未来的一切facts,因为未来是不可完全预知的。构建模型应当致力于解决其中一些核心问题(而非全部),比如失业。

政策。长久以来关于在出现失业时政府应当做些什么,存在较大争议:a)自由放任,什么都不做,b)致力于降低工资以提升就业,c)使用货币政策,d)扩大政府支出。Keynes理论的成功在于它为持有d)立场的人提供了理论依据。New Classical理论为持有a)立场的人提供依据

市场效率。NK认同K的观点,不充分竞争、不完全信息导致市场失灵,从而导致失业的确是资本主义经济面临的核心难题之一。
