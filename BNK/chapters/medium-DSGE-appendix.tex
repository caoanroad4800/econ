%!TEX root = ../DSGEnotes.tex
\begin{subappendices}
\section{scaled variables}
\label{sec:scaled-variables}
中性技术冲击$z_t$的增速
\begin{equation}
  \label{eq:scaled-growth-neutral-shock}
  \mu_{z,t} \equiv \frac{z_t}{z_{t-1}}.
\end{equation}

investment-specific技术冲击$\Psi_t$的增速
\begin{equation}
  \label{eq:scaled-growth-investment-shock}
  \mu_{\Psi,t} \equiv \frac{\Psi_t}{\Psi_{t-1}}.
\end{equation}

结合式\eqref{eq:ztplus-zt-Psi}和式  \eqref{eq:scaled-growth-investment-shock}可得,生产成本系数$z_t^+$的增速
\begin{equation}
  \label{eq:scaled-growth-fixed-cost-shock}
  \mu_{z^+,t} \equiv \frac{z^+_{t+1}}{z^+_{t}} = \mu_{\Psi,t}^{\frac{\alpha}{1-\alpha}} \cdot \mu_{z,t}.
\end{equation}

physical capital stock
\begin{equation}
  \label{eq:scaled-physical-capital}
  \bar{k}_{t+1} \equiv \frac{\bar{K}_{t+1}}{z_t^{+} \cdot \Psi_{t}}.
\end{equation}

physical capital service
\begin{equation}
  \label{eq:scaled-physical-capital-service}
  k_{t+1} \equiv \frac{K_{t+1}}{z_{t}^+ \cdot \Psi_{t}}.
\end{equation}

投资品
\begin{equation}
  \label{eq:scaled-investment-goods}
  i_t \equiv \frac{I_t}{z_t^+ \cdot \Psi_{t}}.
\end{equation}

消费品
\begin{equation}
  \label{eq:scaled-consumption-goods}
  c_t \equiv \frac{C_t}{z_t^+}.
\end{equation}

政府支出
\begin{equation}
  \label{eq:scaled-government-consumption}
  g_t \equiv \frac{G_t}{z_t^+}.
\end{equation}

产出
\begin{equation}
  \label{eq:scaled-product}
  y_t \equiv \frac{Y_t}{z_t^+}.
\end{equation}

实际工资水平
\begin{equation}
  \label{eq:scaled-real-wage}
  \bar{w}_t \equiv \frac{W_t}{z_t^+ \cdot P_t}.
\end{equation}

实际资本租金
\begin{equation}
  \label{eq:real-rental-rate-capital}
  \bar{r}_t^k \equiv \Psi_t \cdot r_t^k.
\end{equation}

t+1时刻形成的资本存量,在t时期的价格
\begin{equation}
  \label{eq:scaled-physical-capital-price}
  p_{k',t} \equiv \Psi_t \cdot P_{k',t}.
\end{equation}

\subsection{几个没写完的说明}

生产成本系数的调整
\begin{equation}
  \label{eq:scaled-produc-cost-adj-coef}
  \psi_{z_t^+, t} \equiv v_t \cdot P_t \cdot z_t^+,
\end{equation}
其中$v_t$表示家庭优化问题中,名义预算约束条件的Lagrangian乘子,它等于额外1单位货币收入带来的边际效用;$v_t \cdot P_t$因此等于额外1单位消费带来的边际效用。

中间产品生产厂商调整后的产品价格
\begin{equation}
  \label{eq:scaled-intm-good-price}
  \tilde{p}_t \equiv \frac{\tilde{P}_t}{P_t}.
\end{equation}

劳动者联盟调整后的工资
\begin{equation}
  \label{eq:scaled-labor-union-price}
  \tilde{w}_t \equiv \frac{\tilde{W}_t}{W_t}.
\end{equation}

通货膨胀率
\begin{equation}
  \label{eq:scaled-price-inflation}
  1+\pi_t \equiv \frac{P_t}{P_{t-1}}.
\end{equation}

工资增速
\begin{equation}
  \label{eq:scaled-wage-inflation}
  1+\pi_{w,t} \equiv \frac{W_t}{W_{t-1}}
\end{equation}


\section{Frisch elasticity of labor supply}
\label{sec:Frish-elasticity}
\subsection{定义}
Frisch elasticity of labor supply是指在保持财富的边际效用不变的情况下,工资和劳动力供应之间的弹性\citep{Frisch:1932wk}\footnote{此外还有Marshallian elasticity of labor supply(保持收入不变的情况下),和Hicksian elasiticity of labor supply(保持效用水平不变的情况下)等。}。Frisch弹性$\eta^{\lambda}$的定义式可以表示如下
\begin{equation*}
  \eta^{\lambda} = \frac{\partial n}{\partial w} \cdot \frac{w}{n} ||_{\lambda},
\end{equation*}
其中$||_{\lambda}$表示保持财富的边际效用不变。

定义式。考虑如下consumer problem
\begin{align*}
  &\max_{\{c_t,a_{t+1},n_t\}} \sum_{t=0}^{\infty} \beta^t \cdot E_t U(c_t,n_t), \\
    &st. \quad c_t + a_{t+1} = (1+r) \cdot a_t + w_t \cdot n_t.
\end{align*}

建Lagrangian
\begin{equation*}
  \mathcal{L} = \sum_{t=0}^{\infty} \beta^t E_t \left\{U(c_t,n_t) + \lambda_t \cdot \left[c_t + a_{t+1} = (1+r) \cdot a_t + w_t \cdot n_t \right]\right\} .
\end{equation*}

FOCs
\begin{align}
\label{eq:Frisch-FOC-c}
  &\frac{\partial \mathcal{L}}{\partial c_t}=0 \Rightarrow U_{c,t}=\lambda_t,\\
\label{eq:Frisch-FOC-n}
  &\frac{\partial \mathcal{L}}{\partial n_t}=0 \Rightarrow U_{n,t}=-\lambda_t \cdot w_t,\\
\label{eq:Frisch-FOC-a}
  &\frac{\partial \mathcal{L}}{\partial a_{t+1}}=0 \Rightarrow \beta \cdot E_t \lambda_{t+1} \cdot (1+r) =\lambda_t.
\end{align}

式\eqref{eq:Frisch-FOC-c}-\eqref{eq:Frisch-FOC-c}$\Rightarrow$
\begin{align}
\label{eq:Frisch-FOC-c-lambda-w}
  &\frac{\partial U(c_t(\lambda_t, w_t),n(\lambda_t, w_t))}{\partial c_t} = \lambda_t,\\
\label{eq:Frisch-FOC-n-lambda-w}
  &\frac{\partial U(c_t(\lambda_t, w_t),n(\lambda_t, w_t))}{\partial n_t} = -\lambda_t \cdot w_t.
\end{align}

式\eqref{eq:Frisch-FOC-c}-\eqref{eq:Frisch-FOC-n}对$w_t$求导得
\begin{align}
\label{eq:Frisch-FOC-U-c-w}
  &\frac{\partial U_{c,t}}{\partial w_t} = 0 \Rightarrow U_{cc,t} \cdot \frac{\partial c_t}{\partial w_t} + U_{cn,t} \cdot \frac{\partial n_t}{\partial w_t} = 0,\\
\label{eq:Frisch-FOC-U-n-w}
  &\frac{\partial U_{n,t}}{\partial w_t} = -\lambda_t \Rightarrow U_{cn,t} \cdot \frac{\partial c_t}{\partial w_t} + U_{nn,t} \cdot \frac{\partial n_t}{\partial w_t} = -\lambda_t.
\end{align}

可见式\eqref{eq:Frisch-FOC-U-c-w}-\eqref{eq:Frisch-FOC-U-n-w}是有两个未知变量$\left(\frac{\partial c_t}{w_t},\frac{\partial n_t}{\partial w_t}\right)$的两个方程组。利用式\eqref{eq:Frisch-FOC-n}替代$\lambda_t$,解得
\begin{equation}
  \label{eq:Frisch-ela}
  \frac{\partial n_t}{\partial w_t} = \frac{\lambda_t \cdot U_{cc,t}}{U_{cn}^2 - U_{cc,t} \cdot U_{nn,t}} = \frac{- \frac{U_{n,t}}{w_t} \cdot U_{cc,t}}{U_{cn}^2 - U_{cc,t} \cdot U_{nn,t}}
\end{equation}

因此Frisch elasticity等于
\begin{equation}
  \label{eq:Frisch-ela-consumption-example}
  \eta^{\lambda} = \frac{\partial n_t}{\partial w_t} \cdot \frac{w_t}{n_t} = \frac{U_{n,t}}{n_t \cdot \left[U_{nn,t}-\left(\frac{U_{cn,t}^2}{U_{cc,t}}\right)\right]}.
\end{equation}

\subsection{举例}
举例,消费者问题:$\max \beta^t \cdot E_t U_{c_t,n_t}$,$U_{c_t,n_t} = \ln c_t - \alpha \cdot \frac{n_t^{1+\frac{1}{\nu}}}{1+\frac{1}{\nu}}$, subject to $c_t + a_{t+1} = (1+r) \cdot a_t + w_t \cdot n_t$。

FOCs:
\begin{align}
\label{Frisch-example-FOC-c}
\frac{1}{c_t} &= \lambda_t,\\
\label{Frisch-example-FOC-n}
n_t &= \left(\frac{\lambda_t \cdot w_t}{\alpha}\right)^{\nu}.
\end{align}

\begin{equation}
  \label{eq:Frisch-example-partial-n-w}
  \frac{\partial n_t}{\partial w_t} = \nu \cdot \left(\frac{\lambda_t}{\alpha}\right)^{\nu} \cdot w_t^{\nu -1}.
\end{equation}

Frisch elasticity of labor supply因此等于
\begin{equation}
  \label{eq:Frisch-example-ela}
  \eta^{\lambda} = \frac{\partial n_t}{\partial w_t} \cdot \frac{w_t}{n_t} = \nu.
\end{equation}

\section{调节成本的常见设定形式及比较}
\label{sec:adjustment-cost-types-compar}
经验研究需要将调节成本函数$F(\cdot)$改写为显函数形式,常见的有三种形式
\begin{enumerate}
\item  线性调节成本。设$S$近似为常数,式\eqref{eq:adjustment-cost-func-CEE}改写为
\begin{equation}
\label{eq:adjustment-cost-lin}
F = constant \cdot I_t,
\end{equation}
即调节成本和投资之间呈线性关系,$constant$是常数。
\item 投资调节成本。最早可见\cite{Lucasjoin:1971hx},后经\cite{Hayashi:1982bc}作进一步模型化。
\begin{equation}
\label{eq:adjustment-cost-capital}
F = I_t - \frac{S''}{2} \cdot \left(\frac{I_t}{\bar{K}_t} - \delta \right)^2 \cdot \bar{K}_{t},
\end{equation}
\item 资本调节成本。最早可见\cite{Christiano:2005ib},如式\eqref{eq:adjustment-cost-func-CEE}所示。
\end{enumerate}



\subsection{家庭部门优化条件}
\label{sec:adj-cost-hh-optimisation}
在capital adjustment cost设定下,家庭部门优化条件表示为通过选择投入组合$\{ C_t, I_t, \Delta_t, B_{t+1}, \bar{K}_{t+1}\}$,基于给定的预算约束式\eqref{eq:HH-opt-prob-budget-constraint}、资本积累式\eqref{eq:phy-cap-accumu-cap-adj-cost}和资本净收益式\eqref{eq:capital-net-payment},来追求式\eqref{eq:hh-utility-C-t-h-int}效用函数最大化。

建Lagrangian
\begin{align*}
\mathcal{L}_t =&E_0 \sum_{t=0}^{\infty} \beta^t \{\left[\log(C_t - b \cdot C_{t-1}) - A \cdot \int_{0}^{1} \frac{h_{t}(j)^{1+\eta}}{1+\eta} dj\right] \\
&+\lambda_{1,t} \cdot \left[
\int_{0}^{1} W_{t}(j) \cdot h_{t}(j) dj + X^k_t \cdot \bar{K}_t
 + R_{t-1} \cdot B_t - P_t \cdot \left(C_t + \frac{1}{\Psi_t} \cdot I_t \right) - B_{t+1} - P_t \cdot P_{k',t} \cdot \Delta_t
\right] \\
&+\lambda_{2,t} \cdot \left[
(1-\delta) \cdot \bar{K}_t + \left[I_t - \frac{S''}{2} \cdot \left(\frac{I_t}{\bar{K}_t} - \delta \right)^2  \cdot \bar{K}_{t}\right] + \Delta_t - \bar{K}_{t+1}
\right] \\
&+\lambda_{3,t} \cdot \left[
 u_{t+1} \cdot P_{t+1} \cdot r_{t+1}^k - \frac{P_{t+1}}{\psi_{t+1}} \cdot a(u_{t+1}) - X_{t+1}^k
\right]\}
\end{align*}

FOCs:
\begin{equation}
\label{HH-capadj-prob-FOC-C}
\frac{\mathcal{L}_t}{\partial C_t} =0 \Rightarrow \lambda_{1,t} = \frac{U_{C,t}}{P_t},
\end{equation}

\begin{equation}
\label{HH-capadj-prob-FOC-I}
\frac{\mathcal{L}_t}{\partial I_t} =0 \Rightarrow \lambda_{1,t} \cdot P_t = \lambda_{2,t} \cdot \left[1-S''\cdot \left(\frac{I_t}{\bar{K}_t}- \delta \right)\right],
\end{equation}

\begin{equation}
\label{HH-capadj-prob-FOC-Delta}
\frac{\mathcal{L}_t}{\partial \Delta_t} =0 \Rightarrow \lambda_{1,t} \cdot P_t \cdot P_{k',t} = \lambda_{2,t},
\end{equation}

\begin{equation}
\label{HH-capadj-prob-FOC-B}
\frac{\mathcal{L}_t}{\partial B_{t+1}} =0 \Rightarrow \beta \cdot E_t \lambda_{1,t+1} \cdot R^k_t = \lambda_{1,t},
\end{equation}

\begin{align}
\label{HH-capadj-prob-FOC-barK}
&\frac{\mathcal{L}_t}{\partial \bar{K}_{t+1}} =0 \Rightarrow \beta \cdot E_{t} \lambda_{1,t+1} \cdot X^k_{t+1} = \lambda_{2,t} \nonumber \\
&+ \beta \cdot E_t \lambda_{2,t+1} \cdot \left[
\frac{S''}{2} \cdot \left(
\frac{I_{t+1}}{\bar{K}_{t+1}} - \delta
\right)^2
- S'' \cdot \left(\frac{I_{t+1}}{\bar{K}_{t+1}} - \delta\right) \cdot \frac{I_{t+1}}{\bar{K}_{t+1}}
-(1-\delta)
\right].
\end{align}

联立式\eqref{HH-capadj-prob-FOC-I}、式\eqref{HH-capadj-prob-FOC-Delta}得
\begin{equation}
\label{eq:HH-capadj-prob-FOC-pkt-q}
P_{k',t} = \frac{1}{1-S'' \cdot \left(\frac{I_t}{\bar{K}_t} - \delta\right)}.
\end{equation}

式\eqref{HH-capadj-prob-FOC-barK}改写为
\begin{align*}
\lambda_{2,t} &= \beta \cdot E_t \lambda_{1,t+1} \cdot X^k_{t+1} + \beta \cdot E_t \cdot \lambda_{2,t+1 }\cdot \mathcal{H}_{t+1}, \nonumber \\
&\mathcal{H}_{t+1}\equiv \left[
(1-\delta) +S'' \cdot \left(\frac{I_{t+1}}{\bar{K}_{t+1}} - \delta\right) \cdot \frac{I_{t+1}}{\bar{K}_{t+1}}-\frac{S''}{2} \cdot \left(
\frac{I_{t+1}}{\bar{K}_{t+1}} - \delta
\right)^2
\right],
\end{align*}
$\mathcal{H}_{t+1}$反映每1单位$\bar{K}_{t+1}$的增加,可能导致$\bar{K}_{t+2}$的增加。等式两侧同时除以$\lambda_{1,t}$得
\begin{equation*}
\frac{\lambda_{2,t}}{\lambda_{1,t}} = \frac{\beta \cdot E_t \cdot \lambda_{1,t+1}}{\lambda_{1,t}}\cdot X^k_{t+1} +\frac{\beta \cdot E_{t} \lambda_{2,t+1}}{\beta \cdot E_{t} \lambda_{2,t+1}} \cdot \frac{\beta \cdot E_{t} \lambda_{2,t+1}}{\lambda_{1,t}} \cdot \mathcal{H}_{t+1},
\end{equation*}
进一步整理得
\begin{equation}
\label{HH-capadj-prob-FOC-lambda1-2}
\frac{U_{C,t}}{\beta \cdot E_t U_{C,t+1}} = \frac{\frac{X^k_{t+1}}{P_{t+1}} + P_{k',t+1} \cdot %\left[
%(1-\delta) +S'' \cdot \left(\frac{I_{t+1}}{\bar{K}_{t+1}} - \delta\right) \cdot \frac{I_{t+1}}{\bar{K}_{t+1}}-\frac{S''}{2} \cdot \left(
%\frac{I_{t+1}}{\bar{K}_{t+1}} - \delta
%\right)^2
%\right]
\mathcal{H}_{t+1}
}{P_{k',t}}
\end{equation}

\subsection{比较线性和非线性调节成本}
\label{sec:adj-cost-comp-lin-nonlin}
式\eqref{HH-capadj-prob-FOC-lambda1-2}以Euler equation的形式反映家庭的跨期消费行为决策。定义投资的回报率(rental rate of return on investment) $R^k_{t+1}$
\begin{equation}
\label{eq:adj-cost-return-investment}
R^k_{t+1} = \frac{x^k_{t+1} + \left[
(1-\delta) + S''\cdot \left(\frac{I_{t+1}}{\bar{K}_{t+1}} - \delta \right) \cdot \frac{I_{t+1}}{\bar{K}_{t+1}} - \frac{S''}{2} \cdot \left(\frac{I_{t+1}}{\bar{K}_{t+1}} - \delta \right)^2
\right] \cdot P_{k',t+1}}{P_{k',t}},
\end{equation}
其中$x^k_{t+1} \equiv \frac{X^k_{t+1}}{P_{t+1}}$表示实际的cash payment,$X^k_{t+1}$由式\eqref{eq:capital-net-payment}给出。RHS分母$P_{k',t}$表示每一单位新增$\bar{K}_t$的市场价格,由式\eqref{eq:HH-capadj-prob-FOC-pkt-q}给出。RHS分子中$P_{k',t+1}$是以消费品形式表示的每一单位$\bar{K}_{t+2}$资本存量的市场价格,乘以$[\cdot]$之后转化为消费品形式。

在充分竞争的要素市场上,额外一单位资本的价格取决于其边际产出,即
\begin{equation}
\begin{split}
P_{k',t} &= - \frac{d C_t}{d \bar{K}_{t+1}} \nonumber \\
&=  - \frac{d C_t}{d I_t} \cdot \frac{d I_t}{d \bar{K}_{t+1}} \nonumber \\
&=\frac{1}{\frac{d \bar{ K}_{t+1}}{d I_t}} \nonumber \\
&=1\text{(线性调节成本),或者} \nonumber \\
&\frac{1}{1-S''\cdot\left(\frac{I_t}{\bar{K}_{t} - \delta}\right)}\text{(资本调节成本)},
\end{split}
\end{equation}
其中$\frac{d C_t}{d I_t}$为消费和实物资本积累(投资)之间的边际技术替代率(MRTS),由式\eqref{eq:output-expenditures}-\eqref{eq:investment-goods-homo}给出;$\frac{I_t}{\bar{K}_{t+1}}$为投资和实物资本积累之间的边际技术替代率(MRTS),由式\eqref{eq:phy-cap-accumu-cap-adj-cost}给出。这样,模型中Tobin's q与$P_{k',t}$相关,为资本的市场价格除以投资品的价格,其中资本的市场价格等于完全竞争市场假定下的资本的边际成本。或者定义
\begin{equation}
\label{eq:tobin-q-sims-2015}
q_t \equiv \frac{\lambda_{2,t}}{\lambda_{1,t}} = P_t \cdot P_{k',t},
\end{equation}
后一等号由式\eqref{HH-capadj-prob-FOC-Delta}得。拉格朗日乘子$\lambda_{1,t}$、$\lambda_{2,t}$分别表示额外1单位实物资本和消费的边际效用(影子价格)。$\lambda_{2,t}/\lambda_{1,t}$因此表示为了在明天多拥有1单位的实物资本,个人愿意放弃多少单位的当前消费,即资本相对于消费的价格。

\subsection{比较投资调节成本和资本调节成本}
\label{sec:adj-cost-comp-inv}
VAR-based evidence:通过对actual data的observation可以看出,在一个正的货币冲击发生后,一方面实际利率持久走低,另一方面投资却呈现出hump-shaped pattern。为了让模型能够更好解释这一现象,就需要在建构模型的时候,使得\eqref{eq:adj-cost-return-investment}中的真实回报率$R^k_{t}$随着扩张性货币政策冲击而降低。情境分析如下。
\begin{enumerate}
\item 如果$S''=0$,即不存在adjustment cost。此时$q_t = p_{k',t}=0$,式\eqref{eq:adj-cost-return-investment}中唯一能让$R^k_{t+1}$下降的便是$x^k_{t+1}$,此时
\begin{enumerate}
\item 若rate of return on capital $RROC \approx (1-\alpha) \cdot K_{t+1}^{\alpha-1} \cdot H_{t+1}^{1-\alpha} + (1-\delta) \approx 1/\beta$,稳定状态下,不考虑经济增长情况下的年均$RROC \approx 1.03$,剩余的内生部分的RROC只占全部RROC的很小一部分。这意味着,为了让RROC下降一小部分,需要让内生RROC($K_{t+1}^{\alpha-1} \cdot H_{t+1}^{1-\alpha}$)下降非常大的一部分,这意味着需要让投资大幅度增加——而在在现实中很难做到:一方面投资的大幅度上升会压迫消费发生大幅度下降,可是现实世界中消费并没有发生如此大的降低\footnote{膏按:habit formation? 见第\ref{sec:hh-consumption-habit-formation}节。};另一方面作为流量的投资的大幅度变化,也只能导致作为存量的实物资本的小幅度提升。
\item VAR based evidence表明,正的货币政策冲击发生后,就业上升,这意味着hours worked $H_t$上升,这会迫使内生RROC上升,压迫实物资本存量,调低投资的回报率。
\end{enumerate}

总之,不引入调节成本的模型,会产生反事实结论:正的货币政策冲击发生后,投资反而会大幅度扩张,而现实数据观测中并未支持这一结论。
\item 如果$S'' >0$,则内生RROC要比$S''=0$情况下的内生RROC大得多,这是由于当调节成本存在时,资本这就速度加快,实物资本存量降低,资本的边际产出、进而边际回报变大。分两种情况来说明。
\begin{enumerate}
\item 资本调节成本+$S'' >0$的设定如式\eqref{eq:adjustment-cost-capital},一次正货币政策冲击所产生的响应有
\begin{enumerate}
\item contemporary response of investment $I_t$,即投资响应中最大的一次响应发生在冲击产生的当期而非随后。
\item a hump-shaped response of capital price $P_{k',t}$。这是由于受到即时变化的投资流响应的影响:一个正的货币政策冲击会产生持续数个时间段的$P_{k',t+1}/pP_{k',t}$上升。将这一现象带回式\eqref{eq:adj-cost-return-investment}可见,对未来$P_{k',t+1}$的语气上升,及对投资收益率的预期上升,会导家庭产生追加投资的冲动,且短期内这种冲动更强,出现hump shaped,这使得在均衡状态下,冲击出现当期的投资响应最强,随后减弱。
\end{enumerate}

\item 投资调节成本$+ S'' >0$的设定如式\eqref{eq:adjustment-cost-func-CEE},投资响应是hump shaped且最大响应发生在shock出现的时间之后而非当期。这是由于相比资本调节成本设定而言,投资调节成本设定下,当期投资相对于上期投资若想要发生大规模变动,其调节成本会变得更高。hump shaped impulse responses of investment导致当期growth rate相对于上期growth rate呈现更强的正相关,且至少会持续一小段时期。
\end{enumerate}
\end{enumerate}
\end{subappendices}
