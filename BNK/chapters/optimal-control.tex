%!TEX root = ../DSGEnotes.tex
\chapter{线性二次最优控制}
\label{sec:linear-quardratic-control-intro}
多数情况下,均衡条件较为复杂,无法直接得出解析解。我们需要引入线性二次最优控制理论(optimal linear-quadratic control, LQ),将均衡条件用状态-空间形式表现出来(state-space representation)。这里对线性二次最优控制论问题做简要介绍。

假定一个经济系统,由一系列状态变量的向量$x_t$和一系列控制变量的向量$u_t$构成;$x_t$受到外生随机冲击$\varepsilon_t$的干扰,外生冲击的方差-协方差矩阵为$\sum$。这样一个线性二次最优控制问题可以表述如下:经济系统的决策者通过选择$u_t$来影响$x_t$的走向,由此我们来构建目标方程和状态转移方程。

\section{目标方程和状态转移方程}
目标方程(objective function):
\begin{equation}
  \label{eq:LQC-obj-function}
  \max_{\{u_t\}} E \sum_{t=0}^{\infty} \beta^t \left[ x_t^T R x_t + u_t^T Q u_t\right],
\end{equation}
其中$\beta$是时间贴现系数,对称矩阵$R$和$Q$分别对应目标方程中状态变量和控制变量的权重。

状态转移方程(transition function):
\begin{equation}
  \label{eq:LQC-trans-function}
  x_{t+1}=A x_t + B u_t + \varepsilon_{t+1},
\end{equation}
当期状态由上期状态变量、上期控制变量所共同决定,二者的权重分别由矩阵$A$和$B$所反映。此外,当期状态还受到当期随机冲击的影响,假定是一个Gaussian过程$\{\varepsilon_{t}\}_{t=0}^{\infty} \sim \mathcal{N}(0,\Sigma)$。

\section{价值方程}
在给定时间贴现系数$\beta$的情况下,如果已知系数矩阵$A,B,R,Q$,我们可以求得系统的解。而对于系数矩阵未知的情况,问题则较为复杂。我们引入动态规划(dynamic programming, DP)的思路,定义价值方程$V(x_t)$表示当期状态变量的价值。结合状态转移方程\eqref{eq:LQC-trans-function}和目标方程\eqref{eq:LQC-obj-function},迭代形式表现的价值方程(value function)优化问题如下
\begin{align}
  \label{eq:LQC-value-object}
  V(x_t) &= \max_{\{u_t\}} \left\{
  x_t^T R x_t + u_t^T Q u_t + \beta E[V(x_{t+1})]
  \right\} \nonumber \\
  &=\max_{\{u_t\}} \left\{
  x_t^T R x_t + u_t^T Q u_t + \beta E[V(Ax_{t} + B u_{t} + \varepsilon_{t+1})]
  \right\},
\end{align}
为了求解线性二次最优控制问题  \eqref{eq:LQC-value-object},我们需要两方面信息。一是政策方程(policy function)的具体形式 $u_t = g(x_t)$,即基于当期状态,决策者如何选择最优控制变量,见第\ref{sec:linear-policy-function}节。二是基于当期状态,价值方程的具体形式。

对于后者,方案设计如下:首先我们假设
\begin{equation}
  \label{eq:LQC-value-function}
  V(x_t) = x_t^T P x_t + d,
\end{equation}
即价值方程是关于状态变量的二项式形式(第\ref{sec:LQC-lin-policy-quad-value-impli}节验证价值方程的二项式形式假设是否成立),系数表示为幂等矩阵$P$,满足$P=P^T=P^2$,$d$是个常数矩阵。

随后,将\eqref{eq:LQC-value-function}代回\eqref{eq:LQC-value-object}可得Bellman equation形式的价值方程
\begin{align}
  V(x_t)&=\max_{\{u_t\}} \left\{
  x_t^T R x_t + u_t^T Q u_t + \beta E \left[
  \left(A x_t + B u_t + \varepsilon_{t+1}\right)^T P \left(A x_t + B u_t + \varepsilon_{t+1}\right)  \right] + \beta d
  \right\} \nonumber \\
\label{eq:LQC-value-function-min-problem}
  &\equiv \max_{\{u_t\}} \left\{
  x_t^T R x_t + u_t^T Q u_t + \beta E \left[
  \mathcal{X}  \right] + \beta d
  \right\},
\end{align}
其中
\begin{align*}
  E [\mathcal{X}] = & x_t^T A^T P A x_t + x_t^T A^T P B u_t + u_t^T B^T P A x_t + u_t^T B^T P B u_t \\
  & + \left(x_t^T A^T P \varepsilon_{t+1}  + u_t^T B^T P \varepsilon_{t+1} + \varepsilon_{t+1}^T P A x_t + \varepsilon_{t+1}^T P B u_t \right)+ \varepsilon_{t+1}^T P \varepsilon_{t+1},
\end{align*}
根据定义式$E[\varepsilon] = 0$,上式进一步简化为
\begin{equation}
  \label{eq:LQC-value-function-auxiliary-mathcalX}
  E [\mathcal{X}] =  x_t^T A^T P A x_t + x_t^T A^T P B u_t + u_t^T B^T P A x_t + u_t^T B^T P B u_t + \varepsilon_{t+1}^T P \varepsilon_{t+1}.
\end{equation}

\subsection{一阶条件}
决策者最优行为可由\eqref{eq:LQC-value-function-min-problem}的一阶条件求得
\begin{align}
  \frac{\partial V(x_t)}{\partial u_t} &=\frac{\partial u_t^T Q u_t}{\partial u_t} + \beta \frac{E[\mathcal X]}{\partial u_t} \nonumber \\
  &= \frac{\partial u_t^T Q u_t}{\partial u_t} + \beta \left(
  \frac{\partial x_t^T A^T P B u_t}{\partial u_t} + \frac{\partial u_t^T B^T P A x_t}{\partial u_t} + \frac{u_t^T B^T P B u_t}{\partial u_t} \right) \nonumber \\
  \label{eq:LQC-value-function-FOC}
  &=2Q u_t + 2 \beta B^T P A x_t + 2 \beta B^T P B u_t= 0,
\end{align}
其中为了求得最后一行等式,我们做一些矩阵运算见第\ref{sec:LQC-value-function-FOC-math}节。
\subsubsection{计算一阶条件所需的部分矩阵运算}
\label{sec:LQC-value-function-FOC-math}

对于对称的常系数矩阵$\Gamma$,首先我们有\footnote{这是由于\begin{align*}
  tr(d(u^T \Gamma u)) &= tr(d((u^T \Gamma)(u))) = tr(u^T \Gamma du + d(u^T \Gamma) u) = tr(u^T \Gamma du + d(u^T \Gamma)^T u) \\
  &=tr(u^T \Gamma du) + tr(u^T + d(u^T \Gamma)) = tr(u^T \Gamma du) + tr(u^T \Gamma^T du) = tr(u^T (\Gamma + \Gamma^T) du).
\end{align*}}
\begin{equation*}
  \frac{d}{du}(u^T \Gamma u)= u^T (\Gamma + \Gamma^T) = 2 \Gamma u,
\end{equation*}
于是
\begin{align*}
  &\frac{\partial u_t^T Q u_t}{\partial u_t} = 2 Q u_t, \\
  &\frac{\partial u_t^T B^T P B u_t}{\partial u_t} = 2 B^T P B u_t.
\end{align*}

其次我们有\footnote{这是由于
\begin{equation*}
  tr(d u^T \Gamma) = tr(d(\Gamma^T u)^T) = tr(d (\Gamma^T u)).
\end{equation*}}
\begin{align*}
  &\frac{d}{d u} \Gamma u = \Gamma^T, \\
  &\frac{d}{d u} u^T \Gamma = \Gamma,
\end{align*}
于是
\begin{equation*}
  \frac{\partial x_t^T A^T P B u_t}{\partial u_t} + \frac{\partial u_t^T B^T P A x_t}{\partial u_t} = (x_t^T A^T P B)^T + (B^T P A x_t) = 2 B^T P A x_t.
\end{equation*}

\section{政策方程}
\label{sec:linear-policy-function}
重新整理\eqref{eq:LQC-value-function-FOC},可以得到政策方程$u_t=g(x_t)$的近似线性表达形式
\begin{equation}
  \label{eq:LQC-linear-policy-func}
  u_t = -\left(Q + \beta B^T P B\right)^{-1} \beta B^T P A x_t.
\end{equation}

换句话说,决策者的最优行为可以描述为如下政策方程:
\begin{align}
  \label{eq:LQC-linear-policy-function}
  &u_t = -F x_t, \quad \text{其中}\\
  \label{eq:LQC-linear-policy-F}
  &F = \beta \left(Q + \beta B^T P B \right)^{-1} B^T P A.
\end{align}

基于政策方程\eqref{eq:LQC-linear-policy-function},控制变量$u_t$随着观测到的状态$x_t$做线性调整,调整依据是系数矩阵$F$。$F$的确是一个非线性的函数形式,其值取决于两组矩阵:基础矩阵$A, B, Q$和(假设为二次形式)价值方程中的$P$。一旦我们算出$P$,便可以进一步算出$F$的值,从而求得政策方程的完整形式\footnote{另一种方案是,先求得F,根据F测算出P,如\cite[Ch.2]{Hansen:2004va}}。

\subsection{政策方程满足确定性等价条件}
\label{sec:LQC-policy-function-certainty-equivalence}
政策方程的政策意义价值还在于,它不受外生随机冲击$\varepsilon$的干扰\footnote{除非以下情况出现:如第一,外生冲击彼此相关(所以我们要在模型设定中假设不相关)。第二,目标方程\eqref{eq:LQC-obj-function}并不是二次形式。},这是由于在这样一个线性二次系统中,确定性等价(certainty equivalence)成立,见第\ref{sec:LQC-policy-function-certainty-equivalence}、\ref{sec:LQC-value-function-certainty-equivalence}节。

\section{线性政策方程和二次价值方程}
\label{sec:LQC-lin-policy-quad-value-impli}
如前文所述,线性政策方程
\eqref{eq:LQC-linear-policy-function}-\eqref{eq:LQC-linear-policy-F}
是基于(假设的)二次价值方程
\eqref{eq:LQC-value-function-min-problem}-\eqref{eq:LQC-value-function-auxiliary-mathcalX}
得出的。现在我们反证,基于这样的线性政策方程的确可以得到二次形式的价值方程。一旦反证成功,我们便可以进一步求得$P$和$d$。

将\eqref{eq:LQC-linear-policy-function}代入\eqref{eq:LQC-value-function-min-problem}\footnote{第三个等号所需条件:根据定义$P=P^T$,因此$x_t^T F^T B^T P A x_t$和$x_t^T A^T P B F x_t$都是标量,且满足\begin{equation*}
x_t^T F^T B^T P A x_t = x_t^T A^T P B F x_t.
\end{equation*}}
\begin{align}
  \label{eq:LQC-value-function-min-expand}
  V(x_t) =& x_t^T R x_t + \left(-F x_t\right)^T Q \left(-F x_t\right) + \beta d  + \beta E[\varepsilon_{t+1}^T P \varepsilon_{t+1}] \nonumber \\
  &+\beta E \left[
  x_t^T A^T P A x_t + x_t^T A^T P B \left(-F x_t\right) + \left(-F x_t\right)^T B^T P A x_t + \left(-F x_t\right)^T B^T P B \left(-F x_t\right) \right] \nonumber \\
  =& x_t^T R x_t + x_t^T F^T Q F x_t + \beta d  + \beta E[\varepsilon_{t+1}^T P \varepsilon_{t+1}] \nonumber \\
  &+ \beta E\left[
  x_t^T A^T P A x_t - x_t^T A^T P B F x_t - x_t^T F^T B^T P A x_t + x_t^T F^T B^T P B F x_t \right] \nonumber \\
  =& x_t^T R x_t + x_t^T F^T Q F x_t + \beta E \left[
  x_t^T A^T P A x_t - 2 x_t^T A^T P B F x_t + x_t^T F^T B^T P B F x_t \right] \nonumber \\
  &+ \beta E[\varepsilon_{t+1}^T P \varepsilon_{t+1}] + \beta d.
\end{align}

联立\eqref{eq:LQC-value-function}和\eqref{eq:LQC-value-function-min-expand},我们有
\begin{equation*}
\begin{cases}
  d = \beta E[\varepsilon_{t+1}^T P \varepsilon_{t+1}] + \beta d, \\
  x_t^T P x_t = x_t^T R x_t + x_t^T F^T Q F x_t + \beta E \left[
  x_t^T A^T P A x_t - 2 x_t^T A^T P B F x_t + x_t^T F^T B^T P B F x_t \right].
\end{cases}
\end{equation*}

\subsection{价值方程满足确定性等价条件}
\label{sec:LQC-value-function-certainty-equivalence}
来看第一个等式,整理可得\footnote{这是由于\begin{equation*}
E[\varepsilon_{t+1}^T P \varepsilon_{t+1}] = tr(E[\varepsilon_{t+1}^T P \varepsilon_{t+1}]) = tr(P E[\varepsilon_{t+1}^T \varepsilon_{t+1}]) = tr(P E[\Sigma_{t+1}]).
\end{equation*}}

\begin{equation}
  \label{eq:LQC-value-function-certainty-equivalence}
  d = \frac{\beta}{1-\beta} tr(P \Sigma),
\end{equation}
可见随机冲击$\varepsilon$尽管对价值方程产生影响,但不是通过$F$至政策方程进而影响价值方程,而是通过常数项$d$影响价值方程的。因此,对于最优价值方程而言,确定性等价条件依然成立。

\subsection{代数矩阵Riccati方程}
第二个等式整理,并引入最优政策方程\eqref{eq:LQC-linear-policy-F}替换$F$得\footnote{最后一个等号的计算依据如下:
\begin{align*}
  &\beta A^T P B \left(\left(Q + \beta B^T P B \right)^{-1}\right)^T \left(Q + \beta B^T P B \right) \left(Q + \beta B^T P B \right)^{-1} \beta B^T P A \\
  &=  \beta A^T P B \left(\left(Q + \beta B^T P B \right)^{T}\right)^{-1} \left(Q + \beta B^T P B \right) \left(Q + \beta B^T P B \right)^{-1} \beta B^T P A \\
  &=  \beta A^T P B \left(\left(Q + \beta B^T P B \right)^{T}\right)^{-1} \left(Q + \beta B^T P B \right)^T \left(Q + \beta B^T P B \right)^{-1} \beta B^T P A \\
  &= \beta^2 A^T P B \left(Q + \beta B^T P B \right)^{-1} B^T P A.
\end{align*}
}
\begin{align}
  P =& R + F^T Q F + \beta \cdot \left(A^T P A - 2 A^T P B F + F^T B^T P B F\right) \nonumber \\
  =& R + \beta A^T P A - 2 \beta A^T P B F + F^T \left(Q + \beta B^T P B \right) F \nonumber \\
  =& R + \beta A^T P A - 2 \beta A^T P B \left(Q + \beta B^T P B \right)^{-1} \beta B^T P A \nonumber \\
  + & \beta A^T P B \left(\left(Q + \beta B^T P B \right)^{-1}\right)^T \left(Q + \beta B^T P B \right) \left(Q + \beta B^T P B \right)^{-1} \beta B^T P A \nonumber \\
  \label{eq:LQC-alg-mat-Riccati-equation}
  =& R + \beta A^T P A - \beta^2 A^T P B \left(Q + \beta B^T P B \right)^{-1} \beta B^T P A.
\end{align}

\eqref{eq:LQC-alg-mat-Riccati-equation}又称线性矩阵Riccati方程(linear matrix Riccati equation),它表明价值方程中的$P$是个与基础矩阵$A,B,R,Q$有关的函数,呈非线性关系。

此外,由\eqref{eq:LQC-alg-mat-Riccati-equation}可以看出,一个线性的最优政策方程\eqref{eq:LQC-linear-policy-function}-\eqref{eq:LQC-linear-policy-F},的确预示着二次形式的价值方程  \eqref{eq:LQC-value-function},\eqref{eq:LQC-alg-mat-Riccati-equation}。

\section{数值方法}
在线性二次控制中,在利用解析法推得线性代数Riccati方程后,需要依赖数值计算的方法,对$P$的矩阵Riccati差分方程做迭代近似。假定期初猜测值为$P_{j}$,经过1次迭代,$P_{j+1}$的值更新至
\begin{equation}
  \label{eq:LQC-mat-Riccati-equation-iteration}
  P_{j+1}=R + \beta A^T P_j A - \beta^2 A^T P_j B \left(Q + \beta B^T P B \right)^{-1} \beta B^T P_j A.
\end{equation}

重复迭代,直至$P$收敛到某一值。根据数值模拟的$P$值计算$F$值,进而得到最优政策方程和价值方程。

\section{范例}
假定这样一个线性二次控制问题。中央银行试图通过控制利率$r_t$来影响通货膨胀率$\pi_t$和产出$y_t$,收益方程(payoff function)
\begin{equation}
  \mathcal{L}_t=\pi_t^2 + y_t^2 + 0.1 r_t^2.
\end{equation}

经济系统的结构,由两个状态变量的运动定律构成
\begin{subequations}
  \begin{align}
    \label{eq:LQC-example-pi-motion}
    \pi_{t+1} &= 0.75 \pi_t - 0.5 r_t + \varepsilon_{\pi,t+1}, \\
    \label{eq:LQC-example-y-motion}
        y_{t+1} &= 0.25 y_t - 0.5 r_t + \varepsilon_{y,t+1}.
  \end{align}
\end{subequations}

作为决策者,中央银行的最大化问题为
\begin{equation*}
  \max_{\{r_t\}} E \sum_{t=0}^{\infty} \beta^t \mathcal{L}_t ,\quad \text{ s.t.}     \eqref{eq:LQC-example-pi-motion}-\eqref{eq:LQC-example-y-motion}
\end{equation*}

\subsection{最优线性二次控制问题}
将这一问题改写为最优线性二次控制的一般形式:
\begin{align*}
  \label{eq:LQC-example-LQ-form}
  &\max_{\{u_t\}} E \sum_{t=0}^{\infty} \beta^t \left(x_t^T R x_t + u_t^T Q u_t\right), \\
  &\text{ s.t. } x_{t+1} = A x_t + B u_t + \varepsilon_t,
\end{align*}
其中状态变量,控制变量,外生扰动变量分别用$x_t,u_t,\varepsilon_t$表示
\begin{equation*}
  x_t \equiv \begin{pmatrix}
    \pi_t \\
    y_t
  \end{pmatrix},\quad
  u_t \equiv r_t,\quad
  \varepsilon_t = \begin{pmatrix}
  \varepsilon_{\pi,t} \\
  \varepsilon_{y,t}
  \end{pmatrix},
\end{equation*}
基础系数矩阵$A,B,R,Q$分别为
\begin{equation*}
  A=\begin{pmatrix}
  0.75 & 0 \\
  0 & 0.25
  \end{pmatrix}, \quad
  B = \begin{pmatrix}
  -0,5 \\ -0.5
  \end{pmatrix}, \quad
  R = \begin{pmatrix}
  1 & 0 \\
  0 & 1
  \end{pmatrix}, \quad
  Q=0.1.
\end{equation*}

如前文所述,我们需要使用数值方法,递归处理线性矩阵Riccati方程以近似价值方程系数矩阵$P$,进而测算政策方程的系数矩阵$F$,对应  \eqref{eq:LQC-mat-Riccati-equation-iteration} \eqref{eq:LQC-linear-policy-F}。

%\subsection{Matlab程序说明}
\subsection{Matlab程序}
\begin{lstlisting}[language=Matlab]
clear;
%时间贴现系数的确定
beta = 0.99;

%定义矩阵
R=zeros(2,2);
Q=zeros(1,1);
A=zeros(2,2);
B=zeros(2,1);

%为矩阵赋值
R(1,1)=1;
R(2,2)=1;
Q(1,1)=0.1;
A(1,1)=0.75;
A(2,2)=0.25;
B(1,1)=-0.5;
B(2,1)=-0.5;

%matrix Riccati equation迭代
%d描述$P_j$和邻近迭代$P_{j+1}$之间元素的最大绝对值偏差
d = 1;
%i记录迭代次数
i = 0;
% 最大绝对值偏差d保存在D;迭代次数i保存在I
D=0;
I=0;
%迭代P的初始值定义为P0
%说明:定义P0和Q的非零值,为了使得linear matrix Riccati equation迭代操作中
%$(Q+\beta B^T P_j B) \neq 0$. 只有这个值不等于0,才能使得该矩阵是可逆的。
P0=zeros(2,2);
P0(1,1)=-0.000001;
P0(2,2)=-0.000001;


format long

%matrix Riccati Equation 迭代
%迭代新生成的$P_{j+1}$写入矩阵P1。
%两个P值的差值,写入矩阵Pd。
%当d<0.0000000001时,迭代终止。否则继续进行。
%每次迭代的d和i写入D和I,以备最终程序输出。
while d > 0.0000000001
P1 = R + beta * A' * P0 * A - beta^2 * A' * P0 * B * (inv(Q+beta * B' * P0 * B)) *  (B' * P0 * A);
Pd = P1 - P0;
d = max(abs(Pd));
d = max(d');
D=[D d];
P0 = P1;
i = i+1;
I = [I i];
end

%根据迭代生成的P值,计算policy function的矩阵F
P=P0;
F = -inv(Q + beta * B' * P * B) * (beta * B' * P * A);
ID = [I(2:length(I))' D(2:length(I))'];
disp('       i                         d');
disp(ID);
disp('  SOLUTIONS');
disp('F');
disp(F);
disp('P');
disp(P);
\end{lstlisting}

\subsection{Matlab程序输出}
运行Matlab程序,输出如下
\begin{lstlisting}
  i                         d
1.000000000000000   1.000000938124847
2.000000000000000   0.325234163624664
3.000000000000000   0.080558191789241
4.000000000000000   0.018875706161742
5.000000000000000   0.004338686564976
6.000000000000000   0.000992730600232
7.000000000000000   0.000226907809828
8.000000000000000   0.000051851748175
9.000000000000000   0.000011848232754
10.000000000000000   0.000002707312049
11.000000000000000   0.000000618616947
12.000000000000000   0.000000141352999
13.000000000000000   0.000000032298933
14.000000000000000   0.000000007380254
15.000000000000000   0.000000001686376
16.000000000000000   0.000000000385334
17.000000000000000   0.000000000088048

SOLUTIONS
F
0.744954171236066   0.175909878487998

P
1.430293038776173  -0.106183304364736
-0.106183304364736   1.044189928712962
\end{lstlisting}

由此可得最优政策方程
\begin{align}
  &u_t = - F x_t, \quad r_t = -F \begin{pmatrix} \pi_t \\ y_t \end{pmatrix}, F = (0.745, 0.176)\\
\label{eq:LQC-example-policy-function}
  & r_t = 0.745 \pi_t + 0.176 y_t.
\end{align}

价值方程
\begin{equation}
  \label{eq:LQC-example-value-function}
  V(\pi_t,y_t)= 1.430 \pi_t^2 + 1.044 y_t^2 - 0.212 y_t.
\end{equation}

根据最优政策方程,当通胀水平和/或产出水平高于稳定状态时,中央银行应当提高利率\footnote{注意式中的变量$z_t = (\tilde{z}_t - z)/z$,表示实际观测到的变量,相对于稳态水平的deviation。}。利率对通胀波动的响应幅度高于对产出波动的响应(0.745相对于0.176),这与模型的假设条件有关:
\eqref{eq:LQC-example-pi-motion}-\eqref{eq:LQC-example-y-motion}假定通胀波动比产出波动更为持久(0.75相对于0.25),从而过去时段的物价波动,比起产出波动来,更有可能对当期经济系统产生影响。

根据价值方程,首先同样地,通胀波动二次项的系数大于产出波动二次项的系数。交互项的系数为负,反映了当通胀和产出的波动同方向变化时,比如通货膨胀伴随产出增加(或通货紧缩伴随产出减少),中央银行更容易通过调节利率这一政策工具来稳定经济运行。反之,如果两种波动反方向变化,稳定经济运行将更为困难。
