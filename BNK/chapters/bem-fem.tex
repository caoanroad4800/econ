%!TEX root = ../DSGEnotes.tex
\chapter{有界元法和有限元法}
\label{sec:bem-fem-methods}

\section{边界值问题:位势方程}
\label{sec:bem-fem-potential-bvp}
我们从二阶偏微分方程入手,介绍边界值问题(boundary value problem)\index{boundary value problem \dotfill 边界值问题}。一个合适的例子是位势方程(potential equation)。

\subsection{偏微分算子及椭圆边界值问题}

定义有界域$\Omega \in \mathbb{R}^d, d=2,3$,边界$\Gamma = \partial \Omega$,外代数单位向量空间(exterior unit normal vector)\index{exterior algebra \dotfill 外代数} $\underline{n}(x)$对于$x \in \Gamma$几乎处处存在。对于$x \in \Omega$,我们考虑一个线性二阶偏微分的自伴随算子\footnote{有限维内积向量空间$V$中,自伴随算子A是一个从$V$到$V$自身的线性映射$\langle A \bm{u}, \bm{\nu} \rangle = \langle \bm{\nu}, A \bm{u} \rangle, \, \forall \nu, w \in V$。}
(self-adjoint operator)\index{self-adjoint operator} $L$
,作用于实值标量方程$u$
\begin{equation}
  \label{eq:bvp-self-adjoint-pde-operator}
  \left( L \, u \right)(x) \coloneqq - \sum_{i,j=1}^d \frac{\partial}{\partial x_j} \left[ a_{ji} (x) \frac{\partial}{\partial x_i} u(x)\right] + a_0(x)\, u(x),
\end{equation}
其中$a_{ji}(x), \, i,j =1,\ldots, d, \, x \in \Omega$表示系数方程,假定为平滑的并满足$a_{ij}(x) = a_{ji}(x)$。由此可以构建一个对称的系数矩阵$A(x)$,满足
\begin{equation*}
  A(x) = \left( a_{ij}(x) \right)_{i,j=1}^{d}, \quad x \in \Omega,
\end{equation*}
对应实数特征根$\lambda_{k}(x)$。

当且仅当$\lambda_{k}(x) > 0$对于所有$k=1,\ldots,d$都成立时,我们称偏微分算子$L$在某一个$x \in \Omega$上是椭圆(elliptic)的。

更进一步,如果$\forall x \in \Omega$该条件都成立,那么我们称$L$在$\Omega$上是椭圆的。

如果存在一个一致下界(uniform lower bound) $\lambda_0 > 0$,满足
\begin{equation}
  \label{eq:bvp-def-uniformly-elliptic}
  \lambda_k (x) \ge \lambda_0, \quad \forall k = 1,\ldots,d, \, \forall x \in \Omega,
\end{equation}
那么我们称$L$在$\Omega$上一致椭圆(uniformly elliptic)。

\subsection{边界条件}

边界条件的分析,可以从散度定理开始。
\begin{theorem}[散度定理]
  \label{theorem:bvp-gauss-divergence-theorem}
  散度定理(divergence theorem)\index{divergence theorem \dotfill 高斯散度定理},又称奥斯特罗格拉德斯基——高斯定理(Ostrogradsky-Gauss theorem)\index{Ostrogradsky-Gauss theorem \dotfill 奥斯特罗格拉德斯基——高斯定理} 、高斯散度定理(Gauss' theorem)\index{Gauss theorem \dotfill 高斯散度定理}等,是指
  \begin{equation}
    \label{eq:bvp-gauss-divergence-theorem}
    \int_{\Omega} \frac{\partial}{\partial x_i} f(x) \, dx = \int_{\Gamma} \left[ \gamma_0^{int} f(x) \right] n_i(x) \, d s_x, \quad i = 1,\ldots,d,
  \end{equation}
  其中$\gamma_0^{int} f(x)$是某个给定方程$f(x), x\in \Omega$的内界迹(interior boundary trace, Theorem \ref{theorem:sobolev-manifold-trace-theorem})\index{trace \dotfill 迹},满足
  \begin{equation}
    \label{eq:bvp-interior-boundary-trace}
    \gamma_0^{int} f(x) \coloneqq \lim_{\Omega \owns \tilde{x} \mapsto x \in \Gamma} f \left( \tilde{x} \right), \quad \forall x \in \Gamma = \partial \Omega.
  \end{equation}
\end{theorem}

假定两个足够光滑的方程$u,\nu \in \Omega$,通过设定$f(x) = u(x) \, \nu(x)$,可以将散度定理\eqref{eq:bvp-gauss-divergence-theorem}改写为分部积分(integration by parts)\index{integration by parts \dotfill 分部积分}的形式
\begin{equation*}
  \int_{\Omega} u(x) \frac{\partial}{\partial x_i} \nu(x) \, x
  + \int_{\Omega} \nu(x) \frac{\partial}{\partial x_i} u(x) \, x
  = \int_{\Gamma}  \left[ \gamma_0^{int} u(x) \right] \left[ \gamma_0^{int} \nu(x) \right] n_i(x) \, d s_x.
\end{equation*}

重新调整上式,将$\nu(x)$视作检测方程(test function),两侧乘以\eqref{eq:bvp-self-adjoint-pde-operator}中的二阶偏微分算子$\left(L\,u\right)(x)$,在$\Omega$中求积
\begin{equation}
  \begin{split}
    &\left( L \, u \right)(x) \, \nu(x) \coloneqq - \sum_{i,j=1}^d \frac{\partial}{\partial x_j} \left[ a_{ji} (x) \frac{\partial}{\partial x_i} u(x)\right] \, \nu(x) + a_0(x)\, \underbrace{u(x) \, \nu(x)}_{\equiv f(x)}, \\
    \hookrightarrow & \int_{\Omega} \left( L \, u \right)(x) \, \nu(x) \, dx = - \sum_{i,j=1}^d \int_{\Omega} \frac{\partial}{\partial x_j} \left[ a_{ji} (x) \frac{\partial}{\partial x_i} u(x)\right] \, \nu(x) \, dx,
  \end{split}
\end{equation}
使用分部积分$\hookrightarrow$
\begin{equation*}
  \begin{split}
    \int_{\Omega} \left( L \, u \right)(x) \, \nu(x) \, dx =&  \underbrace{\sum_{i,j=1}^d \int_{\Omega} a_{ji}(x) \frac{\partial}{\partial x_i} u(x) \, \frac{\partial}{\partial x_j} \nu(x) \, dx}_{\coloneqq a\left(u,\nu \right)} \\
   &- \sum_{i,j=1}^{d} \int_{\Gamma} n_j(x) \left[ \gamma_0^{int} (x) \left( a_{ji}(x) \frac{\partial}{\partial x_i} u(x)\right) \right] \left[ \gamma_0^{int} \nu(x) \right] \, d s_x,
  \end{split}
\end{equation*}

由此,我们由散度定理(Theorem \ref{theorem:bvp-gauss-divergence-theorem})推导出格林第一恒等式(Green's first identity)\index{Green identities!first 格林第一恒等式}
\begin{equation}
  \label{eq:bvp-a-u-nu-inner-prod}
  \begin{split}
  a\left(u,\nu \right) &\coloneqq \sum_{i,j=1}^d \int_{\Omega} a_{ji}(x) \frac{\partial}{\partial x_i} u(x) \, \frac{\partial}{\partial x_j} \nu(x) \, dx \\
  & = \int_{\Omega} \left( L \, u \right)(x) \, \nu(x) \, dx + \sum_{i,j=1}^{d} \int_{\Gamma} \underbrace{n_j(x) \left[ \gamma_0^{\text{int}} (x) \left( a_{ji}(x) \frac{\partial}{\partial x_i} u(x)\right) \right]} \left[ \gamma_0^{\text{int}} \nu(x) \right] \, d s_x,\\
  & =\int_{\Omega} \left( L \, u \right)(x) \, \nu(x) \, dx + \int_{\Gamma} \underbrace{\left[ \gamma_1^{\text{int}} u(x) \right]} \left[ \gamma_0^{\text{int}} \nu(x) \right] \, d s_x,
  \end{split}
\end{equation}
其中定义$\gamma_1^{int}$为内部共形导数(interior co-normal derivative, \cite{Mikhailov:2006vo, Mikhailov:2009wj, Ancona:2009bo})
\begin{equation}
  \label{eq:bvp-int-conformal-derivative}
  \gamma_1^{int}u(x) \coloneqq \lim_{\Omega \owns \tilde{x} \mapsto x \in \Gamma} \left[
\sum_{i,j=1}^{d} n_j(x) a_{ji}\left( \tilde{x} \right) \frac{\partial}{\partial \tilde{x}_i} u \left( \tilde{x} \right)
  \right], \quad x \in \Gamma.
\end{equation}

将格林第一恒等式\eqref{eq:bvp-a-u-nu-inner-prod}中的$u,\nu$互换位置,我们有
\begin{equation*}
  a\left(\nu,u \right) = \int_{\Omega} \left( L \, \nu \right)(x) \, u(x) \, dx + \int_{\Gamma} \left[ \gamma_1^{int} \nu(x) \right] \left[ \gamma_0^{int} u(x) \right] \, d s_x,
\end{equation*}
由上式和\eqref{eq:bvp-a-u-nu-inner-prod}我们有格林第二恒等式(Green's second identity)\index{Green identities!second 格林第二恒等式}: $\forall u,\nu \in \Omega$且$u,\nu$足够平滑
\begin{equation}
  \label{eq:bvp-a-nu-u-green-2nd-identity}
  \begin{split}
    &a(u,\nu) = a(\nu,u) \Longleftrightarrow \\
    &\int_{\Omega} \left( L \, u \right)(x) \, \nu(x) \, dx + \int_{\Gamma} \left[ \gamma_1^{int} u(x) \right] \left[ \gamma_0^{int} \nu(x) \right] \, d s_x
    = \int_{\Omega} \left( L \, \nu \right)(x) \, u(x) \, dx + \int_{\Gamma} \left[ \gamma_1^{int} \nu(x) \right] \left[ \gamma_0^{int} u(x) \right] \, d s_x
  \end{split}
\end{equation}

下面来考虑一个特殊情况,$a_{ij}(x)=\delta_{ij}$,$\delta_{ij}$是克罗内克乘积(Kronecker product)\index{Kronecker product \dotfill 克罗内乘积}。\eqref{eq:bvp-self-adjoint-pde-operator}的二阶偏微分算子$\left(L\,u\right)(x)$变为拉普拉斯算子
\begin{equation}
  \label{eq:bvp-laplace-operator}
  \left( L \, u \right)(x) = - \Delta u(x) \coloneqq - \sum_{i=1}^{d} \frac{\partial^2}{\partial x_i^2} u(x), \quad x \in \mathbb{R}^d.
\end{equation}

内部共形导数$\gamma_1^{int}$ \eqref{eq:bvp-int-conformal-derivative}变为
\begin{equation}
  \label{eq:bvp-laplace-conformal-derivative}
  \gamma_1^{int}u(x) = \frac{\partial}{\partial n_x} u(x) \coloneqq \underline{n}(x) \bigtriangledown u(x), \quad x \in \Gamma.
\end{equation}

对边界域$\Gamma = \partial \Omega$分解成三个不相交集合的并集(disjoint union)
\begin{equation*}
  \Gamma = \overline{\Gamma}_D \cup \overline{\Gamma}_N \cup \overline{\Gamma}_R,
\end{equation*}

对应地,边界值问题变为两部分:第一部分,在$\Omega$中,基于给定的方程$f(x)$,寻找偏微分算子$(L u)(x)$,使得
\begin{equation}
  \label{eq:bvp-extension-omega-cond}
  \left( L \, u \right)(x) = f(x), \quad x \in \Omega.
\end{equation}

第二部分,在$\Gamma$中,基于给定的方程$g(x)$,寻找内界迹$\gamma_0^{int}u(x)$或者内共形导数$\gamma_1^{int}(x)$。随着$\Gamma$的取值范围不同,分为三种情况:
\begin{subequations}
  \begin{equation}
    \label{eq:bvp-extension-gamma-dirichlet}
    \gamma_0^{int} u(x) = g_D(x), \quad x \in \Gamma = \Gamma_D,
  \end{equation}
  \begin{equation}
    \label{eq:bvp-extension-gamma-neumann}
    \gamma_1^{int} u(x) = g_N(x), \quad x \in \Gamma = \Gamma_N,
  \end{equation}
  \begin{equation}
    \label{eq:bvp-extension-gamma-robin}
    \kappa(x) \, \gamma_0^{int} u(x) + \gamma_1^{int} u(x) = g_R(x), \quad x \in \Gamma = \Gamma_R.
  \end{equation}
\end{subequations}


\begin{definition}[边界值条件]
  \label{definition:boundary-value-problem}
  于是我们有以下几种不同的边界值条件:
\begin{itemize}
  \item $\Gamma = \Gamma_D:$ \eqref{eq:bvp-extension-omega-cond} + \eqref{eq:bvp-extension-gamma-dirichlet} $\rightarrow$ 狄利克雷边界值条件(Dirichlet boundary value condition)\index{Dirichlet boundary value condition \dotfill 狄利克雷边界值条件},
  \item $\Gamma = \Gamma_N:$ \eqref{eq:bvp-extension-omega-cond} + \eqref{eq:bvp-extension-gamma-neumann} $\rightarrow$ 诺依曼边界值条件(Neumann boundary value condition)\index{Neumann boundary value condition \dotfill 诺依曼边界值条件},
  \item $\Gamma = \Gamma_R:$ \eqref{eq:bvp-extension-omega-cond} + \eqref{eq:bvp-extension-gamma-robin} $\rightarrow$ 罗宾边界值条件(Robin boundary value condition)\index{Robin boundary value condition \dotfill 罗宾边界值条件},
  \item 混合型边界值条件,以上三种情况的组合。
\end{itemize}
\end{definition}

有时候我们还需要将线性罗宾边界值条件扩展为非线性的情况,\eqref{eq:bvp-extension-gamma-robin} $\rightarrow$
\begin{equation}
  \label{eq:bvp-extension-gamma-robin-nonlinear}
  G\left( \gamma_0^{int} u(x), x \right) + \gamma_1^{int} u(x) = g_R(x), \quad x \in \Gamma = \Gamma_R,
\end{equation}
其中$G(u,\cdot)$是某个给定的非线性方程,如$u(x)^3$。

对于边界值问题的解$u(x)$,还需要注意以下几点
\begin{enumerate}
  \item $u(x)$的存在性和唯一性,相关讨论可参考如\cite{Ladyzhenskaya:1968vq},
  \item 观测到的数据需要是充分平滑的,以确保$u(x)$充分可微(sufficiently differentiable)
  \begin{equation*}
    u \in C^2(\Omega) \cap C^1 \left( \Omega \cup \Gamma_N \cup \Gamma_R \right) \cap C(\Omega \cup \Gamma_D).
  \end{equation*}
\end{enumerate}

\subsection{诺依曼边界值问题}
对于诺依曼边界值条件的解,其存在性和唯一性需要做进一步讨论。

假定$\nu_1(x)=1, x \in \Omega$是关于$\nu_1(x)$的齐次诺依曼边界值问题的一个解,\eqref{eq:bvp-laplace-operator}、\eqref{eq:bvp-laplace-conformal-derivative} $\Rightarrow$关于$u(x)$的诺依曼边界值问题
\begin{equation}
  \label{eq:bvp-neumann-nu-homo}
  \begin{split}
    \left( L \, \nu_1 \right)(x)=0, \quad x \in \Omega,\\
    \gamma_1^{int} \nu_1(x) = 0, \quad x \in \Gamma.
  \end{split}
\end{equation}

\eqref{eq:bvp-neumann-nu-homo}
代入格林第二恒等式\eqref{eq:bvp-a-nu-u-green-2nd-identity}可得正交条件
\begin{equation}
  \label{eq:bvp-neumann-green-2}
  \int_{\Omega} \left( L \, u \right)(x) \, dx + \int_{\Gamma} \gamma_1^{int} u(x) \, d s_x = 0,
\end{equation}

诺依曼边界值条件\eqref{eq:bvp-extension-omega-cond}、 \eqref{eq:bvp-extension-gamma-neumann}$\Rightarrow$
\begin{equation}
  \label{eq:bvp-neumann-cond}
\begin{split}
  \left( L u \right)(x) = f(x), \quad x \in \Omega, \\
  \gamma_1^{int} u(x) = g_N(x), \quad x \in \Gamma.
\end{split}
\end{equation}

对于给定的$f$和$g_N$,根据正交条件\eqref{eq:bvp-neumann-green-2},我们可以假设诺依曼问题的可解性条件(solvability condition)
\begin{equation}
  \label{eq:bvp-neumann-green-2-new}
  \int_{\Omega} f(x) \, dx + \int_{\Gamma} g_N(x) \, d s_x = 0.
\end{equation}

换句话说,如果关于$\nu(x)$的齐次诺依曼边界问题解是$\nu_1(x)=1, x \in \Omega$,那么关于$u$的诺依曼边界值问题\eqref{eq:bvp-neumann-cond}的解并不唯一:不只包括一个解$u(x)$,还包括另一个解$\tilde{u}(x)$,满足关系
\begin{equation*}
  \tilde{u}(x) = u(x) + \alpha, \quad x \in \Omega,
\end{equation*}
其中常数$\alpha \in \mathbb{R}$的值是唯一的,取决于为了使第一个解$u(x)$成为诺依曼边界值问题\eqref{eq:bvp-neumann-cond}的解,而需要在系统中加入的规模调整条件,如
\begin{equation*}
  \int_{\Omega} u(x) \, dx = 0, \quad \text{或者} \quad \int_{\Gamma} \gamma_0^{int}u(x) \, ds_x =0.
\end{equation*}

\section{方程空间}
在进一步介绍边界值问题的弱形式之前,一些与之紧密相关的方程空间的知识是必需的。
相关教材,可参考如\cite{McLean:2000ta, Adams:2003wi, Tartar:2007vm, Mazya:2009vz, Mazya:2009wu}。

%\section{\texorpdfstring{$\varepsilon$}{e}SOA}
\subsection{\texorpdfstring{$C^{k}(\Omega),C^{k,\kappa}(\Omega)$}{CK}空间}

给定$d \in \mathbb{N}$。作如下定义:
\begin{itemize}
  \item 向量(vector) $\alpha = \left( \alpha_1, \alpha_2, \ldots, \alpha_d \right), \alpha_i \in \mathbb{N}_0$。
  \item 多重指标(multi-index)的绝对值 $\left| \alpha \right|=\sum_{i=1}^{d} \alpha_i$。
  \item 阶乘(factorial) $\alpha ! = \alpha_1! \, \alpha_2 ! \,  \ldots \alpha_d !$。
\end{itemize}

给定$x \in \mathbb{R}^d$我们有
\begin{equation*}
  x^{\left| \alpha \right|} = x_1^{\alpha_1} \, x_2^{\alpha_2} \ldots x_d^{\alpha_d}.
\end{equation*}

给定一个充分平滑的实值方程$u$,其相对于$x$的$\alpha$阶偏微分导数
\begin{equation*}
  D^{\alpha} u(x) \coloneqq \left( \frac{\partial }{\partial x_{1}} \right)^{\alpha_1} \left( \frac{\partial }{\partial x_{2}} \right)^{\alpha_2} \ldots \left( \frac{\partial }{\partial x_{d}} \right)^{\alpha_d} u \left( x_1, x_2, \ldots, x_d \right).
\end{equation*}

给定一个开放子集$\Omega \subseteq R^{d}$,对于某个标量$k \in \mathbb{N}_0$。则$C^{k}(\Omega)$表示在$\Omega$域中有界且$k$次连续可导的方程空间。对于某个方程$u \in \Omega$,$u$的范数(norm)\index{norm \dotfill 范数}值是有限的
\begin{equation*}
  \| u \|_{C^{k}(\Omega)} \coloneqq \sum_{\left| \alpha \right| \le k} \sup_{x \in \Omega} \big| D^{\alpha} u(x) \big| < \infty,
\end{equation*}
随着$k \rightarrow \infty$,$C^{\infty}(\Omega)$是个有界且无限阶连续可积的方程空间。

对于方程$u(x), x \in \Omega$,我们将$u$的支撑(support) \index{support \dotfill 支撑}定义为$\text{supp} \, u$
\begin{equation*}
  \text{supp} \, u \coloneqq \overline{x \in \Omega: u(x) \neq 0}.
\end{equation*}

进而定义$C_0^{\infty}(\Omega)$为$C^{\infty}(\Omega)$中的紧支撑(compact support)方程空间。

定义$C^{k,\kappa}, k \in \mathbb{N}_0, \kappa \in (0,1)$为霍德尔连续方程空间(Hölder continuous function space)\index{Hölder space \dotfill 霍德尔空间},对应范数为
\begin{equation*}
  \| u \|_{C^{k,\kappa}(\Omega)} \coloneqq \| u \|_{C^{k}(\Omega)} + \sum_{\left| \alpha \right| = k} \sup_{x,y\in\Omega, x \neq y} \frac{\left| D^{\alpha}u(x)-D^{\alpha}u(y) \right|}{\left| x - y \right|^{\kappa}}
\end{equation*}

当$\kappa=1$时,$C^{k,1}$用来表示第$\left| \alpha \right|=k$次偏导数$D^{\alpha}u(x)$是利普希茨连续方程(Lipschitz continuous)\index{Lipschitz continuous function \dotfill 利普希茨连续方程}的方程$u \in C^{k}(\Omega)$所组成的空间。

我们用$\Gamma$来表示开放集$\Omega \subset \mathbb{R}^d$的边界
\begin{equation*}
  \Gamma \coloneqq \partial \Omega = \bar{\Omega} \cup \left( \mathbb{R}^d \backslash \Omega \right).
\end{equation*}

当$d \ge 2$,$\Gamma = \partial \Omega$可以视作利普希茨方程的局部图,随着所处在$\Gamma$中不同的位置,对应不同的笛卡尔坐标系。一个最简单的例子是假定一个利普希茨方程$\gamma: \mathbb{R}^{d-1} \mapsto \mathbb{R}$,满足
\begin{equation*}
  \Omega \coloneqq \left\{
  x \in \mathbb{R}^d: x_d < \gamma(\tilde{x}), \quad \forall \tilde{x}=(x_1,\ldots,x_{d-1}) \in \mathbb{R}^{d-1}
  \right\}.
\end{equation*}

由利普希茨方程的性质我们有,$\gamma(\cdot)$满足
\begin{equation*}
  \big| \gamma(\tilde x) - \gamma (\tilde y) \big|
  \le L \big| \tilde{x} - \tilde{y} \big|, \quad \forall \tilde{x},\tilde{y} \in \mathbb{R}^{d-1},
\end{equation*}
那么$\Omega$被称为一个利普希茨亚图(Lipschitz hypograph)\index{Lipschitz hypograph \dotfill 利普希茨亚图},对应边界$\Gamma$
\begin{equation*}
  \Gamma = \left\{
  x \in \mathbb{R}^d: x_n = \gamma(\tilde{x}), \quad \forall \tilde{x} \in \mathbb{R}^{d-1}
  \right\}.
\end{equation*}

\begin{definition}[利普希茨域]
  \label{definition:bvp-lipschitz-domain-def}
可参考 \cite{Heinonen:2005wq}。  某一个开集$\Omega \subset \mathbb{R}^d, d \ge 2$在满足如下条件时,被称为利普希茨域(Lipschitz domain)\index{Lipschitz domain \dotfill 利普希茨域},或有利普希茨边界的域(domain with Lipschitz boundary):
  \begin{itemize}
    \item $\Omega$的边界$\Gamma = \partial \Omega$是紧凑的,并且
    \item $\exists$有限的索引族(index family)\index{index family \dotfill 索引族}  $\left\{W_j\right\}$和$\left\{ \Omega_j\right\}$,满足
    \footnote{对于集合$I$和$S$,某个方程$x$
    \begin{equation*}
    \begin{split}
      x:&I \mapsto S \\
      &i \mapsto x_i = x(i)
    \end{split}
  \end{equation*}被称作用$I$索引的$S$中元素的族(family of elements)\index{},也表示为$\left\{ x_i \right\}_{i \in I},\, x_i \subset S$。
  }
  \begin{itemize}
    \item 索引族$\left\{ W_j \right\}$是对$\Gamma$的有限开覆盖(finite open cover)\index{cover(topology) \dotfill 覆盖(拓扑学)}
    \begin{equation*}
      W_j \in \mathbb{R}^d, \quad \& \, \Gamma \subseteq \bigcup_j W_j.
    \end{equation*}
    \item 每一个索引族$\Omega_j, \, \forall j$都可以通过一定的操作变换(transformation)为利普希茨亚图,如旋转(rotation)和平移(translation)等。
    \item $W_j \cap \Omega = W_j \cap \Omega_j, \quad \forall j$。
  \end{itemize}
  \end{itemize}
\end{definition}

需要注意的是,利普希茨边界$\Gamma = \partial \Omega$作局部表达的方案,即对$W_j$和$\Omega_j$的选取,通常来讲并不是唯一的\footnote{如非利普希茨域的例子,可参考\cite{McLean:2000ta}。}。

如果方程$\gamma(\cdot)$中参数的选取使得满足$\gamma \in C^{k}(\mathbb{R}^{d-1})$,我们称这个利普希茨边界$\Gamma$是$k$次可微的;如果满足$\gamma \in C^{k, \kappa}(\mathbb{R}^{d-1})$,我们称$\Gamma$为霍德尔连续(Hölder continuous)\index{Hölder continuous \dotfill 霍德尔连续};如果$\gamma$仅仅是在某个局域内满足该条件,则我们称对应的$\Gamma$为分段平滑边界(piecewise smooth boundary)\index{piecewise smooth boundary \dotfill 分段平滑边界}。

\subsection{勒贝格\texorpdfstring{$L^p(\Omega)$}{(LP)}空间}

数学上$L^{p}(\Omega)$空间($L^{p}(\Omega)$ space) 又称勒贝格空间(Lebesgue space)\index{Lesbegue space \dotfill 勒贝格空间},指$\Omega$上一组测度方程(measurable function)的等价类的集合,这些测度方程都是$p$次勒贝格可积方程(Lebesgue integrable function, Definition \ref{definition:lebesgue-integrable-func-def})\index{Lebesgue integrable function \dotfill 勒贝格可积方程} \footnote{对应地,$\ell^p$空间是由p次可和序列组成的空间。}。

\begin{definition}[勒贝格可积方程]
  \label{definition:lebesgue-integrable-func-def}
  勒贝格可积方程(Lebesgue integrable function)是指该方程的绝对值的$p$次幂的积分是有限的,如
  \begin{equation*}
    \int_{\Omega} \left| u(x) \right|^p d x < \infty.
  \end{equation*}
\end{definition}

\subsubsection{\texorpdfstring{$\sigma$}{SIGMA}代数}
\begin{definition}[sigma代数]
  \label{definition:measure-sigma-algebra}
设$S$是一个非空集合,另有一个集合$\Sigma$中的所有元素都是$S$的子集,那么我们将满足以下条件的$\Sigma$成为$S$上的一个$\sigma$代数($\sigma$-algebra)\index{sigma!algebra \dotfill sigma代数} \citep[p.4]{Bogachev:2007wh, Bogachev:2007tn}
\begin{itemize}
  \item $S$在$\Sigma$中
  \begin{equation*}
    S \in \Sigma,
  \end{equation*}
  \item 如果一个集合$A$在$\Sigma$中,那么它的补集(complement)\index{complement \dotfill 补集} 也在$\Sigma$中
  \begin{equation*}
    A \in \Sigma \Rightarrow A^{\complement} \in \Sigma,
  \end{equation*}
  \item 如果$n$个集合$A_1, \ldots A_n$都在$\Sigma$中,那么他们的并集(union) \index{union \dotfill 并集}也在$\Sigma$中
  \begin{equation*}
    \left( A_n \in \Sigma , \quad  \forall n \in \mathbb{N} \right)
    \Rightarrow \bigcup_{i=1}^{n} A_{i} \in \Sigma.
  \end{equation*}
\end{itemize}
\end{definition}

\begin{definition}[幂集]
  \label{definition:measure-powerset}
  对于任一集合$S$的幂集(powerset)\index{powerset \dotfill 幂集}是指这样的一个集合,包括空集$\varnothing$、$S$本身和$S$的所有集合,常常表示为$\mathcal{P}(S)$或$2^{S}$。

  在公理集合论(axiomatic set theory)\index{axiomatic set theory \dotfill 公理集合论}例如ZFC集合论(ZFC axioms)\index{ZFC axioms \dotfill ZFC集合论}假定了任何集合的幂集均存在。

  $\mathcal{P}(S)$上的全部子集称为$S$上的集族(family of sets over $S$)\index{family of sets \dotfill 集族}。
\end{definition}

假定一个有限集$S$有$n$个元素,表示为$|S|=n$或$card(S)=n$,即$S$的势(cardinality)\index{cardinality \dotfill 势}是$n$。那么$S$的幂集里有 $card(\mathcal{P}(s)) = 2^n$个元素。例如$S=\left\{a,b,c\right\}, card{S}=3$。$S$的全部子集包括
\begin{equation*}
  \begin{cases}
    \left\{ a \right\}, \left\{ b\right\},\left\{c\right\}, \\
    \left\{ a,b \right\},\left\{ a,c \right\},\left\{ b,c \right\},
  \end{cases}
\end{equation*}
因此$\left| \mathcal{P}(s) \right|$为6个$S$的子集,加上$\varnothing$和$S$自身,共$2^3=8$个。

对于非空集合$S$来说,$S$上的一个$\sigma$代数是指其幂集(powerset, 第\ref{definition:measure-sigma-algebra}节) 的一个子集$\Sigma$,$\Sigma$中的元素在经过有限个补集、并集、交集(intersection)\index{intersection \dotfill 交集}这三种运算后,依然属于$\Sigma$。即是说,$\Sigma$对这三种运算是封闭(closed)的。

\subsubsection{测度,可测空间,测度空间}
\label{sec:measure-measure}
\begin{definition}[测度]
  对于一个集合$S$表示函数的定义域,有关于$S$的可测空间$\Sigma$,$\Sigma$中的元素是$S$上的子集族(family of subsets),并且$\Sigma$是一个$\sigma$代数(第\ref{definition:measure-sigma-algebra}节)。沿着扩展实数线(extended real number line) \index{extended real number line \dotfill 扩展实数线}
  \footnote{扩展实数线是指实数集合$\mathbb{R}$加上$-\infty$和$+\infty$,
  常写作$\overline{\mathbb{R}}$或$\left[ -\infty, +\infty \right]$。}
  定义一个测度方程$\mu \in \Sigma$,如果$\mu$满足如下条件,我们称它为一个测度(measure) \index{measure \dotfill 测度}:
  \begin{itemize}
    \item 非负性(non negative)
    \begin{equation*}
      \forall E \in \Sigma: \mu(E) \ge 0,
    \end{equation*}
    \item 空集合的测度为$0$
    \begin{equation}
      \mu(\varnothing) = 0,
    \end{equation}
    \item 可数可加性(countable additivity)\index{countable additivity \dotfill 可数可加性},又称$\sigma$可加性($\sigma$ additivity) \index{sigma!additivity \dotfill sigma可加}:$\left\{ E_i \right\}_{i=1}^{\infty}  = \{ E_1,E_2,\ldots \}$为$\Sigma$中可数个两两不不相交序列的集合(pairwise disjoint sets in $\Sigma$),则所有$E_i$并集的测度等于每个$E_i$的测度之和
    \begin{equation*}
      \mu \left( \bigcup_{i=1}^{\infty} E_i \right) = \sum_{k=1}^{\infty} \mu(E_i).
    \end{equation*}
  \end{itemize}
\end{definition}

\begin{definition}[可测空间]
进而,我们称$\left( S, \Sigma \right)$为一个可测空间(measurable space)\index{measurable space \dotfill 可测空间}。$\Sigma$中的所有元素$\{ E_n \}_{n=1}^{\infty}$ 成为可测集(measurable sets)\index{measurable sets \dotfill 可测集合}。
\end{definition}

\begin{definition}[测度空间]
  \label{definition:measure-measure-space}
一个三元组(triple)\index{triple 三元组}$\left( \mu,  S, \Sigma \right)$称为测度空间(measure space)。测度空间满足如下性质
\begin{itemize}
  \item 测度$\mu$是单调方程(monotonic)
  \begin{equation*}
    \left( \text{可测集合} E_1,E_2 \in \Sigma, \quad E_1 \subseteq E_2 \right) \Rightarrow \left( \mu(E_1) \subseteq \mu(E_2) \right).
  \end{equation*}
  \item 无限个可测集合的并集的测度
  \begin{itemize}
    \item $\mu$是可数的次可加方程(countably subadditive)。在$\Sigma$中的任何可数可测集合$\left\{E_n \right\}_{i=n}^{\infty}$(可以不满足两两不相交),有
    \begin{equation*}
      \mu \left( \bigcup_{n=1}^{\infty} E_n \right) \le \sum_{n=1}^{n} \mu(E_n).
    \end{equation*}
    \item $\mu$是连续(continuous)方程。在$\Sigma$中的任何可数可测集合$\left\{E_n \right\}_{i=n}^{\infty}$(可以不满足两两不相交),满足$E_n \subset E_{n+1} \quad \forall n \in \mathbb{N}$,则集合的并集$\bigcup E_n$也是可测的,并且满足
    \begin{equation*}
      \mu \left( \bigcup_{n=1}^{\infty} \right) = \lim_{n\rightarrow \infty} \mu \left(E_n \right).
    \end{equation*}
  \end{itemize}
  \item 无限个可测集合的交集的测度
  \begin{itemize}
    \item $\mu$是连续方程。在$\Sigma$中的任何可数可测集合$\left\{E_n \right\}_{i=n}^{\infty}$(可以不满足两两不相交),满足$E_n \supset E_{n+1} \quad \forall n \in \mathbb{N}$,则集合的交集$\bigcap E_n$也是可测的,并且满足
    \begin{equation*}
      \mu \left( \bigcap_{n=1}^{\infty} \right) = \lim_{n\rightarrow \infty} \mu \left(E_n \right),
    \end{equation*}
    需要指出的是,对于交集的情况,若无下述假设,该性质一般不成立:可测集合$\left\{ E_n \right\}_{n=1}^{\infty}$中应当至少有一个$E_n$有有限测度。举例来说,如果我们设$E_n = [ n, \infty ) \subset \mathbb{R} \quad \forall n \in \mathbb{N}$,则这些可测集合全都具有无限测度,满足$E_{n} \supset E_{n+1}$,然而
    \begin{equation*}
      \left( \bigcap_{n=1}^{\infty} E_n = \varnothing \right) \Rightarrow \left( \mu \left( \bigcap_{n=1}^{\infty} E_n \right) = \mu \left( \varnothing \right) = 0 \right).
    \end{equation*}
  \end{itemize}
\end{itemize}
\end{definition}

\begin{definition}[计数测度]
  \label{definition:measure-counting-measure}
  在一个测度空间$(S,\Sigma,\mu)$中,任意子集$E \in \Sigma$的计数测度(counting measure) \index{measure!counting \dotfill 计数测度}定义为
  \begin{equation*}
    \mu(E) = \begin{cases}
    card(E), &\text{如果E是有限子集} \\
    +\infty, &\text{如果E是无限子集}
    \end{cases}
  \end{equation*}

  利用计数测度这种直观的方法,我们可以在一个测度空间中,通过将$S$的全部可测子集作$\Sigma$代数,从而将$S$映射进入这个测度空间中。然而,只有当空间$S$是可数的时,它在测度空间$(S,\Sigma,\mu)$中的计数测度才是$\sigma$有限($\sigma$ finite) \index{sigma!finite \dotfill sigma有限}的。
  \end{definition}

  \subsubsection{范数}
  \label{sec:lp-norm}
  范数(norm) \index{norm \dotfill 范数}是一个具有``长度''或``大小''概念的方程,是指赋予一个向量空间(vector space)\index{vector space \dotfill 向量空间} 内每个向量以一个非负的长度或大小;零向量(zero vector)\index{zero vector \dotfill 零向量} 的赋值为0。半范数(seminorm)\index{seminorm! \dotfill 半范数} 是指可以对某些非零向量赋值0。

  \begin{definition}[范数]
    \label{definition-norms}
    假定$V$是域$F$上的向量空间$V \subset F$。$V$上的p范数(p-norm)是指这样一个方程$p: V \mapsto \mathbb{R}$,使得$\forall a \in F, \, \forall \bm{u},\bm{\nu} \in V$,以下4个关系均得到满足
    \begin{itemize}
      \item 绝对齐次(absolutely homogeneous)或称绝对可标量化(absolutely scalable),数乘线性
      \begin{equation*}
        p \left( a \bm{\nu} \right) = | a | \, p(\bm{\nu}),
      \end{equation*}
       \item 次可加(subadditivity)\index{subadditivity \dotfill 次可加} 或称三角不等式(triangle inequality)
      \begin{equation*}
        p\left( \bm{u} + \bm{\nu} \right) \le p\left( \bm{u} \right) + p\left( \bm{\nu} \right),
      \end{equation*}
      \item 严格非负
      \begin{equation*}
        p(\bm{u}) \ge 0,
      \end{equation*}
      \item 确定(definite)。只有零向量的范数是0,反之范数是0的向量是零向量。
      \begin{equation*}
        \left( p(\bm{\nu}) = 0  \right) \Rightarrow \left( \bm{\nu} = 0 \right),
      \end{equation*}
      即$\bm{\nu}$是个空向量。
    \end{itemize}

    只满足前3条关系的方程$p: V \mapsto \mathbb{R}$,我们称之为$V$上的半范数(seminorm)。换句话说,所有范都是半范。
  \end{definition}

  \begin{definition}[商空间]
    \label{definition-quotient-space}
  每个向量空间$V$及其半范$p$都生成一个赋范向量空间(normed vector space)\index{normed vector space \dotfill 赋范向量空间} $\left( \frac{V}{W} \right)$,我们称之为商空间(quotient space)\index{quotient space \dotfill 商空间},其中$W$是$V$的子空间$W \subset V$,包括所有满足$p(\bm{\nu}) = 0$的向量$\bm{\nu} \in V$。

    对应地,商空间中的范数定义为$p (W + \bm{\nu}) = p (\bm{\nu})$。
  \end{definition}

  \begin{definition}[等价范]
    一个向量空间$V$中的两个范(或两个半范) $p$和$q$,当满足下述条件时,可称为等价范(equivalent norms)\index{equivalence of norms \dotfill 范数等价}:
    \begin{equation*}
      c \, q(\bm{\nu}) \le p(\bm{\nu}) \le C \, q(\bm{\nu}), \quad \exists \text{实常数} c,C, \quad \forall \bm{\nu} \in V.
    \end{equation*}
  \end{definition}

  \begin{definition}[平凡半范]
    平凡半范数(trivial seminorm) \index{seminorm!trivial \dotfill 平凡半范数}是指所有满足如下关系的半范
    \begin{equation*}
      p(\bm{\nu}) = 0, \quad \forall \bm{\nu} \in V.
    \end{equation*}
  \end{definition}

  一个向量空间$V$中的每一个线性泛函$f$ (linear form),都定义了1个半范$\bm{\nu} \mapsto | f(\bm{\nu}) |$。

  \subsubsection{有限维度的可数勒贝格空间(p >= 1)}
  $ \bm{x} = \{ x_1, \ldots, x_n \} \in \mathbb{R}^n$的p范数(p-norm)\index{p-norm \dotfill p范数}或者$L^p, \, \forall p \ge 1$\index{LP-norm \dotfill LP范数},可定义为
  \begin{equation*}
    \parallel \bm{x} \parallel_p = \left( \|x_1\|^p + \|x_2\|^p + \ldots + \|x_n\|^p \right)^{\frac{1}{p}},
  \end{equation*}

  包括一些特殊情况如
  \begin{itemize}
    \item $p=1$是个网格距离(grid distance)\index{grid distance \dotfill 网格距离},又称出租车距离(taxicab distance)\index{taxicab geometry \dotfill 出租车距离},曼哈顿距离(Manhattan distance)\index{Manhattan distance \dotfill 曼哈顿距离}等。
    \item $p=2$是个欧几里得范数(Euclidean norm)\index{Euclidean norm \dotfill 欧几里得范数}。
    \item $p=\infty$是个$L^{\infty}$范数($L^{\infty}$ norm)\index{L-infinity norm},又称最大范数(maximum norm)\index{maximum norm \dotfill 最大范数},均匀范数(uniform norm)\index{uniform norm \dotfill 均匀范数},切比雪夫距离(Chebyshev distance)\index{Chebyshev distance \dotfill 切比雪夫距离}等。
  \end{itemize}

  不同范数$p$之间的关系:
\begin{itemize}
  \item $p=1$的曼哈顿范数,从不小于任何欧几里得范数$p=2$。换句话说,任何向量$\bm{x}$的欧几里得范数都受限于它的1范数
  \begin{equation*}
    \| \bm{x} \|_{n} \le \| \bm{x} \|_1, \quad n \ge 1, n\in \mathbb{N}.
  \end{equation*}
  \item [扩展]。任何向量$\bm{x}$的p范数并不随着$p$的增加而增加
  \begin{equation*}
    \| \bm{x} \|_{p+a} \le \| \bm{x} \|_{p}, \quad \forall \text{向量 } \bm{x}, \quad \forall p \ge 1, a \ge 0, p,a \in \mathbb{N}.
  \end{equation*}
  \item [扩展].柯西——施瓦茨不等式(Cauchy-Schwarz inequality, Definition \ref{definition:cauchy-schwarz-inequality})可得
  \begin{equation*}
  \| \bm{x} \|_{1} \le \sqrt{n} \| \bm{x} \|_{2}, \quad n = dim(\bm{x}).
  \end{equation*}
  \item   [扩展].
    \begin{equation*}
      \| \bm{x} \|_{p} \le \| \bm{x} \|_{r} \le n^{\left(\frac{1}{r} - \frac{1}{p} \right)} \| \bm{x} \|_{p}, \quad \forall \bm{x} \in \mathbb{C}^n, \quad 0 < r <p, r,p \in \mathbb{N}.
    \end{equation*}
\end{itemize}

\subsubsection{有限维度的可数勒贝格空间(0 <= p <= 1)}
略。

\subsubsection{无限维度的勒贝格空间(p不可数)}
我们设$p\le 1$。$p$范数可以扩展到分析由无数个元素构成的向量,向量集合构成可数无限维的p范数列空间,用$l^p$表示。一些特殊情况如
\begin{itemize}
  \item $l^1$由绝对收敛(absolute convergence)序列构成的空间,如
  \begin{itemize}
    \item $\sum_{n=0}^{\infty} \left| a_n \right| = L$,其中$L$是某个实数,$\{a_n\}_{n=0}^{\infty}$是一个实数或复数序列。
    \item $\int_{0}^{\infty} \left| f(x) \right| \, dx = L$,其中$ \int_{0}^{\infty}  f(x)  \, dx$ 是关于某个方程$f(x)$的不定积分。
  \end{itemize}
  \item $l^2$是一个由平方可加数列构成的空间,即一个希尔伯特空间(Hilbert space),见第\ref{sec:lp-hilbert-space}节。
  \item $l^{\infty}$是一个由有界数列(bounded sequence)构成的空间。
\end{itemize}

$l^p$数列空间反映这样的向量空间结构:通过一个坐标一个坐标的向量加总(或标量相乘)而组成。如可数无限维实数(复数)数列$\bm{x} = \left\{ x_n \right\}_{n=1}^{\infty}, \bm{y} = \left\{ y_n \right\}_{n=1}^{\infty}$中的向量和和标量乘
\begin{equation*}
  \begin{split}
    & \left( x_1, \ldots, x_n, x_{n+1}, \ldots \right) +
    \left( y_1, \ldots, y_n, y_{n+1}, \ldots \right)
    = \left( x_1+y_1, \ldots, x_n+y_n, x_{n+1}+y_{n+1}, \ldots \right), \\
    & \lambda \left( x_1, \ldots, x_n, x_{n+1} \ldots , \right) =
    \left( \lambda x_1, \ldots, \lambda x_n, \lambda x_{n+1} , \ldots \right),
  \end{split}
\end{equation*}

对应的$p$范数
\begin{equation*}
  \| x\|_p = \left( |x_1|^p + \ldots + |x_n|^p + |x_{n+1}|^p + \ldots \right)^{\frac{1}{p}}.
\end{equation*}

则我们有定义
\begin{definition}[可数无限维数列空间]
  \label{definition:lp-lp-infinite-def}
  定义$l^p$为一个包括所有实(复)数无限数列的空间(countably infinite dimensional sequence space)\index{sequence space!countably infinite dimensional \dotfill 可数无限维数列空间},并且这些数列的$p$范数必须是有限的:这是因为存在着$p$范数是$\infty$的无限数列,他们不应当包括在$l^{p}$空间中,如$\left( 1,1,1,\ldots \right),\, 1 \le p < \infty$。
\end{definition}

随着$p$值的增加,$l^p$集合的大小增加的更快。例如数列$\bm{x} = \left(1, \frac{1}{2}, \ldots, \frac{1}{n}, \frac{1}{n+1},\ldots \right)$为例,它不在$l^1$空间中($p=1$时范数不收敛),而可能在某个$p>1$的$l^{p}$的空间中(范数收敛)
\begin{equation*}
\begin{split}
  &\| x \|_1 = \left( 1 + \frac{1}{2} + \ldots + \frac{1}{n}, \frac{1}{n+1} + \ldots \right) \rightarrow \infty, \\
  &\| x \|_p = \left( 1 + \left(\frac{1}{2}\right)^p + \ldots + \left( \frac{1}{n} \right)^p, \left( \frac{1}{n+1}\right)^p + \ldots \right)^{\frac{1}{p}}, \quad p >1.
\end{split}
\end{equation*}

$p = \infty$时,$l^{\infty}$范数
\begin{equation*}
  \| \bm{x} \|_{\infty} = \sup \left( |x_1|, \ldots, |x_n|, |x_{n+1}|, \ldots  \right),
\end{equation*}
对应$l^{\infty}$数列空间,包括全部实(复)数无限维数列,但要求他们都是有界的,即$\infty$范数收敛。

\subsubsection{无限维度的不可数勒贝格空间}
\label{sec:lp-spaces-banach}
上面我们讨论了有限维度和可数无限维度的勒贝格空间。然而当空间维度无限并且不可数(即不存在可数的基)时,我们无法像前文的方法来定义范数、进而描述空间。但如果该空间是勒贝格可积的(Lebesgue integrable, 见Definition \ref{definition:lebesgue-integrable-func-def}),仍然可以利用下述办法进行描述。

  给定可测空间$(\Omega,\Sigma,x)$ (见Definition \ref{definition:measure-measure-space}),以及$p \in \mathbb{R}, p \ge 1$。考虑所有从$\Omega$到域$\mathbb{F}=\left(\mathbb{R},\mathbb{C} \right)$的可测方程(measurable function)\todo{补一个measurable function的词条}集合$u(x)$,方程绝对值的$p$次幂在$\Omega$上有界,可积
  \begin{equation*}
    L^p(\Omega) = \left\{u(x);  \| u \|_{L^p(\Omega)} \equiv \left( \int_{\Omega} |u(x)|^p \, d x \right)^{\frac{1}{p}} < \infty \right\},
  \end{equation*}
  集合中的方程$u,\nu \in L^p(\Omega)$具有性质
\begin{itemize}
  \item 可加
  \begin{equation*}
    (u+\nu)(x) = u(x) + \nu(x),
  \end{equation*}
  \item 数乘线性
  \begin{equation*}
    u(\lambda \, x) = \lambda u(x), \forall \text{标量} \lambda \in \mathbb{F},
  \end{equation*}
  \item 范数满足不等式
  \begin{equation*}
    \| u + \nu \|_p^p \le 2^{p-1} \left( \| u \|_{p}^{p} + \| \nu \|_{p}^{p} \right),
  \end{equation*}

\end{itemize}
  如果存在某一个集合$K$构成了零测度(zero measure) $\mu(K) = 0$,那么只有当在这个零测度下$u,\nu \in L^p(\Omega)$才有所区别时,我们说$u$和$\nu$互相识别。

  事实上,2个可以扩展到更多个方程的情况,比如3个,即是说三角不等式(triangle inequality) \index{triangle inequality \dotfill 三角不等式}对于范数形式的可积方程$\| \cdot \|_{p}$依然成立,可由闵可夫斯基不等式(Minkowski inequality) \index{Minkowski inequality \dotfill 闵可夫斯基不等式}证得。它表明$L^p$空间是一个赋范向量空间(normed vector space)\index{vector space!normed \dotfill 赋范向量空间}。对于一个测度空间$\Omega \in L^{p}$,设$1 \le p \le \infty$,$u, \nu \in L^{p}(\Omega)$中的元素,此时我们有

\begin{definition}[三角不等式]
  \label{definition:triangle-inequality-def}
  $L^p(\Omega)$中的三角不等式(triangle inequality) \index{triangle inequality \dotfill 三角不等式}
  \begin{equation*}
  \| u + \nu \|_p \le \| u \|_p + \| v \|_p, \quad 1 < p < \infty
  \end{equation*}
式中等号存在的条件当且仅当$u$和$\nu$是严格线性相关的,即$ \exists \lambda \ge 0 \Rightarrow u = \lambda \nu$,或者$\nu = 0$。
\end{definition}

\begin{definition}[闵可夫斯基不等式]
  \label{sec:minkowski-ineq-def}
通过取可数测度(见Definition \ref{definition:measure-counting-measure}节),闵可夫斯基不等式(Minkowski inequality)可表示为数列和向量的形式
\begin{equation*}
  \left( \sum_{k=1}^{n} | x_k + y_k | ^p \right)^{\frac{1}{p}}
  \le \left( \sum_{k=1}^{n} | x_k | ^p \right)^{\frac{1}{p}}
  + \left( \sum_{k=1}^{n} | y_k | ^p \right)^{\frac{1}{p}}, \quad \forall \left\{ \bm{x} \right\}_{n}, \left\{ \bm{y} \right\}_{n} = \mathbb{R}\text{或}\mathbb{C}, \quad n=dim(S).
\end{equation*}
\begin{proof}
首先要证明如果$u$和$\nu$都有有限的$p$范数,那么$(u+\nu)$的$p$范数也是有限的,且满足不等式关系
\begin{equation*}
  | u + \nu |_p \le 2^{p-1} \left( |u| ^{p} + |\nu| ^{p}\right)
\end{equation*}
证明方式为:给定$p>1$,则$h(x) = x^p$是一个在$\mathbb{R}^{+}$上的凸方程(convex)。由方程的凸性质可得
\begin{equation*}
  \big| \frac{1}{2} u + \frac{1}{2} \nu \big|^p
  \le \big|\frac{1}{2} |u| + \frac{1}{2} |\nu| \big|^p
  \le \frac{1}{2} |u|^p + \frac{1}{2} |\nu|^p,
\end{equation*}
$\hookrightarrow$
\begin{equation*}
  \big| u + \nu \big|^p \le \frac{1}{2} |2u|^p + \frac{1}{2} |2\nu|^p
  = 2^{p-1} \left( |u|^p + |\nu|^p \right),
\end{equation*}
可见$\|u+\nu\|_p$是有限范数。

在证明了$\|u+\nu\|_p$范数有限后,我们有:如果$|u + \nu|_p=0$,闵可夫斯基不等式直接变为等号且成立。现在假定$|u + \nu|_p \neq 0$,使用三角不等式和霍德尔不等式(Hölder's inequality, Definition \eqref{definition:hoelder-inequality-def}) \index{Hölder's inequality \dotfill 霍德尔不等式},我们有
\begin{equation*}
\begin{split}
    \|u + \nu \|_p^p &= \left[ \left( \int |u + \nu|^p \, d \mu \right)^{\frac{1}{p}} \right]^{p} \\
    &= \int |u + \nu| \, |u + \nu|^{p-1} \, d \mu \\
    &\le \int \left( |u| + |\nu| \right) \, |u + \nu|^{p-1} \, d \mu\\
    & = \int |u| \, |u + \nu|^{p-1} \, d \mu + \int |\nu| \, |u + \nu|^{p-1} \, d \mu \\
    &=\left[ \left( \int |u|^p \, d \mu \right)^{\frac{1}{p}}  + \left( \int |\nu|^p \, d \mu \right)^{\frac{1}{p}}\right] \left(
    \int \left( \big| u + \nu \big|^{\left(p-1\right) \cdot \left( \frac{p}{p-1} \right)}  \right) \, d \mu
    \right)^{1-\frac{1}{p}}\\
    &= \left( \|u \|_p + \| \nu \|_p \right) \frac{\|u+\nu\|_{p}^{p}}{\| u + \nu \|_p},
\end{split}
\end{equation*}
$\hookrightarrow$
\begin{equation*}
  \| u + \nu \|_p \le \|u \|_p + \| \nu \|_p.
\end{equation*}
\end{proof}
\end{definition}

  有时我们需要积分形式的闵可夫斯基不等式(Minkowski integral inequality)\index{Minkowski inequality!integral \dotfill 闵可夫斯基积分不等式}:
\begin{definition}[闵可夫斯基积分不等式]
  \label{sec:minkowski-ineq-int-def}
  假定存在两个$\sigma$可测度空间$(S_1, \mu_1)$和$(S_2, \mu_2)$,并且$F:S_1 \times S_2 \mapsto \mathbb{R}$是可测方程,那么我们有
  \begin{equation*}
    \left[ \int_{S_2} \Big| \int_{S_1} F(x,y) \mu_1(d x) \Big|^{p} \mu_2 (d y)\right]^{\frac{1}{p}} \le
    \left[ \int_{S_1} \Big| \int_{S_2} F(x,y) \mu_2(d y) \Big|^{p} \mu_1 (d x)\right]^{\frac{1}{p}}, \quad p < \infty,
  \end{equation*}
\end{definition}

满足该条件的$\| \cdot \|_p$构成半范数(seminorm, Definition \ref{definition-norms}),对应半赋范向量空间(seminormed vector space)\index{seminormed vector space \dotfill 半赋范向量空间} $\mathcal{L}^p(\Omega,\mu)$。之所以称之为半范数,是因为该空间中存在非零向量$u$满足$\| u \|_p = 0$。

我们可以用标准的拓扑方法,从半范向量空间$\mathcal{L}^p(\Omega,\mu)$中得到一个赋范向量空间。在$\mathcal{L}^p(\Omega,\mu)$中,考虑所有使$\| u \|_p = 0$的向量集合
\begin{equation*}
  \mathcal{N} = \left\{ u;  \| u \|_p = 0 \right\}.
\end{equation*}
$\mathcal{N}$可以看作是一个映射$f \mapsto \| u \|_p$的零向量空间。则对于可测度方程$u$而言:
\begin{equation*}
  \| u \|_p = 0 \Longleftrightarrow \mu(u \neq 0) \Longleftrightarrow u_{\mu\text{-几乎处处}} = 0,
\end{equation*}
$\mu\text{-几乎处处}$表示在测度$\mu$的意义上几乎处处有界(almost everywhere)。从这个意义上来看,$\mathcal{N}$是一个kernel $\| \cdot \|_p$,并且不依赖于$p$
\begin{equation*}
  \mathcal{N} \equiv ker\left( \| \cdot \|_p \right) = \left\{ u: u _{\mu\text{-几乎处处}}=0 \right\}.
\end{equation*}

则我们可以定义一个关于$\mathcal{L}^p(\Omega,\mu)$和kernel $\mathcal{N}$的商空间
\begin{equation*}
  L^p(\Omega,\mu) \equiv \frac{\mathcal{L}^p (\Omega,\mu)}{\mathcal{N}},
\end{equation*}
商空间$L^p(\Omega,\mu)$中的某个$u$可以看做是与$\mathcal{L}^p(\Omega,\mu)$中的$f$相差1个$\mathcal{N}$中对应元素的等价类。

由此可见,$L^p(\Omega,\mu)$就是$\Omega$上关于测度$\mu$的$L^p$空间。对应的$\| \cdot \|_p$成为$L^p(\Omega,\mu)$的$p$范数。需要指出的是,严格来说$L^p$空间中的元素并非某个具体方程,而是由一个方程族构成的等价类。当我们取出$L^p$中的元素作计算的时候,参与计算的其实是从这个方程组中抽取的一个代表方程。

$p = \infty$时对应的空间$L^{\infty}(S,\mu)$也可以用类似方法求得:
\begin{equation*}
  \begin{split}
    &\| f \|_{\infty} \equiv \inf \left\{ C \ge 0: \left| f(x) \right| \le C \text{对于几乎所有}x \right\},\\
    &\exists q < \infty \Rightarrow f \in L^{\infty}(S,\mu) \bigcap L^{q}(S,\mu) \Rightarrow \| f \|_{\infty} = \lim_{p \rightarrow \infty} \| f \|_{p}.
  \end{split}
\end{equation*}

% 随着$p=+\infty$,方程空间$\mathcal{L}^{\infty}(\Omega,\mu)$为一组可测度方程$u(x)$的集合,这些方程在测度$\mu$的意义上几乎处处有界(almost everywhere)
% \begin{equation*}
%   \| u \|_{L^{\infty}(\Omega)} \coloneqq \text{ess } \sup_{x \in \Omega} \left\{ \left| u(x) \right| \right\} \coloneqq \inf_{K \subset \Omega, \mu(K) = 0} \sup_{x \in \Omega \backslash K} \left|  u(x) \right|.
% \end{equation*}

勒贝格空间$L^{p}(S,\mu)$的完备性(completeness)通常称为里兹——费舍定理(Riesz-Fischer theorem)。证明略\footnote{完备性(completeness)的含义,见第\pageref{footnote:completeness-def}页脚注。}。

对于$1 \le p \le \infty$的情况,勒贝格空间$L^{p}(S,\mu)$是一个完备赋范向量空间,常称为巴拿赫空间(Banach space)\index{Banach space \dotfill 巴拿赫空间}。所有$L^{p}$空间都是巴拿赫空间。

\subsubsection{加权勒贝格空间}
\label{sec:lp-weightd-lp}
有时候会遇到加权勒贝格空间的情况。

\begin{definition}[加权勒贝格空间]
  \label{definition:lp-weightd-lp}
  考虑一个测度空间$L^p \left( S, \sigma, \mu \right)$,其中有一个可测方程$w : S \rightarrow [ 0, \infty)$。有时我们也将$L^p \left(S, w \, d \mu \right)$称为$w-$加权勒贝格空间(w-weighted Lebesgue space)\index{Lebesgue space!weighted \dotfill 加权勒贝格空间},其中测度$d \nu \equiv w \, d \mu$。由此我们有测度的定义
  \begin{equation*}
    \nu (A) \equiv \int_A w(x) \, d \mu(x), \quad \forall A \in \Sigma.
  \end{equation*}

在此基础上,加权勒贝格空间$L^p \left(S, w \, d \mu \right)$的范数
\begin{equation*}
  w = \frac{d \nu}{d \mu} \Rightarrow \Big\| \mu \Big\|_{L^p(S, w \, d\mu)} \equiv \left(
  \int_s w(x) \left| \mu(x) \right|^p \, d \mu(x)
  \right)^{\frac{1}{p}}.
\end{equation*}

从这个角度上说,$L^p (s, d \nu) \equiv L^p ( s, w d\nu)$.
\end{definition}

\subsection{希尔伯特(H)空间}
\label{sec:lp-hilbert-space}
有且只有$p=2$时的特殊形式空间$L^2(\Omega)$,是希尔伯特空间(Hilbert space)\index{Hilbert space \dotfill 希尔伯特空间}。

作为(完备赋范的)内积向量空间(inner product space, Definition \ref{definition:inner-product-space}),希尔伯特空间是有限维欧几里得空间的一个扩展:从$\mathbb{R}$扩展到$\mathbb{R}$和$\mathbb{C}$,从有限维度到无限维度,但保留了完备性(completeness)特征(一般来说,非欧几里得空间往往破坏了完备性) \label{footnote:completeness-def}\footnote{完备性(completeness)是指,希尔伯特空间中所有柯西数列(Cauchy sequences) \index{Cauchy sequences \dotfill 柯西数列}都会收敛到此空间中的一点(一个元素),即这些数列与某个元素的范数差的极限为$0$。}。希尔伯特空间与欧几里得空间相仿,有长度和角度的概念,因而可以引申出正交性和垂直性,从而为基于正交多项式的傅里叶级数等提供表达方式。

任何一个希尔伯特空间都是巴拿赫空间,反之则未必。

\subsubsection{例:欧几里得空间}
\label{sec:hilbert-space-eucilidean-examples}
假设所有的希尔伯特空间都是复数(实际应用中大多数是实数)。二维欧几里得空间$\mathbb{R}^2$中,向量$\bm{x},\bm{y}$构成一个希尔伯特空间
\begin{equation*}
  \bm{x} \cdot \bm{y} = \sum_{k=1}^{n} \overline{x_k} y_k,
\end{equation*}
对应范数
\begin{equation*}
\| \cdot \| = \sqrt{\langle \bm{x}, \rangle{\bm{y}}}.
\end{equation*}

三维欧几里得空间$R^3$中,以笛卡尔坐标系表示的$\bm{x},\bm{y}$向量的点乘为
\begin{equation*}
  \bm{x} \cdot \bm{y} = \left( x_1, x_2, x_3 \right) \cdot \left( y_1,y_2,y_3 \right) = x_1 y_1 + x_2 y_2 + x_3 y_3,
\end{equation*}
点乘具有如下性质:
\begin{itemize}
  \item 对称性
  \begin{equation*}
    \bm{x} \cdot \bm{y} = \bm{y} \cdot \bm{x},
  \end{equation*}
  \item 首项线性
  \begin{equation*}
    \left( a \bm{x} + b \bm{y} \right) \cdot \bm{z} = a \bm{x} \cdot \bm{z} + b \bm{y} \cdot \bm{z}, \quad a,b \text{是任意标量}, \quad  \bm{x},\bm{y},\bm{z} \text{是任意向量},
  \end{equation*}
  \item 正定
  \begin{equation*}
    \bm{x} \cdot \bm{x} \begin{cases}
     \ge 0 & \forall \bm{x} \ge 0, \\
     =0 & \text{iff. } \bm{x} = 0.
    \end{cases}
  \end{equation*}
\end{itemize}

\begin{definition}[内积空间]
\label{definition:inner-product-space}
满足上述三个条件的(实数)向量乘称为(实数)内积(inner product)\index{inner product \dotfill 内积},用$\langle .,. \rangle$表示。给定一个实数或复数域$\mathbb{F}$中的向量空间$V$,则我们将内积形式的向量空间$\langle .,.\rangle : V \times V \mapsto \mathbb{F}$表示为内积空间(inner product space) \index{inner product space \dotfill 内积空间}。内积空间满足三个性质:$\forall \text{向量} \bm{x},\bm{y},\bm{z} \in V$,以及$\forall \text{标量} a \in \mathbb{F}$
\begin{itemize}
  \item 共轭对称(conjugate symmetry)
  \begin{equation*}
    \langle \bm{x} , \bm{y}\rangle =
    \begin{cases}
      \langle \bm{y}, \bm{x} \rangle & \mathbb{F}=\mathbb{R}, \\
      \overline{\langle \bm{y}, \bm{x} \rangle} & \mathbb{F}=\mathbb{C}.
    \end{cases}
  \end{equation*}
  其中标有上横线$\overline{(\cdot)}$的部分表示复数共轭(complex conjugate)。
  \item 首项线性
  \begin{equation*}
  \begin{split}
    &\langle a \bm{x}, \bm{y} \rangle = a \langle \bm{x}, \bm{y} \rangle, \\
    & \langle \bm{x} + \bm{y}, \bm{z} \rangle = \langle \bm{x}, \bm{z} \rangle + \langle \bm{y}, \bm{z} \rangle.
  \end{split}
  \end{equation*}
  \item 正定
  \begin{equation*}
    \begin{split}
      &\langle \bm{x}, \bm{x} \rangle > 0, \\
      &\langle \bm{x}, \bm{x} \rangle = 0 \Leftrightarrow \bm{x} = 0.
    \end{split}
  \end{equation*}
\end{itemize}
\end{definition}

任何有限维内积空间都也是希尔伯特空间。在欧几里得空间内,两个向量的内积大小与两方面因素有关,一为向量的长度(即范数)$\| \bm{x} \|$,一为$\bm{x}$和$\bm{y}$之间的夹角$\theta$,满足
\begin{equation*}
  \bm{x} \cdot \bm{y} = \|\bm{x}\| \, \|\bm{y}\| \cos \theta.
\end{equation*}

欧几里得空间$\mathcal{R}^3$中,对$n \in \mathcal{N}$个向量$\bm{x}_n$求和构成一个数学级数$\sum_{n=0}^{\infty} \bm{x}_n$,当每个向量的范之和收敛到一个小于正无穷的向量$L$时,我们称这个级数仍然是绝对收敛(absolutely convergence) \index{convergence!absolute \dotfill 绝对收敛}的
\begin{equation*}
  \sum_{k=0}^{N} \| \bm{x}_k \| < \infty.
\end{equation*}

一个绝对收敛的向量数列$\sum_{k=0}^{N} \bm{x}$,收敛至某个极限向量$\bm{L} \in R^3$
\begin{equation*}
  \Big\| \bm{L} -  \sum_{k=0}^{N} \bm{x}\Big\| \rightarrow 0, \text{随着} N \rightarrow \infty,
\end{equation*}
这称为欧几里得空间的完备性(completeness of Euclidean space)\index{completeness!Euclidean space \dotfill 完备性(欧几里得空间)}。

类似地,在欧几里得空间中,复数平面(complex plane)\index{complex plane \dotfill 复数平面} $\mathbb{C}$由量(magnitude)的形式予以反映,即复绝对值(complex modulus)\index{complex modulus \dotfill 复数绝对值} $\left| z \right|$, 定义为$z$与其共轭复数(complex conjugate)\index{conjugate \dotfill 共轭复数} $\bar{z}$乘积的平方根
\begin{equation*}
  \left| z \right| = \sqrt{z \, \bar{z}}.
\end{equation*}

如果$z=x+y, \, x=\Re{(z)}, \, y=Im{(z)}$,复绝对值为常见的二元欧几里得空间的长度
\begin{equation*}
  \| z \| = \sqrt{\Re{(z)^2} + \Im{(z)^2}} = \sqrt{x^2 + y^2}.
\end{equation*}

两个复数$z,w$的内积
\begin{equation*}
  \langle z, w \rangle = z \bar{w},
\end{equation*}
或者对于复数空间$z, w \in \mathbb{C}^2$,即$z=(z_1,z_2),w=(w_1,w_2)$,对应内积
\begin{equation*}
  \langle z, w \rangle = z_1 \bar{w}_1 + z_2 \bar{w}_2,
\end{equation*}
其中$\Re(\langle z, w \rangle) \in \mathbb{R}^4$。这个内积埃米特对称(Hermitian symmetric)\index{Hermitian symmetric \dotfill 埃米特对称},即是说
\begin{equation*}
  \langle w,z \rangle = \overline{\langle z,w \rangle}.
\end{equation*}

希尔伯特空间$H$是一个实数(或复数)内积向量空间,其中的向量可以内积形式表示为$\langle \bm{x},\bm{y} \rangle$,满足如下特性:
\begin{itemize}
  \item 对称性
  \begin{equation*}
    \langle \bm{y}, \bm{x} \rangle = \begin{cases}
    \langle \bm{x},\bm{y} \rangle  & \text{实数向量}, \\
    \overline{\langle \bm{x},\bm{y} \rangle}  & \text{复数向量}.
    \end{cases}
  \end{equation*}
  \item 首项线性
  \begin{equation*}
    \left( a \bm{x} + b \bm{y} \right) \cdot \bm{z} = a \bm{x} \cdot \bm{z} + b \bm{y} \cdot \bm{z}, \quad a,b \text{是任意标量}, \quad  \bm{x},\bm{y},\bm{z} \text{是任意向量},
  \end{equation*}
  \item 正定\footnote{省略部分复数形式的表述,以使方程结构更紧凑。}
  \begin{equation*}
    \bm{x} \cdot \bm{x} \begin{cases}
     \ge 0 & \forall \bm{x} \ge 0, \\
     =0 & \text{iff. } \bm{x} = 0.
    \end{cases}
  \end{equation*}
\end{itemize}
由对称性和首项线性可得第二项系数是反线性的(antilinear):
\begin{equation*}
  \langle \bm{x}, a \bm{y} + b \bm{z} \rangle = \bar{a} \langle \bm{x}, \bm{y} \rangle + \bar{b} \langle \bm{x}, \bm{z} \rangle.
\end{equation*}

希尔伯特空间的范是一个实值方程
\begin{equation*}
  \| \cdot \| = \sqrt{\langle \bm{x}, \bm{y} \rangle}
\end{equation*}

\begin{definition}[对偶空间]
已知内积定义
\begin{equation*}
  \langle u,\nu \rangle_{\Omega} \coloneqq \int_{\Omega} u(x) \, \nu(x) \, dx,
\end{equation*}
根据闵可夫斯基不等式(Definition \ref{sec:minkowski-ineq-def}) ,$L^p(\Omega)$中的三角不等式可以扩展到更一般的形式:
\begin{equation*}
  \| \nu \|_{L^{q}(\Omega)} = \sup_{0 \neq u \in L^p(\Omega) } \frac{
  \left| \langle u,\nu \rangle_{\Omega} \right|
  }{
  \| u \|_{L^p(\Omega)}
  }, 1 \le p < \infty,
\end{equation*}
其中$p,q$是伴随参数,满足
\begin{equation*}
\quad \frac{1}{p} + \frac{1}{q} = 1,
\end{equation*}
不难看出,RHS满足三角不等式关系。则$L^q(\Omega)$和$L^p(\Omega)$构成一组对偶空间(dual space)\index{dual space \dotfill 对偶空间}。
\end{definition}

当$p=q=2$时,$L^2(\Omega)$就成为包括全部平方可积方程的空间。此时霍德尔不等式(Definition \ref{definition:hoelder-inequality-def})就变成了柯西——施瓦茨不等式。
\begin{definition}[柯西——施瓦茨不等式]
  \label{definition:cauchy-schwarz-inequality}
  内积空间(见Definition \ref{definition:inner-product-space}) $\langle .,. \rangle : V \times V \mapsto \mathbb{F}$中,对于任意两个向量$\forall \bm{u}, \bm{\nu} \in V$,内积绝对值的平方,满足三角不等式
  \begin{equation*}
    \big| \langle \bm{u},\bm{\nu} \rangle \big|^2 \le \langle \bm{u},\bm{u} \rangle \cdot \langle \bm{\nu},\bm{\nu} \rangle,
  \end{equation*}
  两侧同时开平方根,将RHS改写为向量范的形式,我们有柯西——施瓦茨不等式(Cauchy-Schwarz inequality)\index{Cauchy-Schwarz inequality \dotfill 柯西——施瓦茨不等式}
  \begin{equation*}
\begin{split}
      &\big| \langle \bm{u},\bm{\nu} \rangle \big| \le \| \bm{u} \| \, \| \bm{\nu} \|,\\
      \hookrightarrow & \int_{\Omega} \left| u(x) \, \nu(x) \right| \, d x \le \| u \|_{L^2(\Omega)} \, \| \nu \|_{L^2(\Omega)},\\
      \hookrightarrow & \big| \langle u, u \rangle \big|_{L^2(\Omega)} = \| u \|^2_{L^2(\Omega)}, \forall u=\nu, u \in L^2(\Omega).
\end{split}
  \end{equation*}

  其中等号成立的条件,只有一下两种之一:$\bm{u},\bm{\nu}$线性无关(linearly independent),即平行;$\bm{\nu}$是零向量或是标量。
\end{definition}
\begin{proof}
  柯西——施瓦茨不等式的证明方法有很多种\citep{Wu:2011uc},我们取其中一种。

  $\bm{\nu} = 0 \Rightarrow \big| \langle \bm{u},\bm{\nu} \rangle \big| = \| \bm{u} \| \, \| \bm{\nu} \| \forall \bm{u} \in V$。

  $\bm{u} \neq 0, \bm{\nu} \neq 0$。设一个向量$\bm{z}$满足
  \begin{equation*}
    \bm{z} := \bm{u} - \bm{u_{\nu}} = \bm{u} - \underbrace{\frac{\langle \bm{u}, \bm{\nu} \rangle}{\langle \bm{\nu}, \bm{\nu} \rangle}}_{\text{标量}} \bm{\nu}.
  \end{equation*}

  对$\bm{z}$和$\bm{\nu}$作内积,由内积空间的性质之一——首项线性得
  \begin{equation*}
    \begin{split}
      \langle \bm{z}, \bm{\nu} \rangle &= \left\langle \bm{u} - \frac{\langle \bm{u}, \bm{\nu} \rangle}{\langle \bm{\nu}, \bm{\nu} \rangle} \bm{\nu}, \bm{\nu} \right\rangle \\
      &= \langle \bm{u}, \bm{\nu} \rangle - \left\langle \frac{\langle \bm{u}, \bm{\nu} \rangle}{\langle \bm{\nu}, \bm{\nu} \rangle} \bm{\nu}, \bm{\nu} \right\rangle \\
      &= \langle \bm{u}, \bm{\nu} \rangle - \frac{\langle \bm{u}, \bm{\nu} \rangle}{\langle \bm{\nu}, \bm{\nu} \rangle} \left\langle  \bm{\nu}, \bm{\nu} \right\rangle \\
      &=0.
    \end{split}
  \end{equation*}

  $\langle \bm{z}, \bm{\nu} \rangle = 0 \Rightarrow \bm{z}=0$,作为$\bm{u}$向$\bm{\nu}$所在平面(plane)所做的正交映射,反映了$\bm{u}$和$\bm{\nu}$线性无关。因此我们对$\bm{z}$的定义式继续使用勾股定理
  \begin{equation}
    \begin{split}
      \bm{u} & = \frac{\langle \bm{u}, \bm{\nu} \rangle}{\langle \bm{\nu}, \bm{\nu} \rangle} \bm{\nu} + \bm{z} \\
      \hookrightarrow \left\| \bm{u} \right\|^2 &= \Big| \frac{\langle \bm{u}, \bm{\nu} \rangle}{\langle \bm{\nu}, \bm{\nu} \rangle} \Big|^2  \| \bm{\nu} \|^2 + \| \bm{z}\|^2 \\
      &= \frac{
      \big| \langle \bm{u}, \bm{\nu} \rangle \big|^2
      }{
      \left( \| \bm{\nu} \|^2 \right)^2 }
      \| \bm{\nu} \|^2  + \| \bm{z} \|^2 \\
      &= \frac{
      \big| \langle \bm{u}, \bm{\nu} \rangle \big|^2
      }{
       \| \bm{\nu} \|^2 } + \| \bm{z} \|^2 \\
       & \ge \frac{
       \big| \langle \bm{u}, \bm{\nu} \rangle \big|^2
       }{
        \| \bm{\nu} \|^2 }\\
        \hookrightarrow \left| \langle \bm{u}, \bm{\nu}  \rangle \right| & \le \| \bm{u} \| \, \| \bm{\nu} \|
    \end{split}
  \end{equation}
\end{proof}

\subsection{索伯列夫\texorpdfstring{$W^{k,p}(\Omega)$}{(W)}空间}

\subsubsection{微分的类以及平滑方程}
$H^s(\Omega)$中包括$L^p$空间中的具有弱可导性的平滑方程
\footnote{``平滑''方程(smoothness)的定义包括很多种,由弱到强有
\begin{itemize}
  \item 连续性,
  \item 可导性(可导方程必连续),
  \item 它的最高一阶导数也是连续的,
\end{itemize}
等等。索伯列夫空间中方程设定为``弱''可导形式,是为了使得空间完备,是一个巴拿赫空间。
},常用于求解偏微分方程PDEs。

\begin{definition}[微分的类以及平滑方程]
  \label{eq:soblev-differentiability-classification}
  我们可以根据方程的微分性质,对方程作分类(differentiability classification)\index{differentiability classification \dotfill 方程微分的类}。一个实数集合$\mathbb{R}$上的开区间中,实值方程$f \in \mathbb{R}$。

  如果微分方程$f',f'',\ldots,f^{(k)}$都存在并且$f',f'',\ldots,f^{(k-1)}$连续,我们称$f$属于$\mathbb{C}^{k}$类方程。当$k\rightarrow \infty$时$f$的所有$k$次微分都存在且连续,我们称之为$\mathbb{C}^{\infty}$类方程,无限可微方程(infinitely differentiable),或者称之为平滑方程(smooth function)。

  如果$f$是平滑的,并且$f$沿着域中任意一点作泰勒级数展开都收敛至该点,则我们称$f$是$\mathbb{C}^{\omega}$类方程,或称之为解析方程(analytic function)。可见$\mathbb{C}^{\omega} \subset \mathbb{C}^{\infty}$。

  举例来说,
  \begin{itemize}
    \item $\mathbb{C}^0$中包括所有连续方程,
    \item $\mathbb{C}^1$中包括所有一次可微方程,并且这些方程的一次导数是连续的,称连续可导(continuously differentiable)。进而
    \begin{itemize}
      \item  $\forall f \in \mathbb{C}^1 \Rightarrow f'\text{存在且}f' \in \mathbb{C}^0$
      \item $\forall f \in \mathbb{C}^k \Rightarrow f', f'', \ldots f^{k} \text{存在且}f' \in \mathbb{C}^{k-1}$
    \end{itemize}
    \end{itemize}
\end{definition}

\subsubsection{分部积分}
\begin{definition}[分部积分公式]
  \label{definition:sobolev-integration-by-parts}
  分部积分公式(integration by parts formula)\index{integration by parts \dotfill 分部积分公式}是指,如果${u} = u(x), u'(x) = d u / d x$,以及$\nu = \nu(x), \nu'(x) = d \nu / d x$,那么
  \begin{equation*}
    \begin{split}
      \int_{a}^{b} u(x) \nu'(x) \, dx &= \left[ u(x) \nu(x) \right]_a^b - \int_a^b u'(x) \nu(x) \, dx \\
      &= \left[ u(b) \nu(b) - u(a) \nu(a) \right] - \int_a^b u'(x) \nu(x) \, dx,
  \end{split}
  \end{equation*}
  或者用更紧凑的表现形式
  \begin{equation*}
    \int u d(\nu) = u \nu - \int \nu d(u).
  \end{equation*}
\end{definition}

\subsubsection{广义积分}
\label{sec:generalized-integration}
我们定义$L^{1,\text{loc}}(\Omega)$为局部可积(locally integrable)的方程空间,即方程$u \in L^{1,\text{loc}}(\Omega)$在任意一个封闭有界子集$K \subset \Omega$中可导。

例,设$\Omega = (0,1)$,$u(x)=\frac{1}{x}$。由于
  \begin{equation*}
    \int_0^1 u(x) \, d x \approx \lim_{\epsilon \rightarrow 0} \int_{\epsilon}^1 \frac{1}{x} \, d x \approx \lim_{\epsilon \rightarrow 0} \log \frac{1}{\epsilon} = \infty,
  \end{equation*}
  可见$u \notin L^1(\Omega)$。可由Mathematica算得
  \begin{lstlisting}
  Limit[Integrate[1/x, {x, ee, 1}], ee -> 0]
  Limit[Log[1/x], x -> 0]
  \end{lstlisting}
对于任一闭区间$K \coloneqq [a,b] \subset (0,1) = \Omega, \quad 0 < a < b < 1$我们有
\begin{equation*}
  \int_{K} u(x)\, dx = \int_a^b \frac{1}{x} \, dx = \ln \frac{b}{a} < \infty,
\end{equation*}
可见$u \in L^{1, \text{loc}}(\Omega)$。

此外对于$\phi,\psi \in C_0^{\infty}(\Omega)$,根据分部积分(Definition \ref{definition:sobolev-integration-by-parts})\index{integration by parts \dotfill 分部积分} 我们有
\begin{equation*}
  \int_{\Omega} \phi(x) \frac{\partial}{\partial x_i} \psi(x) \, dx = - \int_{\Omega} \frac{\partial}{\partial x_i} \phi(x) \, \psi(x) \, d x,
\end{equation*}
上式对于哪怕是非平滑方程$\phi,\psi$也适用。由此可得广义偏导数的定义

\begin{definition}[广义偏导数]
  \label{definition:generalized-partial-derivative-def}
  设$u \in L^{1,\text{loc}}(\Omega)$。如果$\exists \, \nu \in L^{1,\text{loc}}(\Omega)$,使得满足
  \begin{equation}
    \label{eq:generalized-partial-derivative-higher}
   \int_{\Omega} \nu(x) \varphi(x) \, dx =   -\int_{\Omega} u(x) \frac{\partial}{\partial x_i} \varphi(x) \, dx,
  \end{equation}
  其中$\varphi(x) \in C^{\infty}_0(\Omega)$,那么我们说$\nu(x)$是$u(x)$在$\Omega$中关于$x_i$的广义偏导数(generalized partial derivative)\index{generalized partial derivative \dotfill 广义偏导数} ,写作$\nu(x) \coloneqq \partial u(x)/ \partial x_i$。

  类似地,$u$的第$\alpha$阶广义偏导数$\nu(x) = D^{\alpha}u(x)$记作
  \begin{equation*}
    \int_{\Omega} u(x) D^{\alpha} \varphi(x) \, dx = (-1)^{\left| \alpha \right|} \int_{\Omega} \nu(x) \varphi(x) \, dx,
  \end{equation*}
  其中多重指数(multi-index) $\alpha = (\alpha_1,\ldots,\alpha_n), x=(x_1,\ldots,x_n),\left|\alpha\right|=\alpha_1 + \ldots + \alpha_n$,积分操作符$D^{\alpha}$是下述形式的缩写
  \begin{equation*}
    D^{\alpha}= \frac{\partial^{\left| \alpha \right|}}{\partial x_1^{\alpha_1} \ldots x_n^{\alpha_n}},
  \end{equation*}
\end{definition}

例,设$u(x) = \left| x \right|$,$x \in \Omega = (-1,1)$。对于任一$\varphi \in C_0^{\infty}(\Omega)$,我们有
\begin{equation*}
  \begin{split}
    &\int_{-1}^{1} u(x) \frac{\partial}{\partial x}\varphi(x) \, dx\\ &= - \int_{-1}^{0} x \frac{\partial}{\partial x}\varphi(x) \, dx  + \int_{0}^{1} x \frac{\partial}{\partial x}\varphi(x) \, dx \\
    &= \left\{
    - \left[ x \, \varphi(x) \right]_{-1}^{0} + \int_{-1}^{0} \varphi(x) \, dx
    \right\} +
    \left\{
      \left[ x \, \varphi(x) \right]_{0}^{1} - \int_{0}^{1} \varphi(x) \, dx
    \right\}\\
    &=\int_{-1}^{0} \varphi(x) \, dx - \int_{0}^{1} \varphi(x) \, dx \\
    &= - \int_{-1}^{1} \sgn(x) \varphi(x) \, dx,
  \end{split}
\end{equation*}
其中
\begin{equation*}
  \sgn(x) \coloneqq \begin{cases}
    1 & x >0,\\
    -1 & x <0.
  \end{cases}
\end{equation*}

则方程$u(x) = \left| x \right|$的1阶广义偏导数$\nu(x)$为
\begin{equation*}
  \nu(x) \coloneqq \frac{\partial}{\partial x}u(x) = \sgn(x) \in L^{1,\text{loc}}(\Omega).
\end{equation*}

递归方法计算2阶广义偏导数:
\begin{equation}
  \label{eq:sobolev-distribution-sgnx}
  \int_{-1}^{1}\sgn(x) \frac{\partial}{\partial x} \varphi(x) = -\int_{-1}^{0} \frac{\partial}{\partial x} \varphi(x) \, dx
  + \int_{0}^1 \frac{\partial}{\partial x} \varphi(x) \, dx = -2 \varphi(0),
\end{equation}
然而$\nexists \nu \in L^{1,\text{loc}}(\Omega)$满足
\begin{equation*}
  \int_{-1}^{1} \nu(x) \, \varphi(x) \, dx = 2 \varphi(0), \quad \forall \, \phi \in C_0^{\infty}(\Omega).
\end{equation*}
我们将在随后讨论$\sgn(x)$作为分布概念时的广义积分,见\pageref{sec:sobolev-space-distributions}页第\ref{sec:sobolev-space-distributions}节。


\subsubsection{整数阶次的单维索伯列夫空间}

索伯列夫相关教材,可参考如\cite{Adams:2003wi, Tartar:2007vm, Mazya:2009vz, Mazya:2009wu}等。


在$\mathbb{R}^n$的一个开放子集$\Omega$中,对于一个给定的非负整数$k$,我们有索伯列夫空间或$W^{k,p}(\Omega)$。它是一种希尔伯特空间的特例,其中
\begin{itemize}
  \item (内积形式表示的)方程向量空间都是可微的,
  \item 范数是可微方程范数的组合,包括方程本身的$L^p$范数,以及方程直到某一给定次导数的范数的组合。
\end{itemize}

单维索伯列夫空间$W^{k,p}$中的方程即是$\mathbb{R}$中的方程,可定义为单维勒贝格空间$L^{p}(\mathbb{R})$中方程$f$的子集,$f$满足如下特征:对于给定的$p \in \mathbb{N}, 1 \le p \le \infty$,其中的$f^{(k-1)}$需要几乎处处可导,并且几乎处处等于其勒贝格积分的$k-1$次导数。

$f$本身以及$f$的直至第$k$阶弱导数是有限的$L^p$范数
\begin{equation*}
  \begin{split}
      \big\| f \big\|_{W^{k,p}} &= \left( \sum_{i=0}^{k} \big\| f^{(i)} \big\|_p^p \right)^{\frac{1}{p}} \\
      &= \sum_{i=0}^{k} \left( \int \left| f^{(i)}(t) \right|^p \, dt \right)^{\frac{1}{p}} \\
      &= \big\| f^{(k)} \big\|_p + \big\| f \big\|_p,
  \end{split}
\end{equation*}
最后一个等式表明,单维索伯列夫空间的范数,等于方程序列自身的范数、以及其最高一阶导数的范数之和。

带有这一范数$\| \cdot \|_{W^{k,p}}$的单维索伯列夫空间$W^{k,p}$是一个巴拿赫空间。

\subsubsection{整数阶次的单维索伯列夫空间(p=2)}
$p=2$的单维索伯列夫空间$W^{k,2}$非常重要,因为它与傅里叶级数关系密切,并且构成了希尔伯特空间$H^k = W^{k,2}$。

$H^k$空间可以定义如下:可由帕塞瓦尔定理(Parseval theorem)\index{Parseval theorem \dotfill 帕塞瓦尔定理}予以证明(证明略。)
\begin{equation*}
  H^k(\mathbb{T}) = \left\{
  f \in L^(\mathbb{T}): \sum_{n=-\infty}^{\infty} \left( 1+n^2+n^4+\ldots+n^{2k} \right) \left| \hat{f}(n) \right|^2 < \infty
  \right\},
\end{equation*}
其中$\hat{f}$是方程$f$的傅里叶级数(Fourier series)\index{Fourier series \dotfill 傅里叶级数},它快速衰减(decay)。$\mathbb{T}$表示环面(torus)。

此时的$H^k$空间可以理解为在$L^2$空间中取内积的形式:
\begin{equation*}
  \langle \bm{u}, \bm{\nu} \rangle_{H^k} = \sum_{i=0}^{k} \langle D^i \bm{u}, D^i \bm{\nu} \rangle_{L^2}.
\end{equation*}

\subsubsection{多维索伯列夫空间}

\begin{definition}[索伯列夫空间]
假设$\exists k \in \mathbb{N}_0, 1 \le p < \infty$,那么多维索伯列夫空间$W^{k,p}(\Omega)$定义为在$\Omega$上的全部方程集合,使得对于每一个多元指数(multiple index) $\alpha$,方程的混合偏导数(分部积分) $f^{(\alpha)}$都存在,并且$f^{(\alpha)} \in L^p(\Omega), \big\| f \big\|_{L^p(\Omega)} < \infty$。从此意义上我们有多维索伯列夫空间$W^{k,p}(\Omega)$的定义式:
\begin{equation}
  \label{eq:sobolev-space-def}
\begin{split}
    W^{k,p}(\Omega) &\coloneqq \overline{C^{\infty}(\Omega)}^{\|\cdot\|_{W^{k,p}(\Omega)}}\\
    &= \left\{ u \in L^p(\Omega): D^{\alpha} u \in L^p (\Omega), \quad \forall \left| \alpha \right| \le k, \quad k \in \mathbb{N} \right\}.
\end{split}
\end{equation}

多维索伯列夫空间$W^{k,p}(\Omega)$的范数定义方式有多重,最常见的两种如下(并且这两种设定是等价的)
\begin{equation}
  \label{eq:sobolev-space-norm-def}
    \|u\|_{W^{k,p}(\Omega)} := \begin{cases}
    \left( \sum_{\left| \alpha \right| \le k} \Big\| D^{\alpha} u \big\|^p_{L^p(\Omega)} \right)^{\frac{1}{p}} & 1 \le p \le +\infty,\\
    \max_{\left| \alpha \right| \le k} \Big\| D^{\alpha} u \Big\| _{L^p(\Omega)} & p = +\infty.
  \end{cases}
\end{equation}

\begin{equation*}
  \|u\|'_{W^{k,p}(\Omega)} :=
  \begin{cases}
    \sum_{\left| \alpha \right| \le k} \Big\| D^{\alpha} u \Big\|_{L^p(\Omega)} & 1 \le p \le +\infty, \\
    \sum_{\left| \alpha \right| \le k} \big\|D^{\alpha} u \Big\|_{L^{\infty}}(\Omega) & p = + \infty.
  \end{cases}
\end{equation*}
\end{definition}

有着上述定义和范数的无限维索伯列夫空间$W^{k,p}(\Omega)$是一个巴拿赫空间。同时,对于$p < \infty$的情况而言,它也是一个可分空间(separable space)\footnote{膏按:FEM!!!!}。

习惯上,我们将1个索伯列夫空间$W^{k,2}(\Omega)$写作希尔伯特空间形式$H^{k}(\Omega)$,对应范数$\big\| \cdot \big\|_{W^{k,2}}(\Omega)$。




\subsubsection{分数次阶索伯列夫空间:Sobolev-Slobodeckij空间法}
\label{sec:sobolev-slobodeckij-space}
关于分数次阶索伯列夫空间,可参考\cite{DiNezza:2012wk}。
前面介绍的几种索伯列夫空间的情况,均假定$k \in \mathcal{N}$。然而有时候我们需要处理$k$是分数的情况(fractional order Sobolev space)。大致说来,有两种方法可以处理分数次阶索伯列夫空间,一种是Sobolev-Slobodeckij空间法,一种是贝塞尔位势空间法,后者需要一些关于分布的傅里叶变换知识,我们将在随后介绍,见\pageref{definition:bessel-potential-space}页 Definition \ref{definition:bessel-potential-space}。

Sobolev–Slobodeckij 空间\index{Sobolev-Slobodeckij!space \dotfill Sobolev–Slobodeckij空间}是勒贝格空间中的霍德尔条件(Hölder condition, Definition \ref{definition:hoelder-inequality-def})\index{Hölder condition \dotfill 霍德尔条件} 的广义化。

根据$W^{k,p(\Omega)}$或者$\mathring{W}^{k,p}(\Omega)$空间的定义\eqref{eq:sobolev-space-def}以及相应的范数\eqref{eq:sobolev-space-norm-def},
对于$0 < s \in \mathbb{R}, $的情况,设$s = k + \kappa, k \in \mathbb{N}_0$,我们有Sobolev-Slobodeckij范数(Sobolev-Slobodeckij norm)\index{Sobolev-Slobodeckij!norm \dotfill Sobolev-Slobodeckij范数}
\begin{equation*}
  \|u\|_{W^{s,p}(\Omega)} \coloneqq \left\{
  \|u\|^{p}_{W^{k,p}(\Omega)} + \left| u \right|^{p}_{W^{s,p}(\Omega)}
  \right\}^{\frac{1}{p}},
\end{equation*}
其中$\left| u \right|^p_{W^{s,p}(\Omega)}$是Sobolev-Slobodeckij半范数(Sobolev-Slobodeckij seminorm)\index{Sobolev-Slobodeckij!seminorm \dotfill Sobolev-Slobodeckij半范数}
\begin{equation*}
  \left| u \right|^p_{W^{s,p}(\Omega)} = \sum_{\left| \alpha \right| = k} \int_{\Omega} \int_{\Omega} \frac{
  \big| D^{\alpha}u(x) - D^{\alpha} u(y) \big|^{p}
  }{
  \big| x - y\big|^{d + p \kappa}
  }\, dx \, dy.
\end{equation*}

对于$p=2$的情况,$W^{s,2}(\Omega)$成为一个内积形式的希尔伯特空间:
\begin{equation}
  \label{eq:sobolev-slobodeckij-innerp-k}
  \langle u,\nu\rangle_{W^{k,2}(\Omega)} \coloneqq \sum_{\left| \alpha \right| \le k} \int_{\Omega} D^{\alpha}u(x) \, D^{\alpha}\nu(x) \, dx, \quad s=k\in \mathbb{N}_0,
\end{equation}
\begin{equation}
  \label{eq:sobolev-slobodeckij-innerp-s}
  \begin{split}
    \langle u,\nu\rangle_{W^{s,2}(\Omega)} \coloneqq & \langle u,\nu\rangle_{W^{k,2}(\Omega)} +
    \sum_{\left| \alpha \right  | = k} \int_{\Omega} \int_{\Omega} \frac{
    \left[D^{\alpha} u(x) - D^{\alpha} u(y) \right]
    \left[D^{\alpha} \nu(x) - D^{\alpha} \nu(y) \right]
    }{
    \left| x - y \right|^{d + 2 \kappa}
    } \, dx \, dy, \\
    & s=k+\kappa, \, k\in \mathbb{N}_0, \, \kappa \in (0,1).
  \end{split}
\end{equation}

对于$s < 0, 1<p<\infty$的情况,索伯列夫空间$W^{s,p}(\Omega)$通过对偶空间\index{dual space \dotfill 对偶空间}$\mathring{W}^{-s,q}(\Omega)$形式得以定义,其中$\frac{1}{p} + \frac{1}{q} = 1$,对应范数
\begin{equation*}
  \| u \| _{W^{s,p}(\Omega)} \coloneqq \sup_{0 \neq \nu \in \mathring{W}^{-s,q}(\Omega)} \frac{
  \big| \langle u,\nu \rangle_{\Omega} \big|
  }{
  \| \nu \|_{W^{-s,q}(\Omega)}
  },
\end{equation*}

同样的,$\mathring{W}^{-s,p}(\Omega)$是$W^{-s,q}(\Omega)$的对偶空间。

\subsubsection{索伯列夫空间的性质:嵌入定理}
介绍一些索伯列夫空间$W^{s,p}(\Omega)$的性质,这些性质有助于更好理解下文介绍的有界元法和有限元法。第一个性质可表示为索伯列夫空间的嵌入定理,我们先从霍德尔不等式入手。

\begin{definition}[霍德尔不等式]
  \label{definition:hoelder-inequality-def}
  霍德尔不等式(Hölder inequality)\index{Hölder inequality \dotfill 霍德尔不等式}可表示为
  \begin{equation*}
    \left| u(x) \nu(x) \right|_{1} \, \le \|u\|_p \,  \|\nu\|_q, \quad 1 \le p, q \le \infty, \frac{1}{p} + \frac{1}{q} = 1.
  \end{equation*}
\end{definition}
\begin{proof}
  见Definition \ref{definition:hoelder-inequality-generalization-def}。
\end{proof}
\begin{definition}[广义霍德尔不等式]
  \label{definition:hoelder-inequality-generalization-def}
  霍德尔不等式(Definition \ref{definition:hoelder-inequality-def}中向量的维度$n=1$。对于$n \ge 2$的情况,如$\left\{ u_i \right\}_{i=1}^{n}$,我们有广义霍德尔不等式(generalized Hölder inequality)\index{Hölder inequality!generalization \dotfill 广义霍德尔不等式}
  \begin{equation}
    \label{eq:hoelder-inequality-generalization-def}
    \int_{\Omega} \big| \prod_{i=1}^{n} u(x_i)\big| \le \prod_{i=1}^{n} \big\| u \big\|_{p_i}, \quad p_i \ge 1, \sum_{i=1}^{n} \frac{1}{p_i}=1.
  \end{equation}
\end{definition}
\begin{proof}
  $n=1$时。略。

  $n \ge 2$并且若$(n-1)$的情况。已知满足霍德尔不等式条件,我们的目标就成了,希望知道$n$的情况是否依然满足。分两种情况来分析。首先来看$n < \infty$,对应$p_n < \infty$。设
  \begin{equation*}
    p \coloneqq \frac{p_n}{p_n -1}, \, q \coloneqq p_n \Rightarrow \frac{1}{p} + \frac{1}{q} = 1.
  \end{equation*}

  此时我们有
  \begin{equation*}
    \begin{split}
      \big\| \prod_{i=1}^{n} u_i \big\|_{1} & \le
      \big\| \prod_{i=1}^{n-1} u_i \big\|_{p} \,
      \big\| \prod_{i=1}^{n} u_n \big\|_{q} \\
      & \le
      \left(
      \big\| \prod_{i=1}^{n-1} u_i^p  \big\|_{1}
      \right)^{\frac{1}{p}} \,
      \big\| u_n \big\|_{p_n},
    \end{split}
  \end{equation*}
我们设$p_i^{'} \coloneqq \frac{p_i}{p}$,上式进一步变为
\begin{equation*}
\big\| \prod_{i=1}^{n} u_i \big\|_{1}  \le
\left(
\big\| \prod_{i=1}^{n-1} u_i^p  \big\|_{p_i^{'}}
\right)^{\frac{1}{p}} \,
\big\| u_n \big\|_{p_n},
\end{equation*}
进一步,已知
\begin{equation*}
  \begin{split}
    \left(
    \big\| \prod_{i=1}^{n-1} u_i^p  \big\|_{p_i^{'}}
    \right)^{\frac{1}{p}} \equiv \left[
    \prod_{i=1}^{n-1} \big| u_i \big|^{\left( p \cdot p_i^{'} \right)}
    \right] ^{\left( \frac{1}{p_i^{'}}  \cdot \frac{1}{p} \right)} =
    \prod_{i=1}^{n-1} \big\| u_i \big\|_{p_i},
  \end{split}
\end{equation*}
则上式进一步变为
\begin{equation*}
  \begin{split}
    \big\| \prod_{i=1}^{n} u_i \big\|_{1} & \le
    \prod_{i=1}^{n-1} \big\| u_i  \big\|_{p_i} \, \big\| u_n \big\|_{p_n} = \prod_{i=1}^{n} \big\| u \big\|_{p_i}, \quad 2 \le n < \infty.
  \end{split}
\end{equation*}

$n = \infty$的情况,
\begin{equation*}
  \because p_n = \infty \Rightarrow \sum_{i=1}^{n-1} \frac{1}{p_i} = 1,
\end{equation*}

\begin{equation*}
  \begin{split}
      \therefore \big\| \prod_{i=1}^{n} u_i \big\| &\le \big\| \prod_{i=1}^{n-1} u_i \big\|_{1} \, \big\| u_n \|_{\infty}\\
      &\le \prod_{i=1}^{n} \left\| u \right\|_{p_i}, \quad n = \infty.
  \end{split}
\end{equation*}
\end{proof}


第一个性质可表示为索伯列夫空间的嵌入定理(embedding theorem of Sobolev)\index{embedding theorem of Sobolev space \dotfill 索伯列夫空间的嵌入定理}。

\begin{definition}[嵌入]
  \label{definition:sobolev-spaces-embeddings}
  数学上,嵌入(embedding) \index{embeddings \dotfill 嵌入}是指某个物件(instance) $X$被嵌入到另一个物件$Y$中去,用保留结构的映射(structure-preserving map)$f: X \mapsto Y$表示。这里的物件指数学结构,如群、子群等。所保留的具体数学``结构''因物件$X$和$Y$的种类而异。如在范畴论(category theorem)中,一个保留结构的映射往往称为一个态射(morphism)。
\end{definition}

\begin{theorem}[索伯列夫空间的嵌入定理$C=C(n,p)$]
  \label{theorem:sobolev-embedding-theorem}
给定有界的开放集$\Omega \subset \mathbb{R}^n$,则
\begin{equation*}
  W_0^{1,p}(\Omega) \subset L^{\frac{np}{n-p}}(\Omega), \quad n \ge 3, \,  1 \le p < n,
\end{equation*}
并且$W_0^{1,p}(\Omega)$被连续嵌入到空间$L^{\frac{n-p}{np}}(\Omega)$中,以下一组关系始终满足
\begin{equation}
  \label{eq:sobolev-embeddings-theorem-constant-def}
  \big\| u \big\|_{L^{\frac{np}{n-p}}(\Omega)} \le C(n,p)\, \big\| D u \big\|_{L^{p}(\Omega)}, \forall u \in W_0^{1,p}(\Omega),
\end{equation}
其中$C(n,p) \in (0,+\infty)$是个和$n,p$有关的常数,$D f = \left( D^{e_1} u, \ldots, D^{e_n} u\right) \in L^p(\Omega) \times \ldots \times L^p(\Omega)$。
\end{theorem}
\begin{proof}
  参考\cite[Theorem 1.4.6]{Brenner:2008hf},\cite[Theorem 3.26]{McLean:2000ta}。

\begin{enumerate}

  \item 证明对于$\forall \, u \in C_0^{\infty}$来说,\eqref{eq:sobolev-embeddings-theorem-constant-def}成立。由广义霍德尔不等式(Definition \ref{definition:hoelder-inequality-generalization-def})我们有,如果存在方程$u(x) \in L^{p}(\Omega)$,满足$u_j(x) \in L^{p_j}(\Omega), j=1,\ldots,m$,并且$\sum_{j=1}^{m}\frac{1}{p_j} =1$,那么
\begin{equation}
  \label{eq:sobolev-embedding-theorem-generalize-hoelder-ineq}
  \int_{\Omega} u_1(x) \ldots u_m(x) \, dx \le \big\| u_1 \big\|_{L^{p_1} (\Omega)} \ldots \big\| u_m \big\|_{L^{p_m} (\Omega)}
\end{equation}

\item 在$p=1$时,由$u \in C_0^{\infty}(\Omega)$可知,方程$u(x)$可以表示为下述积分形式
\begin{equation*}
  \begin{split}
  &u(x) = \int_{-\infty}^{x_i} D^{e_i} u \left(
  x_1, \ldots, x_{i-1}, t, x_{i+1}, \ldots, x_n
  \right) \, dt, \\
  \hookrightarrow & \big| u(x) \big| \le \int_{-\infty}^{x_i} \big| D^{e_i} u \big| \, dt \le \int_{-\infty}^{\infty} \big| D^{e_i} u \big| \, d x_i, \\
  \hookrightarrow & \big| u(x) \big|^{n} \le \prod_{i=1}^{n} \int_{-\infty}^{\infty} \big| D^{e_i} u \big| \, d x_i,
\end{split}
\end{equation*}
从而我们有
\begin{equation}
  \label{eq:sobolev-embedding-ux-partial-n-1}
  \big| u(x) \big|^{\frac{n}{n-1}} \le \left(
  \prod_{i=1}^{n} \int_{-\infty}^{\infty} \big| D^{e_i} u \big| \, d x_i
  \right)^{\frac{1}{n-1}}.
\end{equation}

下面对数列$x = \left\{ x_i \right\}_{i=1}^{n}$求偏导。先从$x_1$开始,根据\eqref{eq:sobolev-embedding-ux-partial-n-1}我们有
\begin{equation}
  \label{eq:sobolev-embedding-ux-partial-n-1-x1}
  \begin{split}
    \int_{-\infty}^{\infty}  \big| u(x) \big|^{\frac{n}{n-1}} \, d x_1
    & \le \int_{-\infty}^{\infty}
    \left(
    \prod_{i=1}^{n} \int_{-\infty}^{\infty} \big| D^{e_i} u \big| \, d x_i
    \right)^{\frac{1}{n-1}}
    \, dx_1 \\
    & = \left(
    \int_{-\infty}^{\infty} \big| D^{e_1} f \big| \, d x_i
    \right)^{\frac{1}{n-1}} \,
    \int_{-\infty}^{\infty}
    \left(
    \prod_{i=2}^{n} \int_{-\infty}^{\infty} \big| D^{e_i} u \big| \, d x_i
    \right)^{\frac{1}{n-1}}
    \, dx_1 \\
    & \le \left(
    \int_{-\infty}^{\infty} \big| D^{e_1} f \big| \, d x_i
    \right)^{\frac{1}{n-1}} \,
    \prod_{i=2}^{n}
    \left(
     \int_{-\infty}^{\infty} \int_{-\infty}^{\infty} \big| D^{e_i} u \big| \, d x_i \, dx_1
    \right)^{\frac{1}{n-1}}.
  \end{split}
\end{equation}

在\eqref{eq:sobolev-embedding-ux-partial-n-1-x1}的基础上,继续对$x_2,\ldots x_n$作偏导,可得
\begin{equation}
  \label{eq:sobolev-embedding-ux-partial-n-1-xn}
  \int_{\mathbb{R}^n} \big| u(x) \big|^{\frac{n}{n-1}} \, d x
  \le \left(
  \prod_{i=1}^{n} \int_{\mathbb{R}^n} \big| D^{e_i} u \big| \, d x_i
  \right)^{\frac{1}{n-1}}.
\end{equation}

\eqref{eq:sobolev-embedding-ux-partial-n-1-xn} $\Rightarrow$
\begin{equation*}
\label{eq:sobolev-embedding-ux-partial-n-1-xn-p1}
  \begin{split}
    \big\| u(x) \big\|_{L^{\frac{n}{n-1}}(\Omega)} & \le \left(
    \prod_{i=1}^{n} \int_{\mathbb{R}^n} \big| D^{e_i} u \big| \, d x_i
    \right)^{\frac{1}{n}} \\
    &\le \int_{\Omega} \left( \sum_{i=1}^{n} \big|
    D^{e_i} u
    \big| \right) \, dx \\
    & \le \frac{1}{\sqrt{n}} \, \int_{\Omega} \big| D u \big| \, dx \\
    & = \frac{1}{\sqrt{n}} \, \big\| D u\big\|_{L^{1}(\Omega)}, \quad \forall u \in C_0^{\infty}(\Omega).
  \end{split}
\end{equation*}

\item 下面考虑$1<p<n$的情况。已知$u \in C_0^{\infty}(\Omega)$,则用$\left| u \right|^{\gamma}, \gamma >1$代替\eqref{eq:sobolev-embedding-ux-partial-n-1-xn-p1}中的$u$,我们有
\begin{equation}
\label{eq:sobolev-embedding-ux-partial-n-1-xn-p1n}
  \big\| \left| u \right|^{\gamma} \big\|_{L^{\frac{n}{n-1}}(\Omega)}  \le \frac{1}{\sqrt{n}} \, \int_{\Omega} \big| D \left| u \right|^{\gamma} \big| \, dx = \frac{\gamma}{\sqrt{n}} \, \int_{\Omega} \big| u \big| ^{\gamma -1} \, \big| D u \big| \, dx,
\end{equation}
设$\frac{1}{p}+\frac{1}{q}=1, \, p,q > 0$,由霍德尔不等式(Definition \ref{definition:hoelder-inequality-generalization-def})得,\eqref{eq:sobolev-embedding-ux-partial-n-1-xn-p1n}$ \Rightarrow$
\begin{equation}
  \label{eq:sobolev-embedding-ux-partial-n-1-xn-p1n-gamma}
  \begin{split}
    &\big\| \left| u \right|^{\gamma} \big\|_{L^{\frac{n}{n-1}}(\Omega)}
    \le \frac{\gamma}{\sqrt{n}} \, \int_{\Omega} \big| u \big| ^{\gamma -1} \, \big| D u \big| \, dx
    \le \frac{\gamma}{\sqrt{n}} \,
    \Big\| \left| u \right|^{\gamma -1} \Big\|_{q} \,
    \Big\| D u \Big\|_{p}, \\
    \hookrightarrow &
    \big\| \left| u \right| \big\|_{L^{\frac{\gamma n}{n-1}}(\Omega)}^{\gamma}
    \le \frac{\gamma}{\sqrt{n}} \,
    \Big\| \left| u \right| \Big\|_{L^{q (\gamma - 1)}(\Omega)}^{\gamma -1} \,
    \Big\| D u \Big\|_{p}.
  \end{split}
\end{equation}

下面选取$\gamma$的值,使得
\begin{equation*}
  \gamma \frac{n}{n-1} \equiv (\gamma - 1) q \Longrightarrow \gamma = \frac{p(n-1)}{n-p} \Longrightarrow \gamma \frac{n}{n-1} \equiv (\gamma - 1) q = \frac{np}{n-p},
\end{equation*}
代入\eqref{eq:sobolev-embedding-ux-partial-n-1-xn-p1n-gamma},调整得
\begin{equation*}
  \big\| u \big\|_{L^{\frac{np}{n-p}}(\Omega)} \le \frac{\gamma}{\sqrt{n}} \, \big\| D u \|_{p}, \quad \forall u \in C_0^{\infty}(\Omega),
\end{equation*}
联系\eqref{eq:sobolev-embeddings-theorem-constant-def}可得$C(n,p)$的值
\begin{equation*}
  C(n,p)=\frac{np-n}{\sqrt{n} (n-p)}.
\end{equation*}

\item 前面的论证过程均假设$u \in C_0^{\infty}$。如果现在假设$u_{\ell} \in W^{1,p}_{0}(\Omega)$,则我们利用$u_m \in C_0^{\infty}(\Omega)$来近似$u_{\ell} \in W^{1,p}(\Omega)$,将$u_{\ell}-u_m$代入\eqref{eq:sobolev-embeddings-theorem-constant-def}中。可见$\{ u_m \} \in L^{\frac{np}{n-p}}(\Omega)$是一个柯西数列。因此我们也能证明$u_{\ell} \in L^{\frac{np}{n-p}}(\Omega)$,并且满足条件\eqref{eq:sobolev-embeddings-theorem-constant-def}。
\end{enumerate}
\end{proof}

前面讨论的都是$k=1$时,$n,p$变化,$C=C(n,p)$。现在来将$k$的变化也考虑进来,$C=C(k,n,p)$。
\begin{corollary}[索伯列夫空间的嵌入定理$C=C(k,n,p)$]
如果$kp < n$,则索伯列夫空间$W_0^{k,p}(\Omega)$连续嵌入到$L^{\frac{np}{n-kp}}(\Omega)$中,对应常数$C(k,n,p)$满足
\begin{equation}
  \label{eq:sobolev-embedding-theorem-c-k-n-p}
  \big\| u \big\|_{L^{\frac{np}{n-kp}}(\Omega)} \le C(k,n,p) \, \big\| u \big\|_{W^{k,p}_0(\Omega)}.
\end{equation}
\end{corollary}
\begin{proof}
假设$k p < n$,$u \in W_0^{k,p}(\Omega)$。

\begin{enumerate}
  \item 根据$D^{\alpha} u \in L^{p}(\Omega), \, \forall \, \left| \alpha \right| \le k$,从索伯列夫不等式可得,对于$|\beta| \le k-1$
  \begin{equation*}
    \big\| D ^{\beta} u \big\|_{L^{p^{*}}(\Omega)} \le C \, \big\| u \big\|_{W^{k,p}(\Omega)}, \quad p^{*} \coloneqq \frac{np}{n-p}
  \end{equation*}
  因此有$u \in W^{k-1,p^{*}}(\Omega)$。

  \item 用类似的方法,我们可以进一步证明$u \in W^{k-2,p^{**}}(\Omega)$,对于$|\gamma| \le k-2$
  \begin{equation*}
    \big\| D ^{\gamma} u \big\|_{L^{p^{**}}(\Omega)} \le C \, \big\| u \big\|_{W^{k,p^{*}}(\Omega)}, \quad p^{**} \coloneqq \frac{1}{p^{*}} - \frac{1}{n} = \frac{1}{p} - \frac{2}{n}.
  \end{equation*}
  \item 以此类推,最终经过$k$次迭代后,我们得以证明,对于$u \in W^{0,q}(\Omega) = L^{q}(\Omega)$,\eqref{eq:sobolev-embedding-theorem-c-k-n-p}成立
  \begin{equation*}
    \big\| u \big\|_{L^{\frac{np}{n-kp}}(\Omega)} \le C \, \big\| u \big\|_{W^{k,p}_0(\Omega)}, \quad \frac{1}{q} = \frac{1}{p} - \frac{k}{n}.
  \end{equation*}
\end{enumerate}
\end{proof}

前面讨论的是将$W_0^{k,p}(\Omega)$嵌入到$L^{p}(\Omega)$中去。如果$\Omega$是一个$C^{k}-domain$,则我们可以利用延拓算子$E$,将索伯列夫嵌入定理从$W_0^{k,p}(\Omega)$延伸到$W^{k,p}(\Omega)$空间。

即是说,如果$u \in W^{k,p}(\Omega)$,则可以考虑一个$E u \in W_0^{k,p}(\Omega')$,其中$\Omega'$是$\Omega$的延拓(extension),$\Omega$是$\Omega'$的限制(restriction):$\Omega' \supset \Omega$。进而,如果$kp < n$,那么$W_0^{k,p}(\Omega') \in L^{\frac{np}{n-kp}}(\Omega')$。

因此$u$属于限制$\Omega$所对应的勒贝格空间:$u \in L^{\frac{np}{n-kp}}(\Omega)$。

一方面在$\Omega$中我们有$E u = u$,另一方面根据$\Omega$的具体情况我们有$\big\| E u \big\|_{W^{k,p} (\Omega')} \le C \, \big\| u \big\|_{W^{k,p}(\Omega)}$。所以我们有
\begin{corollary}[索伯列夫空间的嵌入定理$C=C(k,n,p,\Omega)$]
设$\Omega \subset \mathbb{R}^n$是一个有界的$C^k-domain$。如果$kp < n$,一系列索伯列夫空间$W^{k,p}(\Omega)$被依次嵌入到对应的$L^{\frac{np}{n-kp}}(\Omega)$空间中去。这几是说,存在某个常数$C=C(k,n,p,\Omega)$,使得
\begin{equation}
  \label{eq:sobolev-embedding-theorem-c-k-n-p-omega}
  \big\| u \big\|_{L^{\frac{np}{n-kp}}(\Omega)} \le C \, \big\| u \big\|_{W^{k,p}(\Omega)}, \quad \forall \, u \in W^{1,p}(\Omega).
\end{equation}
\end{corollary}
\begin{proof}

\end{proof}




















\subsubsection{索伯列夫空间的性质:范数等价}
第二个性质是索伯列夫空间的范数等价定理(norm equivalence theorem of Sobolev)\index{norm equivalence theorem of Sobolev space}。先来介绍范数等价的定义,以及一般意义上的范数等价定理。
\begin{definition}[范数等价]
  \label{definition:equivalence-norm-def}
  一个向量空间$V \in \mathbb{F}=\mathbb{R}或\mathbb{C}$中有两个范数$\| \cdot \|_a$和$\| \cdot \|_b$。范数等价(equivalence of norms)\index{equivalence of norms \dotfill 范数等价}是指,$\| \cdot \|_b$总是在$\| \cdot \|_a$的某乘数倍范围之内,换句话说:存在正常数$c,C$使得$\forall x \in V$,都满足
  \begin{equation*}
    c \, \|x\|_a \le \|x\|_b \le C \, \| x \|_a.
  \end{equation*}
\end{definition}

\begin{theorem}[范数等价定理]
\label{theorem:equivalence-norm-theorem}
$\exists V \in \mathbb{F}^{\infty}$,其中$\nu_1, \ldots \nu_n$是$V$的基,$n$是$V$的维度。因此每个$x \in V$都有唯一的表现形式,如
\begin{equation}
  \label{eq:norm-equivalence-def}
  x = \sum_{i=1}^{n} x_i \nu_i,
\end{equation}
上述内积形式中,$\left\{ \nu_i \right\}_{i=1}^{n}$是标准正交基,$\left\{ x_i \right\}_{i=1}^{n}$是一组关于$\left\{ \nu_i \right\}_{i=1}^{n}$的坐标(标量)。

定义$V$中的范数为
\begin{equation*}
  \| x \|_{*} \coloneqq \max_{i=1,\ldots,n} \left| x_i \right|,
\end{equation*}
我们将有范数$  \| x \|_{*}$的$V$称为完整向量空间(complete)。

则范数等价定理为:$V \in \mathbb{F}^{\infty}$中的所有范是等价的。
\end{theorem}
\begin{proof}
  使$\| x \|$为$V$中任一范数。我们的目标是证明$  \| x \|$和$  \| x \|_{*}$等价,即存在常数$c,C$使得不等式\eqref{eq:norm-equivalence-def}成立。分两步来证明。

  第一步来看对$\|x\| \le C \, \|x\|_{*}$的证明。根据定义我们有
  \begin{equation*}
    \begin{split}
      \| x \| &= \big\| \sum_{i=1}^n x_i \nu_i \big\| \\
      &\le \sum_{i=1}^{n} \big\| x_i \nu_i \big\| =\sum_{i=1}^{n} \left| x_i \right| \, \| \nu_i \| \\
      &\le \underbrace{n \, \max_{i=1,\ldots,n} \| \nu_i \|}_{\text{常数}} \, \max_{i=1,\ldots,n} \left| x_i \right|
    \end{split}
  \end{equation*}

设常数$C \coloneqq n \, \max_{i=1,\ldots,n} \| \nu_i \|$,则上式变为
\begin{equation*}
  \begin{split}
    &\| x \| \le C \, \max_{i=1,\ldots,n} \left| x_i \right|, \\
    \hookrightarrow & \| x \| \le C \, \| x \|_{*}.
  \end{split}
\end{equation*}

第二步,来看$c \, \| x \|_{*} \le \| x \|$的证明。首先需要证明$S=\left\{ x: \| x \|_{*}=1 \right\}$是个紧密空间(compact space),方法为证明$S$完整且完全有界(totally bounded)。

完整性的证明:由三角不等式(Definition \ref{definition:triangle-inequality-def})得
\begin{equation*}
  \| x \|_{*} - \| y \|_{*} \le \| x-y \|_{*},
\end{equation*}

$\hookrightarrow$方程$x:x \mapsto \|x\|_{*}$是连续的。

$\hookrightarrow$连续方程$x$构成的向量空间$S$是$V$中的封闭子集。

$\hookrightarrow$已知$V$是个完整空间,$\therefore V$的封闭子集$S$也是个完整空间。

完全有界的证明:设$\varepsilon >0$,选则常数$m > 1/\varepsilon$。因此,$S$被$O(m^d) < \infty$个球体$B$的集合所覆盖,这些球都是半径为$1/m$,范数为$\| \cdot \|_{*}$,表示为
\begin{equation*}
  B_{j_1,\ldots,j_n} \coloneqq \left\{
  \sum_{i=1}^{n} x_i \nu_i : \frac{j_i}{m}-\varepsilon < x_i < \frac{j_i}{m} + \varepsilon, \quad i=1,\ldots,m, \quad j_i = -m, \ldots,m,
  \right\}
\end{equation*}

$\therefore S$完全有界。

其次,设方程$f(x)=\|x\|, x \in [0,\infty)$。对于$x \in S$,我们有$f(x)$也是一个连续方程,这是由于
\begin{equation*}
  \begin{split}
    \big\| f(x) - f(y) \big\| & \le \big\| \|x\| - \|y\| \big\|\\
    & \le \big\| x  - y \big\|\\
    & \le C \, \big\| x - y \big\|_{*}
    \end{split}
\end{equation*}

$\therefore$紧密空间$S$中的方程$f$有最小值。

最后,由$x \in S$得,设$c \coloneqq \min_{x \in S} f(x) > 0$。$x \in S \Rightarrow \frac{x}{\| x \|_{*}} \in S \Rightarrow$
\begin{equation*}
  f \left( \frac{x}{\| x \|_{*}} \right) \equiv \Big\| \frac{x}{\| x \|_{*}} \Big\| \ge c,
\end{equation*}
$\therefore \|x \| \ge c \|x\|_{*}.$
\end{proof}

当$k=1,p=2$时,索伯列夫空间$W^{1,2}(\Omega)$的范数\eqref{eq:sobolev-space-norm-def}变为
\begin{equation*}
\begin{split}
  \| u \|_{W^{1,2}(\Omega)}
  &= \left\{ \big\| D \, u \big\|_{L^{2}(\Omega)}^2 \right\}^{\frac{1}{2}}\\
  &= \left\{
  \|u\|_{L^{2}(\Omega)}^2 + \| \triangledown u \|_{L^{2}(\Omega)}^2
  \right\}^{\frac{1}{2}},
\end{split}
\end{equation*}
并且半范数为
\begin{equation*}
  \left| u \right|_{W^{1,2}(\Omega)} = \| \triangledown u \|_{L^{2}(\Omega)},
\end{equation*}
则我们有索伯列夫空间的范数等价定理。

\begin{theorem}[索伯列夫空间的范数等价定理]
  \label{theorem:sobolev-equivalence-norm-theorem}
  设一个有界的线性方程$f:W^{1,2}(\Omega) \mapsto \mathbb{R}$,满足
  \begin{equation*}
    0 \le \left| f(u) \right| \le c_f \big\| u \big\|_{W^{1,2}(\Omega)}, \quad \forall u \in W^{1,2}(\Omega),
  \end{equation*}
$c_f >0$是个常数。如果对于某个常数$\iota$,$f(\iota) = 0 \, \text{iff.} \, \iota \equiv 0$,那么在$W^{1,2}(\Omega)$空间中,所有满足上述条件的方程$f$的范数是等价的:
\begin{equation}
  \label{eq:sobolev-norm-equivalence-theorem}
  \big\| u \big\|_{W^{1,2}(\Omega), f} \coloneqq \left\{ \left| f(u) \right|^2 + \big\| \triangledown u \big\|_{L^{2}(\Omega)}^2 \right\}^{\frac{1}{2}}.
\end{equation}
\end{theorem}

\begin{proof}
  首先证明$f(u) \le c_f \big\| u \big\|_{L^{2}(\Omega)}$:根据前提假定,$f$是一个线性有界方程,则
  \begin{equation*}
      \big\| u \big\|_{W^{1,2}(\Omega), f}^2 = \underbrace{
      \left| f(u) \right|^2}_{\le c_f^2 \big\| u \big\|^2_{W^{1,2}(\Omega) }} + \underbrace{\big\| \triangledown u \big\|^2_{L^{2}(\Omega)}}_{\le \big\| u \big\|^2_{W^{1,2}(\Omega)}} \le \left(1+c_f^2\right) \big\| u \big\|^2_{L^{2}(\Omega)},
  \end{equation*}
则我们有
\begin{equation*}
  f(u) \le c_f \big\| u \big\|_{L^{2}(\Omega)}.
\end{equation*}


  第二步,对$\left|f(u) \ge 0\right|$的证明,较为间接。已知$\nexists c_0$使得满足下述条件
  \begin{equation*}
    \| u \|_{W^{1,2}(\Omega)} \le c_0 \underbrace{\|u \|_{W^{1,2}(\Omega), f}}_{ \equiv \|f(u)\|},
  \end{equation*}
则假设存在一个数列$\{u_n\}_{n \in \mathbb{N}} \in W^{1,2}(\Omega)$,满足
\begin{equation*}
n \le \frac{
\big\|u_n\big\|_{W^{1,2}(\Omega)}
}{\big\|u_n\big\|_{W^{1,2}(\Omega),f}}.
\end{equation*}

把$\{u_n\}_{n \in \mathbb{N}}$标准化为$\left\{\bar{u}_n\right\}_{n \in \mathbb{N}}$
\begin{equation*}
  \left\{ \bar{u}_n \right\}_{n \in \mathbb{N}} \coloneqq \frac{
  u_n
  }{
  \big\| u_n \big\|_{W^{1,2}(\Omega)}
  },
\end{equation*}
从而使得我们有
\begin{equation*}
  \begin{split}
    &\big\| \bar{u}_n \big\|_{W^{1,2}(\Omega)}=1,\\
    &\big\| \bar{u}_n \big\|_{W^{1,2}(\Omega), f} =
    \frac{
    \big\| u_n \big\|_{W^{1,2}(\Omega), f}
    }{
    \big\| u_n \big\|_{W^{1,2}(\Omega)}
    } \le \frac{1}{n},
  \end{split}
\end{equation*}

那么随着$n \rightarrow \infty$,
\begin{equation*}
  \begin{split}
    \lim_{n \rightarrow \infty} \big\| \bar{u}_n \big\|_{W^{1,2}(\Omega), f} = \lim_{n \rightarrow \infty} \big| f(\bar{u}_n) \big| \le \lim_{n \rightarrow \infty} \left( \frac{1}{n} \right) =0,
  \end{split}
\end{equation*}
代回\eqref{eq:sobolev-norm-equivalence-theorem}有
\begin{equation*}
\begin{split}
  &\lim _{\bar{u}_n \rightarrow \infty} \big\| u \big\|_{W^{1,2}(\Omega), f} = \lim_{n \rightarrow \infty} \left\{ \left| f(\bar{u}_n) \right|^2 + \big\| \triangledown \bar{u}_n \big\|_{L^{2}(\Omega)}^2 \right\}^{\frac{1}{2}} ,\\
  \hookrightarrow & \lim_{n \rightarrow \infty} \big\| \triangledown \bar{u}_n \big\|_{L^2(\Omega)}=0.
\end{split}
\end{equation*}

由于标准化数列$\left\{ \overline{u}_n \right\}_{n \in \mathbb{N}}$在$W^{1,2}(\Omega)$中有界,以及由于$W^{1,2}(\Omega) \hookrightarrow L^2{\Omega}$的嵌入是紧凑的,则$\exists$ 子数列 $\left\{ \overline{u}_n' \right\}_{n' \in \mathbb{N}}  \subset \left\{ \overline{u}_n \right\}_{n \in \mathbb{N}} $,在$L^2(\Omega)$中收敛。

若是定义
\begin{equation*}
  \overline{u} \coloneqq \lim_{n' \rightarrow \infty}\overline{u}_{n'} \in L^{2}(\Omega),
\end{equation*}
则我们有
\begin{equation*}
  \begin{split}
    & \overline{u} \in W^{1,2}(\Omega), \\
    & \big\| \triangledown \overline{u} \big\|_{L^2(\Omega)} = 0,
  \end{split}
\end{equation*}
换句话说,$\overline{u}$是个常数。此外
\begin{equation*}
  0 \le \big| f(\overline{u}) \big| = \big| f \left( \lim_{n' \rightarrow \infty}  \overline{u}_{n'} \right) \big| =  \lim_{n' \rightarrow \infty} \big| f(\overline{u}_{n'})\big| = 0.
\end{equation*}

已知$\overline{u}$是个常数,$f(\overline{u}) = 0$,则我们有$\overline{u}=0$。\footnote{然而需要注意的是,这里有
\begin{equation*}
\big\|  \overline{u} \big\|_{W^{1,2}(\Omega)} = \lim_{n' \rightarrow \infty} \big\| \overline{u}_{n'} \big|_{W^{1,2}(\Omega)} = 1.
\end{equation*}}

\begin{equation*}
  \therefore \big| f\left(u\right) \big| \ge \big| f\left(\overline{u} \right)\big| = 0.
\end{equation*}

对应地,已知$W^{1,2}(\Omega),W^{1,2}(\Gamma)$中的等价范分别为
\begin{equation}
  \label{eq:sobolev-equivalence-norm-w12omega}
  \begin{split}
    &\big\| u \big\|_{W^{1,2}(\Omega), \Omega} \coloneqq \left\{
    \left[
    \int_{\Omega} u(x) \, dx
    \right]^2
    + \big\| \triangledown u \big\|^2_{L^2(\Omega)}
     \right\}^{\frac{1}{2}}, \\
     &\big\| u \big\|_{W^{1,2}(\Omega), \Gamma} \coloneqq \left\{
     \left[
     \int_{\Gamma} u(x) \, ds_x
     \right]^2
     + \big\| \triangledown u \big\|^2_{L^2(\Omega)}
      \right\}^{\frac{1}{2}},
  \end{split}
\end{equation}
那么$\mathring{W}^{1,2}(\Omega)$中的等价范为$\big\| \triangledown u \big\|_{L^2(\Omega)}$。
\end{proof}

下面需要介绍庞加莱不等式。
\begin{theorem}[庞加莱不等式$(u \in W_0^{k,p}(\Omega))$]
  \label{theorem:poincare-inequality-def}
  设一个有界域$\Omega \subset \mathbb{R}^n$。则存在$C_p=C(p,n,M) > 0$,使得$\quad \forall \, u \in W_0^{k,p}(\Omega)$都满足庞加莱不等式(Poincaré inequality) \index{Poincaré inequality \dotfill 庞加莱不等式}
  \begin{equation}
    \label{eq:poincare-inequality-def}
    \big\| u \big\| _{L^p(\Omega)} \le C_p \big\| \triangledown u \big\|_{L^p(\Omega)}.
  \end{equation}
\end{theorem}
\begin{proof}
  \begin{enumerate}
  \item 证明对于$\forall \, u \in C_0^{\infty}(\Omega),$\eqref{theorem:poincare-inequality-def}成立。

  \item 若$u \in W^{1,p}(\Omega)$,则从$u$中选取某一数列$\left\{ u_k \right\} \subset C_0^{\infty}(\Omega)$,使得该数列随着$k \rightarrow \infty$而在$W^{1,p}$范数上收敛至$u$,即
  \begin{equation*}
\begin{split}
  &\lim_{k \rightarrow \infty} \big\| u_k - u \big\|_{L^p(\Omega)}  = 0, \\
  &\lim_{k \rightarrow \infty} \big\| \triangledown u_k - \triangledown u \big\|_{L^p(\Omega)} =0.
\end{split}
  \end{equation*}

上式等价于
\begin{subequations}
\begin{equation}
  \label{eq:poincare-inequality-uk-u}
\lim_{k \rightarrow \infty} \big\| u_k \big\|_{L^p(\Omega)}
= \big\| u \big\|_{L^p(\Omega)},
\end{equation}
\begin{equation}
  \label{eq:poincare-inequality-uk-u-delta}
\lim_{k \rightarrow \infty} \big\| \triangledown u_k \big\|_{L^p(\Omega)}
= \big\| \triangledown u \big\|_{L^p(\Omega)}。
\end{equation}
\end{subequations}

\item 既然对于数列$\left\{ u_k \right\}$中的每一个$u_k$都满足\eqref{eq:poincare-inequality-def},则我们有以下一组不等式成立
\begin{equation*}
  \big\| u_k \big\| _{L^p(\Omega)} \le C_p \big\| \triangledown u_k \big\|_{L^p(\Omega)}, \quad \forall \, u_k \in C_0^{\infty},
\end{equation*}
从而当$k \rightarrow \infty$时,\eqref{eq:poincare-inequality-def}对$\forall \, u \in C_0^{\infty}(\Omega)$都成立。
\end{enumerate}

基于上述思路,来看$C_p$的值。由于在$\Gamma=\partial \Omega$上有$u(\bm{x})=0$,以$u=0$为边界,由散度定理(divergence theorem, Theorem \ref{theorem:bvp-gauss-divergence-theorem})有
\begin{equation}
  \label{eq:sobolev-poincare-divergence-use}
  \begin{split}
    &\int_{\Omega \cap \left\{ u > 0 \right\}} div(u^p \bm{x}) \, d \bm{x} = 0,\\
    &\int_{\Omega \cap \left\{ u < 0 \right\}} div(u^p \bm{x}) \, d \bm{x} = 0,
  \end{split}
\end{equation}
其中
\begin{equation*}
  div(u^p \bm{x}) = p \, u \, \triangledown u \cdot \bm{x} + n u^p.
\end{equation*}

\eqref{eq:sobolev-poincare-divergence-use}$\Rightarrow$
\begin{equation}
  \begin{split}
    &\int_{\Omega \cap \left\{ u > 0 \right\}} u^p d \bm{x} = - \frac{p}{n} \int_{\Omega \cap \left\{ u > 0 \right\}} u^{p-1} \triangledown u \cdot \bm{x} \, d \bm{x}, \\
    &\int_{\Omega \cap \left\{ u > 0 \right\}} u^p d \bm{x} = - \frac{p}{n} \int_{\Omega \cap \left\{ u > 0 \right\}} u^{p-1} \triangledown u \cdot \bm{x} \, d \bm{x}.
  \end{split}
\end{equation}

$\Omega$有界 $\Rightarrow M \coloneqq \max_{\bm{x} \in \Omega}  \left| \bm{x} \right| < \infty$,则由施瓦茨不等式(Schwarz inequality) \index{Schwarz inequality \dotfill 施瓦茨不等式}我们有
\begin{equation*}
\begin{split}
    \int_{\Omega \cap \left\{ u > 0 \right\}} u^p \, d \bm{x} &=
    \Big| \frac{p}{n} \int_{\Omega \cap \left\{ u > 0 \right\}} u^{p-1} \, \triangledown u \cdot \bm{x} \, d \bm{x} \Big| \\
    &\le \frac{p M}{n}
    \left(
    \int_{\Omega \cap \left\{ u > 0 \right\}} \big| u^{p-1} \big|^q \, d \bm{x}
    \right)^{\frac{1}{q}} \,
    \left(
    \int_{\Omega \cap \left\{ u > 0 \right\}} \big| \triangledown u \big|^p \, d \bm{x}
    \right)^{\frac{1}{p}}\\
    &\le \frac{p M}{n}
    \left(
    \int_{\Omega \cap \left\{ u > 0 \right\}} \big| u^{p} \big|^q \, d \bm{x}
    \right)^{\frac{1}{q}} \,
    \big\| \triangledown u \big\|_{L^P(\Omega \cap \left\{ u > 0 \right\})},
\end{split}
\end{equation*}
类似地,我们也有
\begin{equation*}
  \int_{\Omega \cap \left\{ u < 0 \right\}} u^p \, d \bm{x} \le \frac{p M}{n}
  \left(
  \int_{\Omega \cap \left\{ u < 0 \right\}} \big| u^{p} \big|^q \, d \bm{x}
  \right)^{\frac{1}{q}} \,
  \big\| \triangledown u \big\|_{L^P(\Omega \cap \left\{ u < 0 \right\})}.
\end{equation*}

上两式代回\eqref{eq:poincare-inequality-def},可得常数$C_p$的值
\begin{equation*}
  C_p = \frac{p M}{n}
\end{equation*}
\end{proof}

上面的分析主要针对$u \in W^{k,p}_{0}(\Omega)$的情况。我们可以进一步将庞加莱不等式扩展到$u \in W^{k,p}(\Omega)$的分析中。
\begin{corollary}[庞加莱不等式$(u \in W^{k,p}(\Omega))$]
  \label{corollary:poincare-inequality-unbounded-def}
  $k=1$时,由
  \begin{equation*}
    \big\| u \big\|_{W^{1,p}(\Omega)} = \big\| \triangledown u \big\|_{L^p(\Omega)},
  \end{equation*}
  可得
  \begin{equation*}
    \big\| u \big\|_{W^{1,p}(\Omega)} =
    \left(
    \big\| u \big\|^{p}_{L^p(\Omega)}
    +\big\| \triangledown u \big\|^p_{L^p(\Omega)}
    \right)^{\frac{1}{p}},
  \end{equation*}
  因此庞加莱不等式为
  \begin{equation}
    \label{eq:poincare-inequality-unbounded-def}
    \big\| \triangledown u \big\|^p_{L^p(\Omega)} \
    \le \big\| u \big\|_{W^{1,p}(\Omega)}
    \le \left(C_p^p + 1 \right)^{\frac{1}{p}}
    \big\| \triangledown u \big\|_{L^p(\Omega)}, \quad \forall \, u \in W^{1,p}(\Omega).
  \end{equation}
\end{corollary}

\subsubsection{索伯列夫空间的Bramble-Hilbert引理}
\label{sec:sobolev-bramble-hilbert-lemma}
第三个特征是索伯列夫空间的Bramble-Hilbert引理,它有助于我们分析(分段)多项式试探空间(trial space)的近似属性\citep[Sec 2.3.1]{Jovanovic:2014iy}。

则根据庞加莱不等式\eqref{theorem:poincare-inequality-def},等价范\eqref{eq:sobolev-equivalence-norm-w12omega}可以改写为
\begin{equation}
  \label{eq:sobolev-poincare-inequality}
  \int_{\Omega} \big| u(x) \big|^2 \, dx  \le c_P \left\{
  \left[
  \int_{\Omega} u(x) \, dx
  \right]^2
  + \int_{\Omega} \left| \triangledown u(x) \right|^2 \, dx
  \right\}, \quad \forall \, u \in W^{1,2}(\Omega).
\end{equation}

则我们有Bramble-Hilbert引理(Bramble-Hilbert Lemma)\index{Bramble-Hilbert Lemma \dotfill Bramble-Hilbert引理}
\begin{lemma}[Bramble-Hilbert引理]
  \label{lemma:bramble-hilbert-lemma}
  设$k \in \mathbb{N}_0$,一个有界线性泛函$f:W^{k+1,2}(\Omega) \mapsto \mathbb{R}$满足
  \begin{equation*}
    \left| f(\nu) \right| \le c_f \, \| \nu \|_{W^{k+1,2}(\Omega)}, \quad \forall \, \nu \in W^{k+1,2}(\Omega).
  \end{equation*}

用$\mathcal{P}(\Omega)$表示所有在$\Omega$中定义的$k$次多项式。如果以下条件得到满足
\begin{equation*}
  f(q) = 0, \quad \forall \, q \in \mathcal{P}(\Omega),
\end{equation*}
那么我们可得
\begin{equation}
  \label{eq:bramble-hilbert-lemma}
  \big| f(\nu) \big| \le C(C_p) \, C_f \, \big| \nu \big|_{W^{k+1},2}(\Omega),
\end{equation}
其中$C(C_p)$是一个与庞加莱不等式\eqref{eq:sobolev-poincare-inequality}中系数$C_p$有关的常数。
\end{lemma}
\begin{proof}
  以$k=1$的情况为例做出证明($k >1$的情况,证明过程与之相似),对应$\mathcal{P}_{1}(\Omega)$是在$\Omega$中定义的线性方程空间。此时我们有
  \begin{equation*}
    \big| f(\nu) \big| = \big| f(\nu) + f(q) \big| = \big| f(\nu + q) \big| \le C_f \big\| \nu + q \big\|_{W^{2,2}(\Omega)},
  \end{equation*}
  其中最后一个不等式来自于庞加莱不等式\eqref{eq:sobolev-poincare-inequality}。

由索伯列夫空间的定义有
\begin{equation*}
\begin{split}
    \big\| \nu + q \big\|_{W^{2,2}(\Omega)}^2 &=
    \big\| \nu + q \big\|_{L^{2}(\Omega)}^2
    + \big| \nu + q \big|_{W^{1,2}(\Omega)}^2
    + \big| \nu + q \big|_{W^{2,2}(\Omega)}^2 \\
    &= \big\| \nu + q \big\|_{L^{2}(\Omega)}^2
    + \big\| \triangledown \left( \nu + q \right) \big\|_{L^{2}(\Omega)}^2
    + \big| \nu \big|_{W^{2,2}(\Omega)}^2,
\end{split}
\end{equation*}
其中最后一个等式,根据线性方程$q(x) \in \mathcal{P}_{1}(\Omega)$的二阶导数为0。再次使用庞加莱不等式\eqref{eq:sobolev-poincare-inequality},上式变为
\begin{equation}
  \label{eq:bramble-hilbert-middle-vq}
  \begin{split}
    \big\| \nu + q \big\|_{W^{2,2}(\Omega)}^2
    &=\big\| \nu + q \big\|_{L^{2}(\Omega)}^2
    + \big\| \triangledown \left( \nu + q \right) \big\|_{L^{2}(\Omega)}^2
    + \big| \nu \big|_{W^{2,2}(\Omega)}^2 \\
    & \le C_p \left\{
    \left[ \int_{\Omega} \left( \nu + q \right) \right]^2
    + \big\| \triangledown \left( \nu + q \right) \big\|_{L^{2}(\Omega)}^2
    \right\}
    + \big\| \triangledown \left( \nu + q \right) \big\|_{L^{2}(\Omega)}^2
    + \big| \nu \big|_{W^{2,2}(\Omega)}^2 \\
    &= C_p \left[ \int_{\Omega} \left( \nu + q \right) \right]^2
    +\left( 1+C_p \right) \underbrace{\big\| \triangledown \left( \nu + q \right) \big\|_{L^{2}(\Omega)}^2}
    +\big| \nu \big|_{W^{2,2}(\Omega)}^2,
  \end{split}
\end{equation}
对划线标记部分再次使用庞加莱不等式,有
\begin{equation*}
  \begin{split}
    \big\| \triangledown \left( \nu + q \right) \big\|_{L^{2}(\Omega)}^2
    &= \sum_{i=1}^{d} \int_{\Omega} \Big| \frac{\partial}{\partial x_i} \left[ \nu(x) + q(x) \right] \Big|^2 \, dx\\
    & \le C_p \sum_{i=1}^{d} \left\{
    \left[
    \int_{\Omega} \frac{\partial}{\partial x_i} \left[ \nu(x) + q(x) \right]  \, dx
    \right]^2
    + \sum_{j=1}^{d} \int_{\Omega}
    \left[
    \frac{
    \partial^2
    }{
    \partial x_i \, \partial x_j
    }
    \left[ \nu(x) + q(x) \right]
    \right]^2 \, dx
    \right\}\\
    &= C_p \sum_{i=1}^{d} \left[
    \int_{\Omega} \frac{
    \partial
    }{
    \partial x_i
    }
    \left[ \nu(x) + q(x) \right] \, dx
    \right]^2
    + C_p \sum_{i=1}^{d} \sum_{j=1}^{d} \int_{\Omega}
    \left[
    \frac{
    \partial^2
    }{
    \partial x_i \, \partial x_j
    }
    \left[ \nu(x) + q(x) \right]
    \right]^2 dx \\
    &= C_p \sum_{i=1}^{d} \left[
    \int_{\Omega} \frac{
    \partial
    }{
    \partial x_i
    }
    \left[ \nu(x) + q(x) \right] \, dx
    \right]^2
    + C_p \big| \nu \big|^2_{W^{2,2}(\Omega)},
  \end{split}
\end{equation*}
代回\eqref{eq:bramble-hilbert-middle-vq}变为

\begin{equation}
  \label{eq:bramble-hilbert-middle-vqq}
  \begin{split}
    \big\| \nu + q \big\|_{W^{2,2}(\Omega)}^2 \le & C_p \left[ \underbrace{
    \int_{\Omega} \left( \nu(x) + q(x) \right) dx
    }
    \right]^2
    + \left[ \left( 1 + C_p \right) C_p \right] \sum_{i=1}^{d} \left[
    \underbrace{
    \int_{\Omega} \frac{
    \partial
    }{
    \partial x_i
    }
    \left[ \nu(x) + q(x) \right] \, dx
    }
    \right]^2 \\
    &+ \left[ 1 + \left( 1 + C_p \right) C_p \right]
    \big| \nu \big|^2_{W^{2,2}(\Omega)}.
  \end{split}
\end{equation}

给定$u(x) = \nu(x) + q(x) $,Bramble-Hilbert引理\eqref{eq:sobolev-poincare-inequality}若要成立,需要在\eqref{eq:bramble-hilbert-middle-vqq}中,通过选择$q(x) \in \mathcal{P}_{1}(\Omega)$的值,使得满足下两个条件
\begin{subequations}
  \begin{equation}
    \label{eq:bramble-hilbert-middle-q0}
    \int_{\Omega} \left[ \nu(x) + q(x) \right] dx=0,
  \end{equation}
  \begin{equation}
    \label{eq:bramble-hilbert-middle-q1}
    \int_{\Omega} \frac{
    \partial
    }{
    \partial x_i
    }
    \left[ \nu(x) + q(x) \right] \, dx=0.
  \end{equation}
\end{subequations}

这涉及到系数$a_0$和$a_i, i=1,\ldots,d$的值
\begin{equation*}
  q(x) = a_0 + \sum_{i=1}^{d} a_i x_i.
\end{equation*}

\eqref{eq:bramble-hilbert-middle-q1} $\Rightarrow$
\begin{equation*}
  \begin{split}
    &\int_{\Omega} \frac{\partial}{\partial x_i} \nu(x) \, dx
    + \int_{\Omega} \frac{\partial}{\partial x_i} q(x) \, dx
    = 0, \\
    \hookrightarrow &
    \int_{\Omega} \frac{\partial}{\partial x_i} \nu(x) \, dx
    + \left| \Omega \right| a_i = 0, \\
    \hookrightarrow &
    a_i = - \frac{1}{\left| \Omega \right|} \int_{\Omega} \frac{\partial}{\partial x_i} \nu(x) \, dx, \quad i = 1,\ldots,d.
  \end{split}
\end{equation*}

\eqref{eq:bramble-hilbert-middle-q1}$\Rightarrow$
\begin{equation*}
  \begin{split}
    &\int_{\Omega} \left[\nu(x)  +  a_0 + \sum_{i=1}^{d} a_i x_i \right] \, dx = 0, \\
    \hookrightarrow &
    + \left| \Omega \right| a_0 +  \int_{\Omega} \left[\nu(x)  +  \sum_{i=1}^{d} a_i x_i \right] \, dx = 0, \\
    \hookrightarrow & a_0 = - \frac{1}{\left| \Omega \right|} \int_{\Omega} \left[\nu(x)  +  \sum_{i=1}^{d} a_i x_i \right] \, dx.
  \end{split}
\end{equation*}

对于$k \in \mathbb{N}$的情况,证明过程同上。
\end{proof}

%!TEX root = ../DSGEnotes.tex

\subsubsection{索伯列夫空间的分布}
\label{sec:sobolev-space-distributions}
由第\ref{sec:generalized-integration}节的介绍可见,并不是$L^{loc}(\Omega)$中的所有方程都有广义偏导数。然而可以从分布的角度来重新解读广义偏导数\citep{McLean:2000ta}。

\begin{definition}[分布]
  \label{definition:sobolev-distribution-def}
  对于$\Omega \subseteq \mathbb{R}^d$,定义一个试探空间(test space)\index{test space \dotfill 试探空间} $\mathcal{D}(\Omega) \coloneqq C_0^{\infty}(\Omega)$,空间中的一个复值连续线性泛函$T \in \mathcal{D}(\Omega)$可以称为一个分布(distribution)\index{distribution \dotfill 分布}。$T$在$\mathcal{D}(\Omega)$中连续,是指对于$x \in \Omega$中的任何$\varphi_{k}(x) \rightarrow \varphi(x)$,
  总有$\mathcal{D}'(\Omega)$中的$T(\varphi_k) \rightarrow T(\varphi)$;$\mathcal{D}'(\Omega) \subset \mathcal{D}(\Omega)$表示$\mathcal{D}(\Omega)$空间中,所有分布的集合。

  对于$x \in \Omega, u(x) \in L^{1,loc}(\Omega)$,分布常常定义为
  \begin{equation}
    \label{eq:sobolev-distribution-regular-def}
    T_u(\varphi) \coloneqq \int_{\Omega} u(x) \varphi(x) \, dx, \quad \varphi \in \mathcal{\phi}(\Omega).
  \end{equation}

  符合\eqref{eq:sobolev-distribution-regular-def}形式的分布,又称正则分布(regular distribution)\index{distribution!regular \dotfill 正则分布}。

  正则分布中,局部可积方程$u(x) \in L^{1,loc}(\Omega)$由子集空间$\mathcal{D}'(\Omega)$予以识别。因此,$T_u(\varphi) \in \mathcal{D}'(\Omega)$有时也简写为$u \in \mathcal{D}'(\Omega)$。

  不符合正则类型的分布,称为奇异分布(singular distribution)\index{distribution!singular \dotfill 奇异分布}。一种奇异分布的例子是狄拉克分布(Dirac distribution)\index{Dirac distribution \dotfill 狄拉克分布}:
  \begin{equation*}
    \label{eq:sobolev-distribution-dirac-distribution-def}
    \delta_{x_0}(\varphi) = \varphi(x_0), \quad x_0 \in \Omega, \, \varphi \in \mathcal{D}(\Omega).
  \end{equation*}
\end{definition}

现在我们回到式\eqref{eq:sobolev-distribution-sgnx},来看如何计算$u(x)$的广义偏导数$\nu(x) \coloneqq \frac{\partial u(x)}{\partial x} = \sgn(x)$。根据分部积分法有
\begin{equation*}
  \int_{-1}^{1} \nu(x) \frac{\partial
  }{\partial x} \varphi(x) \, dx = -2 \varphi(0) = - \int_{-1}^{1} \frac{\partial}{\partial x}\nu(x) \varphi(x) \, dx, \quad \forall \varphi(x) \in \mathcal{D}(\Omega),
\end{equation*}
可见分布意义上的$\nu(x)$的广义偏导数变为
\begin{equation*}
  \frac{\partial}{\partial x}\nu(x) = 2 \varphi(0) = \delta_{0}(\varphi) \in \mathcal{D}'(\Omega),
\end{equation*}
更高阶的广义偏导数为
\begin{equation*}
  \left( D^{\alpha} T_u  \right)(\varphi) = (-1)^{\left| \alpha \right|} T_u \left( D^{\alpha} \varphi \right), \quad \varphi \in \mathcal{D}(\Omega).
\end{equation*}

在掌握了分布的基本概念之后,下面介绍在满足一些关于$\Omega$的正则条件假设的情况下,$H^{s}(\Omega)$空间近似等价于$W^{s,2}(\Omega)$空间,其中$H^{s}(\Omega)$的基是分布的傅里叶变换。我们从急减方程(rapidly decreasing functions)空间$\mathcal{S} \left( \mathbb{R}^d \right)$开始,进而介绍傅里叶变换。

\begin{definition}[急减方程空间]
  我们将满足以下形式的方程$\varphi(x) \in C^{\infty}(\mathbb{R}^d)$称为急减方程(rapidly decreasing function)\index{rapidly decreasing function \dotfill 急减方程}:
  \begin{equation*}
    \big\|  \varphi \big\|_{k,\ell} \coloneqq
    \sup_{x \in \mathbb{R}^d} \left( \left| x  \right|^k  \right)
    \sum_{\left| \alpha \right| \le \ell } \big| D^{\alpha} \varphi(x) \big| < \infty, \quad k,\ell \in \mathbb{N}_0,
  \end{equation*}
  即是说,急减方程$\varphi(x)$和它的导数,比任意多项式的减速更快。

  急减方程组成的空间,称为急减空间(rapidly decreasing space)\index{rapidly decreasing space \dotfill 急减空间},表示为$\mathcal{S}(\mathbb{R}^b)$。

  类似于$\mathcal{D}'(\Omega) \subset \mathcal{D}(\Omega)$,我们定义缓增分布空间(tempered distribution space)\index{tempered distribution space \dotfill 缓增分布空间} $\mathcal{S}'(\mathbb{R}^d) \subset \mathcal{S}(\mathbb{R}^d)$,作为全部复值线性泛函$T \in \mathcal{S}(\mathbb{R}^{d})$组成的子集。
\end{definition}

举例来说,对于方程$\varphi(x) \coloneqq \exp(- \left| x \right|^2)$,我们有
\begin{equation*}
    \varphi \in \mathcal{S}(\mathbb{R}^d), \quad \varphi \notin \mathcal{D}(\Omega) = C_0^{\infty}(\mathbb{R}^d).
\end{equation*}

\begin{definition}[傅里叶变换]
  对于方程$\varphi(x) \in \mathcal{S}(\mathbb{R}^d)$,我们将傅里叶变换(Fourier transform)\index{Fourier transform! \dotfill 傅里叶变换} $\widehat{\varphi}(x) \in \mathcal{S}(\mathbb{R}^d)$ 定义如下
  \begin{equation}
    \label{eq:sobo-distri-fourier-transform-def}
    \widehat{\varphi}(\xi) \coloneqq \left( \mathcal{F} \varphi \right)(\xi) =
    \left( 2 \pi \right)^{-\frac{d}{2}} \int_{\mathbb{R}^d}
    \exp \left[ - i \, \langle x, \xi \rangle \right] \,
    \varphi(x) \, dx, \quad \xi \in \mathbb{R}^d,
  \end{equation}
  其中$\mathcal{F}$是可逆映射$\mathcal{F}:\mathcal{S}(\mathbb{R}^d) \mapsto \mathcal{S}(\mathbb{R}^d)$。

  对应地,逆傅里叶变换(inverse Fourier transform)\index{Fourier transform!inverse \dotfill 逆傅里叶变换}为
  \begin{equation}
    \label{eq:sobo-distri-inverse-fourier-transform-def}
    \left( \mathcal{F}^{-1} \widehat{\varphi} \right)(x) =
    \left( 2 \pi \right)^{-\frac{d}{2}} \int_{\mathbb{R}^d}
    \exp \left[ i \, \langle x, \xi \rangle \right] \,
    \widehat{\varphi}(\xi) \, dx, \quad x \in \mathbb{R}^d.
  \end{equation}

通常来说,$\varphi \in \mathcal{D}(\mathbb{R}^d)$并不必然意味着$\widehat{\varphi} \in \mathcal{D}(\mathbb{R}^d)$。

此外对于$\varphi \in \mathcal{S}(\mathbb{R}^d)$,我们进一步有
\begin{subequations}
  \begin{equation}
  \label{eq:sobolev-distribution-fourier-higher-order-f}
  D^{\alpha} \left( \mathcal{F} \varphi \right)(\xi) = (-i)^{\left| \alpha \right|} \mathcal{F}\left( x^{\alpha} \varphi \right) (\xi),
\end{equation}
\begin{equation}
  \label{eq:sobolev-distribution-fourier-higher-order-xi}
  \xi^{\alpha} \left( \mathcal{F} \varphi \right)(\xi) = (-i)^{\left| \alpha \right|} \mathcal{F} \left( D^{\alpha} \varphi \right) (\xi).
\end{equation}
\end{subequations}
\end{definition}

\begin{lemma}[傅里叶变换的旋转对称]
  傅里叶变换保持旋转对称结构(rotational symmetries)\index{Fourier transform!rotational symmetries \dotfill 傅里叶变换的旋转对称},即对于$u \in \mathbb{R}^d$,我们有
  \begin{equation*}
    \widehat{u}(\xi) = \widehat{u}(\left| \xi \right|), \forall \, \xi \in \mathbb{R}^{d}, \quad \text{iff. } \, u(x) = u(\left| x \right|), \forall \, x \in \mathbb{R}^d.
  \end{equation*}
\end{lemma}
\begin{proof}
  先来看$d=2$的证明,对应极坐标系(polar coordinates)\index{polar coordinate system \dotfill 极坐标系}
  \begin{equation*}
    \xi = \begin{pmatrix}
    \left| \xi \right| \, \cos \psi \\
    \left| \xi \right| \, \sin \psi
    \end{pmatrix},
    \quad
    x = \begin{pmatrix}
    r \, \cos \phi \\
    r \, \sin \phi
    \end{pmatrix}.
  \end{equation*}

根据\eqref{eq:sobo-distri-fourier-transform-def}我们有
\begin{equation}
  \label{eq:sobo-fourier-transform-symmetry-2d-0}
\begin{split}
  \widehat{u} \left( \xi \right) &= \widehat{u}\left( \left| \xi \right|, \psi \right) \\
  &=\frac{1}{2\pi} \int_{0}^{\infty} \int_{0}^{2 \pi}
  \exp \left\{
  - i \, r \, \left| \xi \right| \,
  \left[
  \cos \phi \, \cos \psi + \sin \phi \, \sin \psi
  \right]
  \right\} \,
  u(r) \, r \, d \phi \, d r \\
  &=\frac{1}{2\pi} \int_{0}^{\infty} \underbrace{\int_{0}^{2 \pi}
  \exp \left\{
  - i \, r \, \left| \xi \right| \,
  \left[
  \cos \left( \phi - \psi \right)
  \right]
  \right\} \,
  u(r) \, r \, d \phi}_{\coloneqq \mathcal{A}} \, d r.
\end{split}
\end{equation}

现在将坐标旋转$\psi_0 \in [0, 2 \pi]$度,变为$\psi + \psi_0$,定义$\widetilde{\phi} \coloneqq \phi - \psi_0$,  \eqref{eq:sobo-fourier-transform-symmetry-2d-0}变为
\begin{equation}
  \label{eq:sobo-fourier-transform-symmetry-2d-1}
  \begin{split}
    \widehat{u}\left(\left| \xi \right|, \psi + \psi_0 \right) &=
    \frac{1}{2\pi} \int_{0}^{\infty} \int_{0}^{2 \pi}
    \exp \left\{
    - i \, r \, \left| \xi \right| \,
    \left[
    \cos \left( \phi - \psi_0 - \psi \right)
    \right]
    \right\} \,
    u(r) \, r \, d \phi \, d r \\
    &= \frac{1}{2\pi} \int_{0}^{\infty} \underbrace{\int_{-\psi_0}^{2 \pi - \psi_0}
    \exp \left\{
    - i \, r \, \left| \xi \right| \,
    \left[
    \cos \left( \widetilde{\phi} - \psi \right)
    \right]
    \right\} \,
    u(r) \, r \, d \widetilde{\phi} }_{\coloneqq \mathcal{B}}\, d r
  \end{split}
\end{equation}

比较上两式
\begin{equation}
\label{eq:sobo-fourier-transform-symmetry-2d-2}
\begin{split}
  & \int_{-\psi_0}^{0} \exp
  \left[
  - i \, r \, \left| \xi \right| \, \cos \left( \widetilde{\phi} - \psi \right)
  \right] \, d \widetilde{\phi}
  = \int_{2 \pi - \psi_0}^{2 \pi} \exp
  \left[
  - i \, r \, \left| \xi \right| \, \cos \left( \widetilde{\phi} - \psi \right)
  \right] \, d \widetilde{\phi}, \\
  \hookrightarrow & \mathcal{A} = \mathcal{B},\\
  \hookrightarrow & \widehat{u}\left( \left| \xi \right|, \psi \right)
  = \widehat{u} \left( \left| \xi \right|, \psi+\psi_0 \right), \quad \forall \psi_0 \in [0, 2\pi], \\
  \hookrightarrow  & \widehat{u}(\xi) = \widehat{u} \left( \left| \xi \right| \right).
\end{split}
\end{equation}

再来看$d=3$的证明,对应三维坐标系(spherical coordinates)\index{spherical coordinate system \dotfill 三维坐标系}
\begin{equation*}
  \xi = \begin{pmatrix}
  \left| \xi \right| \, \cos psi \, \sin \vartheta\\
  \left| \xi \right| \, \sin psi \, \sin \vartheta\\
  \left| \xi \right| \, \cos \vartheta
  \end{pmatrix}, \quad
  x = \begin{pmatrix}
  r \, \cos \phi \, \sin \theta \\
  r \, \sin \phi \, \sin \theta \\
  r \, \cos \theta
  \end{pmatrix}.
\end{equation*}

根据\eqref{eq:sobo-distri-fourier-transform-def}我们有
\begin{equation*}
\begin{split}
  \widehat{u} \left( \left| \xi \right|, \psi, \vartheta \right)
  = \frac{1}{\left( 2 \pi \right)^{\frac{3}{2}}}
  \int_{0}^{\infty} \int_{0}^{2 \pi} \int_{0}^{\pi}
  & \exp \left\{
  - i \, r \, \left| \xi \right| \,
  \left[
  \cos (\phi - \psi) \, \sin \theta \, \sin \vartheta
  + \cos \theta \, \cos \vartheta
  \right]
  \right\} \, \\
  & u(r) \, r^2 \, \sin \theta \, d \theta \, d \phi \, dr.
\end{split}
\end{equation*}

由$d=2$时证得的\eqref{eq:sobo-fourier-transform-symmetry-2d-2},我们有
\begin{equation*}
  \widehat{u} \left( \left| \xi \right|, \psi + \psi_0, \vartheta \right) = \widehat{u} \left( \left| \xi \right|, \psi , \vartheta \right), \quad \forall \, \psi_0 \in [0,2 \pi).
\end{equation*}

给定半径$\varrho$和一个$\vartheta \in [0,\pi)$,我们有$\widehat{u}(\xi) = \widehat{u} \left( \left| \xi \right| \right) = \widehat{u}(\varrho)$
\begin{equation*}
  \begin{split}
  &\xi_1^2 + \xi_2^2 = \varrho^2 \sin^2 \vartheta,\\
  &\xi_3 = \varrho \cos \vartheta.
  \end{split}
\end{equation*}

对原三维坐标作变形,我们有$\widehat{u}(\xi) = \widehat{u}(\varrho)$
\begin{equation*}
  \begin{split}
    & \xi_1^2 + \xi_3^2 = \varrho^2 \sin^2 \vartheta, \\
    & \xi_2 = \varrho \cos \vartheta.
  \end{split}
\end{equation*}
\end{proof}

\begin{definition}[分布的傅里叶变换]
  \label{definition:distribution-fourier-transform}
  对于某一个缓增分布空间中的分布$T \in \mathcal{S}'(\mathbb{R}^d)$,其傅里叶变换$\widehat{T} \in \mathcal{S}'(\mathbb{R}^d)$可定义如下
  \begin{equation*}
    \widehat{T}(\varphi) \coloneqq T(\widehat{\varphi}), \quad \varphi \in \mathcal{S}(\mathbb{R}^d).
  \end{equation*}

  从$\mathcal{F}:\mathcal{S}'(\mathbb{R}^d) \mapsto \mathcal{S}(\mathbb{R}^d)$是一个可逆映射,逆傅里叶变换定义为
  \begin{equation*}
    \left( \mathcal{F}^{-1} T \right) (\varphi) \coloneqq T \left(\mathcal{F}^{-1} \varphi \right), \quad \varphi \in \mathcal{S}(\mathbb{R}^d).
  \end{equation*}

  分布的傅里叶变换,也满足\eqref{eq:sobolev-distribution-fourier-higher-order-f}-\eqref{eq:sobolev-distribution-fourier-higher-order-xi}。
\end{definition}

\begin{definition}[贝塞尔位势空间]
  \label{definition:bessel-potential-space}
对于$s \in \mathbb{R}, u \in \mathcal{S}(\mathbb{R}^d)$,定义一个有界的线性贝塞尔位势算子(Bessel potential operator)\index{Bessel potential operator \dotfill 贝塞尔位势算子} $\mathcal{J}: \mathcal{S}(\mathbb{R}^d) \mapsto \mathcal{S}(\mathbb{R}^d)$
\begin{equation}
  \label{eq:bessel-potential-operator-def}
  \mathcal{J}^s u(x) \coloneqq \left( 2 \pi \right)^{- \frac{d}{2}}
  \int_{\mathbb{R}^d}
  \left( 1 + \left| \xi \right|^2 \right)^{\frac{s}{2}}
  \widehat{u}(\xi)
  \exp \left[ i \, \langle x, \xi \rangle \right] \, d \xi, \quad x \in \mathbb{R}^d.
\end{equation}
\end{definition}
由傅里叶变换可得
\begin{equation*}
  \left( \mathcal{F} \mathcal{J}^s u \right)(\xi) = \underbrace{\left(
  1 + \left| \xi \right|^2
  \right)^{\frac{s}{2}}}
  \left( \mathcal{F} u \right)(\xi),
\end{equation*}
可见在傅里叶空间中,$\mathcal{J}^s$执行的操作近似于傅里叶乘子(Fourier multiplier),即对原式乘以一个与$\left| \xi \right|^s$有关的方程$\mathcal{O} \left( \left| \xi \right|^s \right)$。从这个意义上来讲,类似于\eqref{eq:sobolev-distribution-fourier-higher-order-xi},我们可以将贝塞尔算子$\mathcal{J}^s$ 近似看做$s$阶微分符号。

\begin{definition}[分布意义上的索伯列夫空间]
  \label{definition:sobo-distribution-def}
  在缓增分布空间(tempered distribution space) $\mathcal{S}'(\mathbb{R}^d)$中,给定分布$T \in \mathcal{S}'(\mathbb{R}^d)$。定义有界线性算子$\mathcal{J}^s: \mathcal{S}'(\mathbb{R}^d) \mapsto \mathcal{S}'(\mathbb{R}^d)$
  \begin{equation*}
  \left( \mathcal{J}^s T \right)(\varphi) \coloneqq T \left( \mathcal{J}^s \varphi \right), \quad \varphi \in \mathcal{S}(\mathbb{R}^d).
  \end{equation*}

  进而,分布意义上的索伯列夫空间$H^s(\mathbb{R}^d)$可以表示为,由全部分布$\nu \in \mathcal{S}'(\mathbb{R}^d)$所组成的内积空间,其中$\nu$满足$ \mathcal{J}^s \nu \in L^2(\mathbb{R}^d)$,内积形式表示为
    \begin{equation*}
    \langle u,\nu \rangle_{H^{s}(\mathbb{R}^d)} \coloneqq \langle \mathcal{J}^s u, \mathcal{J}^s \nu \rangle_{L^2(\mathbb{R}^d)},
  \end{equation*}
  对应范数
  \begin{equation*}
    \big\| u \big\|_{H^{s}(\mathbb{R}^d)}^2 \coloneqq \big\| \mathcal{J}^s \, u \big\|_{H^{s}(\mathbb{R}^d)}^2
    = \int_{\mathbb{R}^d} \left( 1 + \left| \xi \right|^s \right)^{s} \,
    \left| \widehat{u}(\xi) \right|^2 \, d \xi.
  \end{equation*}
\end{definition}

\begin{theorem}[索伯列夫空间]
  对于所有$s \in \mathbb{R}$,下式均成立
  \begin{equation*}
    H^{s}(\mathbb{R}^d) = W^{s,2}(\mathbb{R}^d).
  \end{equation*}

  对于有界域$\Omega \in \mathbb{R}^d$,我们将索伯列夫空间$H^s(\mathbb{R}^d)$定义如下形式
  \begin{equation*}
    \begin{split}
      & H^s(\Omega) \coloneqq \left\{ \nu = \widetilde{\nu} |_{\Omega} : \widetilde{\nu} \in H^{s}(\mathbb{R}^d) \right\}, \\
       & \big\| \nu \big\|_{H^s(\Omega)} \coloneqq \inf_{\widetilde{\nu} \in H^{s}(\mathbb{R}^d),
       \widetilde{\nu} |_{\Omega} = \nu}  \big\| \widetilde{\nu} \big\|_{H^s(\mathbb{R}^d)}.
    \end{split}
  \end{equation*}

    此外还可以对索伯列夫空间作如下定义
    \begin{equation*}
      \begin{split}
        &\widetilde{H}^s(\Omega) := \overline{C_0^{\infty}(\Omega)}^{\| \cdot \|_{H^s(\mathbb{R}^d)}}, \\
        & H_0^s(\Omega) = \overline{C_0^{\infty}(\Omega)}^{\| \cdot \|_{H^s(\Omega)}},
      \end{split}
    \end{equation*}
    根据\cite[Theorem 3.33]{McLean:2000ta},上述定义对于几乎全部$s \in \mathbb{R}_{+}$都成立。
\end{theorem}

\begin{theorem}
  假定有一个利普希茨域(Definition \ref{definition:bvp-lipschitz-domain-def}) $\Omega \in \mathbb{R}^d$。对于$s \ge 0$我们有
  \begin{equation*}
    \widetilde{H}^{s}(\Omega) \in H_0^s(\Omega).
  \end{equation*}

  进一步有
  \begin{equation*}
    \widetilde{H}^{s}(\Omega) = H_0^s(\Omega), \quad s \notin \left\{ \frac{1}{2}, \frac{3}{2}, \frac{5}{2}, \ldots \right\}.
  \end{equation*}

  进一步有
  \begin{equation*}
    \widetilde{H}^{s}(\Omega) = \left[ H^{-s} (\Omega) \right]',
    H^s(\Omega) = \left[ \widetilde{H}^{-s}(\Omega) \right]', \quad \forall s \in \mathbb{R}.
  \end{equation*}
\end{theorem}

\begin{definition}[均匀法锥条件]
\label{definition:uniform-cone-condition}
两个索伯列夫空间$H^s(\Omega)$和$W^{s,2}(\Omega)$相等,即范式等价的充分条件是存在一个有界的线性延拓算子
\begin{equation*}
  E_{\Omega} : W^{2,s}(\Omega) \mapsto W^{2,s}(\mathbb{R}^d),
\end{equation*}
这要求存在一个有边界的域$\Omega \in \mathbb{R}^d$,这称为均匀法锥条件(uniform cone condition)\index{uniform cone condition \dotfill 均匀法锥条件}。
\end{definition}
\begin{proof}
  略。见\cite[Theorem 4.6, 4.7]{Adams:2003wi}。
\end{proof}

\begin{theorem}
  对于有界的利普希茨域$\Omega \subset \mathbb{R}^d$,我们有
  \begin{equation*}
    \begin{split}
      &\big\| q \big\|_{L^2(\Omega)} \le C_1
      \left\{
      \big\| q \big\|_{H^{-1}(\Omega)} + \big\| \triangledown q \big\|_{\left[ H^{-1}(\Omega) \right]^d }
      \right\}, \quad \forall q \in L^{2}(\Omega), \\
      & \big\| q \big\|_{L^2(\Omega)} \le C_2
      \big\| \triangledown q \big\|_{\left[ H^{-1}(\Omega)\right]^d}, \quad \forall q \in L^{2}(\Omega), \, \int_{\Omega} q(x) \, d x = 0.
    \end{split}
  \end{equation*}
\end{theorem}

我们可以用有界的线性算子表现从一个索伯列夫空间到另一个索伯列夫空间的映射,这涉及到插值空间。关于插值空间的介绍可见如\cite{Adams:2003wi}。这里介绍与有限元法关系较为密切的插值定理(interpolation theorem)\index{interpolation theorem \dotfill 插值定理}。
\begin{theorem}[插值定理]
  \label{theorem:sobolev-interpolation-theorem}
  设一个有界的线性算子$A:H^{\alpha_1}(\Omega) \rightarrow H^{\beta}(\Omega)$,其范数为
  \begin{equation*}
    \big\| A \big\|_{\alpha_1,\beta} \coloneqq \sup_{0 \neq \nu \in H^{\alpha_1}(\Omega)} \frac{
    \big\| A \, \nu \big\|_{H^{\beta} (\Omega)}
    }{
    \big\| \nu \big\|_{H^{\alpha_1}(\Omega)}
    }.
  \end{equation*}

现在假设$\alpha_2 > \alpha_1$,对应有界线性算子$A:H^{\alpha_2}(\Omega) \rightarrow H^{\beta}(\Omega)$,范数
\begin{equation*}
  \big\| A \big\|_{\alpha_2,\beta} \coloneqq \sup_{0 \neq \nu \in H^{\alpha_2}(\Omega)} \frac{
  \big\| A \, \nu \big\|_{H^{\beta} (\Omega)}
  }{
  \big\| \nu \big\|_{H^{\alpha_2}(\Omega)}
  }.
\end{equation*}

则$A:H^{\alpha}(\Omega) \mapsto H^{\beta}(\beta), \, \forall \alpha \in \left[ \alpha_1, \alpha_2 \right]$都有界,对应范数
\begin{equation*}
  \big\| A \big\|_{\alpha, \beta} \le
  \left(
  \big\| A \big\|_{\alpha_1,\beta}
  \right)^{\frac{\alpha - \alpha_2}{\alpha_1 - \alpha_2}} \,
  \left(
  \big\| A \big\|_{\alpha_2,\beta}
  \right)^{\frac{\alpha - \alpha_1}{\alpha_2 - \alpha_1}}.
\end{equation*}

同样,设$A:H^{\alpha} \mapsto H^{\beta_1(\Omega)}$有界,对应范数$\big\| A \big\|_{\alpha, \beta_1}$。设$A:H^{\alpha} \mapsto H^{\beta_2(\Omega)}$有界,对应范数$\big\| A \big\|_{\alpha, \beta_2}$,$\beta_1 < \beta_2$。

则$A:H^{\alpha} \mapsto H^{\beta(\Omega)}$有界$\, \forall \beta \in \left[ \beta_1, \beta_2 \right]$,对应范数
\begin{equation*}
  \big\| A \big\|_{\alpha, \beta} \le
  \left(
  \big\| A \big\|_{\alpha,\beta_1}
  \right)^{\frac{\beta - \beta_2}{\beta_1 - \beta_2}} \,
  \left(
  \big\| A \big\|_{\alpha,\beta_2}
  \right)^{\frac{\beta - \beta_1}{\beta_2 - \alpha_1}}.
\end{equation*}
\end{theorem}


\subsubsection{流形索伯列夫空间}
\label{sec:sobolev-manifold-space}
\begin{definition}[流形]
  流形(manifold)是局部具有欧式空间性质的空间,包括各种纬度的曲线曲面,例如球体、弯曲的平面等。流形的局部和欧式空间是同构的(局部线性)。流形学习假设所处理的数据点分布在嵌入于外维欧式空间的一个潜在的流形体上,或者说这些数据点可以构成这样一个潜在的流形体\footnote{从词源上来看,最早由黎曼给出了德文名称mannigfaltigkeit,英文翻译为manifold,字面意义为``多层''。中国第一个拓扑学家江泽涵把这个词翻译为``流形'',取自文天祥《正气歌》,``天地有正气,杂然赋流形'',其原始出处为《易经》,``大哉乾元,万物资始,乃统天。云行雨施,品物流形。'' }。
\end{definition}

设一个有边界的域$\Omega \subset \mathbb{R}^d,d=2,3$,其边界用$\Gamma = \partial \Omega$来表示。对$\Gamma$作局部分段参数化(piecewise parameterization)处理,对应一组低维空间$\mathbb{R}^{d-1}$中的参数域(parameter domain) $\mathcal{T}_i$:
\begin{equation}
  \label{sobolev-manifold-piecewise-parameterization}
\begin{split}
    &\Gamma = \bigcup_{i=1}^{J} \Gamma_i,\\
    & \Gamma_i \coloneqq \left\{
    x \in \mathbb{R}^d : x = \chi_{i}(\xi), \, \xi \in \mathcal{T}_i \subset \mathbb{R}^{d-1}
    \right\}.
\end{split}
\end{equation}

与\eqref{sobolev-manifold-piecewise-parameterization}同步,考虑一个单位数列$\left\{ \varphi_i \right\}_{i=1}^{p}$,其中$\varphi_i \in C_0^{\infty}(\mathbb{R}^d)$是非负的cutoff方程,满足
\begin{equation}
  \label{eq:sobolev-manifold-cutoff-func-def}
  \begin{cases}
    \sum_{i=1}^{J} \varphi_{i}(x)=1 & \text{对于} \, x \in \Gamma \\
    \varphi_i(x) = 0  & \text{对于} \, x \in \Gamma \backslash \Gamma_i.
  \end{cases}
\end{equation}

进而,在$\Gamma$中定义一个方程$\nu(x)$,满足
\begin{equation}
  \label{eq:sobolev-manifold-nu-x-def}
  \begin{split}
    &\nu(x) = \sum_{i=1}^{J} \varphi_i(x) \nu(x) = \sum_{i=1}^{J} \nu_i(x), \quad \text{对于} \, x \in \Gamma, \\
    & \nu_i(x) \coloneqq \varphi_i(x) \, \nu(x).
  \end{split}
\end{equation}

将关于$x$的局部分段参数化\eqref{sobolev-manifold-piecewise-parameterization}代入$\nu_i(x)$的定义式,我们可以定义一组新的方程$\left\{ \widetilde{\nu}_i(\xi) \right\}$,其中
\begin{equation*}
  \begin{split}
    \nu_i(x) &= \varphi_i(x) \, \nu(x) \\
    &= \varphi_i \left( \chi_i(\xi) \right) \, \nu \left( \chi_i(\xi) \right)  \\
    & \eqqcolon \widetilde{\nu}_i(\xi), \quad \xi \in \mathcal{T}_i \subset \mathbb{R}^{d-1}.
  \end{split}
\end{equation*}

通过这种方法,$\Omega \subset \mathbb{R}^{d}$中关于变量$x$的问题$\nu(x)$,被表示为分段参数域$\mathcal{T}_i \subset \mathbb{R}^{d-1}$中关于局部参数$\chi_i(\xi)$的问题$\widetilde{\nu}_i(\xi)$,后者使得我们可能建立相应的索伯列夫空间,进行分析。这要求$\tilde{\nu}_i$可导,即局部参数$\chi_i(\xi)$可导,满足链式法则(chain rule)\index{chain rule \dotfill 链式法则}\footnote{简单说来,链式法则可以表示为
\begin{equation*}
  f(g(x))' = f'(g(x)) \cdot g'(x).
\end{equation*}}。如果要求$\left| s \right| \le k$次的导数,需要假定局部参数$\chi_i \in C^{k-1,1}(\mathcal{T}_i)$,例如,对于利普希茨域$\mathcal{T}_i$中的局部参数$\chi_i \in C^{0,1}(\mathcal{T}_i)$,我们需要研究对应的索伯列夫空间$H^s(\mathcal{T}_i), \left| s \right| \le 1$。

\begin{definition}[流形索伯列夫空间($0 \le s \le k $)]
进而对于$0 \le s \le k$,我们的研究对象是整个索伯列夫空间$H^s(\Gamma)$,其范数定义为
\begin{equation}
  \label{eq:sobolev-manifold-space-def}
  \big\| \nu \big\|_{H_{\chi}^{s}(\Gamma)} \coloneqq \left\{
  \sum_{i=1}^{J} \big\| \widetilde{\nu}_i \big\|_{H^{s}(\mathcal{T}_{i})}^{2}
  \right\}^{\frac{1}{2}}.
\end{equation}
\end{definition}

\begin{lemma}[流形索伯列夫空间的等价范($s=0$)]
$s=0$时,流形索伯列夫空间$H_{\chi}^{0}(\Gamma)$的等价范为
\begin{equation*}
  \| \nu \|_{L^{2}(\Gamma)} \coloneqq \left\{
  \int_{\Gamma} \big| \nu(x) \big|^2 \, d s_x
  \right\} ^{\frac{1}{2}}.
\end{equation*}
\end{lemma}

\begin{proof}
  首先来看$H_{\chi}^{0}(\Gamma)$空间。$s=0$时\eqref{eq:sobolev-manifold-space-def}$\Rightarrow$
  \begin{equation*}
    \begin{split}
      \big\| \nu \big\|_{H_{\chi}^{0}(\Gamma)}^2 &=
      \sum_{i=1}^{J} \big\| \widetilde{\nu}_i \big\|_{H^{0}(\mathcal{T}_{i})}^{2} \\
      &= \sum_{i=1}^{J}
      \int_{\Gamma_i} \left[
      \varphi_i(\chi_i(\xi)) \, \nu(\chi_i(\xi))
      \right]^2 \, d\xi.
    \end{split}
  \end{equation*}

  再来看$L^{2}(\Gamma)$空间$\Rightarrow$
  \begin{equation*}
    \begin{split}
      \big\| \nu \big\|_{L^{2}(\Gamma)}^2
      &= \int_{\Gamma}
      \left[ \nu(x) \right]^2 d s_x \\
      &= \sum_{i=1}^{J} \int_{\Gamma_i} \varphi_i(x) \,
      \left[ \nu(x) \right]^2 d s_x \\
      &= \sum_{i=1}^{J} \int_{\Gamma_i} \varphi_i(\chi_i(\xi)) \,
      \left[ \nu(\chi_i(\xi)) \right]^2 \, \left( \det \chi_i(\xi) \right) \, d \xi.
    \end{split}
  \end{equation*}
\end{proof}

\begin{lemma}[流形索伯列夫空间的等价范($0 < s < 1$)]
  $0 < s < 1$时,流形索伯列夫空间$H_{\chi}^{s}(\Gamma)$的等价范,可以用Sobolev-Slobodeckij范数 (Section \ref{sec:sobolev-slobodeckij-space})\index{Sobolev-Slobodeckij norm \dotfill Sobolev-Slobodeckij范数}来表示
  \begin{equation*}
    \left\| \nu \right\|_{H^s(\Gamma)} \coloneqq
    \left\{
    \big\| \nu \big\|_{L^{2}(\Gamma)}^2
    + \int_{\Gamma} \int_{\Gamma}
    \frac{
    \left[ \nu(x) - \nu(y) \right]^2
    }{
    \left| x - y \right|^{d - 1 + 2s}
    }
    \, d s_x \, d s_y
    \right\}^{\frac{1}{2}},
  \end{equation*}
  需要指出的是,上式并不是唯一一种定义等价范的方法。其他方法如,根据索伯列夫范数等价定理(Theorem \ref{theorem:sobolev-equivalence-norm-theorem} \index{equivalence of norms \dotfill 范数等价}),流形索伯列夫空间$H^{s=1/2}(\Gamma)$的等价范也可以写作
  \begin{equation*}
    \left\| \nu \right\|_{H^{\frac{1}{2}}(\Gamma), \Gamma} \coloneqq \left\{
    \left[
    \int_{\Gamma} \nu(x) \, d s_x
    \right]^2
    + \int_{\Gamma} \int_{\Gamma}
    \frac{
    \left[ \nu(x) - \nu(y) \right]^2
    }{
    \left| x - y \right| ^{d}
    }\,
    d s_x \, d s_y
    \right\}^{\frac{1}{2}}.
  \end{equation*}
\end{lemma}

\begin{lemma}[流形索伯列夫空间的等价范($s < 0$)]
$s<0$时的流形索伯列夫空间$H_{\chi}^{s}(\Gamma)$,可由其对偶空间(dual space)\index{dual space \dotfill 对偶空间} $H^{-s}_{\chi}(\Gamma)$进行分析
\begin{equation*}
  H^s(\Gamma) \coloneqq \left[ H^{-s}(\Gamma) \right]',
\end{equation*}
其范数为
\begin{equation}
  \label{eq:sobolev-manifold-dual-space-norm}
\begin{split}
  \big\| w \big\|_{H^{s}(\Gamma)} &\coloneqq \sup_{0 \neq \nu \in H^{-s} (\Gamma)}
  \frac{
  \langle w,\nu \rangle_{\Gamma}
  }{
  \big\| \nu \big\|_{H^{-s}(\Gamma)}
  }\\
  &= \sup_{0 \neq \nu \in H^{-s} (\Gamma)}
  \frac{
  \int_{\Gamma} w(x) \, \nu(x) \, d s_x
  }{
  \big\| \nu \big\|_{H^{-s}(\Gamma)}
  },
\end{split}
\end{equation}
其中$\langle w,\nu \rangle_{\Gamma}$表示对偶配对(duality pairing)\index{duality pairing \dotfill 对偶配对}。
\end{lemma}

\begin{lemma}[流形索伯列夫空间的开放子集]
  对于充分平滑的边界$\Gamma = \partial \Omega$,设一个开放子集$\Gamma_0 \subset \Gamma$。则$H^s(\Gamma_0)$可以定义如下
\begin{enumerate}

  \item $s\ge0$时的索伯列夫空间$H^{s}(\Gamma_0)$及范数
  \begin{equation*}
    \begin{split}
      &H^{s}(\Gamma_0) \coloneqq \left\{ \nu = \widetilde{\nu} \big|_{\Gamma_0} : \widetilde{\nu} \in H^{s}(\Gamma) \right\},\\
      &\big\| \nu \big\|_{H^s(\Gamma_0)} \coloneqq \inf_{\widetilde{\nu} \in H^{s}(\Gamma) : \widetilde{\nu} |_{\Gamma_0} = \nu}  \big\| \widetilde{\nu} \big\|_{H^s(\Gamma)}
    \end{split}
  \end{equation*}

\item $s<0$时,首先定义一个索伯列夫空间$\tilde{H}^{s}(\Gamma_0)$
\begin{equation*}
  \widetilde{H}^{s}(\Gamma_0) \coloneqq \left\{
  \nu = \widetilde{\nu} \big|_{\Gamma_0}: \widetilde{\nu} \in H^s(\Gamma), \, \supp \widetilde{\nu} \subset \Gamma_0
  \right\},
\end{equation*}
其中$\supp$表示支撑集(support),是一个定义在集合$\Gamma_0$上的实值函数$\widetilde{\nu}$的子集,满足$\widetilde{\nu}$恰好在这个子集上非零\footnote{例如一个拓扑空间(如实值轴)$X$,连续方程$f \in X$。此时$\supp f$定义为这样一个闭集$C$,满足1)$f$在$X \backslash C$中为$0$;2)不存在$C$的真闭子集也满足这一条件,即$C$是所有这样的子集中最小的一个。}。进而利用对偶空间的属性
\begin{equation*}
\begin{split}
    &H^s(\Gamma_0) \coloneqq \left[ \widetilde{H}^{-s} (\Gamma_0) \right]', \\
    &\widetilde{H}^s(\Gamma_0) \coloneqq \left[ H^{-s} (\Gamma_0) \right].
\end{split}
\end{equation*}
\end{enumerate}
\end{lemma}

\begin{lemma}[分段平滑索伯列夫空间]
  对于某一封闭边界$\Gamma = \partial \Omega$,假定它是分段平滑的
  \begin{equation*}
    \Gamma = \bigcup_{i=1}^{J} \overline{\Gamma}_i, \quad \Gamma_i \cap \Gamma_j = \emptyset \,  \forall i \neq j.
  \end{equation*}
  对应的索伯列夫空间$H^s_{\text{pw}}(\Gamma)$及分段平滑方程的范数为:

  \begin{enumerate}
    \item $s>0 \Rightarrow$
    \begin{equation*}
      \begin{split}
        & H^{s}_{\text{pw}}(\Gamma) \coloneqq \left\{ \nu \in L^{2}(\Gamma) : \nu |_{\Gamma_i} \in H^{s}(\Gamma_i), \, i=1,\ldots,J \right\}, \\
        & \big\| \nu \big\|_{H^{s}_{\text{pw}}(\Gamma)} \coloneqq
        \left\{
        \sum_{i=1}^{J} \big\| \nu|_{\Gamma_i} \big\|_{H^s(\Gamma_i)}^2
        \right\}^{\frac{1}{2}}.
      \end{split}
    \end{equation*}

    \item $s < 0 \Rightarrow$
    \begin{align}
      \label{eq:sobolev-piecewise-snegative-def}
      &H^{s}_{\text{pw}}(\Gamma) \coloneqq \Pi_{j=1}^{j} \widetilde{H}^{s}_{\Gamma_j}, \\
      \label{eq:sobolev-piecewise-snegative-norm}
      & \big\| w \big\|_{H^{s}_{\text{pw}(\Gamma)}} \coloneqq
      \sum_{j=1}^{J} \big\| w|_{\Gamma_j} \big\|_{\widetilde{H}_{\text{pw}}(\Gamma_j)}.
    \end{align}
  \end{enumerate}
\end{lemma}

\begin{lemma}
  \eqref{eq:sobolev-piecewise-snegative-def}的分段平滑索伯列夫空间$H^{s}_{\text{pw}}(\Gamma), \, s<0$中,方程$w \in H^{s<0}_{\text{pw}}(\Gamma)$满足
  \begin{equation*}
    \big\| w \big\|_{H^P{s}(\Gamma)} \le \big\| w \big\|_{H^P_{\text{pw}}{s}(\Gamma)}
  \end{equation*}
\end{lemma}
\begin{proof}
  由对偶空间的定义\eqref{eq:sobolev-manifold-dual-space-norm}、扩展三角不等式等有
  \begin{equation*}
    \begin{split}
      \big\| w \big\|_{H^s(\Gamma)} &=
      \sup_{0 \neq \nu \in H^{-s} (\Gamma)}
      \frac{
      \big| \langle w,\nu\rangle_{\Gamma} \big|
      }{
      \big\| \nu \big\|_{H^{-s}(\Gamma)}
      } \\
      & \le \sup_{0 \neq \nu \in H^{-s} (\Gamma)}
      \sum_{j=1}^{J}
      \frac{
      \big| \langle w,\nu\rangle_{\Gamma_j} \big|
      }{
      \big\| \nu \big\|_{H^{-s}(\Gamma_j)}
      } \\
      &\le  \sup_{0 \neq \nu \in H^{-s} (\Gamma)}
      \sum_{j=1}^{J}
      \frac{
      \big| \langle w|_{\Gamma_j},\nu|_{\Gamma_j} \rangle_{\Gamma_j} \big|
      }{
      \big\| \nu|_{\Gamma_j} \big\|_{H^{-s}(\Gamma_j)}
      } \\
      &\le  \sum_{j=1}^{J}
      \sup_{0 \neq \nu \in H^{-s} (\Gamma_j)}
      \frac{
      \big| \langle w|_{\Gamma_j},\nu|_{\Gamma_j} \rangle_{\Gamma_j} \big|
      }{
      \big\| \nu|_{\Gamma_j} \big\|_{H^{-s}(\Gamma_j)}
      } \\
      &= \big\| w \big\|_{H^{s}_{pw}(\Gamma)}.
    \end{split}
  \end{equation*}
\end{proof}

对于一个利普希茨域$\Omega \subset \mathbb{R}^d$,其利普希茨边界可表示为$\Gamma = \partial \Omega$。对于$u(x) \in \Omega$,对应的内界迹(interior boundary trace)\index{interior boundary trace \dotfill 内界迹}为$\gamma_0^{int} u \in \Gamma$,见\eqref{eq:bvp-interior-boundary-trace}。则$H^s(\Omega)$和$H^s(\Gamma)$两个索伯列夫空间的关系,常用下述迹定理、逆迹定理予以描述。
\begin{theorem}[索伯列夫空间的迹定理]
  \label{theorem:sobolev-manifold-trace-theorem}
  设$\Omega \subset \mathbb{R}^d$是一个$C^{k-1,1}$域,$\frac{1}{2} < s \le k$,则内界迹
  \begin{equation*}
    \gamma_0^{\text{int}} : H^s(\Omega) \mapsto H^{s - \frac{1}{2}}(\Gamma)
  \end{equation*}
  是有界的,满足
  \begin{equation*}
    \big\| \gamma_0^{\text{int}} \nu \big\|_{H^{s-\frac{1}{2}}(\Gamma)} \le C_T \, \big\| \nu \big\|_{H^{s}(\Omega)}, \quad \forall \, \nu \in H^{s}(\Omega).
  \end{equation*}
\end{theorem}

\begin{lemma}
  若$k=1, s \in ( \frac{1}{2}, 1]$,则我们可以利用Theorem \ref{theorem:sobolev-manifold-trace-theorem}求得迹算子$\gamma_0^{\text{int}} : H^s(\Omega) \mapsto H^{s - \frac{1}{2}}(\Gamma)$。

  若$s \in (\frac{1}{2}, \frac{3}{2})$,该方法也适用\citep[Theorem 3.38]{McLean:2000ta}。
\end{lemma}

\begin{theorem}[索伯列夫空间的逆迹定理]
  \label{theorem:sobolev-manifold-inverse-trace-theorem}
  设$\Omega \subset \mathbb{R}^d$是一个$C^{k-1,1}$域,$\frac{1}{2} < s \le k$,则内界迹
  \begin{equation*}
    \gamma_0^{\text{int}} : H^s(\Omega) \mapsto H^{s - \frac{1}{2}}(\Gamma)
  \end{equation*}
  有一个连续的右逆算子
  \begin{equation*}
    \mathcal{\varepsilon}: H^{s-\frac{1}{2}}(\Gamma) \mapsto H^{s}(\Gamma),
  \end{equation*}
满足如下关系
\begin{equation*}
  \begin{split}
    &\gamma_0^{\text{int}} \mathbb{\varepsilon} w = w, \quad \forall \, w \in H^{s-\frac{1}{2}}(\Gamma), \\
    &\big\| \mathbb{\varepsilon} w \big\|_{H^{s}(\Omega)} \le C_{IT} \, \big\| w \big\|_{H^{s-\frac{1}{2}}(\Gamma)}, \quad \forall \, w \in H^{s-\frac{1}{2}}(\Gamma).
  \end{split}
\end{equation*}
\end{theorem}

\begin{definition}[索伯列夫迹空间]
从这个意义上来说,对于某个索伯列夫空间$H^{s+\frac{1}{2}}(\Omega), s > 0$来说,它的迹空间可以表示为$H^{s}(\Gamma)$,对应的范数
\begin{equation*}
  \big\| \nu \big\|_{H^s(\Gamma), \gamma_0} \coloneqq
  \inf_{V \in H^{s + \frac{1}{2}(\Omega)}, \gamma_0^{\text{int}} \, V = \nu} \, \big\| V \big\|_{H^{s + \frac{1}{2}}(\Omega)}.
\end{equation*}
\end{definition}

需要指出的是,利普希茨域$\Omega \subset \mathbb{R}^d$中,只有当$|s| <1$时,以下范式等价$\| \nu \|_{H^{s}(\gamma), \gamma_0} = \| \nu \|_{H^{s}(\Gamma)}$。


\section{变分法}
\label{sec:variational-methods}
弱形式边界值问题常常表现为带有算子方程(operator equations)的变分问题。对于变分问题,我们常将其表示为表面积位势和体积位势(surface and volume potentials)的偏微分方程,为了求解方程,就需要求得有界的积分算子方程的解,以建立完备的柯西数列解。本节介绍一些泛函分析的基本知识,进而探讨算子方程解的存在性和唯一性。

\subsection{算子方程}
\label{sec:variational-operator-equations}

假设一个希尔伯特空间$X$,空间中的内积形式$\langle .,.\rangle_{X}$,对应范数$\| \cdot \|_{X} = \sqrt{\langle .,.\rangle_{X}}$。$X$的对偶空间(dual space)表示为$X'$;$X$和$X'$以$\langle .,. \rangle$的形式呈对偶配对(duality pairing)。则我们有
\begin{equation}
  \label{eq:var-operator-norm}
  \big\| f \big\|_{X'} = \sup_{0 \neq \nu \in X}
  \frac{
  \big| \langle f,\nu \rangle \big|
  }{\| \nu \|_{X}}, \quad \forall \, f \in X'.
\end{equation}

定义一个有界的自伴随(self-adjoint)线性算子$A:X \mapsto X'$,满足
\begin{equation}
  \label{eq:var-operator-equation}
  \big\| A \nu \big\|_{X'} \le C_2^A \, \| \nu \|_{X}, \quad \forall \, \nu \in X,
\end{equation}
$A$是自伴随的,是指
\begin{equation}
  \label{eq:var-operator-self-adjoint}
  \langle A u, \nu \rangle = \langle  u, A \nu \rangle, \quad \forall \, u, \nu \in X.
\end{equation}

那么,边界值问题可以表示为,对于某个给定的$f \in X'$,寻找算子方程的解$u \in X$,使满足
\begin{equation}
  \label{eq:var-operator-eq-problem}
  A u = f.
\end{equation}

算子方程\eqref{eq:var-operator-eq-problem}的解,通常难于直接求得。替代方案是建立一个变分问题,寻找变分问题的解$u \in X$,使满足
\begin{equation}
  \label{eq:var-operator-var-problem}
  \langle Au,\nu \rangle = \langle f, \nu \rangle, \quad \forall \, \nu \in X.
\end{equation}

\begin{theorem}
  \label{theorem:var-equivalance-solution-operator-var}
  算子方程\eqref{eq:var-operator-eq-problem}的解$u \in X$,和变分问题\eqref{eq:var-operator-var-problem}的解$u \in X$,二者等价。
\end{theorem}
\begin{proof}
  证明过程分为两个部分。
  \begin{enumerate}
    \item 显然,算子方程\eqref{eq:var-operator-eq-problem}的解$u \in X$,构成变分问题\eqref{eq:var-operator-var-problem}的解。

    \item 反过来,假定已求得变分问题\eqref{eq:var-operator-var-problem}的解$u \in X$,由\eqref{eq:var-operator-norm}可得
\begin{equation*}
  \big\| A u - f \big\| _{X'} = \sup_{0 \neq \nu \in X}
  \frac{
  \big| \langle Au - f, \nu \rangle \big|
  }{
  \big\| \nu \big\|_{X}
  },
\end{equation*}
代回\eqref{eq:var-operator-var-problem}可得
\begin{equation*}
  \big\| A u - f \big\| _{X'} = 0 \Rightarrow Au = f,
\end{equation*}
即是说$u \in X$同时也是算子方程\eqref{eq:var-operator-eq-problem}的解。
\end{enumerate}
\end{proof}

基于算子$A$,可定义一个双线性映射(bilinear form) $a(u,\nu)$
\begin{equation}
\label{var-bilinear-form}
  \begin{split}
      & a(u,\nu) \coloneqq \langle Au, \nu \rangle, \quad \forall \, u,\nu \in X, \\
      & A:X \mapsto X' \Rightarrow a(.,.):X \times X \mapsto \mathbb{R},
  \end{split}
\end{equation}
反之亦然,通过双线性映射$a(u,\nu)$可以定义算子$A:X \mapsto X'$,见Lemma \ref{lemma:var-bilinear-form-to-A}。

\begin{lemma}
  \label{lemma:var-bilinear-form-to-A}
  设一个有界的双线性映射$a(.,.): X \times X \mapsto \mathbb{R}$,满足
  \begin{equation*}
    \big| a(u,\nu) \big| \le C_2^A \, \big\| u \big\|_{X} \, \big\| \nu \big\|_{X}, \quad \forall \, u,\nu \in X.
  \end{equation*}

    对于其中任一$u \in X$,都存在一个元素$A u \in X'$,使得满足
    \begin{equation*}
      \langle Au, \nu \rangle = a(u,\nu), \forall \, \nu \in X.
    \end{equation*}

    则我们有:算子$A:X \mapsto X'$是一个有界的线性算子,满足
    \begin{equation*}
      \big\| A u \big\|_{X'} \le C_2^A \, \big\| u \big\|_{X}, \forall \, u \in X.
    \end{equation*}
\end{lemma}
\begin{proof}
  对于任一给定的$u \in X$,我们定义一个$X$中的有界双线性映射$\langle f_u, \nu \rangle \coloneqq a(u,\nu)$,即我们有$f_u \in X'$。通过映射$u \in X \mapsto f_u \in X'$,我们定义一个线性算子$A: X \mapsto X'$,使得$A u = f_u \in X'$,并且满足
  \begin{equation*}
    \big\| Au \big\|_{X'} = \big\| f_u \big\|_{X'}
    = \sup_{ 0 \neq \nu \in X}
    \frac{
      \big| \langle f_u,\nu \rangle \big|
    }{
    \big\| \nu \big\|_{X}
    }
    = \sup_{ 0 \neq \nu \in X}
    \frac{
      \big| a(u,\nu) \big|
    }{
    \big\| \nu \big\|_{X}
    }
    \le C_2^A \, \big\| u \big\|_{X}.
  \end{equation*}
\end{proof}

因此我们可以把求解变分问题\eqref{eq:var-operator-var-problem}的工作,转化为求解下述最小化问题的工作。
\begin{lemma}
  \label{lemma:var-minimization-problem}
  设线性算子$A:X \mapsto X'$是1)自伴随的,即$\langle A u ,\nu \rangle = \langle u, A \nu \rangle$,和2)正半定的,即$\langle A u, \nu \rangle \ge 0$,$\forall \, \nu \in X$。我们设一个泛函$F$
  \begin{equation}
    \label{eq:var-mini-functional-F}
    F(\nu) \coloneqq \frac{1}{2} \langle A\nu, \nu \rangle - \langle f, \nu \rangle, \quad \forall \, \nu \in X.
  \end{equation}

那么变分问题\eqref{eq:var-operator-var-problem}的解$u \in X$,等价于如下最小化问题的解
\begin{equation}
  \label{eq:var-operator-min-problem}
  F(u) = \min_{\nu \in X} F(\nu).
\end{equation}
\end{lemma}

\begin{proof}
  设$u,\nu \in X$,任一$t \in \mathbb{R}$。进而
  \begin{equation*}
    \begin{split}
      F(u+t \nu)&= \frac{1}{2} \langle A(u+t\nu), u + t \nu \rangle - \langle f, u + t \nu \rangle \\
      &= \frac{1}{2}
      \left[
      \langle A u, u \rangle + \langle A u, t \nu \rangle +
      \langle A \nu, u \rangle + \langle A t \nu, t \nu \rangle
      \right]
      - \left[
      \langle f, u \rangle + \langle f, t \nu \rangle
      \right] \\
      &= \frac{1}{2} \langle A u , u \rangle
      + \frac{1}{2} t \langle A u, \nu \rangle
      + \frac{1}{2} t \langle A u, \nu \rangle
      + \frac{1}{2} t^2 \langle A \nu, \nu \rangle
      - \langle f, u \rangle
      - t \langle f , \nu \rangle \\
      & = \left[
      \frac{1}{2} \langle Au, u \rangle - \langle f,u \rangle
      \right]
      + t \left[ \langle Au,\nu \rangle - \langle f, \nu \rangle \right]
      + \frac{1}{2} t^2 \langle  A \nu, \nu \rangle \\
      &= F(u) + t \left[ \langle Au,\nu \rangle - \langle f, \nu \rangle \right]
      + \frac{1}{2} t^2 \langle  A \nu, \nu \rangle.
    \end{split}
  \end{equation*}

\begin{enumerate}
\item 假设$u \in X$是变分问题\eqref{eq:var-operator-var-problem}的解,那么$\langle Au,\nu \rangle = \langle f, \nu \rangle$,上式变为
\begin{equation*}
  \begin{split}
    &F(u+t \nu) = F(u) +  \underbrace{\frac{1}{2} t^2 \langle  A \nu, \nu \rangle}_{\ge 0}, \\
    \hookrightarrow & F(u) \le F(u + t \nu), \quad \forall \, \nu \in X, t \in \mathbb{R},
  \end{split}
\end{equation*}
由此可见$u \in X$同时也是最小化问题\eqref{eq:var-operator-min-problem}的解。

\item 现在假设$u \in X$是最小化问题\eqref{eq:var-operator-min-problem}的解。那么以下条件也成立
\begin{equation*}
  \frac{d}{d t}F(u + t \nu)\big|_{t=0} = 0, \quad \forall \, \nu \in X,
\end{equation*}
由此可得
\begin{equation*}
  \langle A u , \nu \rangle = \langle f, \nu \rangle, \quad \forall \nu \in X,
\end{equation*}
可见$u \in X$同时也是变分问题\eqref{eq:var-operator-var-problem}的解。
\end{enumerate}
\end{proof}

现在来证明变分问题\eqref{eq:var-operator-var-problem}、最小化问题\eqref{eq:var-operator-min-problem}的解$u \in X$存在且唯一,可由里兹表现定理证明。

\begin{theorem}[里兹表现定理]
  \label{theorem:var-riesz-representation-theorem}
  任一线性有界泛函$f \in X'$均可表现为下述形式
  \begin{equation*}
    \langle f, \nu \rangle = \langle u,\nu \rangle_{X},
  \end{equation*}
  其中$u \in X$由$f \in X'$所唯一确定(uniquely determined),并且满足
  \begin{equation}
    \label{eq:var-riesz-representation-theorem}
    \big\| u \big\|_{X} = \big\| f \big\|_{X'},
  \end{equation}
  这称为里兹表现定理(Riesz representation theorem)\index{Riesz representation theorem \dotfill 里兹表现定理}。
\end{theorem}

\begin{proof}
  设某一给定的泛函$f \in X'$,我们可以通过求解变分问题\eqref{eq:var-operator-var-problem}找到解$u \in X$
  \begin{equation*}
    \langle Au,\nu \rangle = \langle f, \nu \rangle, \quad \forall \, \nu \in X,
  \end{equation*}
并且根据Lemma \ref{lemma:var-minimization-problem},这也等同于求解最小化问题\eqref{eq:var-operator-min-problem}
\begin{equation*}
  F(u) = \min_{\nu \in X} F(\nu),
\end{equation*}
其中泛函$F$由\eqref{eq:var-mini-functional-F}给出
\begin{equation*}
  F(\nu) \coloneqq \frac{1}{2} \langle A\nu, \nu \rangle - \langle f, \nu \rangle, \quad \forall \, \nu \in X.
\end{equation*}

\begin{enumerate}

\item 证明$u \in X$是最小化问题和变分问题的解。\eqref{eq:var-mini-functional-F} $\Rightarrow$
\begin{equation*}
  \begin{split}
    F(\nu) &=  \frac{1}{2} \langle  \nu, \nu \rangle_{X} - \langle f,\nu\rangle\\
    &\ge \frac{1}{2} \big\| \nu \big\|_{X}^2 - \big\| f \big\|_{X'} \, \big\| \nu \big\|_{X} \\
    &= \frac{1}{2} \underbrace{
    \left[
    \big\| \nu \big\|_{X} - \big\| f \big\|_{X'}
    \right]^2}_{\ge 0} - \frac{1}{2} \big\| f \big\|_{X'}^2 \\
    & \ge - \frac{1}{2} \big\| f \big\|_{X'}, \quad \forall \, \nu \in X,
  \end{split}
\end{equation*}
可见$F(\nu), \forall \, \nu \in X$有下界(infimum),定义为$\alpha$
\begin{equation*}
  \alpha \coloneqq \inf_{\nu \in X} F(\nu) \in \mathbb{R}.
\end{equation*}

设存在一个数列$\left\{ u_k \right\}_{k \in \mathbb{N}} \subset X$,随着$k \rightarrow \infty$,满足$F(u_k) \rightarrow \alpha$。由数列的性质可得
\begin{equation*}
  \big\| u_k - u_\ell \big\|_{X}^2 + \big\| u_k + u_\ell \big\|_{X}^2= 2 \left\{
  \big\| u_k \big\|_{X}^{2} + \big\| u_\ell \big\|_{X}^{2}
  \right\},
\end{equation*}
进而我们有$\big\| u_k - u_\ell \big\|_{X}^2 \ge 0$,以及
\begin{equation*}
  \begin{split}
    &\big\| u_k - u_\ell \big\|_{X}^2 = 2   \big\| u_k  \big\|_{X}^2   + 2   \big\| u_\ell \big\|_{X}^2  -   \big\| u_k + u_\ell \big\|_{X}^2 \\
    &= 4 \underbrace{\left\{
    \frac{1}{2} \big\| u_k \big\|_{X}^2
    - \langle f, u_k \rangle
    \right\}}_{ = F(u_k)}
    + 4 \underbrace{\left\{
    \frac{1}{2} \big\| u_\ell \big\|_{X}^2
    - \langle f, u_{\ell} \rangle
    \right\}}_{ = F(u_{\ell})}
    - 8 \left[
    \frac{1}{2} \big\| \frac{1}{2} \left( u_k + u_{\ell} \right) \big\|_{X}^2
    - \langle f, \frac{1}{2} \left( u_k + u_{\ell} \right) \rangle
    \right]
    \\
    &=4 F(u_k) + 4 F(u_{\ell})
        - 8 F \left( \frac{1}{2} \left( u_k + u_{\ell} \right) \right)\\
    & \le 4 \alpha + 4 \alpha - 8 \alpha \rightarrow 0, \quad k,\ell \rightarrow \infty.
  \end{split}
\end{equation*}

由此可见,$\left\{ u_k \right\}_{k \in \mathbb{N}}$是一个柯西数列(Cauchy sequence)。此外由于$X$是一个希尔伯特空间,我们得到极限值
\begin{equation*}
  u = \lim_{k \rightarrow \infty} u_k \in X.
\end{equation*}

对应得泛函$F(u)$的值,求解过程如下。由
\begin{equation*}
  \begin{split}
    \left| F(u_k) - F(u) \right| & \le \big| F(u_k - u) \big| \\
    & \le
    \frac{1}{2} \big| \langle u_k, u_k \rangle_{X} - \langle u, u \rangle_{X} \big|
    + \big| \langle f, u_k - u \rangle  \big| \\
    & = \frac{1}{2}
    \big|
    \langle u_k, u_k - u \rangle_{X} + \langle u, u_k - u \rangle_{X}
    \big|
    + \big|
    \langle f, u_k - u \rangle
    \big| \\
    &\le \left\{
    \frac{1}{2} \big\| u_k \big\|_{X}
    + \frac{1}{2} \big\| u \big\|_{X}
    + \big\| f \big\|_{X'}
    \right\} \, \big\| u_k - u \big\|_{X}
  \end{split}
\end{equation*}
可得
\begin{equation*}
  F(u) = \lim_{k \rightarrow \infty} F(u_k) = \alpha.
\end{equation*}

可见$u \in X$是最小化问题\eqref{eq:var-mini-functional-F}和变分问题\eqref{eq:var-operator-var-problem}的解。

\item 证明$u \in X$是最小化问题和变分问题的唯一解。假定还存在另一个$\tilde{u} \in X$,也是\eqref{eq:var-mini-functional-F}和\eqref{eq:var-operator-var-problem}的解,那么我们有
\begin{equation*}
  \langle \widetilde{u}, \nu \rangle_{X} = \langle f, \nu \rangle, \quad \forall \, \nu \in X.
\end{equation*}

将上式代回变分问题\eqref{eq:var-operator-var-problem}$\Rightarrow$
\begin{equation*}
  \langle u - \widetilde{u}, \nu \rangle_{X} = 0, \quad \forall \, \nu \in X.
\end{equation*}

若设$\nu = u - \tilde{u} \Rightarrow $
\begin{equation*}
  \big\| u - \widetilde{u} \big\|_{X}^2 = 0,
\end{equation*}

因此我们有$u = \widetilde{u}$,即$u \in X$是最小化问题和变分问题的唯一解。

\item 证明范式等价$\big\| u \big\|_{X} = \big\| f \big\|_{X'}$。已知
\begin{equation*}
  \begin{split}
    &\big\| u \big\|_{X}^2 = \langle u,u \rangle_X = \langle  f, u \rangle \le \big\| u \big\|_{X} \, \big\| f \big\|_{X'}, \\
    &\hookrightarrow \big\| u \big\|_{X} \le \big\| f \big\|_{X'},
  \end{split}
\end{equation*}
\begin{equation*}
  \begin{split}
    &\big\| f \big\|_{X'} = \sup_{0 \neq \nu \in X} \frac{
    \big| \langle f, \nu \rangle \big|
    }{
    \big\| \nu \big\|_{X}
    }
    = \sup_{0 \neq \nu \in X}
    \frac{
    \big| \langle u, \nu \rangle _{X} \big|
    }{
    \big\| \nu \big\|_{X}
    }
    \le \big\| u \big\|_{X},
  \end{split}
\end{equation*}

则我们有$\big\| u \big\|_{X} \le  \big\| f \big\|_{X'} \& \big\| f \big\|_{X'} \le \big\| u \big\|_{X} \Rightarrow \big\| u \big\|_{X} =  \big\| f \big\|_{X'}.$
\end{enumerate}
\end{proof}

\begin{definition}[里兹映射]
  \label{definition:var-riesz-map-def}
  若里兹表现定理(Theorem \ref{theorem:var-riesz-representation-theorem})成立,那么我们将映射$J:X' \mapsto X, u = J f$称为里兹映射(Riesz map)\index{Riesz map \dotfill 里兹映射},满足如下变分问题
  \begin{equation}
    \label{eq:var-riesz-map-def}
    \langle Jf, \nu \rangle_{X} = \langle f, \nu \rangle, \quad \forall \, \nu \in X,
  \end{equation}
  并且其范数为
  \begin{equation}
    \label{eq:var-riesz-map-norm}
    \big\| J f \big\|_{X} = \big\| f \big\|_{X'}.
  \end{equation}
\end{definition}

%!TEX root = ../DSGEnotes.tex
\subsection{椭圆算子}
\label{sec:var-elliptic-operators}
里兹表现定理(Theorem \ref{theorem:var-riesz-representation-theorem})探讨了算子方程\eqref{eq:var-operator-eq-problem}及变分问题\eqref{eq:var-operator-var-problem}的解$u \in X$的存在性以及唯一性。除此以外,为了确保解得唯一存在,我们还需要对算子$A$和双线性形式$a(.,.)$做出进一步设定。

\begin{definition}[椭圆算子]
  \label{definition:var-elliptic-operator-def}
一个算子$A:X \mapsto X'$被称作$X$-椭圆算子,如果它满足
\begin{equation}
  \label{eq:var-elliptic-operator-def}
  \langle A \nu, \nu \rangle \ge C_1^A \, \big\|\nu\big\|_{X}^2, \quad \forall \nu \in X,
\end{equation}
其中$0 \le C_1^A \in \mathbb{R}$。
\end{definition}

\begin{theorem}[拉克斯一密格拉蒙定理]
  \label{theorem:lax-milgram-lemma}
  \index{Lax-Milgram theorem \dotfill 拉克斯一密格拉蒙定理}设$A:X \mapsto X'$是一个有界的$X$-椭圆算子。对于任一$f \in X'$,算子方程\eqref{eq:var-operator-eq-problem}都存在一个唯一解$u \in X$,满足
  \begin{equation}
    \label{eq:lax-milgram-sulution-u}
    \| u \|_{X} \le \frac{1}{C_1^A} \, \| f \|_{X'}.
  \end{equation}
\end{theorem}
\begin{proof}
  设存在一个里兹映射(Riesz map)算子$J:X' \mapsto X$,满足\eqref{eq:var-riesz-map-def}定义。那么算子方程\eqref{eq:var-operator-eq-problem}等价于下属定点方程
  \begin{equation*}
    u = u - \varrho J \left( A u - f \right) = T_{\varrho} u + \varrho J f,
  \end{equation*}
其中算子$T_{\varrho} \coloneqq I - \varrho J A : X \mapsto X$,参数$0 < \varrho \in \mathbb{R}$。对应范数
\begin{equation}
  \label{eq:var-lax-milgram-operator-tvarrho}
  \begin{split}
    \big\| T_{\varrho} u \big\|_{X}^2 &= \big\| (I = \varrho J A ) u \big\|_{X}^2 \\
    &= \big\| u \big\|_{X}^2 - 2 \varrho \underbrace{\langle JAu, u \rangle_{X}}_{\eqqcolon \mathcal{A}} + \varrho^2 \underbrace{\big\| JAu\big\|_X^2}_{\eqqcolon \mathcal{B}} \\
    & \le \left[1 - 2 \varrho C_1^A + \varrho^2 \left( C_2^A \right)^2 \right] \, \big\| u \big\|_{X}^2,
  \end{split}
\end{equation}
其中,由里兹映射算子$J$的性质\eqref{eq:var-riesz-map-def}和$X$-椭圆算子$A$的性质\eqref{eq:var-elliptic-operator-def}我们有
\begin{equation*}
  \begin{split}
    \mathcal{A} \coloneqq \langle JAu, u \rangle_{X} = \langle A u, u \rangle \ge C_1^A \, \| u \|_{X}^2,
  \end{split}
\end{equation*}
由里兹映射算子$J$的范数\eqref{eq:var-riesz-map-norm}和$X$-椭圆算子$A$的范数\eqref{eq:var-operator-equation}我们有
\begin{equation*}
  \begin{split}
    \mathcal{B} \coloneqq \big\| J A u \big\|_{X} = \big\| A u \big\|_{X'} \le C_2^A \, \| u \|_{X}.
  \end{split}
\end{equation*}

若设$\varrho \in \left(0, \frac{2 C_1^A}{\left( C_2^A \right)^2} \right)$,则算子$T_{\varrho}$是一个$X$中的收缩映射(contraction mapping)\index{contraction mapping \dotfill 收缩映射},并且由收缩映射定理(Banach's contraction mapping theorem, \cite{Palais:2007bo})可得,算子方程\eqref{eq:var-operator-eq-problem}的解$x \in X$是唯一的。进而,对于唯一的解$u \in X$,根据椭圆算子$A$的定义\eqref{eq:var-elliptic-operator-def}和里兹表现定理我们有
\begin{equation*}
  \begin{split}
    C_1^A \, \big\| u \big\|_{X}^2 &\le \langle Au,u \rangle \\
    &=\langle f, u \rangle \\
    &\le \big\| f \big|_{X'} \, \big\| u \big\|_{X}.
  \end{split}
\end{equation*}
\end{proof}

根据拉克斯一密格拉蒙定理(Theorem \ref{theorem:lax-milgram-lemma}),我们可以定义一个逆算子$A^{-1}:X' \mapsto X$,有
\begin{equation*}
  \big\| A^{-1} f \big\|_{X} \le \frac{1}{C_1^A} \, \big\| f \big\|_{X'}, \quad \forall \, f \in X'.
\end{equation*}
\begin{lemma}[$X$-椭圆算子$A$的逆算子$A^{-1}$也是一个椭圆算子]
设$A:X \mapsto X'$是一个有界\eqref{eq:var-operator-equation},自伴随\eqref{eq:var-operator-self-adjoint}的$X$-椭圆算子\eqref{eq:lax-milgram-sulution-u}。那么对于$\forall \nu \in X$我们有
\begin{equation*}
  \langle A^{-1} f, f \rangle \ge \frac{1}{C_2^A} \, \big\| f \big\|_{X'}^2, \quad \forall \, f \in X'.
\end{equation*}
\end{lemma}

\begin{proof}
定义一个算子$B \coloneqq J A : X \mapsto X$,满足
\begin{equation*}
  \big\| B \nu \big\|_{X} = \big\| J A \nu \big\|_{X} = \big\| A \nu \big\|_{X'} \le C_2^A \, \big\| \nu \big\|_X, \quad \forall \, \nu \in X.
\end{equation*}

由于$\forall \, u,\nu \in X$,都有以下关系成立
\begin{equation*}
  \langle B u, \nu \rangle_{X} = \langle J A u, \nu \rangle = \langle A u, \nu \rangle = \langle u, A \nu \rangle = \langle u, J A \nu \rangle_{X} = \langle u, B \nu \rangle_{X},
\end{equation*}
可见$B$是一个自伴随的椭圆算子,满足
\begin{equation*}
  \langle B \nu, \nu \rangle_X = \langle A \nu, \nu \rangle \ge C_1^A \, \big\| \nu \big\|_{X}^2, \quad \forall \, \nu \in X.
\end{equation*}

因此可以定义一个可逆的自伴随算子$B^{\frac{1}{2}}$,满足$B = B^{\frac{1}{2}} \, B^{\frac{1}{2}}$,逆算子$B^{- \frac{1}{2}} \coloneqq \left( B^{\frac{1}{2}} \right)^{-1}$。对应的范数
\begin{equation*}
\begin{split}
  &\big\| B^{\frac{1}{2}} \nu \big\|_{X}^2 = \langle B \nu, \nu \rangle_{X} \le \big\| B \nu \big\|_{X} \, \big\| \nu \big\|_X \le C_2^A \, \big\| \nu \big\|_X^{2}, \\
  \hookrightarrow & \big\| B^{\frac{1}{2}} \nu \big\|_{X} \le \sqrt{C_2^A} \, \big\| \nu \big\|_X, \quad \forall \, \nu \in X.
\end{split}
\end{equation*}

因此,对于任一$f \in X'$,我们有
\begin{equation*}
  \begin{split}
    \| f \|_{X'} &= \sup_{0 \neq \nu \in X} \frac{
    \big| \langle f,\nu \rangle \big|
    }{
    \big\| \nu \big\|_{X}
    }\\
    &= \sup_{0 \neq \nu \in X}
    \frac{
    \big| \langle J f, \nu \rangle_X \big|
    }{
    \big\| \nu \big\|_X
    } \\
    &= \sup_{0 \neq \nu \in X}
    \frac{
    \big| \langle B^{-\frac{1}{2}} J f, B^{\frac{1}{2}} \nu \rangle_X \big|
    }{
    \big\| \nu \big\|_X
    } \\
    &\le \sup_{0 \neq \nu \in X}
    \frac{
    \big\| B^{-\frac{1}{2}} J f \big\|_{X} \, \big\| B^{\frac{1}{2}} \nu \big\|_{X}
    }{
    \big\| \nu \big\|_X
    } \\
    &\le \sqrt{C_2^A} \, \big\| B^{-\frac{1}{2}} J f \big\|_{X},
  \end{split}
\end{equation*}
进而
\begin{equation*}
  \big\| f \big\|_{X'}^2 \le C_2^A \big\| B^{-\frac{1}{2}} J f \big\|_{X}^2 = C_2^A \langle B^{-1} J f, J f \rangle_X = C_2^A \langle A^{-1} f, f \rangle,
\end{equation*}
其中我们使用到了如下关系
\begin{equation*}
  \begin{split}
    \big\| B^{-\frac{1}{2}} J f \big\|_{X}^2 &= \big\| \left( B^{\frac{1}{2}} \right)^{-1} J f \big\|_{X}^2 \\
    &= \langle B^{-1} J f, J f \rangle_{X} \\
    &= \langle A^{-1} f , f \rangle.
  \end{split}
\end{equation*}
\end{proof}

\subsection{算子与稳定性条件}
\label{sec:var-operator-stability-conditions}

设$\Pi$是一个巴拿赫空间(Banach space)\index{Banach space \dotfill 巴拿赫空间},设$B:X \mapsto \Pi'$是一个有界的线性算子,满足条件
\begin{equation}
  \label{eq:var-stability-operator-B}
  \big\| B \nu \big\|_{\Pi'} \le C_2^B \, \big\| \nu \big\|_X, \quad \forall \nu \in X.
\end{equation}

算子$B$意味着如下双线性形式$b(.,.): X \times \Pi \mapsto \mathbb{R}$
\begin{equation*}
  b(\nu,q) \coloneqq \langle B \nu, q \rangle, \quad (\nu,q) \in X \times \Pi.
\end{equation*}

$B$的核或称零空间(kernel, null space)\index{kernel \dotfill 核}\index{null space \dotfill 零空间}定义为
\begin{equation}
  \label{eq:var-kernel-B-def}
  \ker B \coloneqq \left\{ \nu \in X: B \nu = 0 \right\}.
\end{equation}

$\ker B$在希尔伯特空间$X$中的正交补(orthogonal complement)\index{orthogonal complement \dotfill 正交补}为
\begin{equation}
  \label{eq:var-ker-B-orthogonal-complement}
  \left( \ker B \right)^{\bot} \coloneqq
  \left\{
  w \in X: \langle w, \nu \rangle_{X} = 0, \quad \forall \, \nu \in \ker B
  \right\} \subset X.
\end{equation}

进而我们有$\ker B$的极空间(polar coordinate space)\index{polar coordinate space \dotfill 极坐标空间}
\begin{equation}
  \label{eq:var-ker-B-zero}
  \left( \ker B \right)^{0} \coloneqq
  \left\{
  f \in X': \langle f, \nu \rangle = 0, \quad \forall \, \nu \in \ker B
  \right\} \subset X'.
\end{equation}

对于某一给定的$g \in \Pi'$,我们想要求得以下算子方程的解$u \in X$
\begin{equation}
  \label{eq:var-stability-operator-equation}
  B u = g.
\end{equation}

将$B:X \mapsto \Pi'$的值域或称像(range, image)\index{range \dotfill 值域} \index{iamge \dotfill 像},定义为
\begin{equation*}
  \im_{X}B \coloneqq \left\{ B \nu \in \Pi', \quad \forall \, \nu \in X \right\}.
\end{equation*}

则算子方程\eqref{eq:var-stability-operator-equation}要求是可解的(solvability condition),即要求$g$在B的值域中
\begin{equation}
  \label{eq:var-stability-condition-equation}
  g \in \im_{X}B.
\end{equation}

将$B$的伴随算子(adjoint operator)\index{adjoint operator\dotfill 伴随算子}定义为$B':X \mapsto \Pi'$,满足
\begin{equation*}
  \langle \nu, B'q \rangle \coloneqq \langle B \nu, q \rangle, \quad \forall (\nu,q) \in X \times \Pi.
\end{equation*}

由$B$的性质  \eqref{eq:var-kernel-B-def},\eqref{eq:var-ker-B-orthogonal-complement},\eqref{eq:var-ker-B-zero}可得伴随算子$B'$的性质
\begin{align}
  \label{eq:var-kernel-Badj-def}
  \ker B' & \coloneqq \left\{
  q \in \Pi: \langle B \nu, q \rangle = 0, \forall \, \nu \in X
  \right\}, \\
  \label{eq:var-ker-Badj-orthogonal-complement}
  \left( \ker B' \right)^{\bot} &\coloneqq \left\{
  p \in \Pi : \langle p,q \rangle_{\Pi} = 0, \forall \, q \in \ker B'
  \right\},\\
  \label{eq:var-ker-Badj-zero}
  \left( \ker B' \right)^{0} &\coloneqq \left\{
  g \in \Pi' : \langle g, q \rangle = 0, \quad \forall \, q \in \ker B'
  \right\}.
\end{align}

$\im_{X} B$的性质,由闭值域定理(closed range theorem)\index{closed range theorem \dotfill 闭值域定理}给出
\begin{theorem}[闭值域定理]
  \label{theorem:var-closed-range-theorem}
  设$X$和$\Pi$是巴拿赫空间,有界线性算子$B:X \mapsto \Pi'$。则以下属性等价
  \begin{itemize}
    \item $\im_{X} B$是$\Pi'$中的闭集,
    \item $\im_{\Pi} B'$是$X'$中的闭集,
    \item $\im_{X} B = (\ker B')^0$,
    \item $\im_{\pi} B' = (\ker B)^0$。
  \end{itemize}
\end{theorem}
\begin{proof}
  略。可参考\cite[Proposition 11.30]{Muscat:2014cc}。
\end{proof}

可求解性条件\eqref{eq:var-stability-condition-equation}$\Rightarrow$
\begin{equation}
  \label{eq:var-stability-condition-eqivalence}
  \langle g, q \rangle = 0, \forall q \in \ker B' \subset \pi.
\end{equation}

若可求解性条件\eqref{eq:var-stability-condition-equation}或\eqref{eq:var-stability-condition-eqivalence}得到满足,则算子方程\eqref{eq:var-stability-operator-equation}存在至少一个解$u \in X$。但解并不唯一:我们可以加入任一$u_0 \in \ker B$,使得$u + u_0$也是$B(u + u_0) = g$的解。因此我们需要引入额外的假设条件$u \in (\ker B)^{\bot}$,以确保解的唯一性。

\begin{theorem}[算子方程解的唯一存在性]
  \label{theorem:var-solution-exist-uniq}
  设希尔伯特空间$X$和$\Pi$。有界的线性算子$B: X \mapsto \Pi'$。假定已知稳定性条件
  \begin{equation}
    \label{eq:var-operators-stablility-conditions}
    C_S \, \big\| \nu \|_{X} \le
    \sup_{0 \neq q \in \Pi}
    \frac{
    \langle B \nu, q \rangle
    }{
    \big\| q \big\|_{\Pi}
    }, \quad \forall \, \nu \in \left( \ker B \right)^{\bot},
  \end{equation}
  那么对于一个给定的$g \in \im_{X} B$,算子方程$B u = g$存在一个唯一的解$u \in \left( \ker B \right)^{\bot}$,满足
  \begin{equation*}
    \big\| u \big\|_{X} \le \frac{1}{C_S} \, \big\| g \big\|_{\Pi'}.
  \end{equation*}
\end{theorem}
\begin{proof}
  已知根据假设条件$g \in \im_{X} B$,算子方程$B u = g$存在唯一的一个解$u \in \left( \ker B \right)^{\bot}$,满足
  \begin{equation*}
    \langle B u , q \rangle = \langle g, q \rangle, \quad \forall \, q \in \Pi.
  \end{equation*}

  现在设存在第二个解$\bar{u} \in \left( \ker B \right)^{\bot}$,满足
  \begin{equation*}
    \langle B \bar{u} , q \rangle = \langle g, q \rangle, \quad \forall \, q \in \Pi,
  \end{equation*}
  则我们有
  \begin{equation*}
    \langle B ( u - \bar{u}) , q \rangle = 0, \quad \forall q \in \Pi.
  \end{equation*}

  显然$u - \bar{u} \in \left( \ker B \right)^{\bot} $也满足稳定性条件\eqref{eq:var-operators-stablility-conditions}
  \begin{equation*}
  \begin{split}
    &0 \le C_S \, \big\| u - \bar{u} \|_{X} \le
    \sup_{0 \neq q \in \Pi}
    \frac{
    \langle B \left( u - \bar{u} \right), q \rangle
    }{
    \big\| q \big\|_{\Pi}
    }, \quad \forall \, \nu \in \left( \ker B \right)^{\bot} = 0,\\
    \hookrightarrow & u = \bar{u}.
  \end{split}
  \end{equation*}

  把唯一解$u$代回\eqref{eq:var-operators-stablility-conditions}我们有
  \begin{equation*}
\begin{split}
  C_S \, \big\| u  \|_{X} &\le
  \sup_{0 \neq q \in \Pi}
  \frac{
  \langle B u, q \rangle
  }{
  \big\| q \big\|_{\Pi}
  } \\
  & = \sup_{0 \neq q \in \Pi}
  \frac{
  \langle g, q \rangle
  }{
  \big\| q \big\|_{\Pi}
  } \\
  & \le \big\| g \big\|_{\Pi'}.
\end{split}
  \end{equation*}
\end{proof}

\subsection{含有限制条件的算子方程}
\label{sec:var-constraints}
经验研究中我们常常需要求得带有约束条件$B u = g$的算子方程$A u = f$的解。常见的求解思路分为四步。
\begin{enumerate}
\item 关于限定条件$B u = g$,假定可求解条件\eqref{eq:var-stability-condition-equation}成立
\begin{equation*}
  g \in \im_{X} B \coloneqq \left\{ B \nu \in \Pi', \quad \forall \, \nu \in X \right\}.
\end{equation*}

对于给定的$g \in \Pi'$,定义流形$V_g$
\begin{equation*}
  V_g \coloneqq \left\{ \nu \in X : B \nu = g \right\}.
\end{equation*}

此外我们定义零空间$V_0$
\begin{equation*}
  V_0 = \ker B \coloneqq \left\{ \nu \in X: B \nu = 0 \right\}.
\end{equation*}

\item 关于算子方程$A u = f$,同样假定可求解条件\eqref{eq:var-stability-condition-equation}成立
\begin{equation*}
  f \in \im_{V_g} A \coloneqq \left\{ A \nu \in X', \quad \forall \, \nu \in V_g \right\}.
\end{equation*}

对于给定的$f \in X'$,构建变分问题
\begin{equation}
  \label{eq:var-constrain-variational-problem}
  \langle A u, \nu \rangle = \langle f, \nu \rangle, \quad \forall \, \nu \in V_0,
\end{equation}
求解该问题,得到解$u \in V_g$。

\item 所求得解$u \in V_g$的唯一性,见Theorem \ref{theorem:var-constraint-solution-exist-uniq}。

\item 对于唯一存在解$u \in V_g$,可以将$u$的范数,和给定的$f \in X', g \in \Pi'$的范数联系起来。

假定对于给定的$g \in \im_{X} B$,$\exists \, u_g \in V_g$,我们有
\begin{equation}
  \label{eq:var-constraint-norm-equivalence}
  \| u_g \|_{X} \le C_B \, \| g \|_{\Pi'}, \quad C_B > 0 \in \mathbb{R}.
\end{equation}
 ,则范数之间的关联见Corollary \ref{corollary:var-constraint-norm-equivalence}。
\end{enumerate}

\begin{theorem}[带约束算子方程解的唯一存在性]
\label{theorem:var-constraint-solution-exist-uniq}
  设一个有界线性$V_0$-椭圆算子$A:X \mapsto X'$
  \begin{equation*}
    \langle A \nu, \nu \rangle \ge C_1^A \, \big\| \nu \big\|_{X}^2, \quad \forall /, \nu \in V_0 \coloneqq \ker B,
  \end{equation*}
  其中$B:X \mapsto \Pi'$。并且有给定的$f,g$
  \begin{equation}
    \begin{split}
      &f \in \im_{V_g} A \coloneqq \left\{ A \nu \in X', \quad \forall \, \nu \in V_g \right\}, \\
      & g \in \im_{X} B \coloneqq \left\{ B \nu \in \Pi', \quad \forall \, \nu \in X \right\}.
    \end{split}
  \end{equation}

那么作为带有约束条件$B u \ g$的算子方程$A u = f$,有且只有一个解$u \in X$。
\end{theorem}
\begin{proof}
  由已知条件$g \in \im_{X} B$可得,约束条件存在至少一个解$u_g \in X$满足$B u_g = g$。除此而外,我们还需要求得$u_0 \coloneqq u-u_g \in V_0$,作为以下算子方程的解
  \begin{equation*}
    A u_0 = f - A u_g,
  \end{equation*}
  算子方程等价于以下变分问题
\begin{equation*}
  \langle A u_0, \nu \rangle = \langle f- A u_g, \nu \rangle, \quad \forall \nu \in V_0.
\end{equation*}

[存在性]由已知条件$f \in \im_{V_g} A$可得,$f - A u_g \in \im_{V_0} A$,那么方程$A u_0 = f - A u_g$至少存在一个解$u_0 \in V_0$。

[唯一性]现在来证明$u_0 \in V_0$是唯一的。设$\bar{u}_0 \in V_0$是算子方程的另一个解,满足$A \bar{u}_0 = f - A u_g$。由已知条件$A$的$V_0$-椭圆特性可得
\begin{equation*}
\begin{split}
  &0 \le C_1^A \, \big\| u_0 - \bar{u}_0 \big\|_{X}^{2} \le
  \langle A \left( u_0 - \bar{u}_0 \right), u_0 - \bar{u}_0 \rangle
  = \langle A u_0 - A \bar{u}_0, u_0 - \bar{u}_0 \rangle = 0,\\
  \hookrightarrow & u_0 \in X = \bar{u}_0 \in X.
\end{split}
\end{equation*}

$u_g \in V_g$可能不是唯一的解,但$u \in X = u_0 + u_g$却是变分问题的唯一最终解,并不受(可能是多重的) $u_g \in V_g$的影响。这是由于,对于满足约束条件方程$B \hat{u}_g = g$的某一个解$\hat{u}_g \in X$而言,这意味着存在唯一一个$\hat{u}_0 \in V_0$,构成算子方程$A (\hat{u}_0 + \hat{u}_g) = f$的解,进而
\begin{equation*}
\begin{split}
    & B ( u_g - \hat{u}_g) = B u_g - B \hat{u}_g = g - g = 0 \in \Pi',\\
    \hookrightarrow & u_g - \hat{u}_g \in \ker B = V_0.
\end{split}
\end{equation*}

由于
\begin{equation*}
  \begin{split}
    & A (u_0 + u_g) = f, \\
    & A (\hat{u}_0 + \hat{u}_g) = f,
  \end{split}
\end{equation*}
我们因而有
\begin{equation*}
  A(u_0 + u_g - \hat{u}_0 - \hat{u}_g) = 0.
\end{equation*}

显然,$u_0 - \hat{u}_0 + (u_g - \hat{u_g}) \in V_0$,由$A$的$V_0$-椭圆特性我们有
\begin{equation*}
  u_0 - \hat{u}_0 + \left( u_g - \hat{u}_g \right) = 0,
\end{equation*}
因此我们可得解的唯一性
\begin{equation*}
  u = u_0 + u_g = \hat{u}_0 + \hat{u}_g.
\end{equation*}
\end{proof}

\begin{corollary}
  \label{corollary:var-constraint-norm-equivalence}
  已知$u \in V_g$是有限制条件的算子方程唯一解(Theorem \ref{theorem:var-constraint-solution-exist-uniq}),并且满足假定\eqref{eq:var-constraint-norm-equivalence}。那么$\| u \|_{X}$
  和给定的$f \in X', g\in \Pi'$的范数之间关系为
  \begin{equation*}
    \big\| u \big\|_{X} \le \frac{1}{C_1^A} \, \big\| f \big\|_{X'} + \left(
    1+ \frac{
    C_2^A
    }{
    C_1^A
    }
    \right)
    \, C_B \big\| g \big\|_{\Pi'}.
  \end{equation*}
\end{corollary}
\begin{proof}
  由Theorem \ref{theorem:var-constraint-solution-exist-uniq}得,算子方程$Au = f$的解表现为$u = u_g + u_0$的形式,其中$u_0 \in V_0$是以下变分问题的唯一解
  \begin{equation*}
    \langle A u_0, \nu \rangle = \langle f - A u_g, \nu \rangle, \quad \forall \nu \in V_0.
  \end{equation*}

  由算子$A$的$V_0$-椭圆特性可得
  \begin{equation*}
    \begin{split}
      &C_1^A \, \big\|u_0\big\|_{X}^2 \le \langle A u_0, u_0 \rangle =
      \langle f - A u_0, u_0 \rangle
      \le \big\| f - A u_g \big\|_{X'} \, \big\|u_0 \big\|_{X},\\
      \hookrightarrow & \big\| u_0 \big\|_{X} \le \frac{1}{C_1^A}
      \left[
      \big\| f \big\|_{X'} + C_2^A \big\| u_g \big\|_X
      \right],
    \end{split}
  \end{equation*}

$\Rightarrow$
  \begin{equation*}
    \begin{split}
      \big\| u \big\|_X & = \big\| u_0 + u_g \big\|_X \\
      &\le  \big\| u_0 \big\|_X + \big\| u_g \big\|_X \\
      & \le \frac{1}{C_1^A} \, \big\| f \big\|_{X'} +
      \left( 1 + \frac{C_2^A}{C_1^A} \right) \, \big\| u_g \big\|_{X} \\
      & \le \frac{1}{C_1^A} \, \big\| f \big\|_{X'} +
      \left( 1 + \frac{C_2^A}{C_1^A} \right) C_B \, \big\| g \big\|_{\Pi'}.
    \end{split}
  \end{equation*}
\end{proof}

\subsection{混合算子方程(鞍点变分问题)}
\label{sec:var-mixed-formulations}

第\ref{sec:var-constraints}节讨论了如何构建带有限制条件的算子方程来求解变分问题。除此而外的另一种方法是引入拉格朗日乘子$p \in \Pi$,构建扩展变分问题,对于$\forall \, (\nu, q) \in X \times \Pi$,求解$(u , p ) \in X \time \Pi$,使其满足
\begin{subequations}
  \begin{equation}
    \label{eq:var-mixed-problem-aubv}
    \langle A u, \nu \rangle + \langle B \nu, p \rangle = \langle f, \nu \rangle,
  \end{equation}
  \begin{equation}
    \label{eq:var-mixed-problem-bu}
    \langle B u, q \rangle = \langle g, q \rangle,
  \end{equation}
\end{subequations}
其中$u \in V_g$是$A u = f$的解。

与上节相同,\eqref{eq:var-mixed-problem-bu}同样可以用于描述限制条件$B u = g$。但不同的是\eqref{eq:var-mixed-problem-aubv}可理解为另一个变分法问题:将$\nu \in V_0$作为检验方程,求解$u_0 \in V_0$。显然,这种混合算子求解变分问题的研究思路,可行前提之一是确保存在拉格朗日乘子$p \in \Pi$,使得$\forall \nu \in X$都满足等式\eqref{eq:var-mixed-problem-aubv}。具体来说就是,对于一组$(\nu,q) \in X \times \Pi$,定义一个拉格朗日泛函
\begin{equation*}
  \mathcal{L}(\nu, q) \coloneqq \frac{1}{2} \langle A \nu, \nu \rangle - \langle f, \nu \rangle + \langle B \nu, q \rangle - \langle g, q \rangle,
\end{equation*}
应当使得Theorem \eqref{theorem:var-mixed-lagrange-condition}成立。

\begin{theorem}
  \label{theorem:var-mixed-lagrange-condition}
  设$A:X \mapsto X'$为一个有界线性算子,$\forall \, \nu \in X$都具有自伴随$\langle A \nu , nu \rangle = \langle \nu, A \nu \rangle$、正半定$\langle A \nu, \nu \rangle \ge 0 $的特性。设另一个有界线性算子$B:X \mapsto \Pi'$。

  当且仅当
  \begin{equation}
    \label{eq:var-mixed-lagrange-inequality}
    \mathcal{L}(u,q) \le \mathcal{L}(u,p) \le \mathcal{L}(\nu,p) \quad \forall (\nu,q) \in X \times \Pi
  \end{equation}
  时,$(u,p)$成为变分问题\eqref{eq:var-mixed-problem-aubv}-\eqref{eq:var-mixed-problem-bu}的一个解。
\end{theorem}
\begin{proof}
  假设有一组解$(u,p) \in X \times \Pi$。\begin{enumerate}
  \item 证\eqref{eq:var-mixed-lagrange-inequality}的后半部分。
  \begin{equation*}
  \begin{split}
    &\mathcal{L}(\nu,p) - \mathcal{L}(u,p) \\
    & = \frac{1}{2} \langle A \nu, \nu \rangle - \langle f, \nu \rangle + \langle B \nu, p \rangle - \langle g, p \rangle - \frac{1}{2} \langle A u , u \rangle + \langle f, u \rangle - \langle B u, p \rangle + \langle g,p \rangle \\
   & = \frac{1}{2} \underbrace{\langle A (u - \nu), (u - \nu) \rangle}_{\text{正半定}, \ge 0} +
   \underbrace{ \langle A u, (u - \nu) \rangle + \langle B (u - \nu), p \rangle - \langle f, u - \nu \rangle}_{\eqref{eq:var-mixed-problem-aubv}, =0} \\
   & \ge 0,
  \end{split}
  \end{equation*}
\begin{equation*}
    \therefore \mathcal{L}(u,p) < \mathcal{L}(\nu,p), \quad \forall \, \nu \in X.
\end{equation*}

\item 证\eqref{eq:var-mixed-lagrange-inequality}的前半部分。
\begin{equation*}
\begin{split}
&\mathcal{L}(u,p)-\mathcal{L}(u,q) \\
& = \frac{1}{2}\langle Au,u\rangle - \langle f,u \rangle + \langle B u, p \rangle - \langle g, p \rangle - \frac{1}{2} \langle A u , u \rangle + \langle f, u \rangle - \langle B u , q \rangle + \langle g, p \rangle\\
& = \underbrace{\langle B u, p - q \rangle - \langle g, p - q \rangle}_{\eqref{eq:var-mixed-problem-bu}, =0} \\
& = 0,
\end{split}
\end{equation*}
\begin{equation*}
  \therefore \mathcal{L}(u,q) \le  \mathcal{L}(\nu,p), \quad \forall \, q \in \Pi.
\end{equation*}

\item 假设已知某个$p \in \Pi$是方程的解。构建如下最小化问题
\begin{equation}
  \label{eq:var-mix-minimization-problem}
  \mathcal{L}(u,p) \le \mathcal{L}(\nu,p), \quad \forall \nu \in X,
\end{equation}
求解$u \in X$满足式\eqref{eq:var-mixed-problem-aubv}。

设某个$u \in X$是最小化问题的解,我们有$\forall \, w \in X$满足以下两式
\begin{equation}
  \label{eq:var-mix-minimization-diff-lup}
  \frac{d}{dt} \mathcal{L} \left(u + tw, p \right)|_{t=0} =0.
\end{equation}
以及
\begin{equation}
\label{eq:var-mix-minimization-lup}
\begin{split}
    \mathcal{L} \left(u + tw, p \right) =& \underbrace{\frac{1}{2} \langle A u, u \rangle - \langle f, u \rangle + \langle Bu,p \rangle - \langle g, p \rangle}_{ = \mathcal{L}(u,p)} \\
    &+ \frac{1}{2} t^2 \langle A w, w \rangle + t
    \left[
    \langle A u, w \rangle + \langle B w, p \rangle - \langle f, w \rangle
    \right],
\end{split}
\end{equation}

进而\eqref{eq:var-mix-minimization-diff-lup}可得,\eqref{eq:var-mix-minimization-lup}$\Rightarrow$
\begin{equation*}
  \langle A u, w \rangle + \langle B w, p \rangle - \langle f, w \rangle = 0, \forall \, w \in X,
\end{equation*}
满足式\eqref{eq:var-mixed-problem-aubv}。


\item 对于任一$q \in \Pi$,证明\eqref{eq:var-mixed-problem-bu}。
\begin{enumerate}
  \item 定义$\tilde{q} \coloneqq p + q$,进而
  \begin{equation*}
  \begin{split}
  0 &\le \mathcal{L}(u,p) - \mathcal{L}(u, p + g) \\  &= \frac{1}{2} \langle A u , u \rangle - \langle f, u \rangle + \langle B u, p \rangle - \langle g, p \rangle \\
  & \quad - \frac{1}{2} \langle A u , u \rangle + \langle f, u \rangle - \langle B u, p + q \rangle + \langle g, p + q \rangle \\
  & = - \langle B u, q \rangle + \langle g, q \rangle .
  \end{split}
  \end{equation*}
  \item 定义$\tilde{q} \coloneqq p - q$,进而
  \begin{equation*}
  \begin{split}
  0 &\le \mathcal{L}(u,p) - \mathcal{L}(u, p - g) = \langle B u, q \rangle - \langle g, q \rangle .
  \end{split}
  \end{equation*}
  \item
  \begin{equation}
    \therefore \langle B u , q \rangle = \langle g, q \rangle, \quad \forall \, q \in \Pi.
  \end{equation}
\end{enumerate}
\end{enumerate}
\end{proof}

由此可见,扩展变分问题\eqref{eq:var-mixed-problem-aubv}-\eqref{eq:var-mixed-problem-bu}中,组合$(u,p) \in X \times \Pi$是一个拉格朗日泛函$\mathcal{L}(.,.)$的鞍点。从这意义上说,扩展变分问题也常称为鞍点变分问题。下面来探讨$(u,p)$解的唯一性。

\begin{theorem}[混合算子方程(鞍点变分问题)的解]
  \label{theorem:mixed-saddle-point-variational-problem}
  假设巴拿赫空间$X,\Pi$,有界算子$A:X \mapsto X', \, B: X \mapsto \Pi'$。设$A$满足$V_0$-椭圆特性
  \begin{equation*}
    \langle A \nu, \nu \rangle \ge C_1^A \, \big\| \nu \big\|_{X}^2, \quad \forall \nu \in V_0 = \ker B,
  \end{equation*}
  设稳定性条件
  \begin{equation}
    \label{eq:var-mixed-stability-condition}
    C_s \, \big\| q \big\|_{\Pi} \le \sup_{0 \neq \nu \in X} \frac{
    \langle B \nu, q \rangle
    }{
    \big\| \nu \big\|_{X}
    }, \quad \forall q \in \Pi.
  \end{equation}

  那么对于$g \in \im_{X}B, f \in \im_{V_g}A$,扩展变分问题\eqref{eq:var-mixed-problem-aubv}-\eqref{eq:var-mixed-problem-bu}都存在唯一的解$(u,p) \in X \times \Pi$,满足如下关系
  \begin{align}
    \label{eq:var-mixed-uniqueness-u}
    & \big\| u \big\|_{X} \le \frac{1}{C_1^A}  \big\| f \big\|_{X'} + \left( 1 + \frac{C_2^A}{C_1^A} \right)  C_B \big\| g \big\|_{\Pi'},\\
    \label{eq:var-mixed-uniqueness-p}
    & \big\| p \big\|_{\Pi} \le \frac{1}{C_S} \left( 1 + \frac{C_2^A}{C_1^A} \right)
    \left\{
    \big\| f \big\|_{X'} + C_B C_2^A \big\| g \big\|_{\Pi'}
    \right\}.
  \end{align}
\end{theorem}

\begin{proof}
\begin{enumerate}
  \item 证明对于变分问题
\begin{equation*}
  \begin{split}
    & \langle A u, \nu \rangle = \langle f, \nu \rangle, \quad \forall \, \nu \in V_0, \\
    & \langle B u, q \rangle = \langle g, q \rangle, \quad \forall q \in Pi
  \end{split}
\end{equation*}
存在唯一的解$u \in X$。

唯一解的存在性由Theorem \ref{theorem:var-constraint-solution-exist-uniq}证得。唯一解的范数不等式由Corollary \ref{corollary:var-constraint-norm-equivalence}给出,对应式\eqref{eq:var-mixed-uniqueness-u}。

\item 证明对于变分问题
\begin{equation*}
  \langle B \nu, p \rangle = \langle f - A u, \nu \rangle, \quad \forall \, \nu \in X
\end{equation*}
存在唯一的解$p \in \Pi$。
\begin{enumerate}
\item 解的存在性。我们有$f - A u \in \left( \ker B \right)^0$,进而根据闭值域定理(Theorem \ref{theorem:var-closed-range-theorem})我们有$f - A u \in \im_{\Pi}(B')$,进而变分问题的解是$p \in \Pi$。

\item 解的唯一性。假定变分问题有两个解$p, \hat{p} \in \Pi$,满足
\begin{equation*}
  \begin{split}
    & \langle B \nu , p \rangle = \langle f - A u, \nu \rangle, \quad \forall \, \nu \in X, \\
    & \langle B \nu, \hat{p} \rangle = \langle f - A u, \nu \rangle, \quad \forall \, \nu \in X.
  \end{split}
\end{equation*}

两式相减$\Rightarrow$
\begin{equation*}
  \langle B \nu, p - \hat{p} \rangle =0, \quad \forall \, \nu \in X.
\end{equation*}

代入稳定性条件\eqref{eq:var-mixed-stability-condition}有
\begin{equation*}
\begin{split}
  &0 \le C_s \, \big\| p - \hat{p} \big\|_{\Pi} \le \sup_{0 \neq \nu \in X} \frac{
  \langle B \nu, p - \hat{p} \rangle
  }{
  \big\| \nu \big\|_{X}
  } \le 0, \quad \forall q \in \Pi, \\
  &\hookrightarrow p = \hat{p} \in \Pi.
\end{split}
\end{equation*}

\item 计算唯一解$p \in \Pi$的范数。再次代入稳定性条件\eqref{eq:var-mixed-stability-condition}有
\begin{equation*}
  C_s \, \big\| p  \big\|_{\Pi}
  \le \sup_{0 \neq \nu \in X} \frac{
  \langle B \nu, p  \rangle
  }{
  \big\| \nu \big\|_{X}
  }
  = \sup_{0 \neq \nu \in X} \frac{
  \langle f - A u, \nu  \rangle
  }{
  \big\| \nu \big\|_{X}
  }
  \le \left\| f \right\|_{X'} + C_2^A \big\| u \big\|_{X},
\end{equation*}
代入\eqref{eq:var-mixed-uniqueness-u}替换$\big\| u \big\|_{X}$,我们得\eqref{eq:var-mixed-uniqueness-u}。
\end{enumerate}
\end{enumerate}
\end{proof}

设$A:X \mapsto X'$是个$V_0$-椭圆算子时,Theorem \ref{theorem:mixed-saddle-point-variational-problem}成立。此外,若假设$A$是个$X$-椭圆算子,即
\begin{equation}
  \label{eq:var-mixed-ellipcity-AX}
  \langle A \nu, \nu \rangle \ge C_1^A \, \big\| \nu \big\|_{X}^2, \quad \forall \, \nu \in X,
\end{equation}
Theorem \ref{theorem:mixed-saddle-point-variational-problem}依然成立。

Theorem \ref{theorem:mixed-saddle-point-variational-problem}探讨了求扩展变分问题的解$(u,p)\in X \times \Pi$。事实上求解过程可以进一步简化,$u$是一个关于$p$的方程。已知
\begin{equation*}
  \begin{split}
    & \langle A u, \nu \rangle + \langle B \nu, p \rangle = \langle f, \nu \rangle, \\
    \hookrightarrow & \langle B \nu, p \rangle = \langle \nu, B p \rangle = \langle B' p, \nu \rangle, \\
    \hookrightarrow & \langle A u, \nu \rangle + \langle B' p,  \nu \rangle =  \langle f, \nu \rangle, \\
    \hookrightarrow & A u + B ' p = f, \\
    \hookrightarrow & u = A^{-1} \left( f-B'p \right)\\
    \hookrightarrow & \langle B u, q \rangle = \langle g,q\rangle,\\
    \hookrightarrow & \langle B A^{-1} \left( f-B'p \right), q \rangle = \langle g, q \rangle, \\
    \hookrightarrow & \langle B A^{-1} f - g, q \rangle = \langle B A^{-1} B' p, q \rangle,
  \end{split}
\end{equation*}
即对于任一$p \in \Pi$,都存在唯一的一个解$u = A^{-1} \left( f - B' p \right) \in X$。这样一来,原本寻找$(u,p)\in X \times \Pi$的变分问题,就变成了一个新的(椭圆)变分问题:寻找解$p \in \Pi$使满足
\begin{equation}
  \label{eq:var-mixed-var-probl-givenuforq}
  \langle B A^{-1} f - g, q \rangle = \langle B A^{-1} B' p, q \rangle.
\end{equation}

为了探讨(椭圆)变分问题\eqref{eq:var-mixed-var-probl-givenuforq}的解$u$的唯一性,首先我们检验它是否符合拉克斯一密格拉蒙定理(Theorem \ref{theorem:lax-milgram-lemma})\index{Lax-Milgram theorem \dotfill 拉克斯一密格拉蒙定理} 的前提假;如果是,则应用该定理。检验过程见Lemma \ref{lemma:var-mixed-var-probl-onlyq}.

\begin{lemma}
  \label{lemma:var-mixed-var-probl-onlyq}
  设Theorem \ref{theorem:mixed-saddle-point-variational-problem}的假设条件均得到满足。

  那么算子$S \coloneqq B A^{-1} B'$有界,并且由稳定条件\eqref{eq:var-mixed-stability-condition}可得$S$是$\Pi$-椭圆的,满足
  \begin{equation}
    \label{eq:var-mixed-var-prob-forq}
    \langle Sq, q \rangle \ge C_1^S \, \big\| q \big\|_{\Pi}^2, \quad \forall \, q \in \pi.
  \end{equation}
\end{lemma}

\begin{proof}
  已知即对于任一$p \in \Pi$,对于如下变分问题
  \begin{equation*}
    \langle A u, \nu \rangle = \langle B \nu, q \rangle \quad \forall \nu \in X,
  \end{equation*}
  都存在唯一的一个解$u = A^{-1} \left( f - B' p \right) \in X$.

  \begin{enumerate}
    \item $S$的有界性。已知$A:X \mapsto X'$的$X$-椭圆特性\eqref{eq:var-mixed-ellipcity-AX},那么根据Theorem \ref{theorem:mixed-saddle-point-variational-problem}可证得存在唯一解$u \in X$满足
  \begin{equation}
    \label{eq:var-mixed-u-norm-ineq}
  \begin{split}
  \big\| u \big\|_{X} &= \big\| A^{-1} B' q \big\|_{X} \\
  & \le \frac{1}{C_1^A} \, \big\| B' q \big\|_{X'} \\
  & \le \frac{C_2^B}{C_1^A} \big\| q \big\|_{\Pi}, \quad \forall \, q \in \Pi.
  \end{split}
  \end{equation}

  由\eqref{eq:var-mixed-u-norm-ineq}可得
  \begin{equation*}
    \begin{split}
      \big\| S q \big\|_{\Pi'} &= \big\| B A^{-1} B' q \big\|_{\Pi'} \\
      &= \big\| B u \big\|_{\Pi'} \\
      & \le C_2^B \big\| u \big\|_{X} \\
      & \le \frac{\left( C_2^B \right)^2}{C_1^A} \big\| q \big\|_{\Pi}, \quad \forall \, q \in \Pi,
    \end{split}
  \end{equation*}
  即$S:\Pi \mapsto \Pi'$有界。

\item $S$的椭圆性。
\begin{equation}
  \label{eq:var-mixed-sqq-inner-middle}
\begin{split}
  \langle S q, q \rangle &= \langle B A^{-1} B' q, q \rangle \\
  &= \langle B u, q \rangle \\
  &= \langle A u, u \rangle \\
  & \ge C_1^A \big\| u \big\|_X^2,
\end{split}
\end{equation}

根据稳定条件\eqref{eq:var-mixed-stability-condition},
\begin{equation*}
  \begin{split}
    C_S \, \big\| q \big\|_{\Pi} & \le \sup_{0 \neq \nu \in X}
    \frac{
    \langle B \nu, q \rangle
    }{
    \big\| \nu \big\|_{X}
    } \\
    & = \sup_{0 \neq \nu \in X}
    \frac{
    \langle A u,  \nu \rangle
    }{
    \big\| \nu \big\|_{X}
    } \\
    & \le C_2^A \, \big\| u \big\|_X,
  \end{split}
\end{equation*}
代回\eqref{eq:var-mixed-sqq-inner-middle}最后得
\begin{equation*}
  \begin{split}
    \langle S q, q \rangle  \ge C_1^A \big\| u \big\|_X^2 \ge \frac{\left( C_2^B \right)^2}{C_1^A} \, \big\| q \big\|_{\Pi}^2,
  \end{split}
\end{equation*}
由此证得\eqref{eq:var-mixed-u-norm-ineq},$C_1^S \coloneqq \frac{\left( C_2^B \right)^2}{C_1^A}$。
\end{enumerate}
\end{proof}

进而,我们可以使用拉克斯一密格拉蒙定理(Theorem \ref{theorem:lax-milgram-lemma})
求得椭圆问题\eqref{eq:var-mixed-var-probl-givenuforq}的唯一解$ p \in \Pi$。

在此基础上,回到混合算子方程(鞍点变分问题)    \eqref{eq:var-mixed-problem-aubv}
-\eqref{eq:var-mixed-problem-bu}上来:
\begin{theorem}[混合算子方程(鞍点变分问题)的解(续)]
  \label{theorem:mixed-saddle-point-variational-problem-solution}
  设巴拿赫空间$X,\Pi$。有界算子$A:X \mapsto X', B: X \mapsto \Pi'$。假设$A$是$X$-椭圆的,满足稳定条件\eqref{eq:var-mixed-stability-condition}。对于给定的$f \in X', g \in \Pi'$,混合算子方程(鞍点变分问题)    \eqref{eq:var-mixed-problem-aubv}
  -\eqref{eq:var-mixed-problem-bu}存在唯一的解$(u,p) \in X \times \Pi$,满足
  \begin{align}
    \label{eq:mixed-saddle-point-variational-problem-solution-p}
    & \big\| p \big\|_{\Pi} \le \frac{1}{C_1^S} \big\| B A^{-1} f - g \big\|_{\Pi'} \le \frac{1}{C_1^S} \left[
    \frac{C_2^B}{C_1^A} \, \big\| f \big\|_{X'} + \big\| g \big\|_{\Pi'}
    \right], \\
    \label{eq:mixed-saddle-point-variational-problem-solution-u}
    & \big\| u \big\|_{X} \le \frac{1}{C_1^A} \left[
    1 + \frac{
    \left(C_2^B \right)^2
    }{
    C_1^A \, C_1^S
    }
    \right] \,
    \big\| f \big\|_{X'}
    + \frac{C_2^B}{C_1^A \, C_1^S} \, \big\| g \big\|_{\Pi'}.
  \end{align}
\end{theorem}

\begin{proof}
  \begin{enumerate}
    \item 根据Theorem \ref{theorem:mixed-saddle-point-variational-problem}可证得\eqref{eq:mixed-saddle-point-variational-problem-solution-p}。
    \item 根据拉克斯一密格拉蒙定理(Theorem \ref{theorem:lax-milgram-lemma}),可以证明混合算子方程(鞍点变分问题)\eqref{eq:var-mixed-problem-aubv}
  -\eqref{eq:var-mixed-problem-bu}存在唯一解,满足
  \begin{equation*}
    \langle A u, \nu \rangle = \langle f - B' p, \nu \rangle, \quad \forall \, \nu \in X.
  \end{equation*}

  由算子$A$的$X$-椭圆特性可得
  \begin{equation*}
    C_1^A \, \big\|u\big\|_{X}^2 \le \langle Au, u \rangle = \langle f - B'p, u \rangle \le \big\| f - B' p \big\|_{X'} \, \big\|u \big\|_{X},
  \end{equation*}

  由此我们有
  \begin{equation*}
    \begin{split}
      \big\| u \big\|_{X} & \le \frac{1}{C_1^A}  \big\| f - B' p \big\|_{X'} \\
      & \le \frac{1}{C_1^A} \, \big\| f \big\|_{X'} + \frac{C_2^B}{C_1^A} \, \big\| p \big\|_{\Pi}.
    \end{split}
  \end{equation*}
  将\eqref{eq:mixed-saddle-point-variational-problem-solution-p}代入上式,可得\eqref{eq:mixed-saddle-point-variational-problem-solution-u}。
\end{enumerate}
\end{proof}

\subsection{强制算子方程}
\label{sec:var-coercive}
前面介绍有界线性算子$A:X \mapsto X'$,假设它具有$X$-椭圆的特性,如\eqref{eq:var-elliptic-operator-def}。这一假设过强。在多数情况下,我们用强制算子(coercive operator)予以替代。

\begin{definition}[强制算子]
  \label{definitiuon:var-coercive-def}
  如果存在一个紧凑算子$C:X \mapsto X'$,使得和$A:X \mapsto X'$一道满足Gårding不等式(Gårding inequality)\index{Gårding inequality \dotfill Gårding不等式}
  \begin{equation}
    \label{eq:var-coercive-garding-inequality}
    \langle \left(A + C \right) \nu, \nu \rangle \ge C_1^A \, \big\| \nu \big\|_{X}^2, \quad \forall \nu \in X.
  \end{equation}
  那么我们称$A$是一个强制算子(coercive operator)。
\end{definition}

Gårding不等式的详细介绍,可参考\cite[Theorem 2.4]{Jovanovic:2014iy}。证明可见如\cite[Theorem 8.1.1]{Agranovich:2015cv},\cite[Therorem 9.17]{Renardy:2004tg}。

紧凑算子的定义。对于$C:X \mapsto Y$,若$X$中单位球体(unit sphere)的像(image)在$Y$中相对紧凑,则我们称$C$为紧凑算子(compact operator)\index{compact operator \dotfill 紧凑算子}。一个值得关注的特性是,紧凑算子和有界线性算子的乘也是紧凑的。

根据Riesz-Schauder定理(Riesz-Schauder theorem, \cite[Sec. X.5]{Yosida:1978ul}, \cite[Theorem 14.18]{Muscat:2014cc})\index{Riesz-Schauder theorem \dotfill Riesz-Schauder定理},我们有弗雷德霍姆二择一定理(Fredholm alternative theorem)\index{Fredholm alternative theorem \dotfill 弗雷德霍姆二择一定理}如下。
\begin{theorem}[弗雷德霍姆二择一定理]
\label{theorem:var-coercive-fredhold-alternative}
设$K:X \mapsto X$是一个紧凑算子。那么以下两种情况之一会出现:
\begin{itemize}
  \item 齐次方程(homogeneous equation)
  \begin{equation*}
    \left( I - K \right) u = 0
  \end{equation*}
  有一个不平凡解(nontrivial solution) $u \in X$,或
  \item 非其次方程
  \begin{equation*}
    \left( I - K \right) u = g
  \end{equation*}
  对于每一个给定的$g \in X$,都有唯一对应的解$u \in X$,满足关系
  \begin{equation*}
    \big\| u \big\|_{X} \le c \big\| g \big\|_{X}.
  \end{equation*}
\end{itemize}
\end{theorem}
\begin{proof}
  略。可见\cite[Sec. 18.1]{Agranovich:2015cv}。
\end{proof}

基于弗雷德霍姆二择一定理(Theorem \ref{theorem:var-coercive-fredhold-alternative}),我们可以探讨当有界线性算子$A$是强制的时,算子方程$A u = f$的解。
\begin{theorem}[强制算子方程的解]
  假设一个有界线性算子$A: X \mapsto X'$,具有强制性(Definition \ref{definitiuon:var-coercive-def})、内射性(injective\index{injection \dotfill 内射}, 即$A u = 0 \Rightarrow u = 0$)。那么强制算子方程$A u = f$存在唯一的解$u \in X$,并且满足条件
  \begin{equation*}
    \big\| u \big\|_{X} \le c \, \big\| f \big\|_{X'}.
  \end{equation*}
\end{theorem}
\begin{proof}
  定义一个线性算子$D \coloneqq A + C: X \mapsto X'$,由\eqref{eq:var-coercive-garding-inequality}可得,线性有界算子$D$也是$X$-椭圆的。

  通过拉克斯一密格拉蒙定理(Theorem \ref{theorem:lax-milgram-lemma}拉克斯一密格拉蒙定理)可得,逆算子$D^{-1}: X' \mapsto X$。这样,我们可以将原强制算子方程$Au = f$转换为新的强制算子方程
  \begin{equation}
    \label{eq:var-coercive-new-function}
    \begin{split}
      &B u = D^{-1} A u = D^{-1} f, \text{其中线性有界算子}\\
      &B \coloneqq D^{-1} A = D^{-1} \left( D - C \right) = I - D^{-1} C: X \mapsto X.
    \end{split}
  \end{equation}

  由假设条件$C:X \mapsto X'$是紧凑算子,和$D:X \mapsto X'$、进而$D^{-1}:X' \mapsto X$是线性有界算子,可得$D^{-1} C:X \mapsto X$是紧凑算子。进而,可以根据弗雷德霍姆二择一定理(Theorem \ref{theorem:var-coercive-fredhold-alternative})证得强制算子方程\eqref{eq:var-coercive-new-function}解的唯一存在性。

  由假设条件$A:X \mapsto X'$的内射性,可得齐次方程$D^{-1} A u =0$的所有解$u \in X$ 都是平凡解(trivial solution)。因此,非齐次方程$B u = D^{-1} f$存在唯一解$u \in X$,满足
  \begin{equation*}
    \big\| u \big\|_{X} \le c \big\| D^{-1} f \big\|_{X} \le \tilde{c} \, \big\| f \big\|_{X'}.
  \end{equation*}
\end{proof}


\section{变分法求解边界值问题}
\label{sec:variational-bvp}

在介绍了常见的变分方法(第\ref{sec:variational-methods}节)之后,本届关注如何利用变分法分析求解二阶椭圆边界值问题,尤其是位势方程(potential equations, 第\ref{sec:bem-fem-potential-bvp}节)的边界值。应用变分法构建的边界值问题的若方程,是有限元分析的重要基础之一。

\subsection{位势方程基本介绍}
\label{sec:var-bvp-potential-equation}

如第\ref{sec:bem-fem-potential-bvp}节所讨论,先介绍三个算子,分别为实值标量的偏微分算子\eqref{eq:bvp-self-adjoint-pde-operator}
\begin{equation}
  \label{eq:var-bvp-self-adjoint-pde-operator}
  \left( L \, u \right)(x) \coloneqq - \sum_{i,j=1}^d \frac{\partial}{\partial x_j} \left[ a_{ji} (x) \frac{\partial}{\partial x_i} u(x)\right], \quad x \in \Omega \in \mathbb{R}^d,
\end{equation}

内界迹算子\index{trace \dotfill 迹}\eqref{eq:bvp-interior-boundary-trace}
\begin{equation*}
  \gamma_0^{int} u(x) \coloneqq \lim_{\Omega \owns \tilde{x} \mapsto x \in \Gamma} u \left( \tilde{x} \right), \quad x \in \Gamma = \partial \Omega,
\end{equation*}

与之相对应的内部共形导数\eqref{eq:bvp-int-conformal-derivative}
\begin{equation}
  \label{eq:var-bvp-int-conformal-derivative}
  \gamma_1^{int}u(x) \coloneqq \lim_{\Omega \owns \tilde{x} \mapsto x \in \Gamma} \left[
\sum_{i,j=1}^{d} n_j(x) a_{ji}\left( \tilde{x} \right) \frac{\partial}{\partial \tilde{x}_i} u \left( \tilde{x} \right)
  \right], \quad x \in \Gamma = \partial \Omega.
\end{equation}

当$u, \nu \in H^1(\Omega)$时,$Lu \in \widetilde{H}^{-1}(\Omega)$,格林第一恒等式\index{Green identities!first 格林第一恒等式} \eqref{eq:bvp-a-u-nu-inner-prod}化简为
\begin{equation}
  \label{eq:var-bvp-green-1st-identity}
\begin{split}
    a\left(u,\nu \right) &=\int_{\Omega} \left( L \, u \right)(x) \, \nu(x) \, dx + \int_{\Gamma} \left[ \gamma_1^{int} u(x) \right]  \left[ \gamma_0^{int} \nu(x) \right] \, d s_x \\
    &= \langle Lu, \nu \rangle_{\Omega} + \langle \gamma_1^{\text{int}} u, \gamma_0^{\text{int}} \nu \rangle_{\Gamma},
\end{split}
\end{equation}
其中对称双线性泛函$a(.,.)$的定义
\begin{equation}
  \label{eq:var-bvp-bilinear-form-a-def}
  a \left(u,\nu \right) \coloneqq \langle u, \nu \rangle = \sum_{i,j=1}^d \int_{\Omega} a_{ji}(x) \frac{\partial}{\partial x_i} u(x) \, \frac{\partial}{\partial x_j} \nu(x) \, dx
\end{equation}

\begin{lemma}
  \label{lemma:var-bvp-aform-inequality}
  假设$a_{ij} \in L^{\infty}(\Omega), \, i,j = 1,\ldots,d$满足
  \begin{equation}
    \label{eq:var-bvp-coeff-norm}
    \big\| a \big\|_{L^{\infty}(\Omega)} \coloneqq \max_{i,j=1,\ldots,d} \, \sup_{x \in \Omega} \big| a_{ij}(x) \big|.
  \end{equation}

  那么我们可得线性形泛函$a(.,.): H^{1}(\Omega) \times H^{1}(\Omega) \mapsto \mathbb{R}$有界,并且满足
  \begin{equation}
    \label{eq:var-bvp-a-c2a-inequality}
    \begin{split}
    &\big| a (u, \nu) \big| \le c_2^A \, \big| u \big|_{H^{1}(\Omega)} \, \big| \nu \big|_{H^1(\Omega)}, \quad \forall u, \nu \in H^{1}(\Omega), \\
    & c_2^A \coloneqq d \, \big\| a \big\|_{L^{\infty}(\Omega)}.
  \end{split}
\end{equation}
\end{lemma}

\begin{proof}
  由\eqref{eq:var-bvp-bilinear-form-a-def},\eqref{eq:var-bvp-coeff-norm}可得
  \begin{equation*}
    \begin{split}
      \big| a \left(u,\nu \right) \big| &= \Big| \sum_{i,j=1}^d \int_{\Omega} a_{ji}(x) \frac{\partial}{\partial x_i} u(x) \, \frac{\partial}{\partial x_j} \nu(x) \, dx \Big| \\
      & \le \big\| a \big\|_{L^{\infty}(\Omega)} \,
      \int_{\Omega} \left\{
      \sum_{i=1}^{d} \Big| \frac{\partial}{\partial x_i} \, u(x) \Big| \,
      \sum_{j=1}^{d} \Big| \frac{\partial}{\partial x_j} \, \nu(x) \Big| \right\} \,
      dx,
      \end{split}
  \end{equation*}
连续两次使用柯西——施瓦茨不等式(Definition \ref{definition:cauchy-schwarz-inequality}),上式变为
\begin{equation*}
  \begin{split}
    \big| a \left(u,\nu \right) \big|
    & \le \big\| a \big\|_{L^{\infty}(\Omega)} \,
    \left\{
    \int_{\Omega}
    \left[
    \sum_{i=1}^{d} \Big| \frac{\partial}{\partial x_i} \, u(x) \Big|
    \right]^2  dx
    \right\}^{\frac{1}{2}} \,
    \left\{
    \int_{\Omega}
    \left[
    \sum_{j=1}^{d} \Big| \frac{\partial}{\partial x_j} \, \nu(x) \Big|
    \right]^2  dx
    \right\}^{\frac{1}{2}} \\
    & \le \big\| a \big\|_{L^{\infty}(\Omega)} \,
    \left\{
    \int_{\Omega} d
    \sum_{i=1}^{d}
    \Big|
    \frac{\partial}{\partial x_i} \, u(x) \Big|^2
    dx
    \right\}^{\frac{1}{2}} \,
    \left\{
    \int_{\Omega} d
    \sum_{j=1}^{d}
    \Big|
    \frac{\partial}{\partial x_j} \, \nu(x) \Big|^2
    dx
    \right\}^{\frac{1}{2}} \, \\
    & = \underbrace{d \, \big\| a \big\|_{L^{\infty}(\Omega)}}_{\eqqcolon c_2^A} \,
    \big\| \triangledown u \big\|_{L^{2}(\Omega)} \,
    \big\| \triangledown \nu \big\|_{L^{2}(\Omega)}. \, %\\
    % & = c_2^A
    % \big| u \big|_{H^{1}(\Omega)} \, \big| \nu \big|_{H^1(\Omega)},
  \end{split}
\end{equation*}
\end{proof}

可见,由Lemma \ref{lemma:var-bvp-aform-inequality}可得,\eqref{eq:var-bvp-a-c2a-inequality}又进一步表示为
\begin{equation}
  \label{eq:var-bvp-aform-inequality-norm}
  \big| a \left(u,\nu \right) \, \big| \le c_2^A \, \big\| u \big\|_{H^{1}(\Omega)} \, \big\| \nu \big\|_{H^{1}(\Omega)}, \quad \forall u,\nu \in H^{1}(\Omega).
\end{equation}

\begin{lemma}[双线性算子的半椭圆特性]
  \label{lemma:var-bvp-operator-ellipticity-property}
  设$L$是一个如\eqref{eq:var-bvp-self-adjoint-pde-operator}所定义的一致椭圆偏微分算子。用双线性形式$(a.,.)$ \eqref{eq:var-bvp-bilinear-form-a-def}表示,我们有
  \begin{equation}
    \label{eq:var-bvp-bilinear-a-nunu}
    a(\nu,\nu) \ge \lambda_0 \, \big| \nu \big|_{H^{1}(\Omega)}^2, \quad \forall \nu \in H^{1}(\Omega),
  \end{equation}
  其中常数$\lambda_0 > 0$,见椭圆算子\eqref{eq:bvp-def-uniformly-elliptic}。
\end{lemma}

\begin{proof}
  设一个$w_i(x)$
  \begin{equation*}
    w_i(x) \coloneqq \frac{\partial}{\partial x_i} \nu(x), \quad i=1,\ldots,d,
  \end{equation*}
  \begin{equation*}
    \begin{split}
      \hookrightarrow a(\nu, \nu) & = \sum_{i,j=1}^{d} \int_{\Omega}
      a_{ji}(x) \frac{\partial}{\partial x_i} \nu(x) \frac{\partial}{\partial x_i} \nu(x)  \, dx \\
      &= \int_{\Omega} \left[ A \underline{w}(x), \underline{w}(x) \right] \, dx \\
      & \ge \lambda_0 \int_{\Omega} \left[ \underline{w}(x), \underline{w}(x) \right] \, dx \\
      & = \lambda_0 \, \big\| \triangledown \nu \big\|_{L^{2}(\Omega)}^2
      = \lambda_0 \big\| \nu \big\|_{W^{1,2}(\Omega)}^2
      = \lambda_0 \big\| \nu \big\|_{H^1(\Omega)}^2.
    \end{split}
  \end{equation*}
\end{proof}

\subsection{狄利克雷边界值问题1}
\label{sec:var-bvp-dirichlet}

\subsubsection{将狄利克雷边界值问题改写为变分问题}

回顾一下第\ref{sec:bem-fem-potential-bvp}节的狄利克雷边界值问题\index{Dirichlet boundary value condition \dotfill 狄利克雷边界值条件}
\eqref{eq:bvp-extension-omega-cond}-\eqref{eq:bvp-extension-gamma-dirichlet}:基于给定的$f$和$g$,求解
\begin{equation}
  \label{eq:var-dbvp-problem}
  \begin{split}
    &(L u)(x) = f(x), \quad x \in \Omega, \\
    &\gamma_0^{\text{int}} u(x) = g(x), \quad x \in \Gamma,
  \end{split}
\end{equation}
对应弱形式下的流形空间
\begin{equation}
  \label{eq:var-dbvp-problem-manifold}
\begin{split}
  &V_g \coloneqq
  \left\{
  \nu \in H^{1}(\Omega): \gamma_0^{\text{int}} \nu(x) = g(x), \quad x \in \Gamma
  \right\},\\
  &V_0 = H_{0}^{1}(\Omega).
\end{split}
\end{equation}

根据第\ref{sec:variational-methods}介绍的知识,我们可以将狄利克雷边界值问题\eqref{eq:var-dbvp-problem}-\eqref{eq:var-dbvp-problem-manifold}改写为变分问题:
\begin{equation}
  \label{eq:var-dbvp-variational-problem}
  a(u,\nu) = \langle f, \nu \rangle_{\Omega}, \quad \forall \, \nu \in V_0,
\end{equation}
其中双线性泛函$a(u,\nu)$由格林第一恒等式\eqref{eq:var-bvp-green-1st-identity}定义。研究目标是寻找变分问题的解$u \in V_g$。由此狄利克雷边界问题表现为流形$V_g$中的一个附属条件,因此又称为基本边界条件(essential boundary conditions)。

\subsubsection{解的存在性与唯一性}
由上所述可以看出,狄利克雷边界值问题对应的变分问题,可以归入带限制条件的算子方程问题类型,可用相应的变分方法求解(第\eqref{sec:var-constraints}节)。借助Theorem \ref{theorem:var-constraint-solution-exist-uniq}和Corollary \ref{corollary:var-constraint-norm-equivalence},我们可以证明$u \in V_g$存在且唯一。

\begin{theorem}[狄利克雷边界值问题的弱形式解]
  \label{theorem:var-dvbp-uniq-exist-solution}
  给定$f \in H^{-1}(\Omega), g \in H^{\frac{1}{2}}(\Gamma)$,变分问题\eqref{eq:var-dbvp-problem}存在唯一的解$u \in H^{1}(\Omega)$,满足
  \begin{equation}
    \label{eq:var-dvbp-uniq-exist-solution}
    \big\| u \big\|_{H^{1}(\Omega)} \le \frac{1}{c_1^A} \, \big\| f \big\|_{H^{-1}(\Omega)} + \left( 1 + \frac{c_2^A}{c_1^A} \right) c_{\text{IT}} \, \big\| g \big\|_{H^{\frac{1}{2}}(\Gamma)}.
  \end{equation}
\end{theorem}

\begin{proof}
  与Theorem \ref{theorem:var-constraint-solution-exist-uniq}类似,假设解$u$由两部分相加而得,$u \coloneqq u_0 + u_g$,其中$u_g$是$\gamma_0^{\text{int}} u_g = g$的解,$u_0$是$\gamma_1^{\text{int}} (u_0 + u_g) = f$的解。显然,$u$的唯一存在,可由$u_g$和$u_0$的唯一存在所分别证得。
\begin{enumerate}
  \item 证明$u_g$的唯一存在性。根据给定条件$g \in H^{\frac{1}{2}}(\Gamma)$,应用逆迹定理Theorem \ref{theorem:sobolev-manifold-inverse-trace-theorem},可得存在唯一一个有界延拓$u_g \in H^{\Omega}$,满足
  \begin{equation*}
    \begin{split}
      & \gamma_{0}^{\text{int}} u_g = g, \\
      & \big\| u_g \big\|_{H^{1}(\Omega)} \le c_{\text{IT}} \, \big\| g \big\|_{H^{\frac{1}{2}}(\Gamma)}.
    \end{split}
  \end{equation*}

  \item 在求得唯一解$u_g$的基础上,继续求唯一解$u_0$。变分问题\eqref{eq:var-dbvp-variational-problem}变为,寻找解$u_0 \in V_0 = H_{0}^{1}(\Omega)$,使满足新的变分问题
  \begin{equation}
    \label{eq:var-dbvp-variational-problem-u0}
    a(u_0, \nu) = \langle f, \nu \rangle_{\Omega} - a (u_g, \nu), \quad \forall \, \nu \in V_0.
  \end{equation}
  \begin{enumerate}
    \item 由范数等价定理(Theorem \ref{theorem:equivalence-norm-theorem}-\ref{theorem:sobolev-equivalence-norm-theorem}, \eqref{eq:sobolev-equivalence-norm-w12omega})得,$H^{1}(\Omega)$中的等价范为
    \begin{equation*}
%      \begin{split}
        \big\| u_0 \big\|_{W^{1,2}(\Omega), \Gamma} = \left\{
      \left[
      \int_{\Gamma} \gamma_{0}^{\text{int}} u_0(x) \, ds_x
      \right]^2
      + \big\| \triangledown u_0 \big\|^2_{L^2(\Omega)}
       \right\}^{\frac{1}{2}},
%     \end{split}
    \end{equation*}
    即$a(.,.)$有界。

    \item 由Lemma \ref{lemma:var-bvp-operator-ellipticity-property} \eqref{eq:var-bvp-bilinear-a-nunu}可得
    \begin{equation}
      \label{eq:var-bvp-operator-a-ellipticity}
      \begin{split}
        a(u_0,u_0) &\ge \lambda_0 \, \big| u_0 \big\|_{H^{1}(\Omega)} \\
        &= \lambda_0 \, \big\| \triangledown u_0 \big\|_{W^{1,2}(\Omega), \Gamma}^2 \\
        &\ge c_1^A \big\| u_0 \big\|_{H^1(\Omega)}^2,
      \end{split}
    \end{equation}
    即$a(.,.)$是个$V_0$-椭圆算子。

    \item $a(.,.)$有界且$V_0$-椭圆,满足拉克斯一密格拉蒙定理(Theorem \ref{theorem:lax-milgram-lemma})所需的前提条件,根据该定理,变分问题\eqref{eq:var-dbvp-variational-problem-u0}有唯一解$u_0 \in V_0$,满足
    \begin{equation*}
      \begin{split}
        c_1^A \, \big\| u_0 \big\|_{H^1(\Omega)}^2 & \le a(u_0, u_0) \\
        & = \langle f, u_0 \rangle_{\Omega} - a(u_g, u_0),
      \end{split}
    \end{equation*}
    代入Lemma \ref{lemma:var-bilinear-form-to-A},上式进一步变为
    \begin{equation*}
      \begin{split}
        &c_1^A \, \big\| u_0 \big\|_{H^1(\Omega)}^2  \le
        \left(
        \big\| f \big\|_{H^{-1}(\Omega)} + c_2^A \, \big\| u_g \big\|_{H^{1}(\Omega)}
        \right) \, \big\| u_0 \big\|_{H^{1}(\Omega)},\\
        \hookrightarrow &
        \big\| u_0 \big\|_{H^1(\Omega)} \le \frac{1}{c_1^A} \, \big\|f \big\|_{H^{-1}(\Omega)} + \frac{c_2^A}{c_1^A} \, \big\|u_g \big\|_{H^{1}(\Omega)}.
      \end{split}
    \end{equation*}
  \end{enumerate}

  \item 在此基础上我们有
  \begin{equation}
    \begin{split}
      &\big\| u_0 \big\|_{H^1(\Omega)} + \big\| u_g \big\|_{H^1(\Omega)} \le
      \frac{1}{c_1^A} \, \big\|f \big\|_{H^{-1}(\Omega)} + \left( 1+  \frac{c_2^A}{c_1^A} \right) \, \big\|u_g \big\|_{H^{1}(\Omega)}, \\
      \hookrightarrow &\big\| u \big\|_{H^{1}(\Omega)} \le \frac{1}{c_1^A} \, \big\| f \big\|_{H^{-1}(\Omega)} + \left( 1 + \frac{c_2^A}{c_1^A} \right) c_{\text{IT}} \, \big\| g \big\|_{H^{\frac{1}{2}}(\Gamma)}.
    \end{split}
  \end{equation}
\end{enumerate}
\end{proof}

\subsubsection{共形导数}
变分问题\eqref{eq:var-dbvp-variational-problem}的唯一解$u \in V_g$,常常又称为狄利克雷边界值问题\eqref{eq:var-dbvp-problem}的弱形式解。在此基础上,对于$f \in \widetilde{H}^{-1}(\Omega)$,可通过构建变分问题进一步求解共形导数$\gamma_{1}^{\text{int}} u \in H^{-\frac{1}{2}(\Gamma)}$
\begin{equation}
  \label{eq:var-dbvp-conormal-problem}
  \langle \gamma_{1}^{\text{int}} u, z \rangle _{\Gamma} = a(u,\varepsilon_z) - \langle f, \varepsilon_z \rangle_{\Omega}, \quad \forall \, z \in H^{\frac{1}{2}}(\Gamma),
\end{equation}
其中$\varepsilon_z$是逆迹定理Theorem \ref{theorem:sobolev-manifold-inverse-trace-theorem}所定义的有界延拓算子。\eqref{eq:var-dbvp-conormal-problem}的唯一可解条件由Theorem \ref{theorem:var-solution-exist-uniq}给出,由此我们可以假定如下稳定性条件
\begin{equation}
  \label{eq:var-dbvp-conormal-stability-condition}
  \big\| w \big\|_{H^{-\frac{1}{2}}(\Gamma)} = \sup_{0 \neq z \in H^{\frac{1}{2}}(\Gamma)} \frac{
  \langle w, z \rangle_{\Gamma}
  }{
  \big\| z \big\|_{H^{\frac{1}{2}}(\Gamma)}
  }, \quad \forall \, w \in H^{-\frac{1}{2}(\Gamma)}.
\end{equation}

在此基础上我们有共形导数的解
\begin{lemma}[共形导数的解]
  \label{lemma:var-dbvp-conormal-solution}
  设给定$f \in \widetilde{H}^{-1}(\Omega), g \in H^{\frac{1}{2}}(\Gamma)$,我们有$u \in H^{1}(\Gamma)$是狄利克雷边界值问题\eqref{eq:var-dbvp-variational-problem}的唯一解。

  那么相应的共形导数$\gamma_{1}^{\text{int}} u \in H^{-\frac{1}{2}}(\Gamma)$满足
  \begin{equation}
    \label{eq:var-dbvp-conormal-solution}
    \big\| \gamma_1^{\text{int}} u \big\|_{H^{-\frac{1}{2}}(\Gamma)}
    \le c_{\text{IT}} \,
    \left\{
    \big\| f \big\|_{\widetilde{H}^{-1}(\Omega)} +
    c_2^A \, \big| u \big|_{H^{1}(\Omega)}
    \right\}.
  \end{equation}
\end{lemma}

\begin{proof}
  稳定性条件\eqref{eq:var-dbvp-conormal-stability-condition}$\Rightarrow$
  \begin{equation*}
    \big\| \gamma_1^{\text{int}} u \big\|_{H^{-\frac{1}{2}}(\Gamma)} =
    \sup_{0 \neq z \in H^{\frac{1}{2}}(\Gamma)} \frac{
    \langle \gamma_1^{\text{int}} u , z \rangle_{\Gamma}
    }{
    \big\| z \big\|_{H^{\frac{1}{2}}(\Gamma)}
    },
  \end{equation*}
变分问题\eqref{eq:var-dbvp-conormal-problem}、Lemma \eqref{lemma:var-bilinear-form-to-A}$\Rightarrow$
\begin{equation*}
\begin{split}
  \big\| \gamma_1^{\text{int}} u \big\|_{H^{-\frac{1}{2}}(\Gamma)} &=
  \sup_{0 \neq z \in H^{\frac{1}{2}}(\Gamma)} \frac{
  \big|
  a(u,\varepsilon_z) - \langle f, \varepsilon_z \rangle_{\Omega}
  \big|_{\Gamma}
  }{
  \big\| z \big\|_{H^{\frac{1}{2}}(\Gamma)}
  },\\
  &\le
  \left\{
  c_2^A \, \big| u \big|_{H^{1}(\Omega)} + \big\| f \big\|_{\widetilde{H}^{-1}(\Omega)}
  \right\} \,
  \underbrace{\sup_{0 \neq z \in H^{\frac{1}{2}}(\Gamma)}
  \frac{
  \big\| \varepsilon_z \big\|_{H^{1}(\Omega)}
  }{
  \big\| z \big\|_{H^{\frac{1}{2}}(\Gamma)}
  }}_{\eqqcolon \mathcal{A}}
\end{split}
\end{equation*}

由逆迹定理Theorem \ref{theorem:sobolev-manifold-inverse-trace-theorem}可得
\begin{equation*}
  \mathcal{A} \le c_{\text{IT}},
\end{equation*}

因此
\begin{equation*}
  \big\| \gamma_1^{\text{int}} u \big\|_{H^{-\frac{1}{2}}(\Gamma)}
  \le c_{\text{IT}} \,
    \left\{
    \big\| f \big\|_{\widetilde{H}^{-1}(\Omega)} +
    c_2^A \, \big| u \big|_{H^{1}(\Omega)}
    \right\}.
\end{equation*}
\end{proof}

\subsubsection{带有齐次偏微分方程的狄利克雷边界值问题弱解}
\label{sec:var-dbvp-f0-solution}
来考虑一类特殊的狄利克雷边界值问题,即含有齐次偏微分方程$f \equiv 0$的情况,其解$u$对于边界积分算子的分析具有重要意义。
\begin{corollary}[带有齐次偏微分方程的狄利克雷边界值问题弱解]
  \label{corollary:var-dbvp-f0-solution}
  设$u \in H^{1}(\Omega)$是以下带有齐次偏微分方程的狄利克雷边界值问题的弱形式解
  \begin{equation*}
    \begin{split}
      &\left( L u \right) (x) = 0, \quad x \in \Omega, \\
      &\gamma_0^{\text{int}} u(x) = g(x), \quad x \in \Gamma,
    \end{split}
  \end{equation*}
  其中$L$是一个一致椭圆的二阶偏微分算子。

  那么我们有
  \begin{equation}
    \label{eq:var-dbvp-f0-solution}
    a(u,u) \ge c \big\| \gamma_{1}^{\text{int}} u \big\|_{H^{-\frac{1}{2}}(\Gamma)}^2.
  \end{equation}
\end{corollary}
\begin{proof}
  \begin{enumerate}
    \item 当$f \equiv 0$时,共形导数算子和逆迹的关系,由Lemma \ref{lemma:var-dbvp-conormal-solution}的  \eqref{eq:var-dbvp-conormal-solution}变为
  \begin{equation*}
    \big\| \gamma_1^{\text{int}} u \big\|_{H^{-\frac{1}{2}}(\Gamma)}^2
    \le \left[ c_{\text{IT}} \,
    c_2^A \right]^2
    \, \big| u \big|_{H^{1}(\Omega)}^2.
  \end{equation*}

\item 由双线性算子的半椭圆特性Lemma \ref{lemma:var-bvp-operator-ellipticity-property} \eqref{eq:var-bvp-bilinear-a-nunu}可得
\begin{equation*}
  \lambda_0
  \, \big| u \big|_{H^{1}(\Omega)}^2 \le a(u,u).
\end{equation*}
\end{enumerate}
\end{proof}

\subsubsection{狄利克雷边界值问题的强解简述}
\label{sec:var-bvp-strong-solution}
当$\Omega$是利普希茨域时,基于给定的$f$和$g$,我们可以引入更严格的假设条件,构建强形式的正则解如$u$、$\gamma_{1}^{\text{int}}u$等。

\begin{theorem}
  设$\Omega \in \mathbb{R}^d$是个有界的利普希茨域,边界为$\Gamma = \partial \Omega$。设$u \in H^{1}(\Omega)$是狄利克雷边界值问题的弱形式解
  \begin{equation*}
    \begin{split}
      &(L u)(x) = f(x), \quad x \in \Omega,\\
      &\gamma_{0}^{\text{int}} u(x) = g(x), \quad x \in \Gamma.
    \end{split}
  \end{equation*}

  如果给定的$f$和$g$满足$f \in L^{2}(\Omega), g \in H^{1}(\Gamma)$,那么我们有$u \in H^{\frac{3}{2}}(\Omega), \gamma_{1}^{\text{int}} u \in L^{2}(\Gamma)$,并且
  \begin{equation*}
  \begin{split}
  &\big\| u \big\|_{H^{\frac{3}{2}}(\Omega)} \le c_1
  \left\{
  \big\| f \big\|_{L^{2}(\Omega)} + H^{1}(\Gamma)
  \right\},\\
  & \big\| \gamma_{1}^{\text{int}} u \big\|_{L^{2}(\Gamma)} \le c_2 \left\{
  \big\| f \big\|_{L^{2}(\Omega)} + H^{1}(\Gamma)
  \right\}.
  \end{split}
  \end{equation*}
\end{theorem}

我们甚至可以使假定条件更加严格,如$\Gamma = \partial \Omega$是平滑或分段平滑的边界,$\Omega$是凹的,$f \in L^{2}(\Omega)$等。若$g = \gamma_{0}^{\text{int}} u $是方程解$u_g \in H^2(\Omega)$的迹,则我们有$u \in H^2(\Omega)$。更多强形式解的讨论,可见\cite{Demkowicz:2006ww,Demkowicz:2007ur}。

\subsection{狄利克雷边界值问题2}
\label{sec:var-bvp-dirichlet-lagrange}

如第\ref{sec:var-mixed-formulations}节所述,狄利克雷边界值问题\eqref{eq:var-dbvp-problem}也可以改写为鞍点变分问题,即混合算子方程,共形导数对应拉格朗日乘子\citep{Babuska:1973gu, Bramble:1981vv}。

从格林第一恒等式\eqref{eq:var-bvp-green-1st-identity}入手,设拉格朗日乘子
\begin{equation*}
  \lambda \coloneqq \gamma_1^{\text{int}} u \in H^{-\frac{1}{2}}(\Gamma),
\end{equation*}

进而鞍点变分问题表示为,寻找解$(u,\lambda) \in H^{1}(\Omega) \times H^{-\frac{1}{2}}(\Gamma)$,使得满足
\begin{equation}
  \label{eq:var-bvp-saddle-problem}
  \begin{split}
    a(u,\nu) - b(\nu,\lambda) &= \langle f,\nu \rangle_{\Omega} \quad \forall \nu \in H^{1}(\Omega),\\
    b(u,\mu) &= \langle g, \mu \rangle_{\Gamma} \quad \forall \, \mu \in H^{-\frac{1}{2}}(\Gamma),
  \end{split}
\end{equation}
其中定义了一个新的双线性泛函算子
\begin{equation*}
  b(\nu,\mu) \coloneqq \langle \gamma_{0}^{\text{int}} \nu, \mu \rangle_{\Gamma}, \quad (\nu,\mu) \in H^{1}(\Gamma) \times H^{-\frac{1}{2}}(\Gamma).
\end{equation*}

\subsubsection{解的唯一存在性}
\label{sec:var-bvp-saddle-solution-uniq}
鞍点变分形式的狄利克雷边界值问题\eqref{eq:var-bvp-saddle-problem},解的存在性和唯一性,可由Theorem \ref{theorem:mixed-saddle-point-variational-problem}证得。使用该定理之前,需要确保两个前提条件得到满足。一是双线性泛函$a(.,.)$的椭圆特性,二是解的稳定性条件。
\begin{enumerate}
\item 椭圆性。类似于式\eqref{eq:var-bvp-operator-a-ellipticity},由Lemma \ref{lemma:var-bvp-operator-ellipticity-property} \eqref{eq:var-bvp-bilinear-a-nunu}可得
\begin{equation*}
  a(.,.) \ge c_1^A \, \big\| \cdot \big\|_{H^{1}(\Omega)}^2,
\end{equation*}
此外由于
\begin{equation*}
  \ker B \coloneqq \left\{
  \nu \in H^{1}(\Omega) : \langle \gamma_{0}^{\text{int}} \nu, \mu \rangle_{\Gamma} = 0, \quad \forall \, \mu \in H^{-\frac{1}{2}}(\Gamma)
  \right\} = H_{0}^{1}(\Omega),
\end{equation*}
我们因此有,$a(.,.)$是一个$\ker B$-椭圆(或$H_{0}^{1}$-椭圆)的双线性形。(通常来说,我们需要求得一个扩展双线性形$\tilde{a}(.,.)$,使得满足$H^1(\Omega)$-椭圆性质,相关讨论见第\ref{sec:var-bvp-saddle-modified}节。)

\item 解的稳定性条件可以表示为
\begin{equation}
  \label{sec:var-bvp-saddle-solution-stability}
  c_S \big\| \mu \big\|_{H^{-\frac{1}{2}}(\Gamma)} \le
  \sup_{0 \neq \nu \in H^{1}(\Omega)} \frac{
  \langle \gamma_{0}^{\text{int}} \nu, \mu \rangle_{\Gamma}
  }{
  \big\| \nu \big\|_{H^{1}(\Omega)}
  }, \quad \forall \, \mu \in H^{-\frac{1}{2}}(\Gamma),
\end{equation}
其证明见可见Lemma \ref{lemma:var-bvp-saddle-solution-stability}。

\item 应用定理Theorem \ref{theorem:mixed-saddle-point-variational-problem},求得鞍点变分问题的唯一解。
\end{enumerate}

\begin{lemma}[鞍点变分形式狄利克雷边界值问题解的稳定条件]
  \label{lemma:var-bvp-saddle-solution-stability}
  稳定条件\eqref{sec:var-bvp-saddle-solution-stability}成立。
\end{lemma}
\begin{proof}
  已知给定的任一$\mu \in H^{-\frac{1}{2}}(\Gamma)$。由里兹表现定理(Theorem \ref{theorem:var-riesz-representation-theorem})可得,存在唯一的一个$u_{\mu} \in H^{\frac{1}{2}}(\Gamma)$,满足
  \begin{equation*}
    \begin{split}
      & \langle u_{\mu}, \nu \rangle_{H^{\frac{1}{2}}(\Gamma)} = \langle \mu, \nu \rangle_{\Gamma} \quad \forall \, \nu \in H^{\frac{1}{2}}(\Gamma), \\
      & \big\| u_{\mu} \big\|_{H^{\frac{1}{2}}(\Gamma)} = \big\| \mu \big\|_{H^{\frac{1}{2}}(\Gamma)}.
    \end{split}
  \end{equation*}

  由逆迹定理Theorem \ref{theorem:sobolev-manifold-inverse-trace-theorem}可得,存在一个延拓算子$\varepsilon u_{\mu} \in H^{1}(\Omega)$,满足
  \begin{equation*}
    \big\| \varepsilon u_{\mu} \big\|_{H^{1}(\Omega)} \le c_{\text{IT}} \, \big\| u_{\mu} \big\|_{H^{\frac{1}{2}}(\Gamma)}.
  \end{equation*}

  那么,对于$\nu = \varepsilon u_{\mu} \in H^{1}(\Omega)$我们有
  \begin{equation*}
    \begin{split}
      & \frac{
      \langle \nu, \mu \rangle_{\Gamma}
      }{
      \big\| \nu \big\|_{H^{1}(\Omega)}
      }
      =
      \frac{
      \langle u_{\mu}, \mu \rangle_{\Gamma}
      }{
      \big\| \varepsilon u_{\mu} \big\|_{H^{1}(\Omega)}
      }
      =
      \frac{
      \langle u_{\mu}, u_{\mu} \rangle_{H^{\frac{1}{2}}(\Gamma)}
      }{\big\| \varepsilon u_{\mu} \big\|_{H^{1}(\Omega)}}\\
      & \ge \frac{1}{c_{\text{IT}}} \,
      \big\| u_{\mu} \big\|_{H^{\frac{1}{2}}(\Gamma)}
      = \frac{1}{c_{\text{IT}}} \,
      \big\| u \big\|_{H^{- \frac{1}{2}}(\Gamma)}
    \end{split}
  \end{equation*}

  $\therefore$稳定性条件\eqref{sec:var-bvp-saddle-solution-stability}成立。
\end{proof}

\subsubsection{调整鞍点变分问题}
\label{sec:var-bvp-saddle-modified}
需要注意的是,在鞍点变分问题\eqref{eq:var-dbvp-problem}中的双线性形式算子$a(.,.)$是$H_{0}^{1}(\Omega)$-椭圆的。我们常常需要将它扩展为一个$H^1(\Omega)$-椭圆的算子$\tilde{a}(.,.)$,对应新的调整鞍点变分问题。调整思路如下:

已知,用拉格朗日乘子$\lambda \coloneqq \gamma_{1}^{\text{int}}u \in H^{-\frac{1}{2}}(\Gamma)$来描述问题解$u$的共形导数,那么利用格林第二恒等式\eqref{eq:bvp-a-nu-u-green-2nd-identity},可得狄利克雷边界值问题的正交条件\eqref{eq:bvp-neumann-green-2}
\begin{equation}
\begin{split}
  \label{eq:var-dbvp-modified-orthogonality-condition}
  &\int_{\Omega} f(x) \, dx + \int_{\Gamma} \lambda(x) d s_x = 0, \\
  \hookrightarrow &\int_{\Gamma} \lambda(x) d s_x \, \int_{\Gamma} \mu(x) \, d s_x = - \int_{\Omega} f(x) \, dx \, \int_{\Gamma} \mu(x) \, d s_x , \quad \forall \, \mu \in H^{-\frac{1}{2}}(\Gamma).
\end{split}
\end{equation}

另一方面,根据狄利克雷边界条件有
\begin{equation}
  \label{eq:var-dbvp-modified-dirichlet-condition}
  \begin{split}
    &\gamma_{0}^{\text{int}} u = g, \\
    \hookrightarrow & \int_{\Gamma} \gamma_{0}^{\text{int}} u(x) \, d s_x
    = \int_{\Gamma} g(x) \, d s_x, \\
    \hookrightarrow & \int_{\Gamma} \gamma_{0}^{\text{int}} u(x) \, d s_x \,
    \int_{\Gamma} \gamma_{0}^{\text{int}} \nu(x) \, d s_x
    = \int_{\Gamma} g(x) \, d s_x \,
    \int_{\Gamma} \gamma_{0}^{\text{int}} \nu(x) \, d s_x, \quad \forall \, \nu \in H^{1}(\Omega).
  \end{split}
\end{equation}

  将\eqref{eq:var-dbvp-modified-orthogonality-condition}、\eqref{eq:var-dbvp-modified-dirichlet-condition}代入\eqref{eq:var-bvp-saddle-problem},得到调整鞍点变分问题:寻找解$(u,\lambda) \in H^{1}(\Omega) \times H^{-\frac{1}{2}}(\Gamma)$,使得$\forall \, (\nu, \mu) \in H^{1}(\Omega) \times H^{-\frac{1}{2}}(\Gamma)$均满足
  \begin{equation}
    \label{eq:var-bvp-saddle-modified-problem}
    \begin{split}
      \underbrace{
      \int_{\Gamma} \gamma_{0}^{\text{int}} u(x) \, d s_x \,
    \int_{\Gamma} \gamma_{0}^{\text{int}} \nu(x) \, d s_x + a(u,\nu)
    }_{\eqqcolon \tilde{a}(u,\nu)}
    - b(\nu,\lambda) &= \langle f,\nu \rangle_{\Omega}  + \int_{\Gamma} g(x) \, d s_x \,
    \int_{\Gamma} \gamma_{0}^{\text{int}} \nu(x) \, d s_x,\\
      b(u,\mu) + \int_{\Gamma} \lambda(x) d s_x \, \int_{\Gamma} \mu(x) \, d s_x  &= \langle g, \mu \rangle_{\Gamma} - \int_{\Omega} f(x) \, dx \, \int_{\Gamma} \mu(x) \, d s_x .
    \end{split}
  \end{equation}

下面的问题就是,调整鞍点变分问题\eqref{eq:var-bvp-saddle-modified-problem}的解是否存在,是否唯一,以及是否与原鞍点变分问题\eqref{eq:var-bvp-saddle-problem}的解一致。换句话说,两个鞍点变分问题是否等价。
\begin{theorem}[变分问题等价]
  \label{theorem:var-dbvp-saddle-equivalance}
  调整鞍点变分问题\eqref{eq:var-bvp-saddle-modified-problem}有唯一的解$(u,\lambda) \in H^{1}(\Omega) \times H^{-\frac{1}{2}}(\Gamma)$,并且与鞍点变分问题\eqref{eq:var-bvp-saddle-problem}的解一致。即,两个变分问题等价。
\end{theorem}
\begin{proof}
  \begin{enumerate}
  \item 证明扩展双线性形$\tilde{a}(u,\nu)$有界。

  $a(u,\nu)$有界$\Rightarrow$
  \begin{equation*}
    \tilde{a}(u,\nu) \coloneqq \int_{\gamma} \gamma_{0}^{\text{int}} u(x) d s_x \, \int_{\gamma} \gamma_{0}^{\text{int}} \nu(x) d s_x +  a(u,\nu), \quad \forall \, u,\nu \in H^{1}(\Omega)
  \end{equation*}
  有界。

  \item 由$a(.,.)$的半椭圆性质(Lemma \ref{lemma:var-bvp-operator-ellipticity-property})和\eqref{eq:sobolev-equivalence-norm-w12omega}有
  \begin{equation*}
\begin{split}
  \tilde{a}(\nu,\nu) &=
  \left[
  \int_{\Gamma} \gamma_{0}^{\text{int}} \nu(x) \, d s_x
  \right]^2 + a(\nu,\nu) \\
  & \ge \min\{1,\lambda_0\} \, \big\| \nu \big\|_{W^{1,2}(\Omega), \Gamma}^2 \\
  & \ge c_1^{\tilde{A}} \, \big\| \nu \big\|_{H^1(\Omega)}^2, \quad \forall \, \nu \in H^{1}(\Omega),
\end{split}
  \end{equation*}
  由此可得$\tilde{a}(u,\nu)$是$H^{1}(\Omega)$-椭圆。

  \item 前提条件得到满足,可通过Theorem \ref{theorem:mixed-saddle-point-variational-problem}、Theorem \ref{theorem:mixed-saddle-point-variational-problem-solution}证得,调整鞍点变分问题\eqref{eq:var-bvp-saddle-modified-problem}有唯一解。

  \item 对于$(\nu,\mu) \equiv (1,1)$的特殊情况,\eqref{eq:var-bvp-saddle-modified-problem}变为
  \begin{equation*}
    \begin{split}
      \big| \Gamma \big| \, \int_{\Gamma} \gamma_{0}^{\text{int}} u(x) \, d s_x - \int_{\Gamma} \lambda(x) \, d s_x &= \int_{\Omega} f(x) \, d x + \big| \Gamma \big| \, \int_{\Gamma} g(x) \, d s_x,\\
      \int_{\Gamma} \gamma_{0}^{\text{int}} u(x) \, d s_x
      + \big| \Gamma \big| \, \int_{\Gamma} \lambda(x) \, d s_x &=
      \int_{\Gamma} g(x) \, d s_x, - \big| \Gamma \big| \, \int_{\Omega} f(x) \, d x.
    \end{split}
  \end{equation*}
进而
  \begin{equation*}
    \left( 1 + \big| \Gamma \big|^2 \right) \, \int_{\Gamma} \gamma_{0}^{\text{int}} u(x) d s_x = \left( 1 + \big| \Gamma \big|^2 \right) \, \int_{\Gamma} g(x) \, d s_x
  \end{equation*}
即\eqref{eq:var-dbvp-modified-dirichlet-condition}。从而有

\begin{equation*}
  \big| \Gamma \big\| \int_{\Gamma} \lambda(x) \, d s_x = - \big| \Gamma \big\| \int_{\Gamma} f(x) \, d s_x ,
\end{equation*}
即\eqref{eq:var-dbvp-modified-orthogonality-condition}。

因此可见,$(u,\lambda)$也是鞍点变分问题\eqref{eq:var-bvp-saddle-problem}的解的解;两个问题等价。
\end{enumerate}
\end{proof}

\subsection{诺依曼边界值问题}
\label{sec:var-nbvp-problem}


\subsubsection{将诺依曼边界值问题改写为变分问题}
回顾一下第\ref{sec:bem-fem-potential-bvp}节的诺依曼边界值问题\index{Neumann boundary value condition \dotfill 诺依曼边界值条件}\eqref{eq:bvp-extension-omega-cond}-\eqref{eq:bvp-extension-gamma-neumann}:基于给定的$f$和$g$,求解
\begin{equation}
  \label{eq:var-nbvp-problem}
  \begin{split}
    & \left( L u\right)(x) = f(x), \quad x\in \Omega,\\
    & \gamma_{1}^{\text{int}} u(x) = g(x), \quad x \in \Gamma.
  \end{split}
\end{equation}

假定$f$和$g$满足可求解性条件 \eqref{eq:bvp-neumann-green-2-new}
\begin{equation}
  \label{eq:var-nbvp-solvability-cond}
  \int_{\Omega} f(x) \, dx + \int_{\Gamma} g(x) \, d s_x = 0.
\end{equation}

基于前文的分析可见,诺依曼边界值问题\eqref{eq:var-nbvp-problem}的解$u \in H^{1}(\Omega)$将与某一个常数有关。为了将此常数予以确定,可以在$H^{1}(\Omega)$中定义一个测试空间$H_{*}^{1}(\Omega)$,测试方程$\nu(x)$用于规模调节。
\begin{equation}
  \label{eq:var-nbvp-trial-space}
  H_{*}^{1} (\Omega) \coloneqq \left\{ \nu \in H^{1}(\Omega): \int_{\Omega} \nu(x) \, dx = 0 \right\}.
\end{equation}

从而构建变分问题,求解$u \in H_{*}^{1}(\Omega)$使得满足
\begin{equation}
  \label{eq:var-nbvp-variational-problem}
  a(u,\nu) = \langle f, \nu \rangle_{\Omega} + \langle g, \gamma_{0}^{\text{int}} \nu \rangle_{\Gamma}, \quad \forall \, \nu \in H_{*}^{1}(\Omega).
\end{equation}

\subsubsection{解的存在性与唯一性}
\begin{theorem}[诺依曼边界值问题的变分法求解]
  给定$f \in \tilde{H}^{-1}(\Omega), g \in H^{-\frac{1}{2}}(\Gamma)$,满足可求解性条件\eqref{eq:var-nbvp-solvability-cond}。

  则诺依曼边界值的变分问题\eqref{eq:var-nbvp-variational-problem}存在唯一的解$u \in H_{*}^{1}(\Omega)$,满足
  \begin{equation*}
    \big\| u \big\|_{H^{1}(\Omega)} \le \frac{1}{\tilde{c}_1^A}
    \left\{
    \big\| f \big\|_{\widetilde{H}^{-1}(\Omega)} +
    c_{T} \, \big\| g \big\|_{H^{-\frac{1}{2}}(\Gamma)}.
    \right\}
  \end{equation*}
\end{theorem}

\begin{proof}
\begin{enumerate}
\item 证$a(.,.)$有界且椭圆。

对于$\nu \in H_{*}^{1}(\Omega)$,它在$H^{1}(\Omega)$中的等价范,可由\eqref{eq:sobolev-equivalence-norm-w12omega}求得
\begin{equation*}
  \big\| \nu \big\|_{W^{1,2}(\Omega), \Omega} = \left\{
  \left[
  \int_{\Omega} \nu(x) \, dx
  \right]^2
  + \big\| \triangledown \nu \big\|^2_{L^2(\Omega)}
   \right\}^{\frac{1}{2}},
\end{equation*}
进而利用Lemma \ref{lemma:var-bvp-operator-ellipticity-property}可得,
\begin{equation}
  \label{eq:var-nbvp-ellipticity}
  a(\nu,\nu) \ge \lambda_0 \, \big\| \triangledown \nu \big\|_{W^{1,2}(\Omega), \Omega}^2 \ge \tilde{c}_{1}^{A} \, \big\| \nu \big\|_{H^{1}(\Omega)}^2, \quad \forall \nu \in H_{*}^{1}(\Omega),
\end{equation}
因此可得双线性形式$a(.,.)$的$H_{*}^{1}$-椭圆特性。

\item 根据拉克斯一密格拉蒙定理(Theorem \ref{theorem:lax-milgram-lemma}),
证明变分问题\eqref{eq:var-nbvp-variational-problem}存在唯一解。

\item 将求得的变分问题唯一解$u \in H_{*}^{1}(\Omega)$代回椭圆条件
\eqref{eq:var-nbvp-ellipticity}
\begin{equation*}
  \begin{split}
    \widetilde{c}_{1}^{A} \, \big\| u \big\|_{H^1(\Omega)}^2 & \le a(u,u) = \langle f, u \rangle_{\Omega} + \langle g, \gamma_{0}^{\text{int}} u \rangle_{\Gamma} \\
    & \le \big\| f \big\|_{\widetilde{H}^{-1}(\Omega)} \,
    \big\| u \big\|_{H^{1}(\Omega)}
    + \big\| g \big\|_{H^{- \frac{1}{2}}(\Gamma)} \,
    \big\| \gamma_{0}^{\text{int}} u \big\|_{H^{\frac{1}{2}}(\Gamma)}.
  \end{split}
\end{equation*}

由迹定理 Theorem \ref{theorem:sobolev-manifold-trace-theorem}得,上式变为
\begin{equation*}
  \begin{split}
    \widetilde{c}_{1}^{A} \, \big\| u \big\|_{H^1(\Omega)}
    \le \big\| f \big\|_{\widetilde{H}^{-1}(\Omega)} + \big\| g \big\|_{H^{-\frac{1}{2}}(\Gamma)} \,
    c_{T} \big\| u \big\|_{H^{1}(\Gamma)}.
  \end{split}
\end{equation*}
\end{enumerate}
\end{proof}




\subsubsection{鞍点变分问题}
类似地,我们也可以构建一个与\eqref{eq:var-nbvp-variational-problem}等价的鞍点变分问题。此时,用于规模调节的测试空间$H_{*}^{1}(\Omega)$以副条件(side condition)的情况出现。使用一个拉格朗日乘子,寻找解$(u,\lambda) \in H^{1}(\Omega) \times \mathbb{R}$,满足
\begin{equation}
  \label{eq:var-nbvp-saddle-var}
  \begin{split}
    & a(u,\nu) + \lambda \int_{\Omega} \nu(x) \, d x =
    \langle f, \nu \rangle_{\Omega} +
    \langle g, \gamma_{0}^{\text{int}} \nu \rangle_{\Gamma}, \\
    & \int_{\Omega} u(x) \, dx = 0, \quad \forall \, \nu \in H^{1}(\Omega).
  \end{split}
\end{equation}

\begin{theorem}[诺依曼边界值问题的鞍点变分法求解]
  \label{theorem:var-nbvp-saddle-var-solution}
  诺依曼边界值的鞍点变分问题\eqref{eq:var-nbvp-saddle-var}有唯一解$(u,\lambda) \times H^{1}{\Omega} \times \mathbb{R}$。
\end{theorem}
\begin{proof}
  \begin{enumerate}
    \item 双线性形式$b(.,.)$有界
    \begin{equation*}
      b(\nu,\mu) \coloneqq \mu \, \int_{\Omega} \nu(x) \, dx, \quad \forall \, \nu \in H^{1}(\Omega), \mu \in \mathbb{R},
    \end{equation*}
并且有$\ker B = H_{*}^{1}(\Omega)$。

    \item 进而由椭圆性\eqref{eq:var-nbvp-ellipticity}得,双线性形$a(.,.)$是$\ker B$-椭圆的。

\item 证明满足稳定性条件
\begin{equation}
  \label{eq:var-nbvp-saddle-var-stability}
  c_{S} \, \big| \mu \big| \le \sup_{0 \neq \nu \in H^{1}(\Omega)}
  \frac{
  b(\nu,\mu)
  }{
  \big\| \nu \big\|_{H^{1}(\Omega)}
  }, \quad \forall \, \mu \in \mathbb{R}.
\end{equation}

对于任一给定的$\mu \in \mathbb{R}$, 定义$\nu^{*} \coloneqq \nu \in H^{1}(\Omega)$,可以证得\eqref{eq:var-nbvp-saddle-var-stability},其中
$ c_S = | \Omega |^{-\frac{1}{2}}$。

\item 根据定理Theorem \ref{theorem:mixed-saddle-point-variational-problem}可求得,诺依曼边界值的鞍点变分问题存在唯一解$(u,\lambda) \in H^{1}(\Omega) \times \mathbb{R}$。

\item 对于测试方程$\nu \equiv 1$的特殊情况,根据诺依曼边界值问题的可求解条件\eqref{eq:var-nbvp-solvability-cond}我们有拉格朗日乘子的值
\begin{equation*}
  \lambda = 0.
\end{equation*}
  \end{enumerate}
\end{proof}



\subsubsection{调整鞍点变分问题}
如前所述,鞍点变分问题\eqref{eq:var-nbvp-saddle-var}也可以改写为一个新的鞍点变分问题:基于给定的任一$f \in \widetilde{H}^{-1}(\Omega), g \in H^{-\frac{1}{2}}(\Gamma)$,寻找$(u,\lambda) \in H^{1}(\Omega) \times \mathbb{R}$,使满足
\begin{equation}
\begin{split}
  \label{eq:var-nbvp-saddle-modified-var}
  &a(u,\nu) + \lambda \int_{\Omega} \nu(x) \, dx = \langle f, \nu \rangle_{\Omega} + \langle g, \gamma_{0}^{\text{int}} \nu \rangle_{\Gamma}, \\
  & \int_{\Gamma} u(x) \, dx - \lambda = 0, \quad \forall \, \nu \in H^{1}(\Omega).
\end{split}
\end{equation}

利用第二行等式求得拉格朗日乘子$\lambda \in \mathbb{R}$代入第一行,我们得到一个新的调整变分问题,寻找$u \in H^{1}(\Omega)$使满足
\begin{equation}
  \label{eq:var-nbvp-saddle-modified}
  a(u,\nu) + \int_{\Omega}  u(x)  dx \, \int_{\Omega} \nu(x) dx =
  \langle f,\nu \rangle_{\Omega} +
  \langle g, \gamma_{0}^{\text{int}} \nu \rangle_{\Gamma}, \quad \forall \, \nu \in H^{1}(\Omega).
\end{equation}

\begin{theorem}[诺依曼边界值的调整变分问题解]
  \label{theorem:var-nbvp-saddle-modified-solution}
  基于给定的任一$f \in \widetilde{H}^{-1}(\Omega), g \in H^{-\frac{1}{2}}(\Gamma)$,调整变分问题\eqref{eq:var-nbvp-saddle-modified}有唯一的解$u \in H^{1}(\Omega)$。

  如果给定的$f \in \widetilde{H}^{-1}(\Omega), g \in H^{-\frac{1}{2}}(\Gamma)$满足可求解条件\eqref{eq:var-nbvp-solvability-cond},那么调整变分问题\eqref{eq:var-nbvp-saddle-modified}的解$u \in H_{*}^{1}(\Omega)$;换句话说,调整变分问题\eqref{eq:var-nbvp-saddle-modified}和变分问题\eqref{eq:var-nbvp-saddle-var}等价。
\end{theorem}

\begin{proof}
  \begin{enumerate}
  \item 由\eqref{eq:var-nbvp-saddle-modified}可见,调整双线性形$\widetilde{a}(.,.)$写为
  \begin{equation*}
    \widetilde{a}(u,\nu) \coloneqq a(u,\nu) + \int_{\Omega} u(x) \, dx \, \int_{\Omega} \nu(x) \, dx.
  \end{equation*}

  由$a(.,.)$的半椭圆属性Lemma \ref{lemma:var-bvp-operator-ellipticity-property}我们有
  \begin{equation}
    \label{eq:var-nbvp-modified-ellipticity}
  \begin{split}
      \widetilde{a}(\nu,\nu) &\ge \lambda_0 \, \big\| \triangledown \nu \big\|_{L^{2}(\Omega)}^2 + \left[ \int_{\Omega} \nu(x) dx \right]^2\\
      &\ge \min\{\lambda_0, 1\} \, \big\| \nu \big\|_{W^{1,2}(\Omega), \Omega}^2 \\
      & \ge \hat{c}_1^A \, \big\| \nu \big\|_{W^{1,2}(\Omega), \Omega}^2, \quad \forall \, \nu \in H^{1}(\Omega),
  \end{split}
  \end{equation}
  可得$\widetilde{a}(.,.)$是$H^{1}(\Omega)$-椭圆且有界的。

  \item 满足前提条件后,可由拉克斯一密格拉蒙定理(Theorem \ref{theorem:lax-milgram-lemma})证得,调整变分问题\eqref{eq:var-nbvp-saddle-modified}有唯一解$u \in H^{1}(\Omega)$。

  \item 设测试方程$\nu(x) \equiv 1$。调整变分问题\eqref{eq:var-nbvp-saddle-modified}变为
  \begin{equation*}
    \begin{split}
      \big| \Omega \big| \, \int_{\Omega} u(x) \, dx &=
      \langle f, 1 \rangle_{\Omega}
      + \langle g, 1 \rangle_{\Gamma}, \\
      & = \int_{\Omega} f(x) \, d x + \int_{\Gamma} g(x) \, d s_x = 0,
    \end{split}
  \end{equation*}
  其中最后一个等式用到可求解性条件\eqref{eq:var-nbvp-solvability-cond}。由此我们有$u \in H_{*}^{1}(\Omega)$,也是变分问题\eqref{eq:var-nbvp-saddle-var}的解;换句话说,两个问题等价。
\end{enumerate}
\end{proof}

\subsubsection{诺依曼边界值问题的通解}
利用变分法求解诺依曼边界值问题\eqref{eq:var-nbvp-problem},所得到的$u \in H_{*}^{1}(\Omega)$是弱形式解。更一般意义上的通解$\widetilde{u} \in H^{1}(\Omega)$可写为
\begin{equation*}
  \widetilde{u} \coloneqq u + \alpha,
\end{equation*}
其中$\alpha \in \mathbb{R}$是任意常数。
