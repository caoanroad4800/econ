%!TEX root = ../DSGEnotes.tex
\chapter{有界元法和有限元法}
\label{sec:bem-fem-methods}

\section{边界值问题:位势方程}

我们从二阶偏微分方程入手,介绍边界值问题(boundary value problem)\index{boundary value problem \dotfill 边界值问题}。一个合适的例子是位势方程(potential equation)。

\subsection{偏微分算子及椭圆边界值问题}

定义有界域$\Omega \in \mathbb{R}^d, d=2,3$,边界$\Gamma = \partial \Omega$,外代数单位向量空间(exterior unit normal vector)\index{exterior algebra \dotfill 外代数} $\underline{n}(x)$对于$x \in \Gamma$几乎处处存在。对于$x \in \Omega$,我们考虑一个线性二阶偏微分的自伴随算子\footnote{有限维内积向量空间$V$中,自伴随算子A是一个从$V$到$V$自身的线性映射$\langle A \bm{u}, \bm{\nu} \rangle = \langle \bm{\nu}, A \bm{u} \rangle, \, \forall \nu, w \in V$。}
(self-adjoint operator)\index{self-adjoint operator} $L$
,作用于实值标量方程$u$
\begin{equation}
  \label{eq:bvp-self-adjoint-pde-operator}
  \left( L \, u \right)(x) \coloneqq - \sum_{i,j=1}^d \frac{\partial}{\partial x_j} \left[ a_{ji} (x) \frac{\partial}{\partial x_i} u(x)\right] + a_0(x)\, u(x),
\end{equation}
其中$a_{ji}(x), \, i,j =1,\ldots, d, \, x \in \Omega$表示系数方程,假定为平滑的并满足$a_{ij}(x) = a_{ji}(x)$。由此可以构建一个对称的系数矩阵$A(x)$,满足
\begin{equation*}
  A(x) = \left( a_{ij}(x) \right)_{i,j=1}^{d}, \quad x \in \Omega,
\end{equation*}
对应实数特征根$\lambda_{k}(x)$。

当且仅当$\lambda_{k}(x) > 0$对于所有$k=1,\ldots,d$都成立时,我们称偏微分算子$L$在某一个$x \in \Omega$上是椭圆(elliptic)的。

更进一步,如果$\forall x \in \Omega$该条件都成立,那么我们称$L$在$\Omega$上是椭圆的。

如果存在一个一致下界(uniform lower bound) $\lambda_0 > 0$,满足
\begin{equation*}
  \lambda_k (x) \ge \lambda_0, \quad \forall k = 1,\ldots,d, \, \forall x \in \Omega,
\end{equation*}
那么我们称$L$在$\Omega$上一致椭圆。

\subsection{边界条件}

边界条件的分析,可以从散度定理开始。
\begin{theorem}[散度定理]
  \label{theorem:bvp-gauss-divergence-theorem}
  散度定理(divergence theorem)\index{divergence theorem \dotfill 高斯散度定理},又称奥斯特罗格拉德斯基——高斯定理(Ostrogradsky-Gauss theorem)\index{Ostrogradsky-Gauss theorem \dotfill 奥斯特罗格拉德斯基——高斯定理} 、高斯散度定理(Gauss' theorem)\index{Gauss theorem \dotfill 高斯散度定理}等,是指
  \begin{equation}
    \label{eq:bvp-gauss-divergence-theorem}
    \int_{\Omega} \frac{\partial}{\partial x_i} f(x) \, dx = \int_{\Gamma} \left[ \gamma_0^{int} f(x) \right] n_i(x) \, d s_x, \quad i = 1,\ldots,d,
  \end{equation}
  其中$\gamma_0^{int} f(x)$是某个给定方程$f(x), x\in \Omega$的内界迹(interior boundary trace),满足
  \begin{equation}
    \label{eq:bvp-interior-boundary-trace}
    \gamma_0^{int} f(x) \coloneqq \lim_{\Omega \owns \tilde{x} \mapsto x \in \Gamma} f \left( \tilde{x} \right), \quad \forall x \in \Gamma = \partial \Omega.
  \end{equation}
\end{theorem}

假定两个足够光滑的方程$u,\nu \in \Omega$,通过设定$f(x) = u(x) \, \nu(x)$,可以将散度定理\eqref{eq:bvp-gauss-divergence-theorem}改写为分部积分(integration by parts)\index{integration by parts \dotfill 分部积分}的形式
\begin{equation*}
  \int_{\Omega} u(x) \frac{\partial}{\partial x_i} \nu(x) \, x
  + \int_{\Omega} \nu(x) \frac{\partial}{\partial x_i} u(x) \, x
  = \int_{\Gamma}  \left[ \gamma_0^{int} u(x) \right] \left[ \gamma_0^{int} \nu(x) \right] n_i(x) \, d s_x.
\end{equation*}

重新调整上式,将$\nu(x)$视作检测方程(test function),两侧乘以\eqref{eq:bvp-self-adjoint-pde-operator}中的二阶偏微分算子$\left(L\,u\right)(x)$,在$\Omega$中求积
\begin{equation}
  \begin{split}
    &\left( L \, u \right)(x) \, \nu(x) \coloneqq - \sum_{i,j=1}^d \frac{\partial}{\partial x_j} \left[ a_{ji} (x) \frac{\partial}{\partial x_i} u(x)\right] \, \nu(x) + a_0(x)\, \underbrace{u(x) \, \nu(x)}_{\equiv f(x)}, \\
    \hookrightarrow & \int_{\Omega} \left( L \, u \right)(x) \, \nu(x) \, dx = - \sum_{i,j=1}^d \int_{\Omega} \frac{\partial}{\partial x_j} \left[ a_{ji} (x) \frac{\partial}{\partial x_i} u(x)\right] \, \nu(x) \, dx,
  \end{split}
\end{equation}
使用分部积分$\hookrightarrow$
\begin{equation*}
  \begin{split}
    \int_{\Omega} \left( L \, u \right)(x) \, \nu(x) \, dx =&  \underbrace{\sum_{i,j=1}^d \int_{\Omega} a_{ji}(x) \frac{\partial}{\partial x_i} u(x) \, \frac{\partial}{\partial x_j} \nu(x) \, dx}_{\coloneqq a\left(u,\nu \right)} \\
   &- \sum_{i,j=1}^{d} \int_{\Gamma} n_j(x) \left[ \gamma_0^{int} (x) \left( a_{ji}(x) \frac{\partial}{\partial x_i} u(x)\right) \right] \left[ \gamma_0^{int} \nu(x) \right] \, d s_x,
  \end{split}
\end{equation*}

由此,我们由散度定理(Theorem \ref{theorem:bvp-gauss-divergence-theorem})推导出格林第一恒等式(Green's first identity)\index{Green identities!first 格林第一恒等式}
\begin{equation}
  \label{eq:bvp-a-u-nu-inner-prod}
  \begin{split}
  a\left(u,\nu \right) &\coloneqq \sum_{i,j=1}^d \int_{\Omega} a_{ji}(x) \frac{\partial}{\partial x_i} u(x) \, \frac{\partial}{\partial x_j} \nu(x) \, dx \\
  & = \int_{\Omega} \left( L \, u \right)(x) \, \nu(x) \, dx + \sum_{i,j=1}^{d} \int_{\Gamma} \underbrace{n_j(x) \left[ \gamma_0^{int} (x) \left( a_{ji}(x) \frac{\partial}{\partial x_i} u(x)\right) \right]} \left[ \gamma_0^{int} \nu(x) \right] \, d s_x,\\
  & =\int_{\Omega} \left( L \, u \right)(x) \, \nu(x) \, dx + \int_{\Gamma} \underbrace{\left[ \gamma_1^{int} u(x) \right]} \left[ \gamma_0^{int} \nu(x) \right] \, d s_x,
  \end{split}
\end{equation}
其中定义$\gamma_1^{int}$为内部共形导数(interior co-normal derivative, \cite{Mikhailov:2006vo, Mikhailov:2009wj, Ancona:2009bo})
\begin{equation}
  \label{eq:bvp-int-conformal-derivative}
  \gamma_1^{int}u(x) \coloneqq \lim_{\Omega \owns \tilde{x} \mapsto x \in \Gamma} \left[
\sum_{i,j=1}^{d} n_j(x) a_{ji}\left( \tilde{x} \right) \frac{\partial}{\partial \tilde{x}_i} u \left( \tilde{x} \right)
  \right], \quad x \in \Gamma.
\end{equation}

将格林第一恒等式\eqref{eq:bvp-a-u-nu-inner-prod}中的$u,\nu$互换位置,我们有
\begin{equation*}
  \begin{split}
  a\left(\nu,u \right) = \int_{\Omega} \left( L \, \nu \right)(x) \, u(x) \, dx + \int_{\Gamma} \left[ \gamma_1^{int} \nu(x) \right] \left[ \gamma_0^{int} u(x) \right] \, d s_x,
  \end{split}
\end{equation*}
由上式和\eqref{eq:bvp-a-u-nu-inner-prod}我们有格林第二恒等式(Green's second identity)\index{Green identities!second 格林第二恒等式}: $\forall u,\nu \in \Omega$且$u,\nu$足够平滑
\begin{equation}
  \label{eq:bvp-a-nu-u-green-2nd-identity}
  \begin{split}
    &a(u,\nu) = a(\nu,u) \Longleftrightarrow \\
    &\int_{\Omega} \left( L \, u \right)(x) \, \nu(x) \, dx + \int_{\Gamma} \left[ \gamma_1^{int} u(x) \right] \left[ \gamma_0^{int} \nu(x) \right] \, d s_x
    = \int_{\Omega} \left( L \, \nu \right)(x) \, u(x) \, dx + \int_{\Gamma} \left[ \gamma_1^{int} \nu(x) \right] \left[ \gamma_0^{int} u(x) \right] \, d s_x
  \end{split}
\end{equation}

下面来考虑一个特殊情况,$a_{ij}(x)=\delta_{ij}$,$\delta_{ij}$是克罗内克乘积(Kronecker product)\index{Kronecker product \dotfill 克罗内乘积}。\eqref{eq:bvp-self-adjoint-pde-operator}的二阶偏微分算子$\left(L\,u\right)(x)$变为拉普拉斯算子
\begin{equation}
  \label{eq:bvp-laplace-operator}
  \left( L \, u \right)(x) = - \Delta u(x) \coloneqq - \sum_{i=1}^{d} \frac{\partial^2}{\partial x_i^2} u(x), \quad x \in \mathbb{R}^d.
\end{equation}

内部共形导数$\gamma_1^{int}$ \eqref{eq:bvp-int-conformal-derivative}变为
\begin{equation}
  \label{eq:bvp-laplace-conformal-derivative}
  \gamma_1^{int}u(x) = \frac{\partial}{\partial n_x} u(x) \coloneqq \underline{n}(x) \bigtriangledown u(x), \quad x \in \Gamma.
\end{equation}

对边界域$\Gamma = \partial \Omega$分解成三个不相交集合的并集(disjoint union)
\begin{equation*}
  \Gamma = \overline{\Gamma}_D \cup \overline{\Gamma}_N \cup \overline{\Gamma}_R,
\end{equation*}

对应地,边界值问题变为两部分:第一部分,在$\Omega$中,基于给定的方程$f(x)$,寻找偏微分算子$(L u)(x)$,使得
\begin{equation}
  \label{eq:bvp-extension-omega-cond}
  \left( L \, u \right)(x) = f(x), \quad x \in \Omega.
\end{equation}

第二部分,在$\Gamma$中,基于给定的方程$g(x)$,寻找内界迹$\gamma_0^{int}u(x)$或者内共形导数$\gamma_1^{int}(x)$。随着$\Gamma$的取值范围不同,分为三种情况:
\begin{subequations}
  \begin{equation}
    \label{eq:bvp-extension-gamma-dirichlet}
    \gamma_0^{int} u(x) = g_D(x), \quad x \in \Gamma = \Gamma_D,
  \end{equation}
  \begin{equation}
    \label{eq:bvp-extension-gamma-neumann}
    \gamma_1^{int} u(x) = g_N(x), \quad x \in \Gamma = \Gamma_N,
  \end{equation}
  \begin{equation}
    \label{eq:bvp-extension-gamma-robin}
    \kappa(x) \, \gamma_0^{int} u(x) + \gamma_1^{int} u(x) = g_R(x), \quad x \in \Gamma = \Gamma_R.
  \end{equation}
\end{subequations}


\begin{definition}[边界值条件]
  \label{definition:boundary-value-problem}
  于是我们有以下几种不同的边界值条件:
\begin{itemize}
  \item $\Gamma = \Gamma_D:$ \eqref{eq:bvp-extension-omega-cond} + \eqref{eq:bvp-extension-gamma-dirichlet} $\rightarrow$ 狄利克雷边界值条件(Dirichlet boundary value condition)\index{Dirichlet boundary value condition \dotfill 狄利克雷边界值条件},
  \item $\Gamma = \Gamma_N:$ \eqref{eq:bvp-extension-omega-cond} + \eqref{eq:bvp-extension-gamma-neumann} $\rightarrow$ 诺依曼边界值条件(Neumann boundary value condition)\index{Neumann boundary value condition \dotfill 诺依曼边界值条件},
  \item $\Gamma = \Gamma_R:$ \eqref{eq:bvp-extension-omega-cond} + \eqref{eq:bvp-extension-gamma-robin} $\rightarrow$ 罗宾边界值条件(Robin boundary value condition)\index{Robin boundary value condition \dotfill 罗宾边界值条件},
  \item 混合型边界值条件,以上三种情况的组合。
\end{itemize}
\end{definition}

有时候我们还需要将线性罗宾边界值条件扩展为非线性的情况,\eqref{eq:bvp-extension-gamma-robin} $\rightarrow$
\begin{equation}
  \label{eq:bvp-extension-gamma-robin-nonlinear}
  G\left( \gamma_0^{int} u(x), x \right) + \gamma_1^{int} u(x) = g_R(x), \quad x \in \Gamma = \Gamma_R,
\end{equation}
其中$G(u,\cdot)$是某个给定的非线性方程,如$u(x)^3$。

对于边界值问题的解$u(x)$,还需要注意以下几点
\begin{enumerate}
  \item $u(x)$的存在性和唯一性,相关讨论可参考如\cite{Ladyzhenskaya:1968vq},
  \item 观测到的数据需要是充分平滑的,以确保$u(x)$充分可微(sufficiently differentiable)
  \begin{equation*}
    u \in C^2(\Omega) \cap C^1 \left( \Omega \cup \Gamma_N \cup \Gamma_R \right) \cap C(\Omega \cup \Gamma_D).
  \end{equation*}
\end{enumerate}

\subsection{诺依曼边界值问题}
对于诺依曼边界值条件的解,其存在性和唯一性需要做进一步讨论。

假定$\nu_1(x)=1, x \in \Omega$是关于$\nu_1(x)$的齐次诺依曼边界值问题的一个解,\eqref{eq:bvp-laplace-operator}、\eqref{eq:bvp-laplace-conformal-derivative} $\Rightarrow$
\begin{equation}
  \label{eq:bvp-neumann-nu-homo}
  \begin{split}
    \left( L \, \nu_1 \right)(x)=0, \quad x \in \Omega,\\
    \gamma_1^{int} \nu_1(x) = 0, \quad x \in \Gamma.
  \end{split}
\end{equation}

则关于$u(x)$的诺依曼边界值问题可以描述如下:\eqref{eq:bvp-neumann-nu-homo}$\rightarrow$格林第二恒等式\eqref{eq:bvp-a-nu-u-green-2nd-identity} $\Rightarrow$
\begin{equation}
  \label{eq:bvp-neumann-green-2}
  \int_{\Omega} \left( L \, u \right)(x) \, dx + \int_{\Gamma} \gamma_1^{int} u(x) \, d s_x = 0,
\end{equation}

诺依曼边界值条件\eqref{eq:bvp-extension-omega-cond}、 \eqref{eq:bvp-extension-gamma-neumann}$\Rightarrow$
\begin{equation}
  \label{eq:bvp-neumann-cond}
\begin{split}
  \left( L u \right)(x) = f(x), \quad x \in \Omega, \\
  \gamma_1^{int} u(x) = g_N(x), \quad x \in \Gamma.
\end{split}
\end{equation}

\eqref{eq:bvp-neumann-cond} $\rightarrow$ \eqref{eq:bvp-neumann-green-2}得正交条件
\begin{equation}
  \label{eq:bvp-neumann-green-2-new}
  \int_{\Omega} f(x) \, dx + \int_{\Gamma} g_N(x) \, d s_x = 0.
\end{equation}

换句话说,如果关于$\nu(x)$的齐次诺依曼边界问题解是$\nu_1(x)=1, x \in \Omega$,那么关于$u$的诺依曼边界值问题\eqref{eq:bvp-neumann-cond}的解并不唯一:不只包括一个解$u(x)$,还包括另一个解$\tilde{u}(x)$,满足关系
\begin{equation*}
  \tilde{u}(x) = u(x) + \alpha, \quad x \in \Omega,
\end{equation*}
其中常数$\alpha \in \mathbb{R}$的值是唯一的,取决于为了使第一个解$u(x)$成为诺依曼边界值问题\eqref{eq:bvp-neumann-cond}的解,而需要在系统中加入的规模调整条件,如
\begin{equation*}
  \int_{\Omega} u(x) \, dx = 0, \quad \text{或者} \quad \int_{\Gamma} \gamma_0^{int}u(x) \, ds_x =0.
\end{equation*}

\section{方程空间}
在进一步介绍边界值问题的弱形式之前,一些与之紧密相关的方程空间的知识是必需的。
相关教材,可参考如\cite{McLean:2000ta, Adams:2003wi, Tartar:2007vm, Mazya:2009vz, Mazya:2009wu}。

%\section{\texorpdfstring{$\varepsilon$}{e}SOA}
\subsection{\texorpdfstring{$C^{k}(\Omega),C^{k,\kappa}(\Omega)$}{CK}空间}

给定$d \in \mathbb{N}$。作如下定义:
\begin{itemize}
  \item 向量(vector) $\alpha = \left( \alpha_1, \alpha_2, \ldots, \alpha_d \right), \alpha_i \in \mathbb{N}_0$。
  \item 多重指标(multi-index)的绝对值 $\left| \alpha \right|=\sum_{i=1}^{d} \alpha_i$。
  \item 阶乘(factorial) $\alpha ! = \alpha_1! \, \alpha_2 ! \,  \ldots \alpha_d !$。
\end{itemize}

给定$x \in \mathbb{R}^d$我们有
\begin{equation*}
  x^{\left| \alpha \right|} = x_1^{\alpha_1} \, x_2^{\alpha_2} \ldots x_d^{\alpha_d}.
\end{equation*}

给定一个充分平滑的实值方程$u$,其相对于$x$的$\alpha$阶偏微分导数
\begin{equation*}
  D^{\alpha} u(x) \coloneqq \left( \frac{\partial }{\partial x_{1}} \right)^{\alpha_1} \left( \frac{\partial }{\partial x_{2}} \right)^{\alpha_2} \ldots \left( \frac{\partial }{\partial x_{d}} \right)^{\alpha_d} u \left( x_1, x_2, \ldots, x_d \right).
\end{equation*}

给定一个开放子集$\Omega \subseteq R^{d}$,对于某个标量$k \in \mathbb{N}_0$。则$C^{k}(\Omega)$表示在$\Omega$域中有界且$k$次连续可导的方程空间。对于某个方程$u \in \Omega$,$u$的范数(norm)\index{norm \dotfill 范数}值是有限的
\begin{equation*}
  \| u \|_{C^{k}(\Omega)} \coloneqq \sum_{\left| \alpha \right| \le k} \sup_{x \in \Omega} \big| D^{\alpha} u(x) \big| < \infty,
\end{equation*}
随着$k \rightarrow \infty$,$C^{\infty}(\Omega)$是个有界且无限阶连续可积的方程空间。

对于方程$u(x), x \in \Omega$,我们将$u$的支撑(support) \index{support \dotfill 支撑}定义为$\text{supp} \, u$
\begin{equation*}
  \text{supp} \, u \coloneqq \overline{x \in \Omega: u(x) \neq 0}.
\end{equation*}

进而定义$C_0^{\infty}(\Omega)$为$C^{\infty}(\Omega)$中的紧支撑(compact support)方程空间。

定义$C^{k,\kappa}, k \in \mathbb{N}_0, \kappa \in (0,1)$为霍德尔连续方程空间(Hölder continuous function space)\index{Hölder space \dotfill 霍德尔空间},对应范数为
\begin{equation*}
  \| u \|_{C^{k,\kappa}(\Omega)} \coloneqq \| u \|_{C^{k}(\Omega)} + \sum_{\left| \alpha \right| = k} \sup_{x,y\in\Omega, x \neq y} \frac{\left| D^{\alpha}u(x)-D^{\alpha}u(y) \right|}{\left| x - y \right|^{\kappa}}
\end{equation*}

当$\kappa=1$时,$C^{k,1}$用来表示第$\left| \alpha \right|=k$次偏导数$D^{\alpha}u(x)$是利普希茨连续方程(Lipschitz continuous)\index{Lipschitz continuous function \dotfill 利普希茨连续方程}的方程$u \in C^{k}(\Omega)$所组成的空间。

我们用$\Gamma$来表示开放集$\Omega \subset \mathbb{R}^d$的边界
\begin{equation*}
  \Gamma \coloneqq \partial \Omega = \bar{\Omega} \cup \left( \mathbb{R}^d \backslash \Omega \right).
\end{equation*}

当$d \ge 2$,$\Gamma = \partial \Omega$可以视作利普希茨方程的局部图,随着所处在$\Gamma$中不同的位置,对应不同的笛卡尔坐标系。一个最简单的例子是假定一个利普希茨方程$\gamma: \mathbb{R}^{d-1} \mapsto \mathbb{R}$,满足
\begin{equation*}
  \Omega \coloneqq \left\{
  x \in \mathbb{R}^d: x_d < \gamma(\tilde{x}), \quad \forall \tilde{x}=(x_1,\ldots,x_{d-1}) \in \mathbb{R}^{d-1}
  \right\}.
\end{equation*}

由利普希茨方程的性质我们有,$\gamma(\cdot)$满足
\begin{equation*}
  \big| \gamma(\tilde x) - \gamma (\tilde y) \big|
  \le L \big| \tilde{x} - \tilde{y} \big|, \quad \forall \tilde{x},\tilde{y} \in \mathbb{R}^{d-1},
\end{equation*}
那么$\Omega$被称为一个利普希茨亚图(Lipschitz hypograph)\index{Lipschitz hypograph \dotfill 利普希茨亚图},对应边界$\Gamma$
\begin{equation*}
  \Gamma = \left\{
  x \in \mathbb{R}^d: x_n = \gamma(\tilde{x}), \quad \forall \tilde{x} \in \mathbb{R}^{d-1}
  \right\}.
\end{equation*}

\begin{definition}[利普希茨域]
  \label{definition:bvp-lipschitz-domain-def}
  某一个开集$\Omega \subset \mathbb{R}^d, d \ge 2$在满足如下条件时,被称为利普希茨域(Lipschitz domain)\index{Lipschitz domain \dotfill 利普希茨域},或有利普希茨边界的域(domain with Lipschitz boundary):
  \begin{itemize}
    \item $\Omega$的边界$\Gamma = \partial \Omega$是紧凑的,并且
    \item $\exists$有限的索引族(index family)\index{index family \dotfill 索引族}  $\left\{W_j\right\}$和$\left\{ \Omega_j\right\}$,满足
    \footnote{对于集合$I$和$S$,某个方程$x$
    \begin{equation*}
    \begin{split}
      x:&I \mapsto S \\
      &i \mapsto x_i = x(i)
    \end{split}
  \end{equation*}被称作用$I$索引的$S$中元素的族(family of elements)\index{},也表示为$\left\{ x_i \right\}_{i \in I},\, x_i \subset S$。
  }
  \begin{itemize}
    \item 索引族$\left\{ W_j \right\}$是对$\Gamma$的有限开覆盖(finite open cover)\index{cover(topology) \dotfill 覆盖(拓扑学)}
    \begin{equation*}
      W_j \in \mathbb{R}^d, \quad \& \, \Gamma \subseteq \bigcup_j W_j.
    \end{equation*}
    \item 每一个索引族$\Omega_j, \, \forall j$都可以通过一定的操作变换(transformation)为利普希茨亚图,如旋转(rotation)和平移(translation)等。
    \item $W_j \cap \Omega = W_j \cap \Omega_j, \quad \forall j$。
  \end{itemize}
  \end{itemize}
\end{definition}

需要注意的是,利普希茨边界$\Gamma = \partial \Omega$作局部表达的方案,即对$W_j$和$\Omega_j$的选取,通常来讲并不是唯一的\footnote{如非利普希茨域的例子,可参考\cite{McLean:2000ta}。}。

如果方程$\gamma(\cdot)$中参数的选取使得满足$\gamma \in C^{k}(\mathbb{R}^{d-1})$,我们称这个利普希茨边界$\Gamma$是$k$次可微的;如果满足$\gamma \in C^{k, \kappa}(\mathbb{R}^{d-1})$,我们称$\Gamma$为霍德尔连续(Hölder continuous)\index{Hölder continuous \dotfill 霍德尔连续};如果$\gamma$仅仅是在某个局域内满足该条件,则我们称对应的$\Gamma$为分段平滑边界(piecewise smooth boundary)\index{piecewise smooth boundary \dotfill 分段平滑边界}。

\subsection{勒贝格\texorpdfstring{$L^p(\Omega)$}{(LP)}空间}
