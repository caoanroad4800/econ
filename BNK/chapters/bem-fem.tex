%!TEX root = ../DSGEnotes.tex
\chapter{有界元法和有限元法}
\label{sec:bem-fem-methods}

\section{边界值问题:位势方程}

我们从二阶偏微分方程入手,介绍边界值问题(boundary value problem)\index{boundary value problem \dotfill 边界值问题}。一个合适的例子是位势方程(potential equation)。

\subsection{偏微分算子及椭圆边界值问题}

定义有界域$\Omega \in \mathbb{R}^d, d=2,3$,边界$\Gamma = \partial \Omega$,外代数单位向量空间(exterior unit normal vector)\index{exterior algebra \dotfill 外代数} $\underline{n}(x)$对于$x \in \Gamma$几乎处处存在。对于$x \in \Omega$,我们考虑一个线性二阶偏微分的自伴随算子\footnote{有限维内积向量空间$V$中,自伴随算子A是一个从$V$到$V$自身的线性映射$\langle A \bm{u}, \bm{\nu} \rangle = \langle \bm{\nu}, A \bm{u} \rangle, \, \forall \nu, w \in V$。}
(self-adjoint operator)\index{self-adjoint operator} $L$
,作用于实值标量方程$u$
\begin{equation}
  \label{eq:bvp-self-adjoint-pde-operator}
  \left( L \, u \right)(x) \coloneqq - \sum_{i,j=1}^d \frac{\partial}{\partial x_j} \left[ a_{ji} (x) \frac{\partial}{\partial x_i} u(x)\right] + a_0(x)\, u(x),
\end{equation}
其中$a_{ji}(x), \, i,j =1,\ldots, d, \, x \in \Omega$表示系数方程,假定为平滑的并满足$a_{ij}(x) = a_{ji}(x)$。由此可以构建一个对称的系数矩阵$A(x)$,满足
\begin{equation*}
  A(x) = \left( a_{ij}(x) \right)_{i,j=1}^{d}, \quad x \in \Omega,
\end{equation*}
对应实数特征根$\lambda_{k}(x)$。

当且仅当$\lambda_{k}(x) > 0$对于所有$k=1,\ldots,d$都成立时,我们称偏微分算子$L$在某一个$x \in \Omega$上是椭圆(elliptic)的。

更进一步,如果$\forall x \in \Omega$该条件都成立,那么我们称$L$在$\Omega$上是椭圆的。

如果存在一个一致下界(uniform lower bound) $\lambda_0 > 0$,满足
\begin{equation*}
  \lambda_k (x) \ge \lambda_0, \quad \forall k = 1,\ldots,d, \, \forall x \in \Omega,
\end{equation*}
那么我们称$L$在$\Omega$上一致椭圆。

\subsection{边界条件}

边界条件的分析,可以从散度定理开始。
\begin{theorem}[散度定理]
  \label{theorem:bvp-gauss-divergence-theorem}
  散度定理(divergence theorem)\index{divergence theorem \dotfill 高斯散度定理},又称奥斯特罗格拉德斯基——高斯定理(Ostrogradsky-Gauss theorem)\index{Ostrogradsky-Gauss theorem \dotfill 奥斯特罗格拉德斯基——高斯定理} 、高斯散度定理(Gauss' theorem)\index{Gauss theorem \dotfill 高斯散度定理}等,是指
  \begin{equation}
    \label{eq:bvp-gauss-divergence-theorem}
    \int_{\Omega} \frac{\partial}{\partial x_i} f(x) \, dx = \int_{\Gamma} \left[ \gamma_0^{int} f(x) \right] n_i(x) \, d s_x, \quad i = 1,\ldots,d,
  \end{equation}
  其中$\gamma_0^{int} f(x)$是某个给定方程$f(x), x\in \Omega$的内界迹(interior boundary trace),满足
  \begin{equation}
    \label{eq:bvp-interior-boundary-trace}
    \gamma_0^{int} f(x) \coloneqq \lim_{\Omega \owns \tilde{x} \mapsto x \in \Gamma} f \left( \tilde{x} \right), \quad \forall x \in \Gamma = \partial \Omega.
  \end{equation}
\end{theorem}

假定两个足够光滑的方程$u,\nu \in \Omega$,通过设定$f(x) = u(x) \, \nu(x)$,可以将散度定理\eqref{eq:bvp-gauss-divergence-theorem}改写为分部积分(integration by parts)\index{integration by parts \dotfill 分部积分}的形式
\begin{equation*}
  \int_{\Omega} u(x) \frac{\partial}{\partial x_i} \nu(x) \, x
  + \int_{\Omega} \nu(x) \frac{\partial}{\partial x_i} u(x) \, x
  = \int_{\Gamma}  \left[ \gamma_0^{int} u(x) \right] \left[ \gamma_0^{int} \nu(x) \right] n_i(x) \, d s_x.
\end{equation*}

重新调整上式,将$\nu(x)$视作检测方程(test function),两侧乘以\eqref{eq:bvp-self-adjoint-pde-operator}中的二阶偏微分算子$\left(L\,u\right)(x)$,在$\Omega$中求积
\begin{equation}
  \begin{split}
    &\left( L \, u \right)(x) \, \nu(x) \coloneqq - \sum_{i,j=1}^d \frac{\partial}{\partial x_j} \left[ a_{ji} (x) \frac{\partial}{\partial x_i} u(x)\right] \, \nu(x) + a_0(x)\, \underbrace{u(x) \, \nu(x)}_{\equiv f(x)}, \\
    \hookrightarrow & \int_{\Omega} \left( L \, u \right)(x) \, \nu(x) \, dx = - \sum_{i,j=1}^d \int_{\Omega} \frac{\partial}{\partial x_j} \left[ a_{ji} (x) \frac{\partial}{\partial x_i} u(x)\right] \, \nu(x) \, dx,
  \end{split}
\end{equation}
使用分部积分$\hookrightarrow$
\begin{equation*}
  \begin{split}
    \int_{\Omega} \left( L \, u \right)(x) \, \nu(x) \, dx =&  \underbrace{\sum_{i,j=1}^d \int_{\Omega} a_{ji}(x) \frac{\partial}{\partial x_i} u(x) \, \frac{\partial}{\partial x_j} \nu(x) \, dx}_{\coloneqq a\left(u,\nu \right)} \\
   &- \sum_{i,j=1}^{d} \int_{\Gamma} n_j(x) \left[ \gamma_0^{int} (x) \left( a_{ji}(x) \frac{\partial}{\partial x_i} u(x)\right) \right] \left[ \gamma_0^{int} \nu(x) \right] \, d s_x,
  \end{split}
\end{equation*}

由此,我们由散度定理(Theorem \ref{theorem:bvp-gauss-divergence-theorem})推导出格林第一恒等式(Green's first identity)\index{Green identities!first 格林第一恒等式}
\begin{equation}
  \label{eq:bvp-a-u-nu-inner-prod}
  \begin{split}
  a\left(u,\nu \right) &\coloneqq \sum_{i,j=1}^d \int_{\Omega} a_{ji}(x) \frac{\partial}{\partial x_i} u(x) \, \frac{\partial}{\partial x_j} \nu(x) \, dx \\
  & = \int_{\Omega} \left( L \, u \right)(x) \, \nu(x) \, dx + \sum_{i,j=1}^{d} \int_{\Gamma} \underbrace{n_j(x) \left[ \gamma_0^{int} (x) \left( a_{ji}(x) \frac{\partial}{\partial x_i} u(x)\right) \right]} \left[ \gamma_0^{int} \nu(x) \right] \, d s_x,\\
  & =\int_{\Omega} \left( L \, u \right)(x) \, \nu(x) \, dx + \int_{\Gamma} \underbrace{\left[ \gamma_1^{int} u(x) \right]} \left[ \gamma_0^{int} \nu(x) \right] \, d s_x,
  \end{split}
\end{equation}
其中定义$\gamma_1^{int}$为内部共形导数(interior co-normal derivative, \cite{Mikhailov:2006vo, Mikhailov:2009wj, Ancona:2009bo})
\begin{equation}
  \label{eq:bvp-int-conformal-derivative}
  \gamma_1^{int}u(x) \coloneqq \lim_{\Omega \owns \tilde{x} \mapsto x \in \Gamma} \left[
\sum_{i,j=1}^{d} n_j(x) a_{ji}\left( \tilde{x} \right) \frac{\partial}{\partial \tilde{x}_i} u \left( \tilde{x} \right)
  \right], \quad x \in \Gamma.
\end{equation}

将格林第一恒等式\eqref{eq:bvp-a-u-nu-inner-prod}中的$u,\nu$互换位置,我们有
\begin{equation*}
  \begin{split}
  a\left(\nu,u \right) = \int_{\Omega} \left( L \, \nu \right)(x) \, u(x) \, dx + \int_{\Gamma} \left[ \gamma_1^{int} \nu(x) \right] \left[ \gamma_0^{int} u(x) \right] \, d s_x,
  \end{split}
\end{equation*}
由上式和\eqref{eq:bvp-a-u-nu-inner-prod}我们有格林第二恒等式(Green's second identity)\index{Green identities!second 格林第二恒等式}: $\forall u,\nu \in \Omega$且$u,\nu$足够平滑
\begin{equation}
  \label{eq:bvp-a-nu-u-green-2nd-identity}
  \begin{split}
    &a(u,\nu) = a(\nu,u) \Longleftrightarrow \\
    &\int_{\Omega} \left( L \, u \right)(x) \, \nu(x) \, dx + \int_{\Gamma} \left[ \gamma_1^{int} u(x) \right] \left[ \gamma_0^{int} \nu(x) \right] \, d s_x
    = \int_{\Omega} \left( L \, \nu \right)(x) \, u(x) \, dx + \int_{\Gamma} \left[ \gamma_1^{int} \nu(x) \right] \left[ \gamma_0^{int} u(x) \right] \, d s_x
  \end{split}
\end{equation}

下面来考虑一个特殊情况,$a_{ij}(x)=\delta_{ij}$,$\delta_{ij}$是克罗内克乘积(Kronecker product)\index{Kronecker product \dotfill 克罗内乘积}。\eqref{eq:bvp-self-adjoint-pde-operator}的二阶偏微分算子$\left(L\,u\right)(x)$变为拉普拉斯算子
\begin{equation}
  \label{eq:bvp-laplace-operator}
  \left( L \, u \right)(x) = - \Delta u(x) \coloneqq - \sum_{i=1}^{d} \frac{\partial^2}{\partial x_i^2} u(x), \quad x \in \mathbb{R}^d.
\end{equation}

内部共形导数$\gamma_1^{int}$ \eqref{eq:bvp-int-conformal-derivative}变为
\begin{equation}
  \label{eq:bvp-laplace-conformal-derivative}
  \gamma_1^{int}u(x) = \frac{\partial}{\partial n_x} u(x) \coloneqq \underline{n}(x) \bigtriangledown u(x), \quad x \in \Gamma.
\end{equation}

对边界域$\Gamma = \partial \Omega$分解成三个不相交集合的并集(disjoint union)
\begin{equation*}
  \Gamma = \overline{\Gamma}_D \cup \overline{\Gamma}_N \cup \overline{\Gamma}_R,
\end{equation*}

对应地,边界值问题变为两部分:第一部分,在$\Omega$中,基于给定的方程$f(x)$,寻找偏微分算子$(L u)(x)$,使得
\begin{equation}
  \label{eq:bvp-extension-omega-cond}
  \left( L \, u \right)(x) = f(x), \quad x \in \Omega.
\end{equation}

第二部分,在$\Gamma$中,基于给定的方程$g(x)$,寻找内界迹$\gamma_0^{int}u(x)$或者内共形导数$\gamma_1^{int}(x)$。随着$\Gamma$的取值范围不同,分为三种情况:
\begin{subequations}
  \begin{equation}
    \label{eq:bvp-extension-gamma-dirichlet}
    \gamma_0^{int} u(x) = g_D(x), \quad x \in \Gamma = \Gamma_D,
  \end{equation}
  \begin{equation}
    \label{eq:bvp-extension-gamma-neumann}
    \gamma_1^{int} u(x) = g_N(x), \quad x \in \Gamma = \Gamma_N,
  \end{equation}
  \begin{equation}
    \label{eq:bvp-extension-gamma-robin}
    \kappa(x) \, \gamma_0^{int} u(x) + \gamma_1^{int} u(x) = g_R(x), \quad x \in \Gamma = \Gamma_R.
  \end{equation}
\end{subequations}


\begin{definition}[边界值条件]
  \label{definition:boundary-value-problem}
  于是我们有以下几种不同的边界值条件:
\begin{itemize}
  \item $\Gamma = \Gamma_D:$ \eqref{eq:bvp-extension-omega-cond} + \eqref{eq:bvp-extension-gamma-dirichlet} $\rightarrow$ 狄利克雷边界值条件(Dirichlet boundary value condition)\index{Dirichlet boundary value condition \dotfill 狄利克雷边界值条件},
  \item $\Gamma = \Gamma_N:$ \eqref{eq:bvp-extension-omega-cond} + \eqref{eq:bvp-extension-gamma-neumann} $\rightarrow$ 诺依曼边界值条件(Neumann boundary value condition)\index{Neumann boundary value condition \dotfill 诺依曼边界值条件},
  \item $\Gamma = \Gamma_R:$ \eqref{eq:bvp-extension-omega-cond} + \eqref{eq:bvp-extension-gamma-robin} $\rightarrow$ 罗宾边界值条件(Robin boundary value condition)\index{Robin boundary value condition \dotfill 罗宾边界值条件},
  \item 混合型边界值条件,以上三种情况的组合。
\end{itemize}
\end{definition}

有时候我们还需要将线性罗宾边界值条件扩展为非线性的情况,\eqref{eq:bvp-extension-gamma-robin} $\rightarrow$
\begin{equation}
  \label{eq:bvp-extension-gamma-robin-nonlinear}
  G\left( \gamma_0^{int} u(x), x \right) + \gamma_1^{int} u(x) = g_R(x), \quad x \in \Gamma = \Gamma_R,
\end{equation}
其中$G(u,\cdot)$是某个给定的非线性方程,如$u(x)^3$。

对于边界值问题的解$u(x)$,还需要注意以下几点
\begin{enumerate}
  \item $u(x)$的存在性和唯一性,相关讨论可参考如\cite{Ladyzhenskaya:1968vq},
  \item 观测到的数据需要是充分平滑的,以确保$u(x)$充分可微(sufficiently differentiable)
  \begin{equation*}
    u \in C^2(\Omega) \cap C^1 \left( \Omega \cup \Gamma_N \cup \Gamma_R \right) \cap C(\Omega \cup \Gamma_D).
  \end{equation*}
\end{enumerate}

\subsection{诺依曼边界值问题}
对于诺依曼边界值条件的解,其存在性和唯一性需要做进一步讨论。

假定$\nu_1(x)=1, x \in \Omega$是关于$\nu_1(x)$的齐次诺依曼边界值问题的一个解,\eqref{eq:bvp-laplace-operator}、\eqref{eq:bvp-laplace-conformal-derivative} $\Rightarrow$
\begin{equation}
  \label{eq:bvp-neumann-nu-homo}
  \begin{split}
    \left( L \, \nu_1 \right)(x)=0, \quad x \in \Omega,\\
    \gamma_1^{int} \nu_1(x) = 0, \quad x \in \Gamma.
  \end{split}
\end{equation}

则关于$u(x)$的诺依曼边界值问题可以描述如下:\eqref{eq:bvp-neumann-nu-homo}$\rightarrow$格林第二恒等式\eqref{eq:bvp-a-nu-u-green-2nd-identity} $\Rightarrow$
\begin{equation}
  \label{eq:bvp-neumann-green-2}
  \int_{\Omega} \left( L \, u \right)(x) \, dx + \int_{\Gamma} \gamma_1^{int} u(x) \, d s_x = 0,
\end{equation}

诺依曼边界值条件\eqref{eq:bvp-extension-omega-cond}、 \eqref{eq:bvp-extension-gamma-neumann}$\Rightarrow$
\begin{equation}
  \label{eq:bvp-neumann-cond}
\begin{split}
  \left( L u \right)(x) = f(x), \quad x \in \Omega, \\
  \gamma_1^{int} u(x) = g_N(x), \quad x \in \Gamma.
\end{split}
\end{equation}

\eqref{eq:bvp-neumann-cond} $\rightarrow$ \eqref{eq:bvp-neumann-green-2}得正交条件
\begin{equation}
  \label{eq:bvp-neumann-green-2-new}
  \int_{\Omega} f(x) \, dx + \int_{\Gamma} g_N(x) \, d s_x = 0.
\end{equation}

换句话说,如果关于$\nu(x)$的齐次诺依曼边界问题解是$\nu_1(x)=1, x \in \Omega$,那么关于$u$的诺依曼边界值问题\eqref{eq:bvp-neumann-cond}的解并不唯一:不只包括一个解$u(x)$,还包括另一个解$\tilde{u}(x)$,满足关系
\begin{equation*}
  \tilde{u}(x) = u(x) + \alpha, \quad x \in \Omega,
\end{equation*}
其中常数$\alpha \in \mathbb{R}$的值是唯一的,取决于为了使第一个解$u(x)$成为诺依曼边界值问题\eqref{eq:bvp-neumann-cond}的解,而需要在系统中加入的规模调整条件,如
\begin{equation*}
  \int_{\Omega} u(x) \, dx = 0, \quad \text{或者} \quad \int_{\Gamma} \gamma_0^{int}u(x) \, ds_x =0.
\end{equation*}

\section{方程空间}
在进一步介绍边界值问题的弱形式之前,一些与之紧密相关的方程空间的知识是必需的。
相关教材,可参考如\cite{McLean:2000ta, Adams:2003wi, Tartar:2007vm, Mazya:2009vz, Mazya:2009wu}。

%\section{\texorpdfstring{$\varepsilon$}{e}SOA}
\subsection{\texorpdfstring{$C^{k}(\Omega),C^{k,\kappa}(\Omega)$}{CK}空间}

给定$d \in \mathbb{N}$。作如下定义:
\begin{itemize}
  \item 向量(vector) $\alpha = \left( \alpha_1, \alpha_2, \ldots, \alpha_d \right), \alpha_i \in \mathbb{N}_0$。
  \item 多重指标(multi-index)的绝对值 $\left| \alpha \right|=\sum_{i=1}^{d} \alpha_i$。
  \item 阶乘(factorial) $\alpha ! = \alpha_1! \, \alpha_2 ! \,  \ldots \alpha_d !$。
\end{itemize}

给定$x \in \mathbb{R}^d$我们有
\begin{equation*}
  x^{\left| \alpha \right|} = x_1^{\alpha_1} \, x_2^{\alpha_2} \ldots x_d^{\alpha_d}.
\end{equation*}

给定一个充分平滑的实值方程$u$,其相对于$x$的$\alpha$阶偏微分导数
\begin{equation*}
  D^{\alpha} u(x) \coloneqq \left( \frac{\partial }{\partial x_{1}} \right)^{\alpha_1} \left( \frac{\partial }{\partial x_{2}} \right)^{\alpha_2} \ldots \left( \frac{\partial }{\partial x_{d}} \right)^{\alpha_d} u \left( x_1, x_2, \ldots, x_d \right).
\end{equation*}

给定一个开放子集$\Omega \subseteq R^{d}$,对于某个标量$k \in \mathbb{N}_0$。则$C^{k}(\Omega)$表示在$\Omega$域中有界且$k$次连续可导的方程空间。对于某个方程$u \in \Omega$,$u$的范数(norm)\index{norm \dotfill 范数}值是有限的
\begin{equation*}
  \| u \|_{C^{k}(\Omega)} \coloneqq \sum_{\left| \alpha \right| \le k} \sup_{x \in \Omega} \big| D^{\alpha} u(x) \big| < \infty,
\end{equation*}
随着$k \rightarrow \infty$,$C^{\infty}(\Omega)$是个有界且无限阶连续可积的方程空间。

对于方程$u(x), x \in \Omega$,我们将$u$的支撑(support) \index{support \dotfill 支撑}定义为$\text{supp} \, u$
\begin{equation*}
  \text{supp} \, u \coloneqq \overline{x \in \Omega: u(x) \neq 0}.
\end{equation*}

进而定义$C_0^{\infty}(\Omega)$为$C^{\infty}(\Omega)$中的紧支撑(compact support)方程空间。

定义$C^{k,\kappa}, k \in \mathbb{N}_0, \kappa \in (0,1)$为霍德尔连续方程空间(Hölder continuous function space)\index{Hölder space \dotfill 霍德尔空间},对应范数为
\begin{equation*}
  \| u \|_{C^{k,\kappa}(\Omega)} \coloneqq \| u \|_{C^{k}(\Omega)} + \sum_{\left| \alpha \right| = k} \sup_{x,y\in\Omega, x \neq y} \frac{\left| D^{\alpha}u(x)-D^{\alpha}u(y) \right|}{\left| x - y \right|^{\kappa}}
\end{equation*}

当$\kappa=1$时,$C^{k,1}$用来表示第$\left| \alpha \right|=k$次偏导数$D^{\alpha}u(x)$是利普希茨连续方程(Lipschitz continuous)\index{Lipschitz continuous function \dotfill 利普希茨连续方程}的方程$u \in C^{k}(\Omega)$所组成的空间。

我们用$\Gamma$来表示开放集$\Omega \subset \mathbb{R}^d$的边界
\begin{equation*}
  \Gamma \coloneqq \partial \Omega = \bar{\Omega} \cup \left( \mathbb{R}^d \backslash \Omega \right).
\end{equation*}

当$d \ge 2$,$\Gamma = \partial \Omega$可以视作利普希茨方程的局部图,随着所处在$\Gamma$中不同的位置,对应不同的笛卡尔坐标系。一个最简单的例子是假定一个利普希茨方程$\gamma: \mathbb{R}^{d-1} \mapsto \mathbb{R}$,满足
\begin{equation*}
  \Omega \coloneqq \left\{
  x \in \mathbb{R}^d: x_d < \gamma(\tilde{x}), \quad \forall \tilde{x}=(x_1,\ldots,x_{d-1}) \in \mathbb{R}^{d-1}
  \right\}.
\end{equation*}

由利普希茨方程的性质我们有,$\gamma(\cdot)$满足
\begin{equation*}
  \big| \gamma(\tilde x) - \gamma (\tilde y) \big|
  \le L \big| \tilde{x} - \tilde{y} \big|, \quad \forall \tilde{x},\tilde{y} \in \mathbb{R}^{d-1},
\end{equation*}
那么$\Omega$被称为一个利普希茨亚图(Lipschitz hypograph)\index{Lipschitz hypograph \dotfill 利普希茨亚图},对应边界$\Gamma$
\begin{equation*}
  \Gamma = \left\{
  x \in \mathbb{R}^d: x_n = \gamma(\tilde{x}), \quad \forall \tilde{x} \in \mathbb{R}^{d-1}
  \right\}.
\end{equation*}

\begin{definition}[利普希茨域]
  \label{definition:bvp-lipschitz-domain-def}
  某一个开集$\Omega \subset \mathbb{R}^d, d \ge 2$在满足如下条件时,被称为利普希茨域(Lipschitz domain)\index{Lipschitz domain \dotfill 利普希茨域},或有利普希茨边界的域(domain with Lipschitz boundary):
  \begin{itemize}
    \item $\Omega$的边界$\Gamma = \partial \Omega$是紧凑的,并且
    \item $\exists$有限的索引族(index family)\index{index family \dotfill 索引族}  $\left\{W_j\right\}$和$\left\{ \Omega_j\right\}$,满足
    \footnote{对于集合$I$和$S$,某个方程$x$
    \begin{equation*}
    \begin{split}
      x:&I \mapsto S \\
      &i \mapsto x_i = x(i)
    \end{split}
  \end{equation*}被称作用$I$索引的$S$中元素的族(family of elements)\index{},也表示为$\left\{ x_i \right\}_{i \in I},\, x_i \subset S$。
  }
  \begin{itemize}
    \item 索引族$\left\{ W_j \right\}$是对$\Gamma$的有限开覆盖(finite open cover)\index{cover(topology) \dotfill 覆盖(拓扑学)}
    \begin{equation*}
      W_j \in \mathbb{R}^d, \quad \& \, \Gamma \subseteq \bigcup_j W_j.
    \end{equation*}
    \item 每一个索引族$\Omega_j, \, \forall j$都可以通过一定的操作变换(transformation)为利普希茨亚图,如旋转(rotation)和平移(translation)等。
    \item $W_j \cap \Omega = W_j \cap \Omega_j, \quad \forall j$。
  \end{itemize}
  \end{itemize}
\end{definition}

需要注意的是,利普希茨边界$\Gamma = \partial \Omega$作局部表达的方案,即对$W_j$和$\Omega_j$的选取,通常来讲并不是唯一的\footnote{如非利普希茨域的例子,可参考\cite{McLean:2000ta}。}。

如果方程$\gamma(\cdot)$中参数的选取使得满足$\gamma \in C^{k}(\mathbb{R}^{d-1})$,我们称这个利普希茨边界$\Gamma$是$k$次可微的;如果满足$\gamma \in C^{k, \kappa}(\mathbb{R}^{d-1})$,我们称$\Gamma$为霍德尔连续(Hölder continuous)\index{Hölder continuous \dotfill 霍德尔连续};如果$\gamma$仅仅是在某个局域内满足该条件,则我们称对应的$\Gamma$为分段平滑边界(piecewise smooth boundary)\index{piecewise smooth boundary \dotfill 分段平滑边界}。

\subsection{勒贝格\texorpdfstring{$L^p(\Omega)$}{(LP)}空间}

数学上$L^{p}(\Omega)$空间($L^{p}(\Omega)$ space) 又称勒贝格空间(Lebesgue space)\index{Lesbegue space \dotfill 勒贝格空间},指$\Omega$上一组测度方程(measurable function)的等价类的集合,这些测度方程都是$p$次勒贝格可积方程(Lebesgue integrable function, Definition \ref{definition:lebesgue-integrable-func-def})\index{Lebesgue integrable function \dotfill 勒贝格可积方程} \footnote{对应地,$\ell^p$空间是由p次可和序列组成的空间。}。

\begin{definition}[勒贝格可积方程]
  \label{definition:lebesgue-integrable-func-def}
  勒贝格可积方程(Lebesgue integrable function)是指该方程的绝对值的$p$次幂的积分是有限的,如
  \begin{equation*}
    \int_{\Omega} \left| u(x) \right|^p d x < \infty.
  \end{equation*}
\end{definition}

\subsubsection{\texorpdfstring{$\sigma$}{SIGMA}代数}
\begin{definition}[sigma代数]
  \label{definition:measure-sigma-algebra}
设$S$是一个非空集合,另有一个集合$\Sigma$中的所有元素都是$S$的子集,那么我们将满足以下条件的$\Sigma$成为$S$上的一个$\sigma$代数($\sigma$-algebra)\index{sigma!algebra \dotfill sigma代数} \citep[p.4]{Bogachev:2007wh, Bogachev:2007tn}
\begin{itemize}
  \item $S$在$\Sigma$中
  \begin{equation*}
    S \in \Sigma,
  \end{equation*}
  \item 如果一个集合$A$在$\Sigma$中,那么它的补集(complement)\index{complement \dotfill 补集} 也在$\Sigma$中
  \begin{equation*}
    A \in \Sigma \Rightarrow A^{\complement} \in \Sigma,
  \end{equation*}
  \item 如果$n$个集合$A_1, \ldots A_n$都在$\Sigma$中,那么他们的并集(union) \index{union \dotfill 并集}也在$\Sigma$中
  \begin{equation*}
    \left( A_n \in \Sigma , \quad  \forall n \in \mathbb{N} \right)
    \Rightarrow \bigcup_{i=1}^{n} A_{i} \in \Sigma.
  \end{equation*}
\end{itemize}
\end{definition}

\begin{definition}[幂集]
  \label{definition:measure-powerset}
  对于任一集合$S$的幂集(powerset)\index{powerset \dotfill 幂集}是指这样的一个集合,包括空集$\varnothing$、$S$本身和$S$的所有集合,常常表示为$\mathcal{P}(S)$或$2^{S}$。

  在公理集合论(axiomatic set theory)\index{axiomatic set theory \dotfill 公理集合论}例如ZFC集合论(ZFC axioms)\index{ZFC axioms \dotfill ZFC集合论}假定了任何集合的幂集均存在。

  $\mathcal{P}(S)$上的全部子集称为$S$上的集族(family of sets over $S$)\index{family of sets \dotfill 集族}。
\end{definition}

假定一个有限集$S$有$n$个元素,表示为$|S|=n$或$card(S)=n$,即$S$的势(cardinality)\index{cardinality \dotfill 势}是$n$。那么$S$的幂集里有 $card(\mathcal{P}(s)) = 2^n$个元素。例如$S=\left\{a,b,c\right\}, card{S}=3$。$S$的全部子集包括
\begin{equation*}
  \begin{cases}
    \left\{ a \right\}, \left\{ b\right\},\left\{c\right\}, \\
    \left\{ a,b \right\},\left\{ a,c \right\},\left\{ b,c \right\},
  \end{cases}
\end{equation*}
因此$\left| \mathcal{P}(s) \right|$为6个$S$的子集,加上$\varnothing$和$S$自身,共$2^3=8$个。

对于非空集合$S$来说,$S$上的一个$\sigma$代数是指其幂集(powerset, 第\ref{definition:measure-sigma-algebra}节) 的一个子集$\Sigma$,$\Sigma$中的元素在经过有限个补集、并集、交集(intersection)\index{intersection \dotfill 交集}这三种运算后,依然属于$\Sigma$。即是说,$\Sigma$对这三种运算是封闭(closed)的。

\subsubsection{测度,可测空间,测度空间}
\label{sec:measure-measure}
\begin{definition}[测度]
  对于一个集合$S$表示函数的定义域,有关于$S$的可测空间$\Sigma$,$\Sigma$中的元素是$S$上的子集族(family of subsets),并且$\Sigma$是一个$\sigma$代数(第\ref{definition:measure-sigma-algebra}节)。沿着扩展实数线(extended real number line) \index{extended real number line \dotfill 扩展实数线}
  \footnote{扩展实数线是指实数集合$\mathbb{R}$加上$-\infty$和$+\infty$,
  常写作$\overline{\mathbb{R}}$或$\left[ -\infty, +\infty \right]$。}
  定义一个测度方程$\mu \in \Sigma$,如果$\mu$满足如下条件,我们称它为一个测度(measure) \index{measure \dotfill 测度}:
  \begin{itemize}
    \item 非负性(non negative)
    \begin{equation*}
      \forall E \in \Sigma: \mu(E) \ge 0,
    \end{equation*}
    \item 空集合的测度为$0$
    \begin{equation}
      \mu(\varnothing) = 0,
    \end{equation}
    \item 可数可加性(countable additivity)\index{countable additivity \dotfill 可数可加性},又称$\sigma$可加性($\sigma$ additivity) \index{sigma!additivity \dotfill sigma可加}:$\left\{ E_i \right\}_{i=1}^{\infty}  = \{ E_1,E_2,\ldots \}$为$\Sigma$中可数个两两不不相交序列的集合(pairwise disjoint sets in $\Sigma$),则所有$E_i$并集的测度等于每个$E_i$的测度之和
    \begin{equation*}
      \mu \left( \bigcup_{i=1}^{\infty} E_i \right) = \sum_{k=1}^{\infty} \mu(E_i).
    \end{equation*}
  \end{itemize}
\end{definition}

\begin{definition}[可测空间]
进而,我们称$\left( S, \Sigma \right)$为一个可测空间(measurable space)\index{measurable space \dotfill 可测空间}。$\Sigma$中的所有元素$\{ E_n \}_{n=1}^{\infty}$ 成为可测集(measurable sets)\index{measurable sets \dotfill 可测集合}。
\end{definition}

\begin{definition}[测度空间]
  \label{definition:measure-measure-space}
一个三元组(triple)\index{triple 三元组}$\left( \mu,  S, \Sigma \right)$称为测度空间(measure space)。测度空间满足如下性质
\begin{itemize}
  \item 测度$\mu$是单调方程(monotonic)
  \begin{equation*}
    \left( \text{可测集合} E_1,E_2 \in \Sigma, \quad E_1 \subseteq E_2 \right) \Rightarrow \left( \mu(E_1) \subseteq \mu(E_2) \right).
  \end{equation*}
  \item 无限个可测集合的并集的测度
  \begin{itemize}
    \item $\mu$是可数的次可加方程(countably subadditive)\todo{subadditivity, triangle inequality}。在$\Sigma$中的任何可数可测集合$\left\{E_n \right\}_{i=n}^{\infty}$(可以不满足两两不相交),有
    \begin{equation*}
      \mu \left( \bigcup_{n=1}^{\infty} E_n \right) \le \sum_{n=1}^{n} \mu(E_n).
    \end{equation*}
    \item $\mu$是连续(continuous)方程。在$\Sigma$中的任何可数可测集合$\left\{E_n \right\}_{i=n}^{\infty}$(可以不满足两两不相交),满足$E_n \subset E_{n+1} \quad \forall n \in \mathbb{N}$,则集合的并集$\bigcup E_n$也是可测的,并且满足
    \begin{equation*}
      \mu \left( \bigcup_{n=1}^{\infty} \right) = \lim_{n\rightarrow \infty} \mu \left(E_n \right).
    \end{equation*}
  \end{itemize}
  \item 无限个可测集合的交集的测度
  \begin{itemize}
    \item $\mu$是连续方程。在$\Sigma$中的任何可数可测集合$\left\{E_n \right\}_{i=n}^{\infty}$(可以不满足两两不相交),满足$E_n \supset E_{n+1} \quad \forall n \in \mathbb{N}$,则集合的交集$\bigcap E_n$也是可测的,并且满足
    \begin{equation*}
      \mu \left( \bigcap_{n=1}^{\infty} \right) = \lim_{n\rightarrow \infty} \mu \left(E_n \right),
    \end{equation*}
    需要指出的是,对于交集的情况,若无下述假设,该性质一般不成立:可测集合$\left\{ E_n \right\}_{n=1}^{\infty}$中应当至少有一个$E_n$有有限测度。举例来说,如果我们设$E_n = [ n, \infty ) \subset \mathbb{R} \quad \forall n \in \mathbb{N}$,则这些可测集合全都具有无限测度,满足$E_{n} \supset E_{n+1}$,然而
    \begin{equation*}
      \left( \bigcap_{n=1}^{\infty} E_n = \varnothing \right) \Rightarrow \left( \mu \left( \bigcap_{n=1}^{\infty} E_n \right) = \mu \left( \varnothing \right) = 0 \right).
    \end{equation*}
  \end{itemize}
\end{itemize}
\end{definition}

\begin{definition}[计数测度]
  \label{definition:measure-counting-measure}
  在一个测度空间$(S,\Sigma,\mu)$中,任意子集$E \in \Sigma$的计数测度(counting measure) \index{measure!counting \dotfill 计数测度}定义为
  \begin{equation*}
    \mu(E) = \begin{cases}
    card(E), &\text{如果E是有限子集} \\
    +\infty, &\text{如果E是无限子集}
    \end{cases}
  \end{equation*}

  利用计数测度这种直观的方法,我们可以在一个测度空间中,通过将$S$的全部可测子集作$\Sigma$代数,从而将$S$映射进入这个测度空间中。然而,只有当空间$S$是可数的时,它在测度空间$(S,\Sigma,\mu)$中的计数测度才是$\sigma$有限($\sigma$ finite) \index{sigma!finite \dotfill sigma有限}的。
  \end{definition}

  \subsubsection{范数}
  \label{sec:lp-norm}
  范数(norm) \index{norm \dotfill 范数}是一个具有``长度''或``大小''概念的方程,是指赋予一个向量空间(vector space)\index{vector space \dotfill 向量空间} 内每个向量以一个非负的长度或大小;零向量(zero vector)\index{zero vector \dotfill 零向量} 的赋值为0。半范数(seminorm)\index{seminorm! \dotfill 半范数} 是指可以对某些非零向量赋值0。

  \begin{definition}[范数]
    \label{definition-norms}
    假定$V$是域$F$上的向量空间$V \subset F$。$V$上的p范数(p-norm)是指这样一个方程$p: V \mapsto \mathbb{R}$,使得$\forall a \in F, \, \forall \bm{u},\bm{\nu} \in V$,以下4个关系均得到满足
    \begin{itemize}
      \item 绝对齐次(absolutely homogeneous)或称绝对可标量化(absolutely scalable),数乘线性
      \begin{equation*}
        p \left( a \bm{\nu} \right) = | a | \, p(\bm{\nu}),
      \end{equation*}
       \item 次可加(subadditivity)\index{subadditivity \dotfill 次可加} 或称三角不等式(triangle inequality)
      \begin{equation*}
        p\left( \bm{u} + \bm{\nu} \right) \le p\left( \bm{u} \right) + p\left( \bm{\nu} \right),
      \end{equation*}
      \item 严格非负
      \begin{equation*}
        p(\bm{u}) \ge 0,
      \end{equation*}
      \item 确定(definite)。只有零向量的范数是0,反之范数是0的向量是零向量。
      \begin{equation*}
        \left( p(\bm{\nu}) = 0  \right) \Rightarrow \left( \bm{\nu} = 0 \right),
      \end{equation*}
      即$\bm{\nu}$是个空向量。
    \end{itemize}

    只满足前3条关系的方程$p: V \mapsto \mathbb{R}$,我们称之为$V$上的半范数(seminorm)。换句话说,所有范都是半范。
  \end{definition}

  \begin{definition}[商空间]
    \label{definition-quotient-space}
  每个向量空间$V$及其半范$p$都生成一个赋范向量空间(normed vector space)\index{normed vector space \dotfill 赋范向量空间} $\left( \frac{V}{W} \right)$,我们称之为商空间(quotient space)\index{quotient space \dotfill 商空间},其中$W$是$V$的子空间$W \subset V$,包括所有满足$p(\bm{\nu}) = 0$的向量$\bm{\nu} \in V$。

    对应地,商空间中的范数定义为$p (W + \bm{\nu}) = p (\bm{\nu})$。
  \end{definition}

  \begin{definition}[等价范]
    一个向量空间$V$中的两个范(或两个半范) $p$和$q$,当满足下述条件时,可称为等价范(equivalent norms)\index{equivalent norms \dotfill 等价范}:
    \begin{equation*}
      c \, q(\bm{\nu}) \le p(\bm{\nu}) \le C \, q(\bm{\nu}), \quad \exists \text{实常数} c,C, \quad \forall \bm{\nu} \in V.
    \end{equation*}
  \end{definition}

  \begin{definition}[平凡半范]
    平凡半范数(trivial seminorm) \index{seminorm!trivial \dotfill 平凡半范数}是指所有满足如下关系的半范
    \begin{equation*}
      p(\bm{\nu}) = 0, \quad \forall \bm{\nu} \in V.
    \end{equation*}
  \end{definition}

  一个向量空间$V$中的每一个线性泛函$f$ (linear form),都定义了1个半范$\bm{\nu} \mapsto | f(\bm{\nu}) |$。

  \subsubsection{有限维度的可数勒贝格空间(p >= 1)}
  $ \bm{x} = \{ x_1, \ldots, x_n \} \in \mathbb{R}^n$的p范数(p-norm)\index{p-norm \dotfill p范数}或者$L^p, \, \forall p \ge 1$\index{LP-norm \dotfill LP范数},可定义为
  \begin{equation*}
    \parallel \bm{x} \parallel_p = \left( \|x_1\|^p + \|x_2\|^p + \ldots + \|x_n\|^p \right)^{\frac{1}{p}},
  \end{equation*}

  包括一些特殊情况如
  \begin{itemize}
    \item $p=1$是个网格距离(grid distance)\index{grid distance \dotfill 网格距离},又称出租车距离(taxicab distance)\index{taxicab geometry \dotfill 出租车距离},曼哈顿距离(Manhattan distance)\index{Manhattan distance \dotfill 曼哈顿距离}等。
    \item $p=2$是个欧几里得范数(Euclidean norm)\index{Euclidean norm \dotfill 欧几里得范数}。
    \item $p=\infty$是个$L^{\infty}$范数($L^{\infty}$ norm)\index{L-infinity norm},又称最大范数(maximum norm)\index{maximum norm \dotfill 最大范数},均匀范数(uniform norm)\index{uniform norm \dotfill 均匀范数},切比雪夫距离(Chebyshev distance)\index{Chebyshev distance \dotfill 切比雪夫距离}等。
  \end{itemize}

  不同范数$p$之间的关系:
\begin{itemize}
  \item $p=1$的曼哈顿范数,从不小于任何欧几里得范数$p=2$。换句话说,任何向量$\bm{x}$的欧几里得范数都受限于它的1范数
  \begin{equation*}
    \| \bm{x} \|_{n} \le \| \bm{x} \|_1, \quad n \ge 1, n\in \mathbb{N}.
  \end{equation*}
  \item [扩展]。任何向量$\bm{x}$的p范数并不随着$p$的增加而增加
  \begin{equation*}
    \| \bm{x} \|_{p+a} \le \| \bm{x} \|_{p}, \quad \forall \text{向量 } \bm{x}, \quad \forall p \ge 1, a \ge 0, p,a \in \mathbb{N}.
  \end{equation*}
  \item [扩展].柯西——施瓦茨不等式(Cauchy-Schwarz inequality, Definition \ref{definition:cauchy-schwarz-inequality})可得
  \begin{equation*}
  \| \bm{x} \|_{1} \le \sqrt{n} \| \bm{x} \|_{2}, \quad n = dim(\bm{x}).
  \end{equation*}
  \item   [扩展].
    \begin{equation*}
      \| \bm{x} \|_{p} \le \| \bm{x} \|_{r} \le n^{\left(\frac{1}{r} - \frac{1}{p} \right)} \| \bm{x} \|_{p}, \quad \forall \bm{x} \in \mathbb{C}^n, \quad 0 < r <p, r,p \in \mathbb{N}.
    \end{equation*}
\end{itemize}

\subsubsection{有限维度的可数勒贝格空间(0 <= p <= 1)}
略。

\subsubsection{无限维度的勒贝格空间(p不可数)}
我们设$p\le 1$。$p$范数可以扩展到分析由无数个元素构成的向量,向量集合构成可数无限维的p范数列空间,用$l^p$表示。一些特殊情况如
\begin{itemize}
  \item $l^1$由绝对收敛(absolute convergence)序列构成的空间,如
  \begin{itemize}
    \item $\sum_{n=0}^{\infty} \left| a_n \right| = L$,其中$L$是某个实数,$\{a_n\}_{n=0}^{\infty}$是一个实数或复数序列。
    \item $\int_{0}^{\infty} \left| f(x) \right| \, dx = L$,其中$ \int_{0}^{\infty}  f(x)  \, dx$ 是关于某个方程$f(x)$的不定积分。
  \end{itemize}
  \item $l^2$是一个由平方可加数列构成的空间,即一个希尔伯特空间(Hilbert space),见第\ref{sec:lp-hilbert-space}节。
  \item $l^{\infty}$是一个由有界数列(bounded sequence)构成的空间。
\end{itemize}

$l^p$数列空间反映这样的向量空间结构:通过一个坐标一个坐标的向量加总(或标量相乘)而组成。如可数无限维实数(复数)数列$\bm{x} = \left\{ x_n \right\}_{n=1}^{\infty}, \bm{y} = \left\{ y_n \right\}_{n=1}^{\infty}$中的向量和和标量乘
\begin{equation*}
  \begin{split}
    & \left( x_1, \ldots, x_n, x_{n+1}, \ldots \right) +
    \left( y_1, \ldots, y_n, y_{n+1}, \ldots \right)
    = \left( x_1+y_1, \ldots, x_n+y_n, x_{n+1}+y_{n+1}, \ldots \right), \\
    & \lambda \left( x_1, \ldots, x_n, x_{n+1} \ldots , \right) =
    \left( \lambda x_1, \ldots, \lambda x_n, \lambda x_{n+1} , \ldots \right),
  \end{split}
\end{equation*}

对应的$p$范数
\begin{equation*}
  \| x\|_p = \left( |x_1|^p + \ldots + |x_n|^p + |x_{n+1}|^p + \ldots \right)^{\frac{1}{p}}.
\end{equation*}

则我们有定义
\begin{definition}[可数无限维数列空间]
  \label{definition:lp-lp-infinite-def}
  定义$l^p$为一个包括所有实(复)数无限数列的空间(countably infinite dimensional sequence space)\index{sequence space!countably infinite dimensional \dotfill 可数无限维数列空间},并且这些数列的$p$范数必须是有限的:这是因为存在着$p$范数是$\infty$的无限数列,他们不应当包括在$l^{p}$空间中,如$\left( 1,1,1,\ldots \right),\, 1 \le p < \infty$。
\end{definition}

随着$p$值的增加,$l^p$集合的大小增加的更快。例如数列$\bm{x} = \left(1, \frac{1}{2}, \ldots, \frac{1}{n}, \frac{1}{n+1},\ldots \right)$为例,它不在$l^1$空间中($p=1$时范数不收敛),而可能在某个$p>1$的$l^{p}$的空间中(范数收敛)
\begin{equation*}
\begin{split}
  &\| x \|_1 = \left( 1 + \frac{1}{2} + \ldots + \frac{1}{n}, \frac{1}{n+1} + \ldots \right) \rightarrow \infty, \\
  &\| x \|_p = \left( 1 + \left(\frac{1}{2}\right)^p + \ldots + \left( \frac{1}{n} \right)^p, \left( \frac{1}{n+1}\right)^p + \ldots \right)^{\frac{1}{p}}, \quad p >1.
\end{split}
\end{equation*}

$p = \infty$时,$l^{\infty}$范数
\begin{equation*}
  \| \bm{x} \|_{\infty} = \sup \left( |x_1|, \ldots, |x_n|, |x_{n+1}|, \ldots  \right),
\end{equation*}
对应$l^{\infty}$数列空间,包括全部实(复)数无限维数列,但要求他们都是有界的,即$\infty$范数收敛。

\subsubsection{无限维度的不可数勒贝格空间}
\label{sec:lp-spaces-banach}
上面我们讨论了有限维度和可数无限维度的勒贝格空间。然而当空间维度无限并且不可数(即不存在可数的基)时,我们无法像前文的方法来定义范数、进而描述空间。但如果该空间是勒贝格可积的(Lebesgue integrable, 见Definition \ref{definition:lebesgue-integrable-func-def}),仍然可以利用下述办法进行描述。

  给定可测空间$(\Omega,\Sigma,x)$ (见Definition \ref{definition:measure-measure-space}),以及$p \in \mathbb{R}, p \ge 1$。考虑所有从$\Omega$到域$\mathbb{F}=\left(\mathbb{R},\mathbb{C} \right)$的可测方程(measurable function)\todo{补一个measurable function的词条}集合$u(x)$,方程绝对值的$p$次幂在$\Omega$上有界,可积
  \begin{equation*}
    L^p(\Omega) = \left\{u(x);  \| u \|_{L^p(\Omega)} \equiv \left( \int_{\Omega} |u(x)|^p \, d x \right)^{\frac{1}{p}} < \infty \right\},
  \end{equation*}
  集合中的方程$u,\nu \in L^p(\Omega)$具有性质
\begin{itemize}
  \item 可加
  \begin{equation*}
    (u+\nu)(x) = u(x) + \nu(x),
  \end{equation*}
  \item 数乘线性
  \begin{equation*}
    u(\lambda \, x) = \lambda u(x), \forall \text{标量} \lambda \in \mathbb{F},
  \end{equation*}
  \item 范数满足不等式
  \begin{equation*}
    \| u + \nu \|_p^p \le 2^{p-1} \left( \| u \|_{p}^{p} + \| \nu \|_{p}^{p} \right),
  \end{equation*}

\end{itemize}
  如果存在某一个集合$K$构成了零测度(zero measure) $\mu(K) = 0$,那么只有当在这个零测度下$u,\nu \in L^p(\Omega)$才有所区别时,我们说$u$和$\nu$互相识别。

  事实上,2个可以扩展到更多个方程的情况,比如3个,即是说三角不等式(triangle inequality) \index{triangle inequality \dotfill 三角不等式}对于范数形式的可积方程$\| \cdot \|_{p}$依然成立,可由闵可夫斯基不等式(Minkowski inequality) \index{Minkowski inequality \dotfill 闵可夫斯基不等式}证得。它表明$L^p$空间是一个赋范向量空间(normed vector space)\index{vector space!normed \dotfill 赋范向量空间}。对于一个测度空间$\Omega \in L^{p}$,设$1 \le p \le \infty$,$u, \nu \in L^{p}(\Omega)$中的元素,此时我们有

\begin{definition}[三角不等式]
  $L^p(\Omega)$中的三角不等式(triangle inequality) \index{triangle inequality \dotfill 三角不等式}
  \begin{equation*}
  \| u + \nu \|_p \le \| u \|_p + \| v \|_p, \quad 1 < p < \infty
  \end{equation*}
式中等号存在的条件当且仅当$u$和$\nu$是严格线性相关的,即$ \exists \lambda \ge 0 \Rightarrow u = \lambda \nu$,或者$\nu = 0$。
\end{definition}

\begin{definition}[霍德尔不等式]
  \label{definition:hoelder-inequality-def}
  霍德尔不等式(Hölder inequality)\index{Hölder inequality \dotfill 霍德尔不等式}可表示为
  \begin{equation*}
    \int_{\Omega} \left| u(x) \nu(x) \right| \, dx \le \|u\|_{L^p(\Omega)} + \|\nu\|_{L^p(\Omega)}
  \end{equation*}
\end{definition}

\begin{definition}[闵可夫斯基不等式]
  \label{sec:minkowski-ineq-def}
通过取可数测度(见Definition \ref{definition:measure-counting-measure}节),闵可夫斯基不等式(Minkowski inequality)可表示为数列和向量的形式
\begin{equation*}
  \left( \sum_{k=1}^{n} | x_k + y_k | ^p \right)^{\frac{1}{p}}
  \le \left( \sum_{k=1}^{n} | x_k | ^p \right)^{\frac{1}{p}}
  + \left( \sum_{k=1}^{n} | y_k | ^p \right)^{\frac{1}{p}}, \quad \forall \left\{ \bm{x} \right\}_{n}, \left\{ \bm{y} \right\}_{n} = \mathbb{R}\text{或}\mathbb{C}, \quad n=dim(S).
\end{equation*}
\begin{proof}
首先要证明如果$u$和$\nu$都有有限的$p$范数,那么$(u+\nu)$的$p$范数也是有限的,且满足不等式关系
\begin{equation*}
  | u + \nu |_p \le 2^{p-1} \left( |u| ^{p} + |\nu| ^{p}\right)
\end{equation*}
证明方式为:给定$p>1$,则$h(x) = x^p$是一个在$\mathbb{R}^{+}$上的凸方程(convex)。由方程的凸性质可得
\begin{equation*}
  \big| \frac{1}{2} u + \frac{1}{2} \nu \big|^p
  \le \big|\frac{1}{2} |u| + \frac{1}{2} |\nu| \big|^p
  \le \frac{1}{2} |u|^p + \frac{1}{2} |\nu|^p,
\end{equation*}
$\hookrightarrow$
\begin{equation*}
  \big| u + \nu \big|^p \le \frac{1}{2} |2u|^p + \frac{1}{2} |2\nu|^p
  = 2^{p-1} \left( |u|^p + |\nu|^p \right),
\end{equation*}
可见$\|u+\nu\|_p$是有限范数。

在证明了$\|u+\nu\|_p$范数有限后,我们有:如果$|u + \nu|_p=0$,闵可夫斯基不等式直接变为等号且成立。现在假定$|u + \nu|_p \neq 0$,使用三角不等式和霍德尔不等式(Hölder's inequality) \index{Hölder's inequality \dotfill 霍德尔不等式},我们有
\begin{equation*}
\begin{split}
    \|u + \nu \|_p^p &= \left[ \left( \int |u + \nu|^p \, d \mu \right)^{\frac{1}{p}} \right]^{p} \\
    &= \int |u + \nu| \, |u + \nu|^{p-1} \, d \mu \\
    &\le \int \left( |u| + |\nu| \right) \, |u + \nu|^{p-1} \, d \mu\\
    & = \int |u| \, |u + \nu|^{p-1} \, d \mu + \int |\nu| \, |u + \nu|^{p-1} \, d \mu \\
    &=\left[ \left( \int |u|^p \, d \mu \right)^{\frac{1}{p}}  + \left( \int |\nu|^p \, d \mu \right)^{\frac{1}{p}}\right] \left(
    \int \left( \big| u + \nu \big|^{\left(p-1\right) \cdot \left( \frac{p}{p-1} \right)}  \right) \, d \mu
    \right)^{1-\frac{1}{p}}\\
    &= \left( \|u \|_p + \| \nu \|_p \right) \frac{\|u+\nu\|_{p}^{p}}{\| u + \nu \|_p},
\end{split}
\end{equation*}
$\hookrightarrow$
\begin{equation*}
  \| u + \nu \|_p \le \|u \|_p + \| \nu \|_p.
\end{equation*}
\end{proof}
\end{definition}

  有时我们需要积分形式的闵可夫斯基不等式(Minkowski integral inequality)\index{Minowski inequality!integral \dotfill 闵可夫斯基积分不等式}:
\begin{definition}[闵可夫斯基积分不等式]
  \label{sec:minkowski-ineq-int-def}
  假定存在两个$\sigma$可测度空间$(S_1, \mu_1)$和$(S_2, \mu_2)$,并且$F:S_1 \times S_2 \mapsto \mathbb{R}$是可测方程,那么我们有
  \begin{equation*}
    \left[ \int_{S_2} \Big| \int_{S_1} F(x,y) \mu_1(d x) \Big|^{p} \mu_2 (d y)\right]^{\frac{1}{p}} \le
    \left[ \int_{S_1} \Big| \int_{S_2} F(x,y) \mu_2(d y) \Big|^{p} \mu_1 (d x)\right]^{\frac{1}{p}}, \quad p < \infty,
  \end{equation*}
\end{definition}

满足该条件的$\| \cdot \|_p$构成半范数(seminorm, Definition \ref{definition-norms}),对应半赋范向量空间(seminormed vector space)\index{seminormed vector space \dotfill 半赋范向量空间} $\mathcal{L}^p(\Omega,\mu)$。之所以称之为半范数,是因为该空间中存在非零向量$u$满足$\| u \|_p = 0$。

我们可以用标准的拓扑方法,从半范向量空间$\mathcal{L}^p(\Omega,\mu)$中得到一个赋范向量空间。在$\mathcal{L}^p(\Omega,\mu)$中,考虑所有使$\| u \|_p = 0$的向量集合
\begin{equation*}
  \mathcal{N} = \left\{ u;  \| u \|_p = 0 \right\}.
\end{equation*}
$\mathcal{N}$可以看作是一个映射$f \mapsto \| u \|_p$的零向量空间。则对于可测度方程$u$而言:
\begin{equation*}
  \| u \|_p = 0 \Longleftrightarrow \mu(u \neq 0) \Longleftrightarrow u_{\mu\text{-几乎处处}} = 0,
\end{equation*}
$\mu\text{-几乎处处}$表示在测度$\mu$的意义上几乎处处有界(almost everywhere)。从这个意义上来看,$\mathcal{N}$是一个kernel $\| \cdot \|_p$,并且不依赖于$p$
\begin{equation*}
  \mathcal{N} \equiv ker\left( \| \cdot \|_p \right) = \left\{ u: u _{\mu\text{-几乎处处}}=0 \right\}.
\end{equation*}

则我们可以定义一个关于$\mathcal{L}^p(\Omega,\mu)$和kernel $\mathcal{N}$的商空间
\begin{equation*}
  L^p(\Omega,\mu) \equiv \frac{\mathcal{L}^p (\Omega,\mu)}{\mathcal{N}},
\end{equation*}
商空间$L^p(\Omega,\mu)$中的某个$u$可以看做是与$\mathcal{L}^p(\Omega,\mu)$中的$f$相差1个$\mathcal{N}$中对应元素的等价类。

由此可见,$L^p(\Omega,\mu)$就是$\Omega$上关于测度$\mu$的$L^p$空间。对应的$\| \cdot \|_p$成为$L^p(\Omega,\mu)$的$p$范数。需要指出的是,严格来说$L^p$空间中的元素并非某个具体方程,而是由一个方程族构成的等价类。当我们取出$L^p$中的元素作计算的时候,参与计算的其实是从这个方程组中抽取的一个代表方程。

$p = \infty$时对应的空间$L^{\infty}(S,\mu)$也可以用类似方法求得:
\begin{equation*}
  \begin{split}
    &\| f \|_{\infty} \equiv \inf \left\{ C \ge 0: \left| f(x) \right| \le C \text{对于几乎所有}x \right\},\\
    &\exists q < \infty \Rightarrow f \in L^{\infty}(S,\mu) \bigcap L^{q}(S,\mu) \Rightarrow \| f \|_{\infty} = \lim_{p \rightarrow \infty} \| f \|_{p}.
  \end{split}
\end{equation*}

% 随着$p=+\infty$,方程空间$\mathcal{L}^{\infty}(\Omega,\mu)$为一组可测度方程$u(x)$的集合,这些方程在测度$\mu$的意义上几乎处处有界(almost everywhere)
% \begin{equation*}
%   \| u \|_{L^{\infty}(\Omega)} \coloneqq \text{ess } \sup_{x \in \Omega} \left\{ \left| u(x) \right| \right\} \coloneqq \inf_{K \subset \Omega, \mu(K) = 0} \sup_{x \in \Omega \backslash K} \left|  u(x) \right|.
% \end{equation*}

勒贝格空间$L^{p}(S,\mu)$的完备性(completeness)通常称为里兹——费舍定理(Riesz-Fischer theorem)。证明略\footnote{完备性(completeness)的含义,见第\pageref{footnote:completeness-def}页脚注。}。

对于$1 \le p \le \infty$的情况,勒贝格空间$L^{p}(S,\mu)$是一个完备赋范向量空间,常称为巴拿赫空间(Banach space)\index{Banach space \dotfill 巴拿赫空间}。所有$L^{p}$空间都是巴拿赫空间。

\subsubsection{加权勒贝格空间}
\label{sec:lp-weightd-lp}
有时候会遇到加权勒贝格空间的情况。

\begin{definition}[加权勒贝格空间]
  \label{definition:lp-weightd-lp}
  考虑一个测度空间$L^p \left( S, \sigma, \mu \right)$,其中有一个可测方程$w : S \rightarrow [ 0, \infty)$。有时我们也将$L^p \left(S, w \, d \mu \right)$称为$w-$加权勒贝格空间(w-weighted Lebesgue space)\index{Lebesgue space!weighted \dotfill 加权勒贝格空间},其中测度$d \nu \equiv w \, d \mu$。由此我们有测度的定义
  \begin{equation*}
    \nu (A) \equiv \int_A w(x) \, d \mu(x), \quad \forall A \in \Sigma.
  \end{equation*}

在此基础上,加权勒贝格空间$L^p \left(S, w \, d \mu \right)$的范数
\begin{equation*}
  w = \frac{d \nu}{d \mu} \Rightarrow \Big\| \mu \Big\|_{L^p(S, w \, d\mu)} \equiv \left(
  \int_s w(x) \left| \mu(x) \right|^p \, d \mu(x)
  \right)^{\frac{1}{p}}.
\end{equation*}

从这个角度上说,$L^p (s, d \nu) \equiv L^p ( s, w d\nu)$.
\end{definition}

\subsection{希尔伯特(H)空间}
\label{sec:lp-hilbert-space}
有且只有$p=2$时的特殊形式空间$L^2(\Omega)$,是希尔伯特空间(Hilbert space)\index{Hilbert space \dotfill 希尔伯特空间}。

作为(完备赋范的)内积向量空间(inner product space, Definition \ref{definition:inner-product-space}),希尔伯特空间是有限维欧几里得空间的一个扩展:从$\mathbb{R}$扩展到$\mathbb{R}$和$\mathbb{C}$,从有限维度到无限维度,但保留了完备性(completeness)特征(一般来说,非欧几里得空间往往破坏了完备性) \label{footnote:completeness-def}\footnote{完备性(completeness)是指,希尔伯特空间中所有柯西数列(Cauchy sequences) \index{Cauchy sequences \dotfill 柯西数列}都会收敛到此空间中的一点(一个元素),即这些数列与某个元素的范数差的极限为$0$。}。希尔伯特空间与欧几里得空间相仿,有长度和角度的概念,因而可以引申出正交性和垂直性,从而为基于正交多项式的傅里叶级数等提供表达方式。

任何一个希尔伯特空间都是巴拿赫空间,反之则未必。

\subsubsection{例:欧几里得空间}
\label{sec:hilbert-space-eucilidean-examples}
假设所有的希尔伯特空间都是复数(实际应用中大多数是实数)。二维欧几里得空间$\mathbb{R}^2$中,向量$\bm{x},\bm{y}$构成一个希尔伯特空间
\begin{equation*}
  \bm{x} \cdot \bm{y} = \sum_{k=1}^{n} \overline{x_k} y_k,
\end{equation*}
对应范数
\begin{equation*}
\| \cdot \| = \sqrt{\langle \bm{x}, \rangle{\bm{y}}}.
\end{equation*}

三维欧几里得空间$R^3$中,以笛卡尔坐标系表示的$\bm{x},\bm{y}$向量的点乘为
\begin{equation*}
  \bm{x} \cdot \bm{y} = \left( x_1, x_2, x_3 \right) \cdot \left( y_1,y_2,y_3 \right) = x_1 y_1 + x_2 y_2 + x_3 y_3,
\end{equation*}
点乘具有如下性质:
\begin{itemize}
  \item 对称性
  \begin{equation*}
    \bm{x} \cdot \bm{y} = \bm{y} \cdot \bm{x},
  \end{equation*}
  \item 首项线性
  \begin{equation*}
    \left( a \bm{x} + b \bm{y} \right) \cdot \bm{z} = a \bm{x} \cdot \bm{z} + b \bm{y} \cdot \bm{z}, \quad a,b \text{是任意标量}, \quad  \bm{x},\bm{y},\bm{z} \text{是任意向量},
  \end{equation*}
  \item 正定
  \begin{equation*}
    \bm{x} \cdot \bm{x} \begin{cases}
     \ge 0 & \forall \bm{x} \ge 0, \\
     =0 & \text{iff. } \bm{x} = 0.
    \end{cases}
  \end{equation*}
\end{itemize}

\begin{definition}[内积空间]
\label{definition:inner-product-space}
满足上述三个条件的(实数)向量乘称为(实数)内积(inner product)\index{inner product \dotfill 内积},用$\langle .,. \rangle$表示。给定一个实数或复数域$\mathbb{F}$中的向量空间$V$,则我们将内积形式的向量空间$\langle .,.\rangle : V \times V \mapsto \mathbb{F}$表示为内积空间(inner product space) \index{inner product space \dotfill 内积空间}。内积空间满足三个性质:$\forall \text{向量} \bm{x},\bm{y},\bm{z} \in V$,以及$\forall \text{标量} a \in \mathbb{F}$
\begin{itemize}
  \item 共轭对称(conjugate symmetry)
  \begin{equation*}
    \langle \bm{x} , \bm{y}\rangle =
    \begin{cases}
      \langle \bm{y}, \bm{x} \rangle & \mathbb{F}=\mathbb{R}, \\
      \overline{\langle \bm{y}, \bm{x} \rangle} & \mathbb{F}=\mathbb{C}.
    \end{cases}
  \end{equation*}
  其中标有上横线$\overline{(\cdot)}$的部分表示复数共轭(complex conjugate)。
  \item 首项线性
  \begin{equation*}
  \begin{split}
    &\langle a \bm{x}, \bm{y} \rangle = a \langle \bm{x}, \bm{y} \rangle, \\
    & \langle \bm{x} + \bm{y}, \bm{z} \rangle = \langle \bm{x}, \bm{z} \rangle + \langle \bm{y}, \bm{z} \rangle.
  \end{split}
  \end{equation*}
  \item 正定
  \begin{equation*}
    \begin{split}
      &\langle \bm{x}, \bm{x} \rangle > 0, \\
      &\langle \bm{x}, \bm{x} \rangle = 0 \Leftrightarrow \bm{x} = 0.
    \end{split}
  \end{equation*}
\end{itemize}
\end{definition}

任何有限维内积空间都也是希尔伯特空间。在欧几里得空间内,两个向量的内积大小与两方面因素有关,一为向量的长度(即范数)$\| \bm{x} \|$,一为$\bm{x}$和$\bm{y}$之间的夹角$\theta$,满足
\begin{equation*}
  \bm{x} \cdot \bm{y} = \|\bm{x}\| \, \|\bm{y}\| \cos \theta.
\end{equation*}

欧几里得空间$\mathcal{R}^3$中,对$n \in \mathcal{N}$个向量$\bm{x}_n$求和构成一个数学级数$\sum_{n=0}^{\infty} \bm{x}_n$,当每个向量的范之和收敛到一个小于正无穷的向量$L$时,我们称这个级数仍然是绝对收敛(absolutely convergence) \index{convergence!absolute \dotfill 绝对收敛}的
\begin{equation*}
  \sum_{k=0}^{N} \| \bm{x}_k \| < \infty.
\end{equation*}

一个绝对收敛的向量数列$\sum_{k=0}^{N} \bm{x}$,收敛至某个极限向量$\bm{L} \in R^3$
\begin{equation*}
  \Big\| \bm{L} -  \sum_{k=0}^{N} \bm{x}\Big\| \rightarrow 0, \text{随着} N \rightarrow \infty,
\end{equation*}
这称为欧几里得空间的完备性(completeness of Euclidean space)\index{completeness!Euclidean space \dotfill 完备性(欧几里得空间)}。

类似地,在欧几里得空间中,复数平面(complex plane)\index{complex plane \dotfill 复数平面} $\mathbb{C}$由量(magnitude)的形式予以反映,即复绝对值(complex modulus)\index{complex modulus \dotfill 复数绝对值} $\left| z \right|$, 定义为$z$与其共轭复数(complex conjugate)\index{conjugate \dotfill 共轭复数} $\bar{z}$乘积的平方根
\begin{equation*}
  \left| z \right| = \sqrt{z \, \bar{z}}.
\end{equation*}

如果$z=x+y, \, x=\Re{(z)}, \, y=Im{(z)}$,复绝对值为常见的二元欧几里得空间的长度
\begin{equation*}
  \| z \| = \sqrt{\Re{(z)^2} + \Im{(z)^2}} = \sqrt{x^2 + y^2}.
\end{equation*}

两个复数$z,w$的内积
\begin{equation*}
  \langle z, w \rangle = z \bar{w},
\end{equation*}
或者对于复数空间$z, w \in \mathbb{C}^2$,即$z=(z_1,z_2),w=(w_1,w_2)$,对应内积
\begin{equation*}
  \langle z, w \rangle = z_1 \bar{w}_1 + z_2 \bar{w}_2,
\end{equation*}
其中$\Re(\langle z, w \rangle) \in \mathbb{R}^4$。这个内积埃米特对称(Hermitian symmetric)\index{Hermitian symmetric \dotfill 埃米特对称},即是说
\begin{equation*}
  \langle w,z \rangle = \overline{\langle z,w \rangle}.
\end{equation*}

希尔伯特空间$H$是一个实数(或复数)内积向量空间,其中的向量可以内积形式表示为$\langle \bm{x},\bm{y} \rangle$,满足如下特性:
\begin{itemize}
  \item 对称性
  \begin{equation*}
    \langle \bm{y}, \bm{x} \rangle = \begin{cases}
    \langle \bm{x},\bm{y} \rangle  & \text{实数向量}, \\
    \overline{\langle \bm{x},\bm{y} \rangle}  & \text{复数向量}.
    \end{cases}
  \end{equation*}
  \item 首项线性
  \begin{equation*}
    \left( a \bm{x} + b \bm{y} \right) \cdot \bm{z} = a \bm{x} \cdot \bm{z} + b \bm{y} \cdot \bm{z}, \quad a,b \text{是任意标量}, \quad  \bm{x},\bm{y},\bm{z} \text{是任意向量},
  \end{equation*}
  \item 正定\footnote{省略部分复数形式的表述,以使方程结构更紧凑。}
  \begin{equation*}
    \bm{x} \cdot \bm{x} \begin{cases}
     \ge 0 & \forall \bm{x} \ge 0, \\
     =0 & \text{iff. } \bm{x} = 0.
    \end{cases}
  \end{equation*}
\end{itemize}
由对称性和首项线性可得第二项系数是反线性的(antilinear):
\begin{equation*}
  \langle \bm{x}, a \bm{y} + b \bm{z} \rangle = \bar{a} \langle \bm{x}, \bm{y} \rangle + \bar{b} \langle \bm{x}, \bm{z} \rangle.
\end{equation*}

希尔伯特空间的范是一个实值方程
\begin{equation*}
  \| \cdot \| = \sqrt{\langle \bm{x}, \bm{y} \rangle}
\end{equation*}

\begin{definition}[对偶空间]
已知内积定义
\begin{equation*}
  \langle u,\nu \rangle_{\Omega} \coloneqq \int_{\Omega} u(x) \, \nu(x) \, dx,
\end{equation*}
根据闵可夫斯基不等式(Definition \ref{sec:minkowski-ineq-def}) ,$L^p(\Omega)$中的三角不等式可以扩展到更一般的形式:
\begin{equation*}
  \| \nu \|_{L^{q}(\Omega)} = \sup_{0 \neq u \in L^p(\Omega) } \frac{
  \left| \langle u,\nu \rangle_{\Omega} \right|
  }{
  \| u \|_{L^p(\Omega)}
  }, 1 \le p < \infty,
\end{equation*}
其中$p,q$是伴随参数,满足
\begin{equation*}
\quad \frac{1}{p} + \frac{1}{q} = 1,
\end{equation*}
不难看出,RHS满足三角不等式关系。则$L^q(\Omega)$和$L^p(\Omega)$构成一组对偶空间(dual space)\index{dual space \dotfill 对偶空间}。
\end{definition}

当$p=q=2$时,$L^2(\Omega)$就成为包括全部平方可积方程的空间。此时霍德尔不等式(Definition \ref{definition:hoelder-inequality-def})就变成了柯西——施瓦茨不等式。 d
\begin{definition}[柯西——施瓦茨不等式]
  \label{definition:cauchy-schwarz-inequality}
  内积空间(见Definition \ref{definition:inner-product-space}) $\langle .,. \rangle : V \times V \mapsto \mathbb{F}$中,对于任意两个向量$\forall \bm{u}, \bm{\nu} \in V$,内积绝对值的平方,满足三角不等式
  \begin{equation*}
    \big| \langle \bm{u},\bm{\nu} \rangle \big|^2 \le \langle \bm{u},\bm{u} \rangle \cdot \langle \bm{\nu},\bm{\nu} \rangle,
  \end{equation*}
  两侧同时开平方根,将RHS改写为向量范的形式,我们有柯西——施瓦茨不等式(Cauchy-Schwarz inequality)\index{Cauchy-Schwarz inequality \dotfill 柯西——施瓦茨不等式}
  \begin{equation*}
\begin{split}
      &\big| \langle \bm{u},\bm{\nu} \rangle \big| \le \| \bm{u} \| \, \| \bm{\nu} \|,\\
      \hookrightarrow & \int_{\Omega} \left| u(x) \, \nu(x) \right| \, d x \le \| u \|_{L^2(\Omega)} \, \| \nu \|_{L^2(\Omega)},\\
      \hookrightarrow & \big| \langle u, u \rangle \big|_{L^2(\Omega)} = \| u \|^2_{L^2(\Omega)}, \forall u=\nu, u \in L^2(\Omega).
\end{split}
  \end{equation*}

  其中等号成立的条件,只有一下两种之一:$\bm{u},\bm{\nu}$线性无关(linearly independent),即平行;$\bm{\nu}$是零向量或是标量。
\end{definition}
\begin{proof}
  柯西——施瓦茨不等式的证明方法有很多种\citep{Wu:2011uc},我们取其中一种。

  $\bm{\nu} = 0 \Rightarrow \big| \langle \bm{u},\bm{\nu} \rangle \big| = \| \bm{u} \| \, \| \bm{\nu} \| \forall \bm{u} \in V$。

  $\bm{u} \neq 0, \bm{\nu} \neq 0$。设一个向量$\bm{z}$满足
  \begin{equation*}
    \bm{z} := \bm{u} - \bm{u_{\nu}} = \bm{u} - \underbrace{\frac{\langle \bm{u}, \bm{\nu} \rangle}{\langle \bm{\nu}, \bm{\nu} \rangle}}_{\text{标量}} \bm{\nu}.
  \end{equation*}

  对$\bm{z}$和$\bm{\nu}$作内积,由内积空间的性质之一——首项线性得
  \begin{equation*}
    \begin{split}
      \langle \bm{z}, \bm{\nu} \rangle &= \left\langle \bm{u} - \frac{\langle \bm{u}, \bm{\nu} \rangle}{\langle \bm{\nu}, \bm{\nu} \rangle} \bm{\nu}, \bm{\nu} \right\rangle \\
      &= \langle \bm{u}, \bm{\nu} \rangle - \left\langle \frac{\langle \bm{u}, \bm{\nu} \rangle}{\langle \bm{\nu}, \bm{\nu} \rangle} \bm{\nu}, \bm{\nu} \right\rangle \\
      &= \langle \bm{u}, \bm{\nu} \rangle - \frac{\langle \bm{u}, \bm{\nu} \rangle}{\langle \bm{\nu}, \bm{\nu} \rangle} \left\langle  \bm{\nu}, \bm{\nu} \right\rangle \\
      &=0.
    \end{split}
  \end{equation*}

  $\langle \bm{z}, \bm{\nu} \rangle = 0 \Rightarrow \bm{z}=0$,作为$\bm{u}$向$\bm{\nu}$所在平面(plane)所做的正交映射,反映了$\bm{u}$和$\bm{\nu}$线性无关。因此我们对$\bm{z}$的定义式继续使用勾股定理
  \begin{equation}
    \begin{split}
      \bm{u} & = \frac{\langle \bm{u}, \bm{\nu} \rangle}{\langle \bm{\nu}, \bm{\nu} \rangle} \bm{\nu} + \bm{z} \\
      \hookrightarrow \left\| \bm{u} \right\|^2 &= \Big| \frac{\langle \bm{u}, \bm{\nu} \rangle}{\langle \bm{\nu}, \bm{\nu} \rangle} \Big|^2  \| \bm{\nu} \|^2 + \| \bm{z}\|^2 \\
      &= \frac{
      \big| \langle \bm{u}, \bm{\nu} \rangle \big|^2
      }{
      \left( \| \bm{\nu} \|^2 \right)^2 }
      \| \bm{\nu} \|^2  + \| \bm{z} \|^2 \\
      &= \frac{
      \big| \langle \bm{u}, \bm{\nu} \rangle \big|^2
      }{
       \| \bm{\nu} \|^2 } + \| \bm{z} \|^2 \\
       & \ge \frac{
       \big| \langle \bm{u}, \bm{\nu} \rangle \big|^2
       }{
        \| \bm{\nu} \|^2 }\\
        \hookrightarrow \left| \langle \bm{u}, \bm{\nu}  \rangle \right| & \le \| \bm{u} \| \, \| \bm{\nu} \|
    \end{split}
  \end{equation}
\end{proof}

\subsection{索伯列夫\texorpdfstring{$W^{k,p}(\Omega)$}{(W)}空间}

\subsubsection{微分的类以及平滑方程}
$H^s(\Omega)$中包括$L^p$空间中的具有弱可导性的平滑方程
\footnote{``平滑''方程(smoothness)的定义包括很多种,由弱到强有
\begin{itemize}
  \item 连续性,
  \item 可导性(可导方程必连续),
  \item 它的最高一阶导数也是连续的,
\end{itemize}
等等。索伯列夫空间中方程设定为``弱''可导形式,是为了使得空间完备,是一个巴拿赫空间。
},常用于求解偏微分方程PDEs。

\begin{definition}[微分的类以及平滑方程]
  \label{eq:soblev-differentiability-classification}
  我们可以根据方程的微分性质,对方程作分类(differentiability classification)\index{differentiability classification \dotfill 方程微分的类}。一个实数集合$\mathbb{R}$上的开区间中,实值方程$f \in \mathbb{R}$。

  如果微分方程$f',f'',\ldots,f^{(k)}$都存在并且$f',f'',\ldots,f^{(k-1)}$连续,我们称$f$属于$\mathbb{C}^{k}$类方程。当$k\rightarrow \infty$时$f$的所有$k$次微分都存在且连续,我们称之为$\mathbb{C}^{\infty}$类方程,无限可微方程(infinitely differentiable),或者称之为平滑方程(smooth function)。

  如果$f$是平滑的,并且$f$沿着域中任意一点作泰勒级数展开都收敛至该点,则我们称$f$是$\mathbb{C}^{\omega}$类方程,或称之为解析方程(analytic function)。可见$\mathbb{C}^{\omega} \subset \mathbb{C}^{\infty}$。

  举例来说,
  \begin{itemize}
    \item $\mathbb{C}^0$中包括所有连续方程,
    \item $\mathbb{C}^1$中包括所有一次可微方程,并且这些方程的一次导数是连续的,称连续可导(continuously differentiable)。进而
    \begin{itemize}
      \item  $\forall f \in \mathbb{C}^1 \Rightarrow f'\text{存在且}f' \in \mathbb{C}^0$
      \item $\forall f \in \mathbb{C}^k \Rightarrow f', f'', \ldots f^{k} \text{存在且}f' \in \mathbb{C}^{k-1}$
    \end{itemize}
    \end{itemize}
\end{definition}

\subsubsection{分部积分}
\begin{definition}[分部积分公式]
  \label{definition:sobolev-integration-by-parts}
  分部积分公式(integration by parts formula)\index{integration by parts \dotfill 分部积分公式}是指,如果${u} = u(x), u'(x) = d u / d x$,以及$\nu = \nu(x), \nu'(x) = d \nu / d x$,那么
  \begin{equation*}
    \begin{split}
      \int_{a}^{b} u(x) \nu'(x) \, dx &= \left[ u(x) \nu(x) \right]_a^b - \int_a^b u'(x) \nu(x) \, dx \\
      &= \left[ u(b) \nu(b) - u(a) \nu(a) \right] - \int_a^b u'(x) \nu(x) \, dx,
  \end{split}
  \end{equation*}
  或者用更紧凑的表现形式
  \begin{equation*}
    \int u d(\nu) = u \nu - \int \nu d(u).
  \end{equation*}
\end{definition}

\subsubsection{广义积分}
我们定义$L^{1,\text{loc}}(\Omega)$为局部可积(locally integrable)的方程空间,即方程$u \in L^{1,\text{loc}}(\Omega)$在任意一个封闭有界子集$K \subset \Omega$中可导。

例,设$\Omega = (0,1)$,$u(x)=\frac{1}{x}$。由于
  \begin{equation*}
    \int_0^1 u(x) \, d x \approx \lim_{\epsilon \rightarrow 0} \int_{\epsilon}^1 \frac{1}{x} \, d x \approx \lim_{\epsilon \rightarrow 0} \log \frac{1}{\epsilon} = \infty,
  \end{equation*}
  可见$u \notin L^1(\Omega)$。可由Mathematica算得
  \begin{lstlisting}
  Limit[Integrate[1/x, {x, ee, 1}], ee -> 0]
  Limit[Log[1/x], x -> 0]
  \end{lstlisting}
对于任一闭区间$K \coloneqq [a,b] \subset (0,1) = \Omega, \quad 0 < a < b < 1$我们有
\begin{equation*}
  \int_{K} u(x)\, dx = \int_a^b \frac{1}{x} \, dx = \ln \frac{b}{a} < \infty,
\end{equation*}
可见$u \in L^{1, \text{loc}}(\Omega)$。

此外对于$\phi,\psi \in C_0^{\infty}(\Omega)$,根据分部积分(Definition \ref{definition:sobolev-integration-by-parts})\index{integration by parts \dotfill 分部积分} 我们有
\begin{equation*}
  \int_{\Omega} \phi(x) \frac{\partial}{\partial x_i} \psi(x) \, dx = - \int_{\Omega} \frac{\partial}{\partial x_i} \phi(x) \, \psi(x) \, d x,
\end{equation*}
上式对于哪怕是非平滑方程$\phi,\psi$也适用。由此可得广义偏导数的定义

\begin{definition}[广义偏导数]
  \label{definition:generalized-partial-derivative-def}
  设$u \in L^{1,\text{loc}}(\Omega)$。如果$\exists \, \nu \in L^{1,\text{loc}}(\Omega)$,使得满足
  \begin{equation*}
   \int_{\Omega} \nu(x) \varphi(x) \, dx =   -\int_{\Omega} u(x) \frac{\partial}{\partial x_i} \varphi(x) \, dx,
  \end{equation*}
  其中$\varphi(x) \in C^{\infty}_0(\Omega)$,那么我们说$\nu(x)$是$u(x)$在$\Omega$中关于$x_i$的广义偏导数(generalized partial derivative)\index{generalized partial derivative \dotfill 广义偏导数} ,写作$\nu(x) \coloneqq \partial u(x)/ \partial x_i$。

  类似地,$u$的第$\alpha$阶广义偏导数$\nu(x) = D^{\alpha}u(x)$记作
  \begin{equation*}
    \int_{\Omega} u(x) D^{\alpha} \varphi(x) \, dx = (-1)^{\left| \alpha \right|} \int_{\Omega} \nu(x) \varphi(x) \, dx,
  \end{equation*}
  其中多重指数(multi-index) $\alpha = (\alpha_1,\ldots,\alpha_n), x=(x_1,\ldots,x_n),\left|\alpha\right|=\alpha_1 + \ldots + \alpha_n$,积分操作符$D^{\alpha}$是下述形式的缩写
  \begin{equation*}
    D^{\alpha}= \frac{\partial^{\left| \alpha \right|}}{\partial x_1^{\alpha_1} \ldots x_n^{\alpha_n}},
  \end{equation*}
\end{definition}

例,设$u(x) = \left| x \right|$,$x \in \Omega = (-1,1)$。对于任一$\varphi \in C_0^{\infty}(\Omega)$,我们有
\begin{equation*}
  \begin{split}
    &\int_{-1}^{1} u(x) \frac{\partial}{\partial x}\varphi(x) \, dx\\ &= - \int_{-1}^{0} x \frac{\partial}{\partial x}\varphi(x) \, dx  + \int_{0}^{1} x \frac{\partial}{\partial x}\varphi(x) \, dx \\
    &= \left\{
    - \left[ x \, \varphi(x) \right]_{-1}^{0} + \int_{-1}^{0} \varphi(x) \, dx
    \right\} +
    \left\{
      \left[ x \, \varphi(x) \right]_{0}^{1} - \int_{0}^{1} \varphi(x) \, dx
    \right\}\\
    &=\int_{-1}^{0} \varphi(x) \, dx - \int_{0}^{1} \varphi(x) \, dx \\
    &= - \int_{-1}^{1} \sgn(x) \varphi(x) \, dx,
  \end{split}
\end{equation*}
其中
\begin{equation*}
  \sgn(x) \coloneqq \begin{cases}
    1 & x >0,\\
    -1 & x <0.
  \end{cases}
\end{equation*}

则方程$u(x) = \left| x \right|$的1阶广义偏导数为
\begin{equation*}
  \frac{\partial}{\partial x}u(x) = \sgn(x) \in L^{1,\text{loc}}(\Omega).
\end{equation*}

递归方法计算2阶广义偏导数:$\because$
\begin{equation*}
  \int_{-1}^{1}\sgn(x) \frac{\partial}{\partial x} \varphi(x) = -\int_{-1}^{0} \frac{\partial}{\partial x} \varphi(x) \, dx
  + \int_{0}^1 \frac{\partial}{\partial x} \varphi(x) \, dx = -2 \varphi(0),
\end{equation*}
然而$\nexists \nu \in L^{1,\text{loc}}(\Omega)$满足
\begin{equation*}
  \int_{-1}^{1} \nu(x) \, \varphi(x) \, dx = 2 \varphi(0), \forall \phi \in C_0^{\infty}(\Omega).
\end{equation*}
我们将在随后讨论$\sgn(x)$作为分布概念时的广义积分\todo{reference}。


\subsubsection{整数阶次的单维索伯列夫空间}

索伯列夫相关教材,可参考如\cite{Adams:2003wi, Tartar:2007vm, Mazya:2009vz, Mazya:2009wu}等。


在$\mathbb{R}^n$的一个开放子集$\Omega$中,对于一个给定的非负整数$k$,我们有索伯列夫空间或$W^{k,p}(\Omega)$。它是一种希尔伯特空间的特例,其中
\begin{itemize}
  \item (内积形式表示的)方程向量空间都是可微的,
  \item 范数是可微方程范数的组合,包括方程本身的$L^p$范数,以及方程直到某一给定次导数的范数的组合。
\end{itemize}

单维索伯列夫空间$W^{k,p}$中的方程即是$\mathbb{R}$中的方程,可定义为单维勒贝格空间$L^{p}(\mathbb{R})$中方程$f$的子集,$f$满足如下特征:对于给定的$p \in \mathbb{N}, 1 \le p \le \infty$,其中的$f^{(k-1)}$需要几乎处处可导,并且几乎处处等于其勒贝格积分的$k-1$次导数。

$f$本身以及$f$的直至第$k$阶弱导数是有限的$L^p$范数
\begin{equation*}
  \begin{split}
      \big\| f \big\|_{W^{k,p}} &= \left( \sum_{i=0}^{k} \big\| f^{(i)} \big\|_p^p \right)^{\frac{1}{p}} \\
      &= \sum_{i=0}^{k} \left( \int \left| f^{(i)}(t) \right|^p \, dt \right)^{\frac{1}{p}} \\
      &= \big\| f^{(k)} \big\|_p + \big\| f \big\|_p,
  \end{split}
\end{equation*}
最后一个等式表明,单维索伯列夫空间的范数,等于方程序列自身的范数、以及其最高一阶导数的范数之和。

带有这一范数$\| \cdot \|_{W^{k,p}}$的单维索伯列夫空间$W^{k,p}$是一个巴拿赫空间。

\subsubsection{整数阶次的单维索伯列夫空间(p=2)}
$p=2$的单维索伯列夫空间$W^{k,2}$非常重要,因为它与傅里叶级数关系密切,并且构成了希尔伯特空间$H^k = W^{k,2}$。

$H^k$空间可以定义如下:可由帕塞瓦尔定理(Parseval theorem)\index{Parseval theorem \dotfill 帕塞瓦尔定理}予以证明(证明略。)
\begin{equation*}
  H^k(\mathbb{T}) = \left\{
  f \in L^(\mathbb{T}): \sum_{n=-\infty}^{\infty} \left( 1+n^2+n^4+\ldots+n^{2k} \right) \left| \hat{f}(n) \right|^2 < \infty
  \right\},
\end{equation*}
其中$\hat{f}$是方程$f$的傅里叶级数(Fourier series)\index{Fourier series \dotfill 傅里叶级数},它快速衰减(decay)。$\mathbb{T}$表示环面(torus)。

此时的$H^k$空间可以理解为在$L^2$空间中取内积的形式:
\begin{equation*}
  \langle \bm{u}, \bm{\nu} \rangle_{H^k} = \sum_{i=0}^{k} \langle D^i \bm{u}, D^i \bm{\nu} \rangle_{L^2}.
\end{equation*}

\subsubsection{多维索伯列夫空间}

\begin{definition}[索伯列夫空间]
假设$\exists k \in \mathbb{N}_0, 1 \le p < \infty$,那么多维索伯列夫空间$W^{k,p}(\Omega)$定义为在$\Omega$上的全部方程集合,使得对于每一个多元指数(multiple index) $\alpha$,方程的混合偏导数(分部积分) $f^{(\alpha)}$都存在,并且$f^{(\alpha)} \in L^p(\Omega), \big\| f \big\|_{L^p(\Omega)} < \infty$。从此意义上我们有多维索伯列夫空间$W^{k,p}(\Omega)$的定义式:
\begin{equation}
  \label{eq:sobolev-space-def}
\begin{split}
    W^{k,p}(\Omega) &\coloneqq \overline{C^{\infty}(\Omega)}^{\|\cdot\|_{W^{k,p}(\Omega)}}\\
    &= \left\{ u \in L^p(\Omega): D^{\alpha} u \in L^p (\Omega), \quad \forall \left| \alpha \right| \le k, \quad k \in \mathbb{N} \right\}.
\end{split}
\end{equation}

多维索伯列夫空间$W^{k,p}(\Omega)$的范数定义方式有多重,最常见的两种如下(并且这两种设定是等价的)
\begin{equation}
  \label{eq:sobolev-space-norm-def}
    \|u\|_{W^{k,p}(\Omega)} := \begin{cases}
    \left( \sum_{\left| \alpha \right| \le k} \Big\| D^{\alpha} u \big\|^p_{L^p(\Omega)} \right)^{\frac{1}{p}} & 1 \le p \le +\infty,\\
    \max_{\left| \alpha \right| \le k} \Big\| D^{\alpha} u \Big\| _{L^p(\Omega)} & p = +\infty.
  \end{cases}
\end{equation}

\begin{equation*}
  \|u\|'_{W^{k,p}(\Omega)} :=
  \begin{cases}
    \sum_{\left| \alpha \right| \le k} \Big\| D^{\alpha} u \Big\|_{L^p(\Omega)} & 1 \le p \le +\infty, \\
    \sum_{\left| \alpha \right| \le k} \big\|D^{\alpha} u \Big\|_{L^{\infty}}(\Omega) & p = + \infty.
  \end{cases}
\end{equation*}
\end{definition}

有着上述定义和范数的无限维索伯列夫空间$W^{k,p}(\Omega)$是一个巴拿赫空间。同时,对于$p < \infty$的情况而言,它也是一个可分空间(separable space)\footnote{膏按:FEM!!!!}。

习惯上,我们将1个索伯列夫空间$W^{k,2}(\Omega)$写作希尔伯特空间形式$H^{k}(\Omega)$,对应范数$\big\| \cdot \big\|_{W^{k,2}}(\Omega)$。

\subsubsection{空间方程的边界}
如上文所述,$k=1,p=2$的索伯列夫空间$W^{1,2}(\Omega)$也可表示为$k=1$的希尔伯特空间$H^1(\Omega)$。在$H^1(\Omega)$中有一个非常重要的子空间$H_0^1 (\Omega)$,它定义为1个$H^1(\Omega)$中的闭包(closure)\index{closure \dotfill 闭包},由$\Omega$中被封闭之称的无限可微方程组成。此时索伯列夫范数由前文的正式定义,缩减为下式
\begin{equation*}
  \Big\| f \Big\|_{H^1} = \left(
  \int_{\Omega}
  \left(
  \left| f \right|^2 + | \triangledown f |^2
  \right)
  \right)^{\frac{1}{2}}
\end{equation*}
对应的有闭包索伯列夫空间$\mathring{W}^{k,p}(\Omega)$
\begin{equation}
  \label{eq:sobolev-space-closure-def}
  \mathring{W}^{k,p}(\Omega) \coloneqq \overline{C^{\infty}_0 (\Omega)}^{
  \| \cdot \|_{W^{k,p}(\Omega)}
  }.
\end{equation}


当$\Omega$有正则边界(regular boundary)时,$H_0^1 (\Omega)$可以表示为由一组$H^1(\Omega)$中的方程组成的空间,这些方程的迹(traces)\index{trace \dotfill 迹}\todo{在下面做一个reference. $\#$Extension by zero}在边界上消失了。当$n=1$时,如果$\Omega=(a,b)$是个有界区间,则1维索伯列夫空间$H_0^1(a,b)$包括闭区间$[a,b]$中所有满足下列形式的连续方程
\begin{equation*}
  f(x) = \int_a^b f'(t) \, dt, \quad x \in [a,b],
\end{equation*}
其中导数在$ f'(t)$在$L^2(a,b)$区间中,且不可积。我们因此由$f(b)=f(a)=0$。

当$\Omega$有界时,由庞加莱不等式(Poincaré inequality)\index{Poincaré inequality \dotfill 庞加莱不等式}\todo{Poincaré inequality}表明存在一个常数$C = C(\Omega)$,使满足
\begin{equation*}
  \int_{\Omega} \left| f \right|^2 \le C^2 \int_{\Omega} \left| \bigtriangledown f \right|^2, \quad f \in H_{0}^1(\Omega).
\end{equation*}

并且对应有界的$\Omega$,从$H_0^1 (\Omega)$到$L^2(\Omega)$的投影是紧凑单射(compact injection)\index{injection \dotfill 单射}。这一现象有助于我们理解狄利克雷问题(Dirichlet problem)\index{Dirichlet problem \index 狄利克雷问题}:事实上存在一个$L^2(\Omega)$的标准正交基方程(orthonormal basis),方程的特征向量是满足狄利克雷边界条件(Dirichlet boundary condition, Definition \ref{definition:boundary-value-problem})的拉普拉斯算子(Laplace operator) $\bigtriangleup f$。

\subsubsection{迹}
如前所述,求解偏微分方程PDEs常用索伯列夫空间,而其中的关键是索伯列夫方程的边界值问题(boundary value problem)\index{Sobolev space!boundary value problem \dotfill 边界值问题(索伯列夫空间)}。如果$u \in C(\Omega)$,那么我们常用$u \big|_{\partial \Omega}$来表示边界值。由于在边界上的n维测度是$0$,我们需要借助下文的迹定理来求$u \in W^{k,p}(\Omega)$的边界值。

\begin{theorem}[迹定理]
  \label{theorem:sobolev-trace-theorem}
假定$\Omega$有利普希茨边界(Lipschitz boundary, Definition \ref{definition:bvp-lipschitz-domain-def},或参考 \cite{Heinonen:2005wq})\index{Lipschitz boundary \dotfill 利普希茨边界}。那么存在1个有界的线性算子$T: W^{1,p}(\Omega) \mapsto L^p(\partial \Omega)$,使
\begin{equation*}
  \begin{split}
    &T \, u = u \big|_{\partial \Omega}, \quad u \in W^{1,p} \cap C(\bar{\Omega}), \\
    &\Big\| T \, u \Big\|_{L^P(\partial \Omega)} \le c(p,\Omega) \Big\| u \Big\|_{W^{1,p}(\Omega)}, \quad u \in W^{1,p}(\Omega),
  \end{split}
\end{equation*}
$T$称为迹算子(trace operator),$Tu$称为$u$的迹(trace)。
\end{theorem}

严格意义来讲,迹定理把限制条件$u \big|_{\partial \Omega}$扩展到一个品行良好(well behaved) $\Omega$所对应的索伯列夫空间$W^{1,p}(\Omega)$中。通常来讲$T$并非满射的(surjective)\index{surjection \dotfill 满射},但对于$1 < p < \infty$的情况,$T$映射到一个Sobolev-Slobodeckij 空间$W^{\left( 1 - \frac{1}{p}, p \right)}\left( \partial \Omega \right)$\todo{Sobolev-Slobodeckij 空间见下文$\#$ noninteger k的情况}。

执行取迹操作使索伯列夫空间$W^{1,p}(\Omega)$损失了$\frac{1}{p}$单位的导数次项;含有零迹$Tu=0$的方程$u$也因此可以表示为
\begin{equation*}
  \begin{split}
    &W_0^{1,p} (\Omega) = \left\{ u \in W^{1,p} (\Omega): Tu = 0\right\}, \text{其中}\\
    &W_0^{1,p}(\Omega) :=
    \left\{
    u \in W^{1,p}(\Omega): \exists \left\{ u_m \right\}_{m=1}^{\infty} \in C_c^{\infty}(\Omega), \quad \text{使得} u_m \rightarrow u \, \text{in} \, W^{1,p}(\Omega)
    \right\}.
  \end{split}
\end{equation*}

换句话说,对于有利普希茨边界的$\Omega$,索伯列夫空间$W^{1,p}(\Omega)$中的零迹方程可由紧凑支撑的平滑方程予以近似。

\subsubsection{分数次阶索伯列夫空间:贝塞尔位势空间法}
关于分数次阶索伯列夫空间,可参考\cite{DiNezza:2012wk}。
前面介绍的几种索伯列夫空间的情况,均假定$k \in \mathcal{N}$。然而有时候我们需要处理$k$是分数的情况(fractional order Sobolev space)。借助傅里叶乘子(Fulier multiplier)我们有
\begin{equation*}
  \begin{split}
    &W^{k,p}(\mathbb{R}^n) = H^{k,p}(\mathbb{R}^n) :=
    \left\{
    f \in L^p(\mathbb{R}^n): \mathcal{F}^{-1}
    \left[
    \left( 1 + \left| \xi \right|^2 \right)^{\frac{k}{2}} \mathcal{F} f
    \right] \in L^p(\mathbb{R}^n)
    \right\},\\
    & \Big\| f \Big\|_{H^{k,p} \left( \mathbb{R}^n \right)} := \Big\| \mathcal{F}^{-1} \left[
     \left( 1 + \left| \xi \right|^2 \right)^{\frac{k}{2}}
    \mathcal{F} f \right] \Big\|_{L^p(\mathbb{R}^n)}.
  \end{split}
\end{equation*}
大致说来,有两种方法可以处理分数次阶索伯列夫空间,一种是贝塞尔位势空间法,一种是Sobolev-Slobodeckij空间法。
\begin{definition}[贝塞尔位势空间]
利用傅里叶分析法去理解索伯列夫空间,$k$可以不必是个正整数,而可以是任意正的实数$s \in \mathbb{R}^{+}$,比如分数。换句话说,非正整数的索伯列夫空间,又称贝塞尔位势空间(Bessel potential space)\index{Bessel potential space \dotfill 贝塞尔位势空间} $H^{s,p}$
\begin{equation*}
  H^{s,p}(\mathbb{R}^n) := \left\{
  f \in L^p(\mathbb{R}^n): \mathcal{F}^-1 \left[
  \left( 1 + \left| \xi \right|^2 \right)^{\frac{k}{2}}
 \mathcal{F} f \right] \in L^p(\mathbb{R}^n)
  \right\}.
\end{equation*}

$H^s,p(\Omega)$可以是个限定方程$f$的集合$:H^{s,p}(\mathbb{R}^n) \mapsto \Omega$,集合中方程的范数为
\begin{equation*}
  \Big\| f \Big\|_{H^{s,p}(\Omega)} := \inf \left\{
  \Big\| g \Big\|_{H^{s,p}(\mathbb{R}^n)}: g \in H^{s,p}(\mathbb{R}^n), g \big|_{\Omega} = f
  \right\}.
\end{equation*}
\end{definition}

利用索伯列夫空间的延拓定理(extension theorem of sobolev space)\index{extension theorem of Sobolev space \dotfill 延拓定理(所布偶列夫空间)} 可得,如果$\Omega$有均匀的$\mathbb{C}^k$边界,$k \in \mathbb{N}, 1 < p < \infty$,也可得$W^{k,p}(\Omega)$和$H^{k,p}(\Omega)$ (在范数的意义上)等价。从嵌入(embeddings, 定义\ref{definition:sobolev-spaces-embeddings}) 的角度看,
\begin{equation*}
  H^{k+1,p}(\mathbb{R}^n) \hookrightarrow H^{s',p}(\mathbb{R}^n) \hookrightarrow H^{s,p}(\mathbb{R}^n) \hookrightarrow H^{k,p}(\mathbb{R}^n), \quad k \le s \le s' \le k+1.
\end{equation*}

\begin{definition}[嵌入]
  \label{definition:sobolev-spaces-embeddings}
  数学上,嵌入(embedding) \index{embeddings \dotfill 嵌入}是指某个物件(instance) $X$被嵌入到另一个物件$Y$中去,用保留结构的映射(structure-preserving map)$f: X \mapsto Y$表示。这里的物件指数学结构,如群、子群等。所保留的具体数学``结构''因物件$X$和$Y$的种类而异。如在范畴论(category theorem)中,一个保留结构的映射往往称为一个态射(morphism)。
\end{definition}

通过这种方法,贝塞尔位势空间$H^{s,p}(\mathbb{R}^n)$以复数插值(complex interpolation space)\index{complex interpolation space \dotfill 复数插值} \todo{complex interpolation space}的形式夹在两个索伯列夫空间$H^{k+1,p}(\mathbb{R}^n)$和$H^{k,p}(\mathbb{R}^n)$之间,即从范数等价的意义上来说,下式成立
\begin{equation*}
\begin{split}
  &\left[W^{k,p}(\mathbb{R}^n), W^{k+1,p}(\mathbb{R}^n))
  \right]_{\theta} = H^{s,p}(\mathbb{R}^n), \\
  &1 \le p \le \infty, \, 0 < \theta <1, \, s = (1-\theta) k + \theta (k+1) = k + \theta.
\end{split}
\end{equation*}

\subsubsection{分数次阶索伯列夫空间:Sobolev-Slobodeckij空间法}
另一种定义分数次阶索伯列夫空间的方法是Sobolev–Slobodeckij 空间\index{Sobolev–Slobodeckij spaces \dotfill 索伯列夫}。它是勒贝格空间中的霍德尔条件(Hölder condition)\index{Hölder condition \dotfill 霍德尔条件}\todo{霍德尔条件} 进一步一般化。

根据$W^{k,p(\Omega)}$或者$\mathring{W}^{k,p}(\Omega)$空间的定义\eqref{eq:sobolev-space-def},\eqref{eq:sobolev-space-closure-def}以及相应的范数\eqref{eq:sobolev-space-norm-def},
对于$0 < s \in \mathbb{R}, $的情况,设$s = k + \kappa, k \in \mathbb{N}_0$,我们有Sobolev-Slobodeckij范数(Sobolev-Slobodeckij norm)\index{Sobolev-Slobodeckij norm \dotfill Sobolev-Slobodeckij范数}
\begin{equation*}
  \|u\|_{W^{s,p}(\Omega)} \coloneqq \left\{
  \|u\|^{p}_{W^{k,p}(\Omega)} + \left| u \right|^{p}_{W^{s,p}(\Omega)}
  \right\}^{\frac{1}{p}},
\end{equation*}
其中$\left| u \right|^p_{W^{s,p}(\Omega)}$是Sobolev-Slobodeckij半范数(Sobolev-Slobodeckij norm)\index{Sobolev-Slobodeckij seminorm \dotfill Sobolev-Slobodeckij半范数}
\begin{equation*}
  \left| u \right|^p_{W^{s,p}(\Omega)} = \sum_{\left| \alpha \right| = k} \int_{\Omega} \int_{\Omega} \frac{
  \big| D^{\alpha}u(x) - D^{\alpha} u(y) \big|^{p}
  }{
  \big| x - y\big|^{d + p \kappa}
  }\, dx \, dy.
\end{equation*}

对于$p=2$的情况,$W^{s,2}(\Omega)$成为一个内积形式的希尔伯特空间:
\begin{equation}
  \label{eq:sobolev-slobodeckij-innerp-k}
  \langle u,\nu\rangle_{W^{k,2}(\Omega)} \coloneqq \sum_{\left| \alpha \right| \le k} \int_{\Omega} D^{\alpha}u(x) \, D^{\alpha}\nu(x) \, dx, \quad s=k\in \mathbb{N}_0,
\end{equation}
\begin{equation}
  \label{eq:sobolev-slobodeckij-innerp-s}
  \begin{split}
    \langle u,\nu\rangle_{W^{s,2}(\Omega)} \coloneqq & \langle u,\nu\rangle_{W^{k,2}(\Omega)} +
    \sum_{\left| \alpha \right  | = k} \int_{\Omega} \int_{\Omega} \frac{
    \left[D^{\alpha} u(x) - D^{\alpha} u(y) \right]
    \left[D^{\alpha} \nu(x) - D^{\alpha} \nu(y) \right]
    }{
    \left| x - y \right|^{d + 2 \kappa}
    } \, dx \, dy, \\
    & s=k+\kappa, k\in mathbb{N}_0, \kappa \in (0,1).
  \end{split}
\end{equation}

对于$s < 0, 1<p<\infty$的情况,索伯列夫空间$W^{s,p}(\Omega)$通过对偶空间\index{dual space \dotfill 对偶空间}$\mathring{W}^{-s,q}(\Omega)$形式得以定义,其中$\frac{1}{p} + \frac{1}{q} = 1$,对应范数
\begin{equation*}
  \| u \| _{W^{s,p}(\Omega)} \coloneqq \sup_{0 \neq \nu \in \mathring{W}^{-s,q}(\Omega)} \frac{
  \big| \langle u,\nu \rangle_{\Omega} \big|
  }{
  \| \nu \|_{W^{-s,q}(\Omega)}
  },
\end{equation*}

同样的,$\mathring{W}^{-s,p}(\Omega)$是$W^{-s,q}(\Omega)$的对偶空间。

\subsubsection{索伯列夫空间的性质}
介绍一些索伯列夫空间$W^{s,p}(\Omega)$的性质,这些性质有助于更好理解下文介绍的有界元法和有限元法。假定$s \in \mathbb{R}, p\in \mathbb{N}$,方程$u\in W^{s,p}(\Omega)$有界且连续。

第一个性质可表示为索伯列夫空间嵌入定理(embedding theorem of Sobolev)\index{embedding theorem of Sobolev space \dotfill 索伯列夫空间的嵌入定理}
\begin{theorem}[索伯列夫空间的嵌入定理]
  \label{theorem:sobolev-embedding-theorem}
  设$\Omega \subset \mathbb{R}^d$是一个有界域,利普希茨边界$\partial \Omega$。空间维度$d$和$s$满足关系
  \begin{equation*}
    \begin{cases}
      d \le s & p=1,\\
      \frac{d}{p} < s & p >1.
    \end{cases}
  \end{equation*}

  那么对于索伯列夫空间中的方程$u \in W^{s,p}(\Omega)$,我们可得$u \in C(\Omega)$,满足关系
  \begin{equation*}
    \| u \|_{L^{\infty}(\Omega)} \le c \| u \|_{W^{s,p}(\Omega)}, \quad \forall u \in W^{s,p}(\Omega).
  \end{equation*}
\end{theorem}
\begin{proof}
  略。参考\cite[Theorem 1.4.6]{Brenner:2008hf},\cite[Theorem 3.26]{McLean:2000ta}。
\end{proof}

当$k=1,p=2$时,索伯列夫空间$W^{1,2}(\Omega)$的范数\eqref{eq:sobolev-space-norm-def}变为
\begin{equation*}
\begin{split}
  \| u \|_{W^{1,2}(\Omega)}
  &= \left\{ \big\| D \, u \big\|_{L^{2}(\Omega)}^2 \right\}^{\frac{1}{2}}\\
  &= \left\{
  \|u\|_{L^{2}(\Omega)}^2 + \| \triangledown u \|_{L^{2}(\Omega)}^2
  \right\}^{\frac{1}{2}},
\end{split}
\end{equation*}
并且半范数为
\begin{equation*}
  \left| u \right|_{W^{1,2}(\Omega)} = \| \triangledown u \|_{L^{2}(\Omega)},
\end{equation*}
则我们有索伯列夫空间的范数等价原理(norm equivalence theorem of Sobolev)\index{norm equivalence theorem of Sobolev space}
\begin{theorem}[索伯列夫空间的范数等价原理]
  设一个有界的线性方程$f:W^{1,2}(\Omega) \mapsto \mathbb{R}$,满足
  \begin{equation*}
    0 \le \left| f(u) \right| \le c_f \big\| u \big\|_{W^{1,2}(\Omega)}, \quad \forall \nu \in W^{1,2}(\Omega),
  \end{equation*}
\end{theorem}
$c_f$是个常数。如果对于某个常数$\iota$,$f(\iota) = 0 \text{iff.} \iota \equiv 0$,那么在$W^{1,2}(\Omega)$空间中,所有满足上述条件的方程$f$的范数是等价的:
\begin{equation}
  \label{eq:sobolev-norm-equivalence-theorem}
  \big\| u \big\|_{W^{1,2}(\Omega), f} \coloneqq \left\{ \left| f(u) \right|^2 + \big\| \triangledown u \big\|_{L^{2}(\Omega)}^2 \right\}^{\frac{1}{2}}.
\end{equation}

\begin{proof}
  根据前提假定,$f$是一个线性有界方程,则
  \begin{equation*}
    \begin{split}
      \big\| u \big\|_{W^{1,2}(\Omega), f}^2 &= \underbrace{
      \left| f(u) \right|^2}_{\le c_f^2 \big\| u \big\|_{W^{1,2}(\Omega)^2 }} + \big\| \triangledown u \big\|^2_{L^{2}(\Omega)} \\
        & \le c_f^2 \big\| u \big\|_{W^{1,2}(\Omega)}^2   +
          \underbrace{\big\| \triangledown u \big\|^2_{L^{2}(\Omega)}}_{ \le \big\| u \big\|^2_{W^{1,2}(\Omega)}} \\
      & \le \left(1+c_f^2 \right) \big\| u \big\|^2_{W^{1,2}(\Omega)},
    \end{split}
  \end{equation*}
  换句话说,上式只有在常数$c_f \equiv 0$的条件下才能成立。

  反方向的证明情况,较为间接\footnote{何止是间接......简直是曲折蜿蜒!}。(略)。
\end{proof}
