%!TEX root = ../DSGEnotes.tex
\subsection{狄利克雷边界值问题2}
\label{sec:var-bvp-dirichlet-lagrange}

如第\ref{sec:var-mixed-formulations}节所述,狄利克雷边界值问题\eqref{eq:var-dbvp-problem}也可以改写为鞍点变分问题,即混合算子方程,共形导数对应拉格朗日乘子\citep{Babuska:1973gu, Bramble:1981vv}。

从格林第一恒等式\eqref{eq:var-bvp-green-1st-identity}入手,设拉格朗日乘子
\begin{equation*}
  \lambda \coloneqq \gamma_1^{\text{int}} u \in H^{-\frac{1}{2}}(\Gamma),
\end{equation*}

进而鞍点变分问题表示为,寻找解$(u,\lambda) \in H^{1}(\Omega) \times H^{-\frac{1}{2}}(\Gamma)$,使得满足
\begin{equation}
  \label{eq:var-bvp-saddle-problem}
  \begin{split}
    a(u,\nu) - b(\nu,\lambda) &= \langle f,\nu \rangle_{\Omega} \quad \forall \nu \in H^{1}(\Omega),\\
    b(u,\mu) &= \langle g, \mu \rangle_{\Gamma} \quad \forall \, \mu \in H^{-\frac{1}{2}}(\Gamma),
  \end{split}
\end{equation}
其中定义了一个新的双线性泛函算子
\begin{equation*}
  b(\nu,\mu) \coloneqq \langle \gamma_{0}^{\text{int}} \nu, \mu \rangle_{\Gamma}, \quad (\nu,\mu) \in H^{1}(\Gamma) \times H^{-\frac{1}{2}}(\Gamma).
\end{equation*}

\subsubsection{解的唯一存在性}
\label{sec:var-bvp-saddle-solution-uniq}
鞍点变分形式的狄利克雷边界值问题\eqref{eq:var-bvp-saddle-problem},解的存在性和唯一性,可由Theorem \ref{theorem:mixed-saddle-point-variational-problem}证得。使用该定理之前,需要确保两个前提条件得到满足。一是双线性泛函$a(.,.)$的椭圆特性,二是解的稳定性条件。
\begin{enumerate}
\item 椭圆性。类似于式\eqref{eq:var-bvp-operator-a-ellipticity},由Lemma \ref{lemma:var-bvp-operator-ellipticity-property} \eqref{eq:var-bvp-bilinear-a-nunu}可得
\begin{equation*}
  a(.,.) \ge c_1^A \, \big\| \cdot \big\|_{H^{1}(\Omega)}^2,
\end{equation*}
此外由于
\begin{equation*}
  \ker B \coloneqq \left\{
  \nu \in H^{1}(\Omega) : \langle \gamma_{0}^{\text{int}} \nu, \mu \rangle_{\Gamma} = 0, \quad \forall \, \mu \in H^{-\frac{1}{2}}(\Gamma)
  \right\} = H_{0}^{1}(\Omega),
\end{equation*}
我们因此有,$a(.,.)$是一个$\ker B$-椭圆(或$H_{0}^{1}$-椭圆)的双线性形。

\item 解的稳定性条件可以表示为
\begin{equation}
  \label{sec:var-bvp-saddle-solution-stability}
  c_S \big\| \mu \big\|_{H^{-\frac{1}{2}}(\Gamma)} \le
  \sup_{0 \neq \nu \in H^{1}(\Omega)} \frac{
  \langle \gamma_{0}^{\text{int}} \nu, \mu \rangle_{\Gamma}
  }{
  \big\| \nu \big\|_{H^{1}(\Omega)}
  }, \quad \forall \, \mu \in H^{-\frac{1}{2}}(\Gamma),
\end{equation}
其证明见可见Lemma \ref{lemma:var-bvp-saddle-solution-stability}。

\item 应用定理Theorem \ref{theorem:mixed-saddle-point-variational-problem},求得鞍点变分问题的唯一解。
\end{enumerate}

\begin{lemma}[鞍点变分形式狄利克雷边界值问题解的稳定条件]
  \label{lemma:var-bvp-saddle-solution-stability}
  稳定条件\eqref{sec:var-bvp-saddle-solution-stability}成立。
\end{lemma}
\begin{proof}
  已知给定的任一$\mu \in H^{-\frac{1}{2}}(\Gamma)$。由里兹表现定理(Theorem \ref{theorem:var-riesz-representation-theorem})可得,存在唯一的一个$u_{\mu} \in H^{\frac{1}{2}}(\Gamma)$,满足
  \begin{equation*}
    \begin{split}
      & \langle u_{\mu}, \nu \rangle_{H^{\frac{1}{2}}(\Gamma)} = \langle \mu, \nu \rangle_{\Gamma} \quad \forall \, \nu \in H^{\frac{1}{2}}(\Gamma), \\
      & \big\| u_{\mu} \big\|_{H^{\frac{1}{2}}(\Gamma)} = \big\| \mu \big\|_{H^{\frac{1}{2}}(\Gamma)}.
    \end{split}
  \end{equation*}

  由逆迹定理Theorem \ref{theorem:sobolev-manifold-inverse-trace-theorem}可得,存在一个延拓算子$\varepsilon u_{\mu} \in H^{1}(\Omega)$,满足
  \begin{equation*}
    \big\| \varepsilon u_{\mu} \big\|_{H^{1}(\Omega)} \le c_{\text{IT}} \, \big\| u_{\mu} \big\|_{H^{\frac{1}{2}}(\Gamma)}.
  \end{equation*}

  那么,对于$\nu = \varepsilon u_{\mu} \in H^{1}(\Omega)$我们有
  \begin{equation*}
    \begin{split}
      & \frac{
      \langle \nu, \mu \rangle_{\Gamma}
      }{
      \big\| \nu \big\|_{H^{1}(\Omega)}
      }
      =
      \frac{
      \langle u_{\mu}, \mu \rangle_{\Gamma}
      }{
      \big\| \varepsilon u_{\mu} \big\|_{H^{1}(\Omega)}
      }
      =
      \frac{
      \langle u_{\mu}, u_{\mu} \rangle_{H^{\frac{1}{2}}(\Gamma)}
      }{\big\| \varepsilon u_{\mu} \big\|_{H^{1}(\Omega)}}\\
      & \ge \frac{1}{c_{\text{IT}}} \,
      \big\| u_{\mu} \big\|_{H^{\frac{1}{2}}(\Gamma)}
      = \frac{1}{c_{\text{IT}}} \,
      \big\| u \big\|_{H^{- \frac{1}{2}}(\Gamma)}
    \end{split}
  \end{equation*}

  $\therefore$稳定性条件\eqref{sec:var-bvp-saddle-solution-stability}成立。
\end{proof}

\subsubsection{调整鞍点变分问题}
\label{sec:var-bvp-saddle-modified}
需要注意的是,在鞍点变分问题


边界值问题的正交条件\eqref{eq:bvp-neumann-green-2-new}
