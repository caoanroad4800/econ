%!TEX root = ../DSGEnotes.tex
\section{经验验证}
\label{sec:stylized-empirics}
结合前文提到的典型DSGE模型及其状态——空间表达形式,本节做经验验证。具体说来,截取美国1984Q1-2007Q4的季度经济数据作样本(对应从大调整时期之后, the Great Moderation到大衰退时间之前, the Great Recession period),模拟全体矩。


数据下载自FRED数据库(Federal Reserve Bank of St. Louis, https://fred.stlouisfed.org/),结合模型实际需要,包括以下诸项(表  \ref{tab:stylized-ssrep-empirics-data}):
\begin{itemize}
  \item 实际总产出(seasonally adjusted GDP at the annual rate, 2009 dollars). GDPC96.
  对季度产出取$\log$后相减。
  \item 劳动收入份额(seasonally adjusted GDP at the annual rate),计算方法为$\log \left(\frac{COE}{GDP}\right)$,COE表示compensation of employees.
  \item 通货膨胀率。implicit price deflator. GDPDEF. 对季度数据取$\log$后相减。
  \item 回报率。Effective Federal Funds Rate. GDPDEF. 是月度数据并且未经seasonally adjusted。将每个季度的三个月度数据取平均数,化为季度数据。
\end{itemize}


\begin{longtable}{p{.20\textwidth}|p{.20\textwidth}p{.20\textwidth}p{.20\textwidth}p{.20\textwidth}}
        \hline
        &$\log\left( X_{t}/X_{t-1} \right)$	&$\log (lsh_{t})$	& $ \pi_{t}$	& $R_{t}$\\
        \hline
        1984-01-01	&0.0197	&-0.7846	&0.0106	&0.0969\\
        1984-04-01	&0.0174	&-0.7866	&0.0084	&0.1056\\
        1984-07-01	&0.0098	&-0.7836	&0.0081	&0.1139\\
        1984-10-01	&0.0079	&-0.7808	&0.0067	&0.0927\\
        1985-01-01	&0.0099	&-0.7843	&0.0114	&0.0848\\
        1985-04-01	&0.0091	&-0.7844	&0.0062	&0.0792\\
        1985-07-01	&0.0154	&-0.7891	&0.0058	&0.0790\\
        1985-10-01	&0.0075	&-0.7820	&0.0058	&0.0810\\
        1986-01-01	&0.0092	&-0.7823	&0.0049	&0.0783\\
        1986-04-01	&0.0046	&-0.7821	&0.0040	&0.0692\\
        1986-07-01	&0.0100	&-0.7820	&0.0040	&0.0621\\
        1986-10-01	&0.0052	&-0.7737	&0.0056	&0.0627\\
        1987-01-01	&0.0070	&-0.7689	&0.0072	&0.0622\\
        1987-04-01	&0.0112	&-0.7703	&0.0067	&0.0665\\
        1987-07-01	&0.0090	&-0.7688	&0.0072	&0.0684\\
        1987-10-01	&0.0164	&-0.7673	&0.0082	&0.0692\\
        1988-01-01	&0.0056	&-0.7653	&0.0078	&0.0666\\
        1988-04-01	&0.0131	&-0.7651	&0.0096	&0.0716\\
        1988-07-01	&0.0058	&-0.7659	&0.0117	&0.0798\\
        1988-10-01	&0.0132	&-0.7694	&0.0080	&0.0847\\
        1989-01-01	&0.0100	&-0.7763	&0.0109	&0.0944\\
        1989-04-01	&0.0078	&-0.7860	&0.0103	&0.0973\\
        1989-07-01	&0.0074	&-0.7896	&0.0072	&0.0908\\
        1989-10-01	&0.0021	&-0.7822	&0.0069	&0.0861\\
        1990-01-01	&0.0109	&-0.7842	&0.0110	&0.0825\\
        1990-04-01	&0.0039	&-0.7803	&0.0103	&0.0824\\
        1990-07-01	&0.0002	&-0.7775	&0.0089	&0.0816\\
        1990-10-01	&-0.0086	&-0.7747	&0.0075	&0.0774\\
        1991-01-01	&-0.0047	&-0.7790	&0.0099	&0.0643\\
        1991-04-01	&0.0077	&-0.7844	&0.0068	&0.0586\\
        1991-07-01	&0.0048	&-0.7875	&0.0073	&0.0564\\
        1992-01-01	&0.0117	&-0.7829	&0.0043	&0.0402\\
        1991-10-01	&0.0044	&-0.7873	&0.0054	&0.0482\\
        1992-04-01	&0.0110	&-0.7869	&0.0064	&0.0377\\
        1992-07-01	&0.0097	&-0.7935	&0.0047	&0.0326\\
        1992-10-01	&0.0100	&-0.7955	&0.0067	&0.0304\\
        1993-01-01	&0.0019	&-0.8035	&0.0057	&0.0304\\
        1993-07-01	&0.0049	&-0.8018	&0.0060	&0.0306\\
        1993-04-01	&0.0059	&-0.8000	&0.0061	&0.0300\\
        1993-10-01	&0.0133	&-0.8052	&0.0052	&0.0299\\
        1994-01-01	&0.0098	&-0.8156	&0.0048	&0.0321\\
        1994-04-01	&0.0136	&-0.8117	&0.0050	&0.0394\\
        1994-07-01	&0.0059	&-0.8127	&0.0054	&0.0449\\
        1994-10-01	&0.0113	&-0.8133	&0.0055	&0.0517\\
        1995-01-01	&0.0034	&-0.8076	&0.0057	&0.0581\\
        1995-04-01	&0.0035	&-0.8052	&0.0044	&0.0602\\
        1995-07-01	&0.0085	&-0.8059	&0.0047	&0.0580\\
        1995-10-01	&0.0071	&-0.8064	&0.0049	&0.0572\\
        1996-01-01	&0.0065	&-0.8048	&0.0054	&0.0536\\
        1996-04-01	&0.0173	&-0.8076	&0.0038	&0.0524\\
        1996-07-01	&0.0092	&-0.8042	&0.0028	&0.0531\\
        1996-10-01	&0.0105	&-0.8044	&0.0051	&0.0528\\
        1997-01-01	&0.0076	&-0.7984	&0.0062	&0.0528\\
        1997-04-01	&0.0150	&-0.8007	&0.0027	&0.0552\\
        1997-07-01	&0.0127	&-0.7993	&0.0036	&0.0553\\
        1997-10-01	&0.0077	&-0.7883	&0.0033	&0.0551\\
        1998-01-01	&0.0098	&-0.7786	&0.0016	&0.0552\\
        1998-04-01	&0.0097	&-0.7734	&0.0021	&0.0550\\
        1998-07-01	&0.0130	&-0.7732	&0.0037	&0.0553\\
        1998-10-01	&0.0163	&-0.7760	&0.0031	&0.0486\\
        1999-01-01	&0.0080	&-0.7696	&0.0050	&0.0473\\
        1999-04-01	&0.0082	&-0.7716	&0.0033	&0.0475\\
        1999-07-01	&0.0125	&-0.7740	&0.0036	&0.0509\\
        1999-10-01	&0.0172	&-0.7718	&0.0046	&0.0531\\
        2000-01-01	&0.0029	&-0.7456	&0.0076	&0.0568\\
        2000-04-01	&0.0187	&-0.7642	&0.0057	&0.0627\\
        2000-07-01	&0.0012	&-0.7527	&0.0065	&0.0652\\
        2000-10-01	&0.0057	&-0.7586	&0.0054	&0.0647\\
        2001-01-01	&-0.0028	&-0.7447	&0.0063	&0.0559\\
        2001-04-01	&0.0053	&-0.7609	&0.0070	&0.0433\\
        2001-07-01	&-0.0032	&-0.7664	&0.0033	&0.0350\\
        2001-10-01	&0.0028	&-0.7726	&0.0030	&0.0213\\
        2002-01-01	&0.0092	&-0.7813	&0.0032	&0.0173\\
        2002-04-01	&0.0055	&-0.7808	&0.0037	&0.0175\\
        2002-07-01	&0.0049	&-0.7881	&0.0044	&0.0174\\
        2002-10-01	&0.0006	&-0.7920	&0.0054	&0.0144\\
        2003-01-01	&0.0052	&-0.8013	&0.0061	&0.0125\\
        2003-04-01	&0.0092	&-0.7998	&0.0032	&0.0125\\
        2003-07-01	&0.0166	&-0.8093	&0.0055	&0.0102\\
        2003-10-01	&0.0116	&-0.8094	&0.0047	&0.0100\\
        2004-01-01	&0.0057	&-0.8181	&0.0087	&0.0100\\
        2004-04-01	&0.0073	&-0.8148	&0.0087	&0.0101\\
        2004-07-01	&0.0090	&-0.8101	&0.0061	&0.0143\\
        2004-10-01	&0.0086	&-0.8186	&0.0070	&0.0195\\
        2005-01-01	&0.0106	&-0.8294	&0.0092	&0.0247\\
        2005-04-01	&0.0052	&-0.8312	&0.0072	&0.0294\\
        2005-07-01	&0.0084	&-0.8317	&0.0093	&0.0346\\
        2005-10-01	&0.0057	&-0.8322	&0.0076	&0.0398\\
        2006-01-01	&0.0119	&-0.8240	&0.0078	&0.0446\\
        2006-04-01	&0.0030	&-0.8285	&0.0080	&0.0491\\
        2006-07-01	&0.0009	&-0.8288	&0.0070	&0.0525\\
        2006-10-01	&0.0078	&-0.8210	&0.0035	&0.0525\\
        2007-01-01	&0.0006	&-0.8058	&0.0112	&0.0526\\
        2007-04-01	&0.0076	&-0.8153	&0.0056	&0.0525\\
        2007-07-01	&0.0067	&-0.8220	&0.0035	&0.0507\\
        2007-10-01	&0.0036	&-0.8183	&0.0043	&0.0450\\
        \hline
%        \centering
      \caption{用于经验验证的观测数据集}

      \small{数据来源:FRED database. 整理后放在\textit{./data/20180403-empirics.xlsx} 中。}
        \label{tab:stylized-ssrep-empirics-data}
\end{longtable}

\subsection{自协方差}
\label{sec:stylized-ssrep-empirics-var}

我们计算样本的协方差$\widehat{\Gamma}_{yy} \left(h \right)$,作为全体自协方差$\Gamma_{yy} \left(h \right)$的近似
\begin{equation}
  \label{eq:stylized-ssrep-empirics-autocovar}
  \widehat{\Gamma}_{yy} \left( h \right) = \frac{1}{T}
  \sum_{t=h}^{T} \left( y_{t} - \hat{\mu}_{y} \right) \,
  \left( y_{t} - \hat{\mu}_{y} \right)^{\top}, \quad \, \mu_{y} = \frac{1}{T} \sum_{t=1}^{T} y_{t}.
\end{equation}

在满足一定的正则条件的前提下,根据强大数法则(strong law of large numbers, SSLN)和中心极限定理(central limit theorem, CLT)样本自协方差,$\widehat{\Gamma}_{yy} \left(h \right)$收敛至全体自协方差$\Gamma_{yy} \left(h \right)$,这些正则条件如
\begin{itemize}
  \item 向量过程$y_{t}$的协方差是平稳的,
  \item $y_{t}$的序列相关的衰减速度足够快,
  \item $y_{t}$的(至少一系列)高阶矩是有界的,等。
\end{itemize}

如果研究目标包括计算自协方差矩阵的数列\footnote{Matlab中可用xcov程序计算自协方差、跨协方差。},一个比较有效率的求解方法是,首先建立一个辅助模型并作参数估计。第二步,将这个辅助模型的估计参数转换为自协方差数列的参数值。辅助模型有许多种类,一个较为合适的种类是线性向量自回归模型(vector autoregressions, VARs)\index{vector autoregressions, VARs \dotfill 向量自回归},一个简单的VAR(1)可表示如下
\begin{equation}
  \label{sec:stylized-ssrep-empirics-var1-def}
  y_{t} = \phi_{1} y_{t-1} + \phi_{0} + u_{t}, \quad u_{t} \sim i.i.d. \, \mathbb{N} \left(0, \Sigma \right),
\end{equation}
其中全体的OLS系数$\phi_{1}$向量,和残差方差$\Sigma$无法直接计算求得,可用样本的对应值$\hat{\phi}_{1}, \widehat{\Sigma}$予以近似,计算方法如下
\begin{align}
  \label{sec:stylized-ssrep-empirics-VAR1-sample-coef-phi1}
  \hat{\phi}_{1} & = \widehat{\Gamma}_{yy} \left( 1 \right)
  \widehat{\Gamma}_{yy}^{-1} \left( 0 \right) + \mathcal{O}_{p} \left( T ^{-1} \right), \\
  \label{sec:stylized-ssrep-empirics-VAR1-sample-coef-resvar}
  \widehat{\Sigma} & = \widehat{\Gamma}_{yy} \left( 0 \right)
  - \widehat{\Gamma}_{yy} \left( 1 \right)
  \widehat{\Gamma}_{yy}^{-1} \left( 0 \right)
  \widehat{\Gamma}_{yy}^{\top} \left( 1 \right)
  + \mathcal{O}_{p} \left( T ^{-1} \right),
\end{align}
其中$\mathcal{O}_{p} \left( T^{-1} \right)$用于反映OLS定义式的系数估计方法,和样本自协方差求和估计式\eqref{eq:stylized-ssrep-empirics-autocovar}之间的微小不一致——对于一组随机变量的数列$X_{t}$,如果随着$T \mapsto \infty$, $T X_{T}$随机有界,我们将数列$X_{t}$表示为$\mathcal{O}_{p} \left( T^{-1} \right)$。

$VAR(1)$过程的自协方差、跨协方差方程,类似于\eqref{eq:stylized-ssrep-autocov},分别表示为$\widehat{\Gamma}_{yy}^{V}(h),\, \widehat{\Gamma}_{ss}^{V}(h), \, \widehat{\Gamma}_{ys}^{V}(h)$,并且$\widehat{\Gamma}_{yy}(0)$和$\widehat{\Gamma}_{yy}(h)$
的计算方式,类似于\eqref{eq:stylized-ssrep-lyapunov-func}\eqref{eq:stylized-ssrep-cov-ss-h};将$\hat{\phi}_{1}$的近似估计\eqref{sec:stylized-ssrep-empirics-VAR1-sample-coef-phi1}插入到估计过程中去,有
\begin{align}
  \label{eq:stylized-ssrep-var1-autocov-yy0}
  \widehat{\Gamma}_{yy}^{V} \left( 0 \right) = & \widehat{\Gamma}_{yy} \left( 0 \right) + \mathcal{O}_{p} \left( T^{-1} \right), \\
  \label{eq:stylized-ssrep-var1-autocov-yyh}
  \widehat{\Gamma}_{yy}^{V} \left( h \right) = &
  \left( \hat{\phi}_{1} \right)^{h} \widehat{\Gamma}_{yy}^{V} \left( 0 \right) + \mathcal{O}_{p} \left( T^{-1} \right)
  = \left[ \widehat{\Gamma}_{yy} \left( 1 \right) \,
  \widehat{\Gamma}_{yy}^{-1} \left( 0 \right)
  \right]^{h}
  \widehat{\Gamma}_{yy} \left( 0 \right)
  + \mathcal{O}_{p} \left( T^{-1} \right).
\end{align}

来比较$\widehat{\Gamma}_{yy}^{V} \left( h \right)$和$\widehat{\Gamma}_{yy} \left( h \right)$:
\begin{equation*}
  \widehat{\Gamma}_{yy}^{V} \left( h \right) =
  \begin{cases}
    \widehat{\Gamma}_{yy} \left( h \right) + \mathcal{O}_{p} \left( T^{-1} \right) & h=0, \\
    \widehat{\Gamma}_{yy} \left( h \right) + \mathcal{O}_{p} \left( T^{-1} \right) & h=1, \\
    \left[ \widehat{\Gamma}_{yy} \left( 1 \right) \,
    \widehat{\Gamma}_{yy}^{-1} \left( 0 \right)
    \right]^{h}
    \widehat{\Gamma}_{yy} \left( 0 \right)
    + \mathcal{O}_{p} \left( T^{-1} \right) & h > 1,
  \end{cases}
\end{equation*}
换句话说,当$h=0,1$时两个自协方差的估计矩阵相等(除了$\mathcal{O}_{p} \left( T^{-1} \right)$的部分);而当$h>1$时二者不相等。\cite{Schorfheide:2005jg}指出,如果实际时间序列数据能够较好契合VAR(1)特征,那么插入法的估计$\widehat{\Gamma}_{yy}^{V} \left( h \right)$要比OLS估计$\widehat{\Gamma}_{yy} \left( h \right)$更有效。

实际研究过程中,有时时间序列数据并不符合VAR(1)特征。就需要在$VAR(p), \, p >1$框架下分析协自协方差
\begin{equation}
  \label{sec:stylized-ssrep-empirics-varp-def}
  y_{t} = \phi_{1} y_{t-1} + \ldots + \phi_{p} y_{t-p} + \phi_{0} + u_{t}, \quad \, u_{t} \sim i.i.d. \, \mathbb{N} \left( 0, \Sigma \right)
\end{equation}

$p$值的选取标准,可通过贝叶斯信息量(Bayesian information criterion, BIC)\index{Bayesian information criterion(BIC) \dotfill 贝叶斯信息量}方法予以估计。此外还有如赤池信息量(Akaike information criterion, AIC)\index{Akaike information criterion(AIC) \dotfill 赤池信息量}等方法。一个直观的处理方法是改写成友矩阵(companion matrix)形式,即将$VAR(p)$时间序列向量$y_{t}$理解为一个关于堆栈向量$\tilde{y}_{t} = \left[ y_{t}^{\top}, y_{t-1}^{\top}, \ldots, y_{t-p+1}^{\top} \right]$的$VAR(1)$过程:
\begin{equation}
  \label{sec:stylized-ssrep-empirics-varp-companion}
\begin{split}
    & \tilde{y}_{t} = \tilde{\phi}_{1} \tilde{y}_{t-1} + \tilde{\phi}_{0} + \tilde{u}_{t}, \quad \tilde{u}_{t} \sim \, i.i.d. \, \mathbb{N} \left( 0, \widetilde{\Sigma} \right),\\
    & \tilde{\phi_{1}} =
    \begin{pmatrix}
    \phi_{1} & \ldots & \phi_{p-1} & \phi_{p} \\
    I_{n \times n} & \ldots & 0_{n \times n} & 0_{n \times n} \\
    \vdots & \ddots & \vdots & \vdots \\
    0_{n \times n} & \ldots & I_{n \times n} & 0_{n \times n}
  \end{pmatrix},
  \quad \tilde{\phi}_{0} =
  \begin{pmatrix}
  \phi_{0} \\
  0_{n \left( p - n \right) \times 1}
  \end{pmatrix}, \\
  & \tilde{\epsilon_{t}} =
  \begin{pmatrix}
    \epsilon_{t} \\
    0_{n \left( p - n \right) \times 1}
  \end{pmatrix},
  \quad \widetilde{\Sigma} =
  \begin{pmatrix}
    \Sigma & 0_{n \times n \left( p - 1 \right)} \\
    0_{n \left( p - 1 \right) \times n} &
    0_{n \left( p - 1 \right) \times n \left( p - 1 \right)}
  \end{pmatrix},
\end{split}
\end{equation}
这样一来,估计的步骤如下:首先利用类似于\eqref{eq:stylized-ssrep-var1-autocov-yy0}\eqref{eq:stylized-ssrep-var1-autocov-yyh}的思路,将$y_{t}$替换为$\tilde{y}_{t}$,算得$\tilde{y}_{t}$的自协方差$\widehat{\Gamma}_{\widetilde{yy}}(0), \widehat{\Gamma}_{\widetilde{yy}}(h)$。
随后根据$y_{t} = M^{\top} \tilde{y}_{t}$,还原$y_{t}$的自协方差矩阵,其中$M^{\top}$是一个选择矩阵(selection matrix)\index{selection matrix \dotfill 选择矩阵}
\begin{equation*}
  M^{\top} = \left[ I_{n}, 0_{n \times n \left( p - 1 \right)}\right].
\end{equation*}

\begin{figure}[htbp]
  \caption{抽样观测数据的跨相关}
  \centering
  \includegraphics[width=12cm]{./Figures/20180404-corss-correlations}
  \label{fig:stylized-ssrep-cross-correlation}
%
%  \small{Source: PBOC.}
\end{figure}

图\eqref{fig:stylized-ssrep-cross-correlation}
画出了基于抽样观测数据(表\ref{tab:stylized-ssrep-empirics-data})的产出增速相对于利率(黑色实线),通货膨胀率(红色段线),劳动收入份额(蓝色点线)的跨相关(cross correlation)值\footnote{Matlab中用xcov计算,跨协方差就是跨相关方程。},横轴表示跨越的时间期$h$。左图使用样本自协方差矩阵$\widehat{\Gamma}_{yy} \left( h \right)$\eqref{eq:stylized-ssrep-empirics-autocovar}测算。右图使用$VAR(1)$测算而得\footnote{Matlab中用varm来作VAR分析。},其中其中根据BIC检验确定$p=1$,根据\eqref{eq:stylized-ssrep-empirics-autocovar}确定自协方差$\widehat{\Gamma}_{yy} \left( h \right)$。两组跨相关性质接近,但量化值不同,这是因为VAR模型更"吝啬"一些,导致VAR生成的相关系数更加平滑。

\subsection{光谱分析}
\label{sec:stylized-ssrep-empirics-spectral}

直观上来看,我们似乎很容易用样本的周期图(periodgram)\index{periodgram \dotfill 周期图}来做光谱密度的近似估计,由\eqref{eq:stylized-ssrep-autocovariance-density}得
\begin{equation}
  \label{eq:spectral-periodgram-def}
\begin{split}
  \hat{f}_{yy} \left( \omega \right)
  & = \frac{1}{2 \pi} \sum_{h=-T+1}^{T-1} \widehat{\Gamma} \left( h \right)  \\
  & = \frac{1}{ 2 \pi}
  \left\{
  \widehat{\Gamma}_{yy} \left( 0 \right)
  + \sum_{h=1}^{T-1}
  \left[
  \widehat{\Gamma}_{yy} \left( h \right)
  + \widehat{\Gamma}_{yy}^{\top} \left( h \right)
  \right]
  \cos \left( \omega h \right)
  \right\},
\end{split}
\end{equation}
然而需要指出的是,尽管样本周期图对应的光谱密度 $\hat{f}_{yy} \left( \omega \right)$是对全体密度$f_{yy} \left( \omega \right)$的渐进无偏估计,但它并不适合用作近似估计,因为它不一致\footnote{
一致性(consistency)是指随着$T \rightarrow \infty$,估计值的偏差(biase)和方差都收敛至$0$,可参考\cite[Sec. 8.2]{Koopmans:1995vn}。}。

为了求得一致估计,需要沿着临近频率(adjacent frequencies)对样本周期图作平滑。基于基准频率(base frequency) $\omega = \frac{2 \pi}{T}$,定义一组基本频率(fundamental frequencies) $\left\{ \omega_{j} \right\}$,是基准频率的整数倍:
\begin{equation*}
  \omega_{j} = j \frac{2 \pi}{T}, \quad j = 1, \ldots, \frac{T-1}{2},
\end{equation*}

那么可以将平滑的周期图表示为$\overline{f}_{yy} \left( \omega \right)$,满足如下关系
\begin{equation}
  \label{eq:spectral-periodgram-smooth}
  \overline{f}_{yy} \left( \omega \right)  =
  \frac{\pi}{ \lambda \frac{T-1}{2}} \sum_{j=1}^{\frac{T-1}{2}}
  \mathcal{K} \left( \frac{\omega_{j} - \omega}{\lambda} \right)
  \hat{f}_{yy} \left( \omega_{j} \right),
\end{equation}
其中$\mathcal{K}(\cdot)$是一个核方程(kernel)\index{kernel \dotfill 核},满足$\int \mathcal{K}(x) \, \mathrm{d} x = 1$。核方程的具体形式有多种,一个简单的定义式可以表示为
\begin{equation}
  \label{eq:spectral-periodgram-smooth-kernel-def}
  \mathcal{K} \left( \frac{\omega_{j} - \omega}{\lambda} \right)
  \hat{f}_{yy} \left( \omega_{j} \right)
  \coloneqq \mathbb{I}
  \left\{
  - \frac{1}{2} < \frac{\omega_{j} - \omega }{\lambda} < \frac{1}{2}
  \right\}
  = \mathbb{I}
  \left\{
    \omega_{j} \in
    \mathbb{B} \left( \omega | \lambda \right)
  \right\},
\end{equation}
其中$\mathbb{I} \left\{ \cdot \right\}$是一个指示方程(indicator function)\index{indicator function \dotfill 指示方程}
见\eqref{eq:euler-complex-indicator-function-def},$\mathbb{B} \left( \omega | \lambda \right)$构成一个频率带(frequency band)\index{frequency band \dotfill 频率带}。关于核方程的简要介绍,可参考第\todo{ref}节。
