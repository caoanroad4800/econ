%!TEX root = ../DSGEnotes.tex
\subsection{超奇异边界积分算子的椭圆性}
\label{sec:bvp-bie-hyper-ellipticity}

第\ref{sec:bvp-hyperbie-operator}节介绍了超奇异边界积分算子$D$。由\eqref{eq:bvp-hyper-bie-D-u0-equiv0}可得,若引入特征解(eigensolution) $u_{0} \equiv 1, x \in \Gamma$,那么$\left( D u_{0}  \right) (x) = 0$。可见我们无法保证超奇异边界积分算子$D$在$H^{\frac{1}{2}}(\Gamma)$中全部都是椭圆的,而只是半椭圆(semi-ellipticity)\index{ellipticity!semi \dotfill 半椭圆性}的。在本节中,首先我们介绍半椭圆性。随后介绍替代方案:在$H^{\frac{1}{2}}(\Gamma)$中构建满足椭圆特征的子空间,两种常见的子空间如下文所示。

\subsubsection{超奇异边界积分算子的半椭圆性}
\label{sec:bvp-bie-hyper-semi-ellipticity}

\begin{theorem}[超奇异边界积分算子的半椭圆性]
\label{theorem:bvp-bie-hyper-semi-ellipticity}
  超奇异边界积分算子$D$是半椭圆的,即
  \begin{equation*}
    \langle D \nu, \nu \rangle_{\Gamma} \ge c_{1}^{\text{D}} \,
    \left\| \nu \right\|_{H^{\frac{1}{2}}(\Gamma)}^{2}, \quad \forall \, \nu \in H_{*}^{\frac{1}{2}}(\Gamma).
  \end{equation*}
\end{theorem}
\begin{proof}
  \begin{enumerate}
  \item 对于$\nu \in H_{*}^{\frac{1}{2}}(\Gamma)$,考虑齐次偏微分方程的解$u(x)$可以表示为如下双层位势
  \begin{equation*}
    u(x) \coloneqq - \left( W u \right)(x) , \quad x \in \Omega \cup \Omega{c}.
  \end{equation*}

  对于$x \in \Gamma$,$u(x)$的迹算子和共法导数算子以伴随双层位势的形式表示如下
  \begin{equation*}
    \begin{cases}
      \gamma_{0}^{\text{int}} u(x) = \left( 1 - \sigma(x) \right)
      \nu(x) - \left( K \nu \right)(x), \\
      \gamma_{1}^{\text{int}} u(x) = \left( D \nu \right) (x), \quad x \in \Gamma,
    \end{cases}
  \end{equation*}
  \begin{equation*}
    \begin{cases}
      \gamma_{0}^{\text{ext}} u(x) = - \sigma(x)
      \nu(x) - \left( K \nu \right)(x), \\
      \gamma_{1}^{\text{ext}} u(x) = \left( D \nu \right) (x), \quad x \in \Gamma.
    \end{cases}
  \end{equation*}

\item 由此可见,方程$u = -W u$构成内界狄利克雷边界值问题的唯一解
\begin{equation*}
  \begin{cases}
    - \Delta u(x) = 0 & x \in \Gamma, \\
    \gamma_{0}^{\text{int}} u(x) = \left( 1- \sigma(x) \right)\nu(x) - \left( K \nu \right)(x) & x \in \Gamma.
  \end{cases}
\end{equation*}

由格林第一恒等式\eqref{eq:bvp-a-u-nu-inner-prod}可得,解$u(x)$满足如下关系
\begin{equation*}
  \int_{\Gamma} \triangledown u(x) \triangledown w(x) \, d x
  = \langle \gamma_{1}^{\text{int}} u, \gamma_{0}^{\text{int}} w \rangle_{\Gamma}, \quad \forall \, w \in H^{1}(\Omega).
\end{equation*}

\item 设一个包围$\Omega$的球体$B_{R}(y_0)$,以$y_0 \in \Omega$为圆心,以$R > 2 \diam(\Omega)$为半径。即$\Omega \subset B_{R}(y_0)$。那么$u = - W u$同时也是以下狄利克雷边界值问题的唯一解
\begin{equation*}
  \begin{cases}
    - \Delta u(x) = 0 & x \in B_{R}(y_0)\backslash \overline{\Omega}, \\
    \gamma_{0}^{\text{ext}} u(x) = - \sigma(x) \nu(x) - \left(K \nu \right)(x) & x \in \Gamma = \partial \Omega,\\
    \gamma_{0} u(x) = \gamma_{0}^{\text{int}} u(x) + \gamma_{0}^{\text{ext}} u(x) = - \left( W \nu \right)(x) & x \in \partial B_{r}(y_0).
  \end{cases}
\end{equation*}

根据格林第一恒等式可得\eqref{eq:bvp-a-u-nu-inner-prod}可得,解$u(x)$表示为如下双线性形式
\begin{equation*}
  \int_{B_{R}(y_0) \backslash \overline{\Omega}}
  \triangledown u(x) \triangledown w(x) \, d x
  = - \langle
  \gamma_{1}^{\text{ext}} u, \gamma_{0}^{\text{ext}} w
  \rangle_{\Gamma}
  + \langle \gamma_{1} u, \gamma_{0} w \rangle_{\partial B_{r}(y_0)}, \quad \forall \, w \in H^{1}(B_{R}(y_0) \backslash \overline{\Omega}).
\end{equation*}

\begin{enumerate}
  \item 对于$x \in \Gamma$,根据定义有
  \begin{equation*}
    u(x) = \frac{1}{2 \left( d - 1 \right) \pi }
    \int_{\Gamma}
    \frac{
    \left( y - x, n_y \right)
    }{
    \left| x - y \right|^{d}
    }
    \nu(y) \, d s_y.
  \end{equation*}
  \item 对于 $x \in \partial B_{R}(y_0)$,有
  \begin{equation*}
    \begin{split}
      \left| u(x) \right| \le c_{1}(\nu) \, R^{1-d}, \\
      \left| \triangledown u(x) \right| \le c_{2}(\nu) R^{-d}.
    \end{split}
  \end{equation*}
\end{enumerate}

那么选取$w = u = - W \nu$,取极限$\lim R \rightarrow \infty$,可得外界域中的格林第一恒等式
\begin{equation*}
  \int_{\Omega^{c}} \left| \triangledown u(x) \right|^{2} \, dx
  = - \langle \gamma_{1}^{\text{ext}} u, \gamma_{0}^{\text{ext}} u \rangle_{\Gamma}.
\end{equation*}

\item 将内、外界域的格林第一恒等式相加求和,并考虑不同边界积分算子的跃动关系,我们有超奇异边界积分算子$D$的双线性形式
\begin{equation*}
  \begin{split}
    \underbrace{
    \langle D \nu, \nu \rangle_{\Gamma}
    }_{\eqqcolon \mathcal{C}}
    & =
    \langle \gamma_{1}^{\text{int}} u,
    \left[ \gamma_{0}^{\text{int}} - \gamma_{0}^{\text{ext}} u \right]
    \rangle_{\Gamma} \\
    & =
    \langle \gamma_{1}^{\text{int}} u, \gamma_{0}^{\text{int}} u \rangle_{\Gamma}
    - \langle \gamma_{1}^{\text{int}} u, \gamma_{0}^{\text{ext}} u \rangle_{\Gamma} \\
    & =
    \int_{\Omega} \left| \triangledown u(x) \right|^{2} \, d x
    + \int_{\Omega^{c}} \left| \triangledown u(x) \right|^{2} \, d x \\
    & = \underbrace{
    \left| u \right|_{H^{1}(\Omega)}^{2}
    }_{\eqqcolon \mathcal{B}}
    + \underbrace{
    \left| u \right|_{H^{1}(\Omega^{c})}^{2}
    }_{\eqqcolon \mathcal{A}}
    .
  \end{split}
\end{equation*}

\begin{enumerate}
  \item 来看$\mathcal{A}$。外界域$\Omega^{c}$中,双层位势$u(x) = - \left(W u \right)(x)$,取$\lim \left| x \right| \rightarrow \infty$,有
  \begin{equation*}
    c_{1}  \, \left\| u \right\|_{H^{1}(\Omega^{c})}^{2} \le
    \left| u \right|_{H^{1}(\Omega^{c})}^{2}.
  \end{equation*}
  \item 来看$\mathcal{B}$。
  \begin{enumerate}
    \item 由以下三点
    \begin{itemize}
    \item 对于$\nu \in H_{*}^{\frac{1}{2}}(\Gamma)$,有自然密度$w_{\text{eq}} \in H^{-\frac{1}{2}}(\Gamma)$,
    \item $V w_{\text{eq}} = 1$,
    \item 有界积分算子呈对称关系\eqref{eq:bvp-bie-operator-relation-3} 成立
    \end{itemize}
    可得
    \begin{equation*}
      \begin{split}
        \langle \gamma_{0}^{\text{int}} u, w_{\text{eq}} \rangle_{\Gamma} & =
        \langle
        \left( \frac{1}{2} I - K \right) \nu,
        w_{\text{eq}}
        \rangle_{\Gamma} \\
        & = \langle \nu, w_{\text{eq}} \rangle_{\Gamma}
        - \langle
        \left( \frac{1}{2} I + K \right) \nu,
        w_{\text{eq}} \rangle_{\Gamma} \\
        & = - \langle
        \left( \frac{1}{2} I + K \right) \nu,
        V^{-1} \rangle_{\Gamma} \\
        & = - \langle
        V ^{-1} \left( \frac{1}{2} I + K \right) \nu,
        1 \rangle_{\Gamma} \\
        & = - \langle
        \left( \frac{1}{2} I + K \right) V^{-1}  \nu,
        1 \rangle_{\Gamma} \\
        & = - \langle
        V^{-1}  \nu,
        \left( \frac{1}{2} I + K \right)  \rangle_{\Gamma} \\
        & = 0.
      \end{split}
    \end{equation*}
    \item 进而,根据索伯列夫空间中的范数等价定理\eqref{eq:sobolev-norm-equivalence-theorem}(Theorem \ref{theorem:sobolev-equivalence-norm-theorem})可得$H^{1}(\Omega)$空间中的一个等价范
    \begin{equation*}
      \left\| u \right\|_{H_{*}^{1}(\Omega)}
      \coloneqq \left\{
      \left[
      \langle
      \gamma_{0}^{\text{int}} u, w_{\text{eq}}
      \rangle_{\Gamma}
      \right]^{2}
      + \left\|
      \triangledown u \right\|_{L^{2}(\Omega)}^{2}
      \right\}^{\frac{1}{2}}.
    \end{equation*}
      \end{enumerate}
    \item 来看$\mathcal{C}$。由迹定理和双层位势的跃动关系可得
    \begin{equation*}
      \begin{split}
        \langle D \nu, \nu \rangle_{\Gamma} & \ge
        c \, \left\{
        \left\| u \right\|_{H^{1}(\Omega)}^{2} +
        \left\| u \right\|_{H^{1}(\Omega^{c})}^{2}
        \right\}\\
        & \ge \widetilde{c} \,
        \left\{
        \left\| \gamma_{0}^{\text{int}} u \right\|_{H^{\frac{1}{2}}(\Gamma)}^{2}
        + \left\| \gamma_{0}^{\text{ext}} u \right\|_{H^{\frac{1}{2}}(\Gamma)}^{2}
        \right\} \\
        & \ge
        \frac{1}{2} \widetilde{c}
        \left\|
        \gamma_{0}^{\text{int}} u
        - \gamma_{0}^{\text{ext}} u
        \right\|_{H^{\frac{1}{2}}(\Gamma)}^{2} \\
        & = c_{1}^{\text{D}} \,
        \left\| \nu \right\|_{H^{\frac{1}{2}}(\Gamma)}^{2}, \quad \forall \, \nu \in H_{*}^{\frac{1}{2}}(\Gamma).
      \end{split}
    \end{equation*}
\end{enumerate}
\end{enumerate}
由此可证得超奇异边界积分算子$D$的$H_{*}^{\frac{1}{2} (\Gamma)}$-椭圆性。
\end{proof}

\subsubsection{椭圆性所处子空间的替代方案}
Theorem \ref{theorem:bvp-bie-hyper-semi-ellipticity}对超奇异边界积分算子$D$的$H_{*}^{\frac{1}{2} (\Gamma)}$-椭圆性的证明,涉及到对解方程所处空间的限定:需要将其限定在一个更适宜的子空间之内——即正交于常数的子空间。显然,随着算子$D$双线性内积类型的不同,对应正交子空间的种类也有所不同。具体说来,根据索伯列夫空间中的范数等价定理\eqref{eq:sobolev-norm-equivalence-theorem}(Theorem \ref{theorem:sobolev-equivalence-norm-theorem}),可定义一个$H^{\frac{1}{2}}(\Gamma)$中的等价范
\begin{equation*}
  \left\| \nu \right\|_{H_{*}^{\frac{1}{2}}(\Gamma)}
  \coloneqq
  \left\{
  \left[
  \langle \nu, w_{\text{eq}} \rangle_{\Gamma}
  \right]^{2}
  + \left| \nu \right|_{H^{\frac{1}{2}}(\Gamma)}^{2}
  \right\}^{\frac{1}{2}},
\end{equation*}
其中$w_{\text{eq}} \in H^{-\frac{1}{2}}(\Gamma)$由自然密度的定义式  \eqref{eq:bvp-bie-single-scaling-equation}给出。那么

\begin{corollary}[超奇异边界积分算子的半椭圆性]
  \label{corollary:bvp-bie-single-semi-ellipticity}
  超奇异积分算子$D$是$H^{\frac{1}{2}}(\Gamma)$-半椭圆(semi-elliptic)\index{ellipticity!semi \dotfill 半椭圆性}的,即
  \begin{equation}
    \label{eq:bvp-bie-single-semi-ellipticity}
    \langle D \nu, \nu \rangle_{\Gamma}
    \ge \overline{c}_{1}^{\text{D}} \, \left| \nu \right|_{H^{\frac{1}{2}}(\Gamma)}^{2}, \quad \forall \nu \in H^{\frac{1}{2}}(\Gamma).
  \end{equation}
\end{corollary}

  Corollary \ref{corollary:bvp-bie-single-semi-ellipticity}中,对子空间$H_{*}^{\frac{1}{2}}(\Gamma)$的定义,涉及到边界积分方程\eqref{eq:bvp-bie-single-scaling-equation}的唯一解,即自然密度$w_{\text{eq}} \in H^{-\frac{1}{2}}(\Gamma)$。然而在实际计算过程中,该定义往往较难直接处理。有鉴于此,需要另外寻求替代方案。

  常见的替代方案之一是定义另一个子空间$H_{**}^{\frac{1}{2}}(\Gamma)$,从而简化内积结构
  \begin{equation*}
    H_{**}^{\frac{1}{2}}(\Gamma) \coloneqq
    \left\{ \nu \in H^{\frac{1}{2}}(\Gamma): \langle \nu, 1 \rangle_{\Gamma} = 0 \right\}.
  \end{equation*}

则由\eqref{eq:bvp-bie-single-semi-ellipticity}得,对于$\nu \in H_{**}^{\frac{1}{2}}(\Gamma)$,内积可以简化为
\begin{equation}
  \label{eq:bvp-bie-single-innnerp-simplify-starstar}
  \begin{split}
  \langle D \nu, \nu \rangle_{\Gamma}
  & \ge \overline{c}_{1}^{\text{D}} \, \left| \nu \right|_{H^{\frac{1}{2}}(\Gamma)}^{2} \\
  & = \overline{c}_{1}^{\text{D}} \,
  \left\{
  \left| \nu \right|_{H^{\frac{1}{2}}(\Gamma)}^{2}
  + \left[
  \langle \nu, 1 \rangle_{\Gamma}
  \right]^2
  \right\}\\
  & \ge \widetilde{c}_{1}^{\text{D}} \,
  \left\| \nu \right\|_{H^{\frac{1}{2}}(\Gamma)}^{2}.
\end{split}
\end{equation}

若是采用子空间$H_{**}^{\frac{1}{2}}(\Gamma)$,我们也可以证明算子$D$的$H_{**}^{\frac{1}{2}}(\Gamma)$-椭圆性。(证明略)。

另一个替代方案是,考虑一个子空间$\Gamma_{0} \subset \Gamma$。对于某一给定的$\nu \in \widetilde{H}^{\frac{1}{2}}(\Gamma_{0})$,定义其延拓$\widetilde{\nu} \in H^{\frac{1}{2}}(\Gamma)$如下
\begin{equation*}
  \widetilde{\nu}(x) =
  \begin{cases}
    \nu(x) & x \in \Gamma_{0}, \\
    0 & \text{否则}.
  \end{cases}
\end{equation*}

根据范数等价定理,$H^{\frac{1}{2}}(\Gamma)$上的等价范可定义为
\begin{equation*}
  \left\| w \right\|_{H^{\frac{1}{2}}(\Gamma), \Gamma_{0}}
  \coloneqq
  \left\{
  \left\| w \right\|_{L^{2}(\Gamma \backslash \Gamma_{0})}^{2}
  + \left| w \right|_{H^{\frac{1}{2}}(\Gamma)}^{2}
  \right\}^{\frac{1}{2}}.
\end{equation*}

那么,对于$\nu \in \widetilde{H}^{\frac{1}{2}}(\Gamma_{0})$,内积简化为
\begin{equation*}
  \begin{split}
    \langle D \nu, \nu \rangle_{\Gamma_{0}}
    & = \langle D \widetilde{\nu}, \widetilde{\nu} \rangle_{\Gamma} \\
    & \ge \overline{c}_{1}^{\text{D}} \,
    \left[
    \left\| \widetilde{\nu} \right\|_{L^{2}(\Gamma \backslash \Gamma_{0})}^{2}
    + \left| \widetilde{\nu} \right|_{H^{\frac{1}{2}}(\Gamma)}^{2}
    \right] \\
    & = \overline{c}_{1}^{\text{D}} \,
    \left\| \widetilde{\nu} \right\|_{H^{\frac{1}{2}}(\Gamma), \Gamma_{0}}^{2} \\
    & \ge \widehat{c}_{1}^{\text{D}} \,
    \left\| \widetilde{\nu} \right\|_{H^{\frac{1}{2}}(\Gamma)}^{2} \\
    & = \widehat{c}_{1}^{\text{D}} \,
    \left\| \nu \right\|_{\widetilde{H}^{\frac{1}{2}}(\Gamma_{0})}^{2}.
  \end{split}
\end{equation*}

同样地,若是采用子空间$\widetilde{H}^{\frac{1}{2}}(\Gamma_{0})$,我们可以证明算子$D$的$\widetilde{H}^{\frac{1}{2}}(\Gamma_{0})$-椭圆性。(证明略)。
