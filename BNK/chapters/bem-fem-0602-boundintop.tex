%!TEX root = ../DSGEnotes.tex

\subsection{超奇异边界积分算子}
\label{sec:bvp-hyperbie-operator}

由第\ref{sec:bvp-double-layer-potential}节关于双层位势的介绍可见,给定密度方程$\nu \in H^{\frac{1}{2}}(\Gamma)$,它的双层位势$\left( W \nu \right)$的余法向导数(conormal derivative)$\gamma_{1}^{\text{int}}\left( W \nu \right)(x)$定义了一个有界算子
\begin{equation*}
  \gamma_{1}^{\text{int}} W : H^{\frac{1}{2}}(\Gamma) \mapsto H^{- \frac{1}{2}}(\Gamma),
\end{equation*}
可表示为
\begin{equation}
  \label{eq:bvp-hypersingular-bie-def}
  \left( D \nu \right)(x) \coloneqq - \gamma_{1}^{\text{int}}\left( W \nu \right)(x)
  = - \lim_{\Omega \ni \widetilde{x} \rightarrow x \in \Gamma}
  n_x \triangledown_{\widetilde{x}} \left( W \nu \right)(\widetilde{x}), \quad x \in Gamma,
\end{equation}
满足
\begin{equation}
  \label{eq:bvp-hypersingular-bie-norm}
  \left\| D \nu \right\|_{H^{-\frac{1}{2}}(\Gamma)}
  \le c_2^{D} \, \left\| \nu \right\|_{H^{\frac{1}{2}}(\Gamma)}, \quad \nu \in H^{\frac{1}{2}}(\Gamma).
\end{equation}

具体说来,由\eqref{eq:bvp-hypersingular-bie-def}可见
\begin{enumerate}
\item 对于$d=2$的二维系统,双层位势表示为
\begin{equation}
  \label{eq:bvp-hyper-bie-d2}
  \left( W \nu \right)(\widetilde{x}) =
  \frac{1}{2 \pi} \lim_{\varepsilon \rightarrow 0}
  \int_{y \in \Gamma: \left| y -x \right| \ge \varepsilon}
  \frac{\left( \widetilde{x} - y, n_y \right)}{\left| \widetilde{x} - y \right|^2} \nu(y) d s_y, \quad \widetilde{x} \in \Omega.
\end{equation}

对于任一给定的$\varepsilon > 0$,我们取极限值$\Omega \ni \widetilde{x} \rightarrow  x \in \Gamma$,将\eqref{eq:bvp-hyper-bie-d2}对$\widetilde{x}$求偏导,代入\eqref{eq:bvp-hypersingular-bie-def}得
\begin{equation}
  \label{eq:bvp-hyper-bie-double-potential-d2}
  \left( D_{\varepsilon} \nu \right)(x)
  = \frac{1}{2 \pi}
  \lim_{\varepsilon \rightarrow 0}
  \int_{y \in \Gamma: \left| y - x \right| \ge \varepsilon}
  \left[
  - \frac{\left( n_x, n_y \right)}{\left| x - y \right|^2}
  + 2 \frac{
  \left( x - y, n_x \right)
  \left( x - y, n_y \right)
  }{
  \left| x - y \right|^4
  }
  \right]
  \nu(y) \, d s_y.
\end{equation}

\item 类似地,对于$d=3$的三维系统,可得
\begin{equation}
  \label{eq:bvp-hyper-bie-double-potential-d3}
  \left( D_{\varepsilon} \nu \right)(x)
  = \frac{1}{4 \pi}
  \int_{y \in \Gamma: \left| y - x \right| \ge \varepsilon}
  \left[
  - \frac{\left( n_x, n_y \right)}{\left| x - y \right|^{3}}
  + 3 \frac{
  \left( y - x, n_x \right)
  \left( y - x, n_y \right)
  }{
  \left| x - y \right|^{5}
  }
  \right]
  \nu(y) \, d s_y.
\end{equation}
\end{enumerate}

然而,无论是二维系统\eqref{eq:bvp-hyper-bie-double-potential-d2}还是三维系统\eqref{eq:bvp-hyper-bie-double-potential-d3}的情况,极限$\varepsilon \rightarrow 0, x \in \Gamma$下的积分都不存在柯西主值(Cauchy principal value)\index{Cauchy principal value \dotfill 柯西主值}。我们将此类算子$D$称为超奇异边界积分算子(hypersingular boundary integral operator),作为柯西积分(Cauchy integral)的一个扩展\footnote{关于超奇异边界积分方程(hypersingular boundary integral equation),一个简要介绍可见\cite[Ch.9]{Mason:2003tc},更多介绍可见\cite{Lifanov:2003va, Ang:2013vv},数值近似算法的探讨可见\cite{Guiggiani:1992cm}。}。

在不存在柯西主值的情况下,为了求得$D$的线性表现式,往往需引入一些正则设定,如将$u_{0}(x) \equiv 1 $引入泊松方程表现式\eqref{eq:bvp-fund-var-ux-uy-repform}中可得(更详细介绍见\pageref{sec:bvp-bie-hyper-ellipticity}页第\ref{sec:bvp-bie-hyper-ellipticity}节。)
\begin{equation*}
  1 = - \int_{\Gamma} \gamma_{1,y}^{\text{int}} U^{*}(\widetilde{x},y) d s_y, \quad \widetilde{x} \in \Omega,
\end{equation*}

\begin{equation*}
  \hookrightarrow \triangledown_{x} \left( W u_{0} \right)(\widetilde{x}) = \underline{0}, \quad \widetilde{x} \in \Omega,
\end{equation*}

\begin{equation}
  \label{eq:bvp-hyper-bie-D-u0-equiv0}
  \hookrightarrow \left( D u_{0} \right) (x) = 0, \quad x \in \Gamma.
\end{equation}

进而可以将\eqref{eq:bvp-hypersingular-bie-def}写为
\begin{equation*}
  \begin{split}
  \left( D \nu \right)(x)
  &= - \lim_{\Omega \ni \widetilde{x} \rightarrow x \in \Gamma}
  n_x \triangledown_{\widetilde{x}} \left( W \nu \right)(\widetilde{x}) \\
  &= - \lim_{\Omega \ni \widetilde{x} \rightarrow x \in \Gamma}
  n_x \triangledown_{\widetilde{x}}
  \int_{\Gamma}
  \gamma_{1,y}^{\text{int}}
  U^{*}(\widetilde{x}, y)
  \left[ \nu(y) - \nu(x) \right]
  d s_y, \quad x \in \Gamma.
\end{split}
\end{equation*}

根据上式,如果密度方程是连续的,我们就可以将超奇异边界积分算子$D$的表现式写为
\begin{equation}
  \label{eq:bvp-hypersingular-operator-representation-form}
  \left( D \nu \right)(x) =
  - \int_{\Gamma}
  \gamma_{1,x}^{\text{int}}
  \gamma_{1,y}^{\text{int}}
  U^{*}(x,y)
  \left[
  \nu(y) - \nu(x)
  \right]
  d s_y, \quad x \in \Gamma.
\end{equation}
下面分二维、三维空间两个情况继续展开分析。

\subsubsection{二维空间}
\label{sec:bvp-hyperbie-operator-d2}
以$d=2$为例。假定$\Gamma = \partial \Omega$是分段平滑的,满足
\begin{equation*}
  \Gamma = \cup_{k=1}^{p} \Gamma_{k},
\end{equation*}
每一分段$\Gamma_{k}$都可由局部参数化表示
\begin{equation}
  \label{eq:bvp-hypersingular-gamma-k}
  \Gamma_{k} \coloneqq y = y(t) = \begin{pmatrix}
  y_1(t) \\ y_2 (t)
  \end{pmatrix}
  , \quad t \in \left( t_k, t_{k+1} \right),
\end{equation}
其中$y_{i}(t),i=1,2$为连续可导方程。

结合\eqref{eq:bvp-hypersingular-operator-representation-form}、\eqref{eq:bvp-hypersingular-gamma-k}我们有
\begin{equation*}
  \begin{split}
    d s_y &= \sqrt{
    \left[ y_{1}'(t)\right]^2
    + \left[ y_{2}'(t)\right]^2
    } \,
    dt \\
    n(y) &= \frac{1}{
    \sqrt{
    \left[ y_{1}'(t)\right]^2
    + \left[ y_{2}'(t)\right]^2
    }
    }
    \begin{pmatrix}
      y_{2}'(t) \\ - y_{1}'(t)
    \end{pmatrix}
    , \quad y \in \Gamma_k,
  \end{split}
\end{equation*}
其中$n(y)$表示外部法向量(exterior normal vector)\index{exterior normal vector \dotfill 外部法向量}。

二维系统中对于$x \in \mathbb{R}^{2}$的情况,某一标量方程$\widetilde{\nu}(x)$的旋转(rotation,或称curl)\index{rotation \dotfill 旋转}可以定义为
\begin{equation*}
  \underline{\curl} \, \widetilde{\nu}(x) \coloneqq
  \begin{pmatrix}
    \frac{\partial}{\partial x_2} \widetilde{\nu}(x) \\
    - \frac{\partial}{\partial x_1} \widetilde{\nu}(x)
  \end{pmatrix},
\end{equation*}
现在来看$\widetilde{\nu}(x)$的计算:给定某一密度方程$\nu(x), x \in \Gamma_{k}$,则我们可以考虑一个在$\Gamma_k$的适宜邻域上的延拓$\widetilde{\nu}$
\begin{equation}
  \widetilde{\nu}(\widetilde{x}) = \nu(x), \quad \widetilde{x} = x + \left( \widetilde{x} - x, \underline{n}(x) \right) \underline{n}(x).
\end{equation}

也即,对于$x \in \Gamma_{k}$,我们可以定义$\nu(x)$的旋转$\curl_{\Gamma_k} \nu(x)$为
\begin{equation*}
  \begin{split}
    \curl_{\Gamma_k} \nu(x) &\coloneqq \underline{n}(x) \,  \underline{\curl} \widetilde{\nu}(x) \\
    &= n_{1}(x) \frac{\partial}{\partial x_{2}} \widetilde{\nu}(x)
    - n_{2}(x) \frac{\partial}{\partial x_{1}} \widetilde{\nu}(x), \quad x \in \Gamma_{k}.
  \end{split}
\end{equation*}

根据上式我们有
\begin{equation*}
  \begin{split}
    \int_{\Gamma_{k}}
    \curl_{\Gamma_{k}} \nu(y)
    d s_y & =
    \int_{\Gamma_{k}}
    \left[
    n_{1}(y) \frac{
    \partial
    }{\partial y_{2}}
    \widetilde{\nu}(y)
    -
    n_{2}(y) \frac{
    \partial
    }{\partial y_{1}}
    \widetilde{\nu}(y)
     \right]
     d s_y \\
     & = \int_{t_{k}}^{t_{k+1}}
     \left[
     y_{2}'(t)
     \frac{
     \partial
     }{\partial y_{2}}
     \nu(y(t))
     +
     y_{1}'(t)
     \frac{
     \partial
     }{\partial y_{1}}
     \nu(y(t))
     \right] \, dt\\
     &= \int_{t_{k}}^{t_{k+1}}
     \frac{d}{dt}
     \nu(y(t)) dt,
  \end{split}
\end{equation*}
通过该式可见,$\curl_{\Gamma_{k}} \nu$与预先选定的延拓$\widetilde{\nu}$无关。
\begin{lemma}
  \label{lemma:bvp-supersingular-curl-gammak}
  设一个开放的有界区间$\Gamma_{k}$,$\Gamma_{k}$由局部参数式\eqref{eq:bvp-hypersingular-gamma-k}所定义。对应的连续可导方程为$y_{i}(t),i=1,2$。

  如果$\nu$和$w$连续可导,则我们有分部积分式
  \begin{equation*}
    \begin{split}
      \int_{\Gamma_{k}} \nu(y) \curl_{\Gamma_{k}} w(y) \, d s_y
      = - \int_{\Gamma_{k}} \curl_{\Gamma_{k}} \nu(y) w(y) d s_{y}
      + \nu(y(t)) w(y(t)) \big|_{t_{k}}^{t_{k+1}}.
    \end{split}
  \end{equation*}
\end{lemma}
\begin{proof}
\begin{equation*}
  \begin{split}
      \int_{\Gamma_{k}} \curl_{\Gamma_{k}}
      \left[ \nu(y) w(y) \right] \, d s_y &=
      \int_{t_{k}}^{t_{k+1}} \frac{d}{dt}
      \left[ \nu(y(t)) w(y(t)) \right] \, dt \\
      &=\left[ \nu(y(t)) w(y(t)) \right]_{t=t_{k}}^{t=t_{k+1}}.
  \end{split}
\end{equation*}
\end{proof}

若$\nu$是沿着闭曲线(closed curve)\index{closed curve \dotfill 闭曲线} $\Gamma$而定义的\footnote{闭曲线是指这样的一种曲线,其起始点同时也是其终结点。},则我们可将$\nu$的旋转$\curl_{\Gamma} \nu(x)$定义为
\begin{equation*}
  \curl_{\Gamma} \nu(x) \coloneqq \curl_{\Gamma_{k}} \nu(x), \quad x \in \Gamma_{k}, k = 1, \ldots, p,
\end{equation*}
那么根据Lemma \ref{lemma:bvp-supersingular-curl-gammak},我们有

\begin{corollary}
  \label{corollary:bvp-supersingular-curl-gammak}
  设$\Gamma$是一个分段平滑的闭曲线,如果方程$\nu$和$w$均分段连续可微,则我们有
  \begin{equation}
    \label{eq:bvp-supersingular-curl-gammak}
    \int_{\Gamma} \nu(y) \curl_{\Gamma} w(y) d s_y
    = - \int_{\Gamma} \curl_{\Gamma} \nu(y) w(y) d s_y
    + \sum_{k=1}^{p} \nu(y(t)) w(y(t)) \big|_{t_{k}}^{t_{k+1}},
  \end{equation}

  进一步,如果$\nu$和$w$全局连续,则
  \begin{equation}
    \label{eq:bvp-supersingular-curl-gammakk}
    \int_{\Gamma} \nu(y) \curl_{\Gamma} w(y) d s_{y}
    = - \int_{\Gamma} \curl \nu(y) w(y) d s_{y}.
  \end{equation}
\end{corollary}

\begin{proof}
略。
\end{proof}

利用分部积分法,我们可以将超奇异边界积分算子$D$的双线性形式,改写为单层位势算子$V$的双线性形式。在二维系统中,这种改写可以表示如下。

\begin{theorem}
  \label{theorem:bvp-hypersingular-bilinear-d2}
  设$\Gamma$是一个分段平滑的封闭曲线。$u,\nu \in \Gamma$是全局连续方程。进一步,设$u,\nu$在$\Gamma$的某一段$\Gamma_{k}$上连续可微。那么我们有
  \begin{equation}
    \label{eq:bvp-hypersingular-d-bilinear-form}
    \langle D u, \nu \rangle_{\Gamma}
    = - \frac{1}{2 \pi}
    \int_{\Gamma} \curl_{\Gamma} \nu(x)
    \int_{\Gamma} \log \left| x - y \right|
    \curl_{\Gamma} u(y)
    d s_y d s_x.
  \end{equation}
\end{theorem}

\begin{proof}
  根据定义式\eqref{eq:bvp-hypersingular-bie-def},超奇异边界积分算子$D$可以定义为双层位势算子$W$的法导数$\gamma_{1}^{\text{int}} W$的负数。对于$\widetilde{x} \in \Omega$,定义
  \begin{equation*}
    w(\widetilde{x}) \coloneqq \left(W u \right)(\widetilde{x})
    = - \frac{1}{2 \pi}
    \int_{\Gamma} u(y) \frac{\partial}{\partial n_{y}}
    \log \left| \widetilde{x} - y \right| d s_y, \quad \widetilde{x} \in \Omega.
  \end{equation*}

一方面$\widetilde{x} \in \Omega, y \in \Gamma \Rightarrow \widetilde{x} \neq y$,另一方面
\begin{equation*}
  \frac{\partial}{\partial y_i} \log \left| \widetilde{x} - y \right|
  = \frac{y_i - \widetilde{x}_{i}}{\left| \widetilde{x} - y \right|^{2}}
  = - \frac{\widetilde{x}_{i} - y_{i}}{\left| \widetilde{x} - y \right|^{2}}
  = - \frac{\partial}{\partial \widetilde{x}_{i}} \log \left| \widetilde{x} - y \right|,
\end{equation*}
结合这两方面可得
\begin{equation*}
  \frac{\partial}{\partial \widetilde{x}_{i}}
  \left(
  \frac{\partial}{\partial n_{y}}
  \log \left| \widetilde{x} - y \right|
  \right)
  = - \underline{n}(y)
  \triangledown_{y}
  \left(
  \frac{\partial}{\partial y_{i}}
  \log \left| \widetilde{x} - y \right|
  \right).
\end{equation*}

由于$\Delta_{y} \log \left| \widetilde{x} - y \right| = 0, \text{对于} y \neq \widetilde{x}$,则我们有以下两个等式
\begin{equation*}
  \begin{split}
    \curl_{\Gamma, y}
    \left(
    \frac{\partial}{\partial y_{1}}
    \log
    \left|
    \widetilde{x} - y
    \right|
    \right)
    &=
    n_{1}(y)
    \frac{\partial}{\partial y_2}
    \frac{\partial}{\partial y_1}
    \log
    \left|
    \widetilde{x} - y
    \right|
    -
    n_{2}(y)
    \frac{\partial}{\partial y_1}
    \frac{\partial}{\partial y_1}
    \log
    \left|
    \widetilde{x} - y
    \right| \\
    & =
    n_{1}(y)
    \frac{\partial}{\partial y_2}
    \frac{\partial}{\partial y_1}
    \log
    \left|
    \widetilde{x} - y
    \right|
    +
    n_{2}(y)
    \frac{\partial}{\partial y_2}
    \frac{\partial}{\partial y_2}
    \log
    \left|
    \widetilde{x} - y
    \right| \\
    & = \underline{n}(y) \triangledown{y}
    \left(
    \frac{\partial}{\partial y_{2}}
    \log
    \left|
    \widetilde{x} - y
    \right|
    \right),
  \end{split}
\end{equation*}

\begin{equation*}
  \begin{split}
    \curl_{\Gamma, y}
    \left(
    \frac{\partial}{\partial y_{2}}
    \log
    \left|
    \widetilde{x} - y
    \right|
    \right)
    &=\underline{n}(y) \triangledown{y}
    \left(
    \frac{\partial}{\partial y_{1}}
    \log
    \left|
    \widetilde{x} - y
    \right|
    \right).
  \end{split}
\end{equation*}

因此,可以将全局连续方程$u$的双层位势$w = \left(W u \right)$的偏导数写为如下两个分部积分形式
\begin{equation*}
\begin{split}
  \frac{\partial}{\partial \widetilde{x}_{1}} w(\widetilde{x})
  &=
  -\frac{1}{2 \pi}
  \int_{\Gamma}
  u(y)
  \frac{\partial}{\partial \widetilde{x}}
  \frac{\partial}{\partial n_{y}}
  \log
  \left|
  \widetilde{x} - y
  \right|
  \, d s_y \\
  &=
  \frac{1}{2 \pi}
  \int_{\Gamma}
  u(y)
  n(y)
  \triangledown_{y}
  \left(
  \frac{\partial}{\partial y_{1}}
  \log
  \left|
  \widetilde{x} - y
  \right|
  \right)
  \, d s_y \\
  &=
  -\frac{1}{2 \pi}
  \int_{\Gamma}
  u(y)
  \curl_{\Gamma,y}
  \left(
  \frac{\partial}{\partial y_{2}}
  \log
  \left|
  \widetilde{x} - y
  \right|
  \right)
  \, d s_y\\
  &=
  \frac{1}{2 \pi}
  \int_{\Gamma}
  \curl_{\Gamma}
  u(y)
  \frac{\partial}{\partial y_{2}}
  \log
  \left|
  \widetilde{x} - y
  \right|
  \, d s_y,
\end{split}
\end{equation*}
\begin{equation*}
  \begin{split}
    \frac{\partial}{\partial \widetilde{x}_{2}} w(\widetilde{x})
    &= - \frac{1}{2 \pi}
    \int_{\Gamma}
    \curl_{\Gamma}
    u(y)
    \frac{\partial}{\partial y_{1}}
    \log
    \left|
    \widetilde{x} - y
    \right|
    \, d s_y.
  \end{split}
\end{equation*}

根据定义,双层位势$w = W u$的法导数
\begin{equation*}
\begin{split}
  & \underline{n}(x) \triangledown_{\widetilde{x}} w(\widetilde{x})\\
  &=
  \frac{1}{2 \pi}
  \int_{\Gamma}
  \curl_{\Gamma}
  u(y)
  \left[
  n_{1}(x)
  \frac{\partial}{\partial y_{2}}
  \log
  \left|
  \widetilde{x} - y
  \right|
  -
  n_{2}(x)
  \frac{\partial}{\partial y_{1}}
  \log
  \left|
  \widetilde{x} - y
  \right|
  \right]
  \, d s_{y} \\
  &=
  \frac{1}{2 \pi}
  \lim_{\varepsilon \rightarrow 0}
  \int_{y \in \Gamma: \left| y - x \right| \ge \varepsilon}
  \curl_{\Gamma}
  u(y)
  \left[
  n_{1}(x)
  \frac{\partial}{\partial y_{2}}
  \log
  \left|
  \widetilde{x} - y
  \right|
  -
  n_{2}(x)
  \frac{\partial}{\partial y_{1}}
  \log
  \left|
  \widetilde{x} - y
  \right|
  \right]
  \, d s_y
\end{split}
\end{equation*}

取极限$\Omega \ni \widetilde{x} \rightarrow x \in \Gamma$,上式变为
\begin{equation*}
  \begin{split}
    & \frac{\partial}{\partial n_{x}} w(x) \\
    & =
    \frac{1}{2 \pi}
    \lim_{\varepsilon \rightarrow 0}
    \int_{y \in \Gamma: \left| y - x \right| \ge \varepsilon}
    \curl_{\Gamma}
    u(y)
    \left[
    n_{1}(x)
    \frac{\partial}{\partial y_{2}}
    \log
    \left|
    x - y
    \right|
    -
    n_{2}(x)
    \frac{\partial}{\partial y_{1}}
    \log
    \left|
    x - y
    \right|
    \right]
    \, d s_y \\
    & =
    - \frac{1}{2 \pi}
    \lim_{\varepsilon \rightarrow 0}
    \int_{y \in \Gamma: \left| y - x \right| \ge \varepsilon}
    \curl_{\Gamma}
    u(y)
    \left[
    n_{1}(x)
    \frac{\partial}{\partial x_{2}}
    \log
    \left|
    x - y
    \right|
    -
    n_{2}(x)
    \frac{\partial}{\partial x_{1}}
    \log
    \left|
    x - y
    \right|
    \right]
    \, d s_y \\
    &=
    - \frac{1}{2 \pi}
    \lim_{\varepsilon \rightarrow 0}
    \int_{y \in \Gamma: \left| y - x \right| \ge \varepsilon}
    \curl_{\Gamma}
    u(y)
    \curl_{\Gamma, x}
    \log
    \left| x - y \right|
    \, d s_y.
  \end{split}
\end{equation*}

进而
\begin{equation*}
  \begin{split}
    & \int_{\Gamma} \nu(x) \frac{\partial}{\partial n_x} w(x) d s_x \\
    & =
    -\frac{1}{2 \pi}
    \int_{x \in \Gamma}
    \nu(x)
    \lim_{\varepsilon \rightarrow 0}
    \int_{y \in \Gamma: \left| y - x \right| \ge \varepsilon}
    \curl_{\Gamma} u(y)
    \curl_{\Gamma, x} \log \left| x - y \right|
    \, d s_y d s_x \\
    & =
    -\frac{1}{2 \pi}
    \int_{y \in \Gamma}
    \curl_{\Gamma} u(y)
    \lim_{\varepsilon \rightarrow 0}
    \int_{x \in \Gamma: \left| x - y \right| \ge \varepsilon}
    \nu(x) \curl_{\Gamma, x}
    \log \left| x - y \right|
    \, d s_x d s_y.
  \end{split}
\end{equation*}

由此可证。
\end{proof}

\subsubsection{三维空间}
超奇异积分算子$D$的分部积分表现式也可以用于三维系统$d=3$的分析中\citep{Dautray:1990hc}。

设一个分段平滑表面$\Gamma = \cup_{k=1}^{p}$,其中每一个分段$\Gamma_{k}$都可以用参数表示如下
\begin{equation*}
  y \in \Gamma_{k}:y(s,t) =
  \begin{pmatrix}
    y_{1}(s,t) \\
    y_{2}(s,t) \\
    y_{3}(s,t)
  \end{pmatrix}, \quad (s,t) \in \tau,
\end{equation*}
$\tau$指某个参考元(reference element)\index{reference element \dotfill 参考元}。

$\mathbb{R}^3$中某个向量方程$\underline{\nu}$的旋转\index{rotation \dotfill 旋转}可以定义为
\begin{equation*}
  \underline{\curl} \underline{\nu}(x) \coloneqq
  \triangledown \times \underline{\nu}(x), \quad x \in \mathbb{R}^{3}.
\end{equation*}

如果$u$是个$\Gamma_{k}$上的标量方程,其表面旋转(surface curl)可定义为
\begin{equation*}
  \underline{\curl}_{\Gamma_{k}} u(x) \coloneqq
  \underline{n}(x) \times \triangledown \widetilde{u}(x), \quad x \in \Gamma_{k},
\end{equation*}
其中$\widetilde{u} \in \Gamma_{k}$是一个给定的$u \in \Gamma_{k}$在$\Gamma_{k}$的三维邻域上的适宜延拓。

此外我们有
\begin{equation*}
  \curl_{\Gamma_{k}} \underline{\nu}(x) \coloneqq
  \underline{n}(x) \times \underline{\curl} \, \underline{ \widetilde{\nu}}(x), \quad x \in \Gamma_{k}.
\end{equation*}

\begin{lemma}
  \label{lemma:bvp-hypersingular-curl-u-nu-d3}
  设$\Gamma$为一个$\mathbb{R}^{3}$中的分段平滑封闭利普希茨表面。假定每个表面分段$\Gamma_{k}$都是平滑的,对应分段平滑边界曲线$\partial \Gamma_{k}$。设$u$和$\nu$均在$\Gamma$中全局连续,并且在每个分段$\Gamma_{k}$中都局部有节且平滑。那么由分部积分可得
  \begin{equation}
    \label{eq:bvp-hypersingular-curl-u-nu-d3}
    \int_{\Gamma} \underline{\curl}_{\Gamma} u(x) \underline{\nu}(x) \, d s_x
    = - \int_{\Gamma}
    u(x)
    \curl_{\Gamma} \underline{\nu}(x) \, d s_x.
  \end{equation}
\end{lemma}
\begin{proof}
根据乘积法则可得
\begin{equation*}
  \triangledown \times \left[ \widetilde{u}(x) \underline{\nu}(x) \right]
  \equiv \triangledown \widetilde{u}(x) \times \underline{\nu}(x)
  + \widetilde{u}(x)
  \left[
  \triangledown \times \underline{\nu} (x)
  \right],
\end{equation*}

LHS \eqref{eq:bvp-hypersingular-curl-u-nu-d3}$\Rightarrow$
\begin{equation*}
  \begin{split}
    \int_{\Gamma_{k}} \underline{\curl}_{\Gamma_{k}} u(x) \underline{\nu}(x) \, d s_x
    & = \int_{\Gamma_{k}}
    \left[
    \underline{n}(x) \times \triangledown \widetilde{u}(x)
    \right]
    \nu(x)
    \, d s_{x}\\
    &= \int_{\Gamma_{k}}
    \left[
    \triangledown \widetilde{u}(x) \times \underline{\nu}(x)
    \right]
    n(x)
    \, d s_{x} \\
    &= \int_{\Gamma_{k}}
    \left\{
    \left[
    \triangledown \times \left( \widetilde{u}(x) \times \underline{\nu}(x) \right)
    \right]
    - \left[
    \widetilde{u}(x)
    \left( \triangledown \times \underline{\nu}(x) \right)
    \right]
    \right\}
    n(x) \, d s_{x} \\
    & = \underbrace{
    \int_{\partial \Gamma_{k}} u(x) \underline{\nu}(x) \underline{t}(x) \, d \sigma
    }_{\eqqcolon 0}
    - \int_{\Gamma_{k}} u(x) \curl_{\Gamma_{k}} \underline{\nu}(x)
    \, d s_{x} \\
    & = - \int_{\Gamma_{k}} u(x) \curl_{\Gamma_{k}} \underline{\nu}(x)
    \, d s_{x}
  \end{split}
\end{equation*}
\end{proof}

\begin{theorem}[超奇异积分算子的双线性表现式]
  \label{theorem:bvp-hypersingular-bilinear-d3}
  设一个分段平滑的封闭表面$\Gamma$。在$\Gamma$中定义全局连续方程$u$和$\nu$,且在$\Gamma_{k}$中可微。那么超奇异积分算子$D$的双线性表现式为
  \begin{equation}
    \label{eq:bvp-hypersingular-bilinear-d3}
    \langle D u, \nu \rangle_{\Gamma}
    = \frac{1}{4 \pi}
    \int_{\Gamma}
    \int_{\Gamma}
    \frac{
    \underline{\curl}_{\Gamma} u(y)
    \underline{\curl}_{\Gamma} \nu(x)
    }{
    \left| x - y \right|
    }
    \, d s_{x} d s_{y}.
  \end{equation}
\end{theorem}
\begin{proof}
  与二维系统的证明过程(Theorem \ref{theorem:bvp-hypersingular-bilinear-d2})类似。根据定义式\eqref{eq:bvp-hypersingular-bie-def},超奇异边界积分算子$D$可以定义为双层位势算子$W$的法导数$\gamma_{1}^{\text{int}} W$的负数。对于$\widetilde{x} \in \Omega$,定义
  \begin{equation*}
    w(\widetilde{x}) \coloneqq
    - \frac{1}{4 \pi} \int_{\Gamma} u(y)
    \underbrace{
    \frac{\partial}{\partial n_{y}}
    \frac{1}{\left| \widetilde{x} - y \right|}
    }_{\text{核方程}}
    \, d s_y.
  \end{equation*}
\begin{enumerate}
\item 求核方程的偏导数。
根据
\begin{equation*}
  \begin{split}
    \frac{\partial}{\partial y_{i}}
    \frac{1}{\left| \widetilde{x} - y \right|}
    &=
    \frac{
    \widetilde{x} - y_{i}
    }{
    \left| \widetilde{x} - y \right|^{3}
    }
    =
    -
    \frac{
    y_{i} - \widetilde{x}
    }{
    \left| \widetilde{x} - y \right|^{3}
    }
    =
    -
    \frac{\partial}{\partial x_{i}}
    \frac{1}{\left| \widetilde{x} - y \right|},
  \end{split}
\end{equation*}
则
\begin{equation*}
  \begin{split}
    \frac{\partial}{\partial \widetilde{x}_{i}}
    \left(
    \frac{\partial}{\partial y_{i}}
    \frac{1}{\left| \widetilde{x} - y \right|}
    \right)
    = - \underline{n}(y)
    \triangledown_{y}
    \left(
    \frac{\partial}{\partial y_{i}}
    \frac{1}{\left| \widetilde{x} - y \right|}
    \right).
  \end{split}
\end{equation*}

\item 设$\underline{e}_{i}$为$\mathbb{R}^{3}$中的第$i$-th个单位向量(unit vector)。由于$\widetilde{x} \neq y$可得向量乘
\begin{equation*}
  \begin{split}
    \underline{\curl}_{y}
    \left(
    \underline{e}_{i}
    \times
    \triangledown_{y} \frac{1}{\left| \widetilde{x} - y \right|}
    \right)
    &=
    \triangledown_{y} \times
    \left(
    \underline{e}_{i}
    \times
    \triangledown_{y} \frac{1}{\left| \widetilde{x} - y \right|}
    \right) \\
    &=
    \left(
    \triangledown_{y}
    \cdot
    \triangledown_{y}
    \frac{1}{\left| \widetilde{x} - y \right|}
    \right)
    \underline{e}_{i}
    -
    \left(
    \triangledown_{y} \cdot \underline{e}_{i}
    \right)
    \triangledown_{y}
    \frac{1}{\left| \widetilde{x} - y \right|}\\
    & =
    \underbrace{\Delta_{y}
    \frac{1}{\left| \widetilde{x} - y \right|}
    \underline{e}_{i}
    }_{ = 0}
    -
    \frac{\partial}{\partial y_{i}}
    \left(
    \triangledown_{y}
    \frac{1}{\left| \widetilde{x} - y \right|}
    \right) \\
    & =
    -
    \frac{\partial}{\partial y_{i}}
    \left(
    \triangledown_{y}
    \frac{1}{\left| \widetilde{x} - y \right|}
    \right).
  \end{split}
\end{equation*}

\item 对双层位势$w(\widetilde{x})$作关于$\widetilde{x}_{i}$的偏微分,调整微分、积分顺序得
\begin{equation*}
  \begin{split}
    \frac{\partial}{\partial \widetilde{x}_{i}} w(\widetilde{x})
    & =
    - \frac{1}{4 \pi}
    \int_{\Gamma}
    u(y)
    \frac{\partial}{\partial \widetilde{x}_{i}}
    \left(
    \frac{\partial}{\partial n_{y}}
    \frac{1}{\left| \widetilde{x} - y \right|}
    \right)
    \, d s_y \\
    & = \frac{1}{4 \pi}
    \int_{\Gamma}
    u(y)
    n_{y} \cdot
    \triangledown_{y}
    \left(
    \frac{\partial}{\partial y_{i}}
    \frac{1}{\left| \widetilde{x} - y \right|}
    \right)
    \, d s_y \\
    & =
    - \frac{1}{4 \pi}
    \int_{\Gamma}
    u(y)
    n(y) \cdot
    \underline{\curl}_{y}
    \left(
    \underline{e}_{i}
    \times
    \triangledown_{y}
    \frac{1}{\left| \widetilde{x} - y \right|}
    \right)
    \, d s_y \\
    & =
    - \frac{1}{4 \pi}
    \int_{\Gamma}
    u(y)
    \cdot
    \curl_{\Gamma, y}
    \left(
    \underline{e}_{i}
    \times
    \triangledown_{y}
    \frac{1}{\left| \widetilde{x} - y \right|}
    \right)
    \, d s_y.
  \end{split}
\end{equation*}

根据Lemma \ref{lemma:bvp-hypersingular-curl-u-nu-d3},上式进一步改写为
\begin{equation*}
  \begin{split}
    \frac{\partial}{\partial \widetilde{x}_{i}} w(\widetilde{x})
    &= \frac{1}{4 \pi}
    \int_{\Gamma}
    \underline{\curl}_{\Gamma, y}
    u(y)
    \cdot
    \left(
    \underline{e}_{i} \times \triangledown_{x} \frac{1}{\left| \widetilde{x} - y \right|}
    \right)
    \, d s_y\\
    & =
    -\frac{1}{4 \pi}
    \int_{\Gamma}
    \underline{e}_{i}
    \left(
    \underline{\curl}_{\Gamma, y} u(y) \times
    \triangledown_{y} \frac{1}{\left| \widetilde{x} - y \right|}
    \right)
    \, d s_y.
  \end{split}
\end{equation*}

由此可得双层位势的梯度(gradient)
\begin{equation*}
  \begin{split}
    \triangledown_{\widetilde{x}} w(\widetilde{x})
    & =
    -\frac{1}{4 \pi}
    \int_{\Gamma}
    \left(
    \underline{\curl}_{\Gamma, y} u(y) \times
    \triangledown_{y} \frac{1}{\left| \widetilde{x} - y \right|}
    \right)
    \, d s_y \\
    & =
    \frac{1}{4 \pi}
    \int_{\Gamma}
    \left(
    \underline{\curl}_{\Gamma, y} u(y) \times
    \triangledown_{\widetilde{x}} \frac{1}{\left| \widetilde{x} - y \right|}
    \right)
    \, d s_y.
  \end{split}
\end{equation*}

两侧同时乘以法向量(normal vector) $\underline{n}(x)$得
\begin{equation*}
  \begin{split}
    \underline{n}(x) \cdot \triangledown_{\widetilde{x}} w(\widetilde{x})
    & =
    \frac{1}{4 \pi}
    \int_{\Gamma}
    \left(
    \underline{\curl}_{\Gamma, y} u(y) \times
    \triangledown_{\widetilde{x}} \frac{1}{\left| \widetilde{x} - y \right|}
    \right) \cdot \underline{n}(x)
    \, d s_y \\
    & =
    - \frac{1}{4 \pi}
    \int_{\Gamma}
    \underline{\curl}_{\Gamma, y} u(y) \cdot
    \left(
    \underline{n}(x)
    \times
    \triangledown_{\widetilde{x}} \frac{1}{\left| \widetilde{x} - y \right|}
    \right)
    \, d s_y \\
    & =
    - \frac{1}{4 \pi}
    \lim_{\varepsilon \rightarrow 0}
    \int_{y \in \Gamma: \left| y - x \right| \ge \varepsilon}
    \underline{\curl}_{\Gamma, y} u(y) \cdot
    \left(
    \underline{n}(x)
    \times
    \triangledown_{\widetilde{x}} \frac{1}{\left| \widetilde{x} - y \right|}
    \right)
    \, d s_y.
  \end{split}
\end{equation*}

取极限$\Omega \ni \widetilde{x} \rightarrow x \in \Gamma$,上式变为
\begin{equation*}
  \begin{split}
  \left( D u \right)
  & =
  - \frac{1}{4 \pi}
  \lim_{\varepsilon \rightarrow 0}
  \int_{y \in \Gamma: \left| y - x \right| \ge \varepsilon}
  \underline{\curl}_{\Gamma, y} u(y) \cdot
  \left(
  \underline{n}(x)
  \times
  \triangledown_{x} \frac{1}{\left| x - y \right|}
  \right)
  \, d s_y \\
  & =
  - \frac{1}{4 \pi}
  \lim_{\varepsilon \rightarrow 0}
  \int_{y \in \Gamma: \left| y - x \right| \ge \varepsilon}
  \underline{\curl}_{\Gamma, y} u(y) \cdot
  \underline{\curl}_{\Gamma, x} \frac{1}{\left| x - y \right|}
  \, d s_y
  \end{split}
\end{equation*}

\item 进而,双线性形式
\begin{equation*}
  \begin{split}
    \langle D u, \nu \rangle_{\Gamma}
    & =
    - \frac{1}{4 \pi}
    \int_{\Gamma}
    \nu(x)
    \lim_{\varepsilon \rightarrow 0}
    \int_{y \in \Gamma: \left| y - x \right| \ge \varepsilon}
    \underline{\curl}_{\Gamma, y} u(y) \cdot
    \underline{\curl}_{\Gamma, x} \frac{1}{\left| x - y \right|}
    \, d s_{y} d s_{x} \\
    & = - \frac{1}{4 \pi}
    \int_{\Gamma}
    \lim_{\varepsilon \rightarrow 0}
    \int_{x \in \Gamma: \left| x - y \right| \ge \varepsilon}
    \left(
    \nu(x)
    \underline{\curl}_{\Gamma, y} u(y)
    \right)
    \underline{\curl}_{\Gamma, x} \frac{1}{\left| x - y \right|}
    \, d s_{x} d s_{y} \\
    & =
    \frac{1}{4 \pi}
   \int_{\Gamma}
   \lim_{\varepsilon \rightarrow 0}
   \int_{x \in \Gamma: \left| x - y \right| \ge \varepsilon}
   \underbrace{
   \curl_{\Gamma, x}
   \left(
   \nu(x)
   \underline{\curl}_{\Gamma, y} u(y)
   \right)
   }_{}
   \frac{1}{\left| x - y \right|}
   \, d s_{x} d s_{y},
  \end{split}
\end{equation*}

其中
\begin{equation*}
  \begin{split}
    \curl_{\Gamma, x}
    \left(
    \nu(x)
    \underline{\curl}_{\Gamma, y} u(y)
    \right)
    & =
    \underline{n}(x) \cdot
    \left[
    \triangledown_{x} \times
    \left(
    \nu(x)
    \underline{\curl}_{\Gamma, y} u(y)
    \right)
    \right] \\
    & =
    \underline{n}(x) \cdot
    \left[
    \triangledown_{x} \nu(x) \times
    \underline{\curl}_{\Gamma, y} u(y)
    \right] \\
    & =
    \left[
    n(x) \times \triangledown_{x} \nu(x)
    \right]
    \cdot
    \underline{\curl}_{\Gamma, y} u(y) \\
    & =
    \underline{\curl}_{\Gamma, x} \nu(x)
    \underline{\curl}_{\Gamma, y} u(y).
  \end{split}
\end{equation*}

因此
\begin{equation*}
\begin{split}
  \langle D u, \nu \rangle_{\Gamma}
  = \frac{1}{4\pi}
  \int_{\Gamma}
  \lim_{\varepsilon \rightarrow 0}
  \int_{x \in \Gamma: \left| x - y \right| \ge \varepsilon}
  \underline{\curl}_{\Gamma, x} \nu(x)
  \underline{\curl}_{\Gamma,y} \nu(y)
  \frac{1}{\left| x - y \right|}
  \, d s_{x} d s_{y}
\end{split}
\end{equation*}

对上式取极限$\varepsilon \rightarrow 0$,证毕。
\end{enumerate}
\end{proof}

\subsection{边界积分算子之间的关系}
\label{bvp-operators-relations}

回顾一下泊松方程的表现式\eqref{eq:bvp-fund-var-ux-uy-repform}
\begin{equation*}
  u(\widetilde{x}) = \int_{\Omega} U^{*}(\widetilde{x},y) f(y) \, dy
  + \int_{\Gamma} U^{*}(\widetilde{x},y) \gamma_{1}^{\text{int}} u(y) \, d s_y
  - \int_{\Gamma} U^{*}(\widetilde{x},y) \gamma_{0}^{\text{int}} u(y) \, d s_y, \quad \widetilde{x} \in \Omega.
\end{equation*}

取极限$\Omega \ni \widetilde{x} \rightarrow x \in \Gamma$,由上式可以得出一系列边界和体积位势的性质。其中需要关注的是两个边界积分方程。第一个是将内界迹算子$\gamma_{0}^{\text{int}}$应用到解$u(x), x \in \Gamma$中
\begin{equation}
  \label{eq:bvp-bie-relation-inttrace}
  \gamma_{0}^{\text{int}} u (x)
  =
  \left(
  V \gamma_{1}^{\text{int}} u
  \right)
  (x)
  + \left(1 - \sigma(x) \right) \gamma_{0}^{\text{int}} u(x)
  - \left( K\gamma_{0}^{\text{int}} u \right)(x)
  + N_{0} f(x).
\end{equation}

第二个是将共法导数算子$\gamma_{1}^{\text{int}}$应用到解$u(x), x \in \Gamma$中
\begin{equation}
  \label{eq:bvp-bie-relation-conormal}
  \gamma_{1}^{\text{int}} u(x)
  = \sigma(x) \gamma_{1}^{\text{int}} u(x)
  + \left( K' \gamma_{1}^{\text{int}} u \right) (x)
  + \left( D \gamma_{0}^{\text{int}} u \right)(x)
  + N_{1} f(x).
\end{equation}

联立\eqref{eq:bvp-bie-relation-inttrace}和\eqref{eq:bvp-bie-relation-conormal},构成一个边界积分方程系统
\begin{equation}
  \label{eq:bvp-bie-system}
  \begin{pmatrix}
    \gamma_{0}^{\text{int}} u(x) \\
    \gamma_{1}^{\text{int}} u(x)
  \end{pmatrix}
  =
  \underbrace{
  \begin{pmatrix}
    \left( 1-\sigma \right) & V \\
    D & \sigma I + K'
  \end{pmatrix}
  }_{\eqqcolon \mathcal{C}}
  \,
  \begin{pmatrix}
    \gamma_{0}^{\text{int}} u(x)\\
    \gamma_{1}^{\text{int}} u(x)
  \end{pmatrix}
  +
  \begin{pmatrix}
    N_{0} f(x) \\
    N_{1} f(x)
  \end{pmatrix}.
\end{equation}

我们将系数矩阵定义为卡尔德隆投影(Calderón projection)\index{Calderón projection \dotfill 卡尔德隆投影}
\begin{equation}
  \label{eq:bvp-bie-system-calderon-projection}
  \mathcal{C} \coloneqq
  \begin{pmatrix}
    \left( 1-\sigma \right) & V \\
    D & \sigma I + K'
  \end{pmatrix}.
\end{equation}

\begin{lemma}[卡尔德隆投影]
  式\eqref{eq:bvp-bie-system-calderon-projection}算子$\mathcal{C}$是一个投影,满足$\mathcal{C} = \mathcal{C}^2$。
\end{lemma}
\begin{proof}
  设任一给定的$\left(\psi, \varphi \right) \in H^{-\frac{1}{2}}(\Gamma) \times H^{\frac{1}{2}}(\Gamma)$。那么齐次偏微分方程的解$u(\widetilde{x}), \widetilde{x} \in \Omega$可以表示为
  \begin{equation*}
    u \left( \widetilde{x} \right)
    \coloneqq \left( V \psi \right) (\widetilde{x})
    -\left( W \varphi \right) (\widetilde{x}), \quad \widetilde{x} \in \Omega.
  \end{equation*}

由$u$的内界迹和共法导数,对应$x \in \Gamma$,我们有边界位势的性质
\begin{equation*}
  \begin{split}
    \gamma_{0}^{\text{int}} &= \left( V \psi \right) (x)
    + \left[ 1 - \sigma(x) \right] \varphi(x) - \left(K \varphi \right)(x), \\
    \gamma_{1}^{\text{int}} &= \sigma \psi(x) + \left(K' \psi \right) (x)  \left( D \varphi \right)(x).
  \end{split}
\end{equation*}

不难看出,$u$是齐次偏微分方程的解。通过$\left[\gamma_{0}^{\text{int}}, \gamma_{1}^{\text{int}} \right], x \in \Gamma$,可以生成一组对应的柯西数(Cauchy data)。这些柯西数构成边界积分方程系统\eqref{eq:bvp-bie-relation-inttrace}-\eqref{eq:bvp-bie-relation-conormal}的解,即
\begin{equation}
  \label{eq:bvp-bie-system-solution}
\begin{split}
  \left( V \gamma_{1}^{\text{int}} u \right)(x) & =
  \left( \sigma I + K \right) \gamma_{0}^{\text{int}} u(x), \\
  \left( D \gamma_{0}^{\text{int}} u \right)(x) & =
  \left( \left(1 - \sigma \right) I - K' \right) \gamma_{1}^{\text{int}} u(x).
\end{split}
\end{equation}

上式等价于
\begin{equation}
  \label{eq:bvp-bie-system-solution-matrix}
  \begin{pmatrix}
    \gamma_{0}^{\text{int}} u(x) \\
    \gamma_{1}^{\text{int}} u(x)
  \end{pmatrix}
  =
  \begin{pmatrix}
    \left( 1-\sigma \right) & V \\
    D & \sigma I + K'
  \end{pmatrix}
  \,
  \begin{pmatrix}
    \gamma_{0}^{\text{int}} u(x) \\
    \gamma_{1}^{\text{int}} u(x)
  \end{pmatrix}.
\end{equation}

联系
\begin{equation*}
  \begin{pmatrix}
    \gamma_{0}^{\text{int}} u(x) \\
    \gamma_{1}^{\text{int}} u(x)
  \end{pmatrix}
  =
  \begin{pmatrix}
    \left( 1-\sigma \right) & V \\
    D & \sigma I + K'
  \end{pmatrix}
  \,
  \begin{pmatrix}
    \varphi(x) \\
    \psi(x)
  \end{pmatrix}
\end{equation*}
可见,$\mathcal{C}$是一个投影。根据定义,投影矩阵,又称幂等矩阵(idempotent matrix),满足$\mathcal{C} = \mathcal{C}^{2}$ \citep[p.464]{Seber:2003wu}。
\end{proof}

在得到卡尔德隆投影之后,我们有以下有界积分算子之间的关系。
\begin{corollary}[有界积分算子之间的关系]
  所有有界积分算子均满足以下关系
  \begin{align}
    \label{eq:bvp-bie-operator-relation-1}
    V D &= \left( \sigma I + K \right) \, \left[ \left( 1-\sigma \right) I - K \right],\\
    \label{eq:bvp-bie-operator-relation-2}
    D V &=  \left( \sigma I + K' \right) \, \left[ \left( 1-\sigma \right) I - K' \right],\\
    \label{eq:bvp-bie-operator-relation-3}
    V K' &= K V, \\
    \label{eq:bvp-bie-operator-relation-4}
    K' D &= D K.
  \end{align}
\end{corollary}
由\eqref{eq:bvp-bie-operator-relation-3}可见,双层位势算子$K$是对称的——并不是自伴随的对称,而是随着单层位势算子$V$的对称。

此外,边界积分方程系统中,假定单层位势$V$是可逆的(更多讨论见\todo{section 6.1.1}),则我们也可得到牛顿位势$N_{1}f$的表现式如下

\begin{lemma}[牛顿位势算子与其他算子的关系]
  \label{lemma:bvp-bie-relation-newton-potential}
牛顿位势算子$\left( N_{1} f \right)(x), x \in \Gamma$的表现式为
\begin{equation}
  \label{eq:bvp-bie-relation-newton-potential}
  \left(N_{1} f \right)(x) =
  \left( (\sigma - 1) I + K' \right)
  V^{-1}
  \left( N_{0} f \right)(x)
\end{equation}
\end{lemma}
\begin{proof}
  由边界积分方程系统\eqref{eq:bvp-bie-system}第一行可知,假设$V$是可逆的
  \begin{equation*}
    \gamma_{1}^{\text{int}} u(x) =
    V^{-1} \left( \sigma I + K' \right) \gamma_{0}^{\text{int}} u(x)
    - V^{-1} \left( N_{0} f \right)(x), \quad x \in \Gamma.
  \end{equation*}

  代入第二行,可得
  \begin{equation*}
    \begin{split}
      \gamma_{1}^{\text{int}} u(x) &=
      \left(D \gamma_{0}^{\text{int}} u \right)(x)
      + \left( \sigma I + K' \right) \gamma_{1}^{\text{int}} u(x)
      + \left( N_{1} f \right)(x) \\
      & =
      \left(D \gamma_{0}^{\text{int}} u \right)(x)
      + \left( \sigma I + K' \right)
      \left[
      V^{-1} \left( \sigma I + K \right) \gamma_{0}^{\text{int}} u(x)
      - V^{-1} \left( N_{0} f \right)(x)
      \right]
      + \left( N_{1} f \right)(x) \\
      & =
      \left[ D +
      \left( \sigma I + K' \right)
      V^{-1} \left( \sigma I + K \right)
      \right]
      \gamma_{0}^{\text{int}} (x)
      - \left( \sigma I + K' \right)
      V^{-1} \left( N_{0} f \right)(x)
      + \left( N_{1} f \right)(x), \quad x \in \Gamma.
    \end{split}
  \end{equation*}

  整理可得
  \begin{equation*}
    - V^{-1} \left( N_{0} f \right)(x)
    = - \left( \sigma I + K' \right) V^{-1} \left( N_{0} f \right)(x) + \left( N_{1} f \right)(x),
    \end{equation*}
证毕。
\end{proof}

\subsection{单层位势算子的椭圆性和可逆性}
\label{sec:bvp-single-layer-ellipticity}

单层位势算子$V:H^{-\frac{1}{2}}(\Gamma) \mapsto H^{\frac{1}{2}}(\Gamma)$的可逆性可由拉克斯一密格拉蒙定理Theorem \ref{theorem:lax-milgram-lemma}证得。在此基础上,进一步讨论$V$的$H^{\frac{1}{2}}(\Gamma)$-椭圆特性。

回顾,已知方程$u(x) = \left( \widetilde{V} w \right)(x), x \in \Gamma$是以下内界狄利克雷边界值问题的解
\begin{equation*}
  \begin{split}
    - \Delta u (x) &= 0, \quad x \in \Omega, \\
    u(x) &= \gamma_{0}^{\text{int}} \left( \widetilde{V} w \right)(x) = \left( V w \right)(x), \quad x \in \Gamma.
  \end{split}
\end{equation*}

假定$w \in H^{-\frac{1}{2}}(\Gamma)$,则$u = \widetilde{V} w \in H^{1}(\Omega)$。通过选取方程$\nu \in H^{1}(\Omega)$,由\eqref{eq:bvp-a-u-nu-inner-prod}可得内界狄利克雷边界值问题的格林第一恒等式
\begin{equation}
  \label{eq:bvp-bie-interior-dirichlet-green}
  a_{\Omega} (u, \nu) \coloneqq \int_{\Omega} \triangledown u(x) \triangledown \nu(x) \, d x
  = \langle \gamma_{1}^{\text{int}} u, \gamma_{0}^{\text{int}} \nu \rangle_{\Gamma}.
\end{equation}

代入\eqref{eq:var-dbvp-f0-solution} (Corollary \ref{corollary:var-dbvp-f0-solution}),上式改写为不等式形式
\begin{equation}
  \label{eq:bvp-bie-interior-dirichlet-green-ineq}
  a_{\Omega}(u,u) \ge c_{1}^{\text{int}} \,
  \left\| \gamma_{1}^{\text{int}} u \right\|_{H^{-\frac{1}{2}}(\Gamma)}.
\end{equation}

随后我们需要计算外界共法导数$\gamma_{1}^{\text{int}} u \in H^{-\frac{1}{2}}(\Gamma)$,这可以通过分析$\left| x \right| \rightarrow \infty$时单层位势$\left( \widetilde{V} w \right)(x)$的远端特征而得。为此,我们需要引入子空间$H_{*}^{-\frac{1}{2}}(\Gamma)$,子空间中的方程正交于常数
\begin{equation}
  \label{eq:bvp-bie-h-gamma-star-def}
  H_{*}^{-\frac{1}{2}}(\Gamma) \coloneqq
  \left\{
  w \in H^{-\frac{1}{2}}(\Gamma) : \langle w, 1 \rangle_{\Gamma} = 0
  \right\}.
\end{equation}

\begin{lemma}
  \label{lemma:bvp-bie-single-layer-solution-boundary}
  假定
  \begin{equation*}
    \left| x - y_{0} \right| >
    \max \left\{ 1, 2 \diam(\Omega) \right\}, \quad y_{0} \in \Omega, x \in \mathbb{R}^{d}.
  \end{equation*}

  设
  \begin{equation*}
    w \in \begin{cases}
    H^{-\frac{1}{2}}(\Gamma) & d = 3, \\
    H_{*}^{-\frac{1}{2}}(\Gamma) & d = 2.
    \end{cases}
  \end{equation*}

  那么$u = \widetilde{V} w $的边界可以表示为

  \begin{align}
    \label{eq:bvp-bie-single-layer-solution-boundary-u}
    \left| u(x) \right| &= \left| \left( \widetilde{V} w \right) (x) \right| \le c_{1}(w) \, \frac{1}{\left| x - y_{0} \right|},\\
     \label{eq:bvp-bie-single-layer-solution-boundary-pu}
     \left| \triangledown u(x) \right| & = \left| \triangledown \left( \widetilde{V} w \right) (x) \right| \le c_{2}(w) \, \frac{1}{\left| x - y_{0} \right|^2}.
  \end{align}
\end{lemma}

\begin{proof}
  由三角不等式我们有
  \begin{equation*}
    \begin{split}
      \left| x - y_{0} \right| & \le
      \left| x - y \right| + \left| y - y_{0} \right| \\
      & \le
      \left| x - y \right| + \diam(\Omega) \\
      & \le
      \left| x - y \right| + \frac{1}{2} \left| x - y_{0} \right|.
    \end{split}
  \end{equation*}

\begin{equation*}
  \hookrightarrow \left| x - y \right| \ge \frac{1}{2} \left| x - y_{0} \right|.
\end{equation*}

下面分$d=3,2$分别证明。
\begin{enumerate}
\item 三维空间$d=3 \Rightarrow $
\begin{equation*}
  \begin{split}
    \left\| U^{*}(x, \cdot) \right\|_{H^{1}(\Omega)}^{2} & =
    \frac{1}{16 \pi^{2}}
    \int_{\Omega} \frac{1}{\left| x - y \right|^{2}} \, d_y
    + \frac{1}{16 \pi^{2}}
    \int_{\Omega} \frac{1}{\left| x - y \right|^{4}} \, d_y \\
    & \le
    \frac{1}{4 \pi^{2}}
    \int_{\Omega}
    \frac{1}{\left| x - y_{0} \right|^{2}} \, d_y
    + \frac{1}{\pi^{2}}
    \int_{\Omega}
    \frac{1}{\left| x - y_{0} \right|^{4}} \, d_y \\
    & \le \frac{5}{4} \frac{\left| \Omega \right|}{\pi^{2}}
    \frac{1}{\left| x - y_{0} \right|^{2}} \, d_y,
  \end{split}
\end{equation*}
证得\eqref{eq:bvp-bie-single-layer-solution-boundary-u}。

\begin{equation*}
  \begin{split}
  \left| u(x) \right| &=
  \left|
  \langle
  \gamma_{0}^{\text{int}} U^{*}(x, \cdot) , w
  \rangle_{\Gamma} \right| \\
  & \le
  \left\| \gamma_{0}^{\text{int}} U^{*}(x, \cdot) \right\|_{H^{\frac{1}{2}}(\Gamma)}
  \,
  \left\| w \right\|_{H^{-\frac{1}{2}}(\Gamma)} \\
  & \le c_{\text{T}} \,
  \left\| \gamma_{0}^{\text{int}} U^{*}(x, \cdot) \right\|_{H^{1}(\Omega)}
  \,
  \left\| w \right\|_{H^{-\frac{1}{2}}(\Gamma)}.
  \end{split}
\end{equation*}

\begin{equation*}
  \begin{split}
    \hookrightarrow \frac{\partial}{\partial x_{i}} u(x) &=
    \frac{1}{4 \pi} \int_{\Gamma}
    \frac{y_{i} - x_{i}}{\left| x - y \right|^{3}}
    w(y) \, d s_y,
  \end{split}
\end{equation*}
证得\eqref{eq:bvp-bie-single-layer-solution-boundary-pu}。

\item 二维空间$d=2$中,沿着某个适宜的定值$\bar{y} \in \Omega$作泰勒展开
\begin{equation*}
  \log \left| y - x \right|  \approx
  \log \left| y_{0} - x \right| +
  \frac{
  \left( y - y_{0}, \bar{y} - x \right)
  }{
  \left| \bar{y} - x \right|^2
  }.
\end{equation*}

进而根据$w \in H_{*}^{-\frac{1}{2}}(\Gamma)$可得
\begin{equation*}
  u(x) = -\frac{1}{2 \pi} \int_{\Gamma}
  \frac{
  \left( y - y_{0}, \bar{y} - x \right)
  }{
  \left| \bar{y} - x \right|^2
  }
  w(y) \, d s_{y},
\end{equation*}
证得\eqref{eq:bvp-bie-single-layer-solution-boundary-u}。在此基础上证得\eqref{eq:bvp-bie-single-layer-solution-boundary-pu}。
\end{enumerate}
\end{proof}

对于$y_{0} \in \Omega, R \ge 2 \diam(\Omega)$,定义球体
\begin{equation*}
  B_{R}(y_{0}) \coloneqq \left\{ x \in \mathbb{R}^{d} : \left| x - y_{0} \right| < R \right\}.
\end{equation*}

那么此时$u(x) = \left( V w \right)(x), x \in \Omega^{c}$就成为以下狄利克雷边界值问题的唯一解
\begin{align}
  \label{eq:bvp-dirichlet-exterior-problem-1}
  - \Delta u(x) & = 0, \quad x \in B_{R}(y_{0}) \backslash \overline{\Omega}, \\
  \label{eq:bvp-dirichlet-exterior-problem-2}
  u(x) & = \gamma_{0}^{\text{int}} \left( \widetilde{V} w \right) (x) = \left( V w \right)(x), \quad x \in \Gamma, \\
  \label{eq:bvp-dirichlet-exterior-problem-3}
  u(x) & = \left( \widetilde{V} w \right)(x), \quad x \in \partial B_{R}(y_{0}).
\end{align}

其中,对于有界域$ B_{R}(y_{0}) \backslash \overline{\Omega}$ \eqref{eq:bvp-dirichlet-exterior-problem-1} 中的情况,利用格林第一恒等式\eqref{eq:bvp-a-u-nu-inner-prod}可得
\begin{equation}
  \label{eq:bvp-dirichlet-exterior-green1}
  a_{B_{R}(y_{0}) \backslash \overline{\Omega}} (u, \nu)
  = \underbrace{
  - \langle \gamma_{1}^{\text{ext}} u, \gamma_{0}^{\text{ext}} \nu \rangle_{\Gamma}
  }_{\eqqcolon \mathcal{A}}
  + \underbrace{
  \langle \gamma_{1}^{\text{int}} u, \gamma_{0}^{\text{int}} \nu \rangle_{\partial B_{R}(y_{0})}
  }_{\eqqcolon \mathcal{B}}
  .
\end{equation}
\begin{enumerate}
  \item 先来看$\mathcal{B}$。设$u = \nu$。由Lemma \ref{lemma:bvp-bie-single-layer-solution-boundary}可得
  \begin{equation*}
    \begin{split}
      \left| \langle
      \gamma_{1}^{\text{int}} u, \gamma_{0}^{\text{int}} u
      \rangle_{\partial B_{R}(y_{0})} \right|
      & \le c_{1}(w) c_{2}(w)
      \int_{\left| x - y_{0} \right| = R}
      \frac{1}{\left| x - y_{0} \right|^{3}} \, d s_{x} \\
      & \le c \, R^{d-4}.
    \end{split}
  \end{equation*}

  \item 再来看极限$R \rightarrow \infty$的情况。在外界域(exterior domain) $\partial B_{R}(y_{0})$ \eqref{eq:bvp-dirichlet-exterior-problem-3}中,根据格林第一恒等式\eqref{eq:bvp-a-u-nu-inner-prod}可得
  \begin{equation}
    \label{eq:bvp-bie-omega-complement-bilinear}
      a_{\Omega^{c}(u, u)} \coloneqq \int_{\Omega^{c}}
      \triangledown u(x)
      \triangledown u(x)
      \, dx
      = - \langle \gamma_{1}^{\text{ext}} u, \gamma_{0}^{\text{ext}} u \rangle_{\Gamma}.
  \end{equation}
\end{enumerate}

在二维空间中,假设条件$w \in H_{*}^{-\frac{1}{2}}(\Gamma)$即可保证上述结论成立。以类似于Corollary \ref{corollary:var-dbvp-f0-solution},\eqref{eq:var-dbvp-f0-solution}的方法,对于\eqref{eq:bvp-bie-omega-complement-bilinear}的外界狄利克雷边界值问题,满足不等式
\begin{equation}
  \label{eq:bvp-bie-omega-complement-bilinear-ineq}
  a_{\Omega^{c}(u,u)} \ge c_{1}^{\text{ext}} \,
  \left\| \gamma_{1}^{\text{ext}} u \right\|_{H^{-\frac{1}{2}}(\Gamma)}^{2}.
\end{equation}

\begin{theorem}[单层位势算子的椭圆形(三维空间)]
  \label{theorem:bvp-bie-vw-w-gamma}
  设
  \begin{equation*}
    w \in
    \begin{cases}
      H^{-\frac{1}{2}}(\Gamma) & d =3, \\
      H_{*}^{-\frac{1}{2}}(\Gamma) & d = 2.
    \end{cases}
  \end{equation*}

  那么有以下不等式
  \begin{equation}
    \label{eq:bvp-bie-vw-w-gamma}
    \langle V w, w \rangle_{\Gamma} \ge c_{1}^{V} \,
    \left\| w \right\|_{H^{-\frac{1}{2}}(\Gamma)}^{2},
  \end{equation}
  其中常数$c_{1}^{V} \ge 0$。
\end{theorem}
\begin{proof}
根据$u = \widetilde{V} w$,由内界、外界狄利克雷边界值问题的格林第一恒等式\eqref{eq:bvp-bie-interior-dirichlet-green}、\eqref{eq:bvp-bie-omega-complement-bilinear}可知
\begin{equation*}
  \begin{split}
    a_{\Omega}(u,u) & = \langle \gamma_{1}^{\text{int}} u, \gamma_{0}^{\text{int}} u \rangle_{\Gamma}, \\
    a_{\Omega^{c}}(u,u) & = - \langle \gamma_{1}^{\text{ext}} u, \gamma_{0}^{\text{ext}} u \rangle_{\Gamma}.
  \end{split}
\end{equation*}

两式相加
\begin{equation*}
  \underbrace{
  a_{\Omega}(u,u) + a_{\Omega^{c}}(u,u)
  }_{\eqqcolon \mathcal{C}}
  = \langle
  \underbrace{
  \left[ \gamma_{1}^{\text{int}} u - \gamma_{1}^{\text{ext}} u \right]
  }_{\eqqcolon \mathcal{A}}
  ,
  \underbrace{
  \left[ \gamma_{0}^{\text{int}} u - \gamma_{0}^{\text{ext}} u \right]
  }_{\eqqcolon \mathcal{B}}
  \rangle_{\Gamma}.
\end{equation*}


\begin{enumerate}
  \item 由单层位势算子的跃动关系\eqref{eq:bvp-single-layer-jump-relaton}可得
\begin{equation*}
  \mathcal{B} \coloneqq \gamma_{0}^{\text{int}} u - \gamma_{0}^{\text{ext}} u \coloneqq \gamma_{0} u,
\end{equation*}

\item 由单层位势共法导数的跃动关系\eqref{eq:bvp-jump-relation}可得
\begin{equation*}
  \mathcal{A} \coloneqq \gamma_{1}^{\text{int}} u - \gamma_{1}^{\text{ext}} u \coloneqq w,
\end{equation*}

\item 将\eqref{eq:bvp-bie-interior-dirichlet-green-ineq},\eqref{eq:bvp-bie-omega-complement-bilinear-ineq}代入$\mathcal{C}$得
\begin{equation*} \begin{split}
  \mathcal{C} &\coloneqq a_{\Omega}(u,u) + a_{\Omega^{c}}(u,u) \\
  & \ge c_{1}^{\text{int}} \,
  \left\| \gamma_{1}^{\text{int}} u \right\|_{H^{-\frac{1}{2}}(\Gamma)}
  + c_{1}^{\text{ext}} \,
  \left\| \gamma_{1}^{\text{ext}} u \right\|_{H^{-\frac{1}{2}}(\Gamma)}^{2} \\
  & \ge \min \left\{ c_{1}^{\text{int}}, c_{2}^{\text{ext}} \right\}
  \left[
   \left\| \gamma_{1}^{\text{int}} u \right\|_{H^{-\frac{1}{2}}(\Gamma)}
   +
   \left\| \gamma_{1}^{\text{ext}} u \right\|_{H^{-\frac{1}{2}}(\Gamma)}^{2}
  \right]
\end{split}
\end{equation*}
\end{enumerate}

\begin{equation*}
  \begin{split}
    \langle \mathcal{A} , \mathcal{B} \rangle_{\Gamma} &=
    \langle w, \gamma_{0} u \rangle_{\Gamma} =
    \langle Vw, w \rangle_{\Gamma} \\
    & =
    \mathcal{C}
    \ge
    \min \left\{ c_{1}^{\text{int}}, c_{2}^{\text{ext}} \right\}
    \left[
     \left\| \gamma_{1}^{\text{int}} u \right\|_{H^{-\frac{1}{2}}(\Gamma)}
     +
     \left\| \gamma_{1}^{\text{ext}} u \right\|_{H^{-\frac{1}{2}}(\Gamma)}^{2}
    \right].
  \end{split}
\end{equation*}

此外根据\eqref{lemma:bvp-jump-relation}有
\begin{equation*}
  \begin{split}
    \left\| w \right\|_{H^{-\frac{1}{2}}(\Gamma)}^{2}
    & =
    \left\|
    \gamma_{1}^{\text{int}} u - \gamma_{1}^{\text{ext}} u
    \right\|_{H^{-\frac{1}{2}}(\Gamma)}^2 \\
    & \le
    \left[
    \left\|
    \gamma_{1}^{\text{int}} u
    \right\|_{H^{-\frac{1}{2}}(\Gamma)}
    +
    \left\|
    \gamma_{1}^{\text{ext}} u
    \right\|_{H^{-\frac{1}{2}}(\Gamma)}
    \right]^2 \\
    & \le 2 \left[
    \left\|
    \gamma_{1}^{\text{int}} u
    \right\|_{H^{-\frac{1}{2}}(\Gamma)}^2
    +
    \left\|
    \gamma_{1}^{\text{ext}} u
    \right\|_{H^{-\frac{1}{2}}(\Gamma)}^2
    \right],
  \end{split}
\end{equation*}

\begin{equation*}
  \begin{split}
    \langle Vw, w \rangle_{\Gamma}
    & \ge
    \min \left\{ c_{1}^{\text{int}}, c_{2}^{\text{ext}} \right\}
    \left[
     \left\| \gamma_{1}^{\text{int}} u \right\|_{H^{-\frac{1}{2}}(\Gamma)}
     +
     \left\| \gamma_{1}^{\text{ext}} u \right\|_{H^{-\frac{1}{2}}(\Gamma)}^{2}
    \right] \\
    & \ge \underbrace{
    \min \left\{ c_{1}^{\text{int}}, c_{2}^{\text{ext}} \right\}
    }_{\eqqcolon c_{1}^{V} \ge 0}
    \,
    \left\| w \right\|_{H^{-\frac{1}{2}}(\Gamma)}^{2}.
  \end{split}
\end{equation*}
\end{proof}

对于二维系统$d=2$的情况,我们也可从Theorem \ref{theorem:bvp-bie-vw-w-gamma}推得,单层位势$V$的$H_{*}^{-\frac{1}{2}}(\Gamma)$-椭圆特性。具体来说,有两种证明思路。

\subsubsection{鞍点变分问题求证椭圆性}
第一种证明思路是,汇总$d = 2, 3$的情况,构建鞍点问题
\begin{align}
  \label{eq:bvp-bie-sile-ellipticity-saddle-1}
  &\langle V t, \tau \rangle_{\Gamma} - \lambda \langle 1, \tau \rangle_{\Gamma} = 0, \quad \forall \tau \in H^{-\frac{1}{2}}(\Gamma), \\
  \label{eq:bvp-bie-sile-ellipticity-saddle-2}
  & \langle t, 1 \rangle_{\Gamma} = 1,
\end{align}
寻找解$(t,\lambda) \in H^{-\frac{1}{2}}(\Gamma) \times \mathbb{R}$。

将$t$定义如下
\begin{equation*}
  t \coloneqq \widetilde{t} + \frac{1}{\left| \Gamma \right|}, \quad \tau \in H_{*}^{-\frac{1}{2}}(\Gamma),
\end{equation*}
不难看出,如果对于某一$\widetilde{t} \in H_{*}^{-\frac{1}{2}}(\Gamma)$,\eqref{eq:bvp-bie-sile-ellipticity-saddle-1}的解$t$同时也满足关系式\eqref{eq:bvp-bie-sile-ellipticity-saddle-2}。那么求解上述按点问题的关键就变为,寻找合适的$\widetilde{t}$,使满足如下关系
\begin{equation*}
  \langle V \widetilde{t}, \tau \rangle_{\Gamma}
  = - \frac{1}{\left| \Gamma \right|}
  \langle V 1, \tau \rangle_{\Gamma}, \quad \forall \, \tau \in H_{*}^{-\frac{1}{2}}(\Gamma).
\end{equation*}

由Theorem \ref{theorem:bvp-bie-vw-w-gamma}可见,上式中单层位势是$H_{*}^{-\frac{1}{2}}(\Gamma)$-椭圆的。因此该变分问题可解,解唯一。设这个唯一解为$w_{\text{eq}}$,表示自然密度方程(natural density function)\index{natural density \dotfill 自然密度}\todo{更多自然密度的介绍见后}。
\begin{equation}
  w_{\text{eq}} \coloneqq \widetilde{t} + \frac{1}{\left| \Gamma \right|},
\end{equation}

设$\tau = w_{\text{eq}}$,则拉格朗日乘子$\lambda$满足
\begin{equation*}
  \lambda = \langle V w_{\text{eq}}, w_{\text{eq}} \rangle_{\Gamma}.
\end{equation*}

在三维空间$d=3$中,由Theorem \ref{theorem:bvp-bie-vw-w-gamma}可得拉,拉格朗日乘子严格为正$\lambda > 0$,定义为$\Gamma$的容度(capacity)\index{capacity \dotfill 容度},用$\capacity_{\Gamma}$表示。

在二维空间$d=2$中,$\capacity_{\Gamma}$可定义为指数形式
\begin{equation*}
  \capacity_{\Gamma} \coloneqq \exp (- 2 \pi \lambda),
\end{equation*}
基于参数的基本解为
\begin{equation*}
  U^{*}(x,y) \coloneqq \frac{1}{2 \pi} \log r - \frac{1}{2 \pi} \log
  \left| x - y \right|, \quad r \in \mathbb{R}^{+},
\end{equation*}
对应的边界积分算子
\begin{equation*}
  \left( V_{r} w \right)(x) \coloneqq
  \int_{\Gamma} U_{r}^{*} (x,y) w(y) \, d s_{y}, \quad x \in \Gamma.
\end{equation*}

由此可得
\begin{equation*}
  \left( V_{r} w \right)(x) = \frac{1}{2 \pi} \log r + \lambda
  = \frac{1}{2 \pi} \log \frac{r}{\capacity_{\Gamma}},
\end{equation*}
尤其是当$r=1$时,有
\begin{equation}
  \label{eq:bvp-bie-single-lambda-requal1}
  \lambda \coloneqq \frac{1}{2} \log \frac{1}{\capacity_{\Gamma}}.
\end{equation}

由\eqref{eq:bvp-bie-single-lambda-requal1}可见,在二维系统$d=2$中,$r=1$的前提下,只有当指数容度$\capacity_{\Gamma} <1$时,才能确保$\lambda > 0$,进而使得解存在。\cite{Hsiao:1977vf}论证了,使满足$\capacity_{\Gamma} <1$的一个充分条件是假定$\diam(\Omega) < 1$——有时这便需要我们对给定的域$\Omega \in \mathbb{R}^{2}$作规模调节,使得直径小于1的假定得到满足(见\pageref{sec:bvp-bie-single-ellipticity-natural-density-approach}页第\ref{sec:bvp-bie-single-ellipticity-natural-density-approach}节)。

\begin{theorem}[单层位势算子的椭圆形(二维空间)]
  \label{theorem:bvp-bie-single-ellipticity-d2}
  二维空间$d=2$中,设$\diam(\Omega) <1$,从而$\lambda >0$。单层位势$V$是$H^{-\frac{1}{2}}(\Gamma)$-椭圆的,即
  \begin{equation}
    \label{eq:bvp-bie-single-d2-ellipticity}
    \langle V w, w \rangle_{\Gamma}
    \ge
    \widetilde{c}_{1}^{\text{V}} \,
    \left\| w \right\|_{H^{-\frac{1}{2}}(\Gamma)}^{2}, \quad \forall \, w \in H^{-\frac{1}{2}}(\Gamma).
  \end{equation}
\end{theorem}

\begin{proof}
  \begin{enumerate}
    \item 看RHS。对于任一给定$w \in H^{-\frac{1}{2}}(\Gamma)$,都可以对$w$作唯一的分解
  \begin{equation*}
    w = \widetilde{w} + \alpha w_{\text{eq}}, \quad \widetilde{w} \in H_{*}^{-\frac{1}{2}}(\Gamma), \alpha = \langle w, 1 \rangle_{\Gamma}.
  \end{equation*}
\begin{equation*}
\begin{split}
  \hookrightarrow \left\| w \right\|_{H^{-\frac{1}{2}}(\Gamma)}^{2} & =
  \left\|
  \widetilde{w} + \alpha w_{\text{eq}} \right\|_{H^{-\frac{1}{2}}(\Gamma)}^{2} \\
  & \le
  \left[
  \left\| \widetilde{w} \right\|_{H^{-\frac{1}{2}}(\Gamma)}
  + \alpha
  \left\| w_{\text{eq}} \right\|_{H^{-\frac{1}{2}}(\Gamma)}
  \right]^{2} \\
  & \le
  2 \left[
  \left\| \widetilde{w} \right\|_{H^{-\frac{1}{2}}(\Gamma)}^{2}
  + \alpha^{2} \,
  \left\| w_{\text{eq}} \right\|_{H^{-\frac{1}{2}}(\Gamma)}^{2}
  \right] \\
  & \le 2 \max \left\{ 1, \left\| w_{\text{eq}} \right\|_{H^{-\frac{1}{2}}(\Gamma)}^{2} \right\} \,
  \left[
  \left\| \widetilde{w} \right\|_{H^{-\frac{1}{2}}(\Gamma)}^{2}
  + \alpha^{2}
  \right].
\end{split}
\end{equation*}

\item 看LHS。
\begin{equation*}
  \begin{split}
    \langle V w, w \rangle_{\Gamma}
    & = \langle
    V \left(\widetilde{w} + \alpha w_{\text{eq}} \right),
    \widetilde{w} + \alpha w_{\text{eq}}
    \rangle_{\Gamma} \\
    & =
    \underbrace{
    \langle V \widetilde{w}, \widetilde{w} \rangle
    }_{\eqqcolon \mathcal{A}}
    + 2 \alpha
    \underbrace{
    \langle
    \widetilde{V} w_{\text{eq}}, \widetilde{w}
    \rangle_{\Gamma}
    }_{\eqqcolon \mathcal{B}}
    + \alpha^{2}
    \underbrace{
    \langle
    V w_{\text{eq}}, w_{\text{eq}}
    \rangle_{\Gamma}
    }_{\eqqcolon \mathcal{C}},
  \end{split}
\end{equation*}
其中
\begin{enumerate}
  \item 根据Theorem \ref{theorem:bvp-bie-vw-w-gamma}
\begin{equation*}
  \mathcal{A} \coloneqq \langle
  \widetilde{V} w_{\text{eq}}, \widetilde{w}
  \rangle_{\Gamma} \ge c_{1}^{\text{V}} \, \left\| \widetilde{w} \right\|_{H^{-\frac{1}{2}}(\Gamma)}^{2},
\end{equation*}
\item
\begin{equation*}
  \mathcal{B} \coloneqq
  \langle
  \widetilde{V} w_{\text{eq}}, \widetilde{w}
  \rangle_{\Gamma} = 0,
\end{equation*}
\item 根据定义有
\begin{equation*}
  \mathcal{C} \coloneqq
  \langle
  V w_{\text{eq}}, w_{\text{eq}}
  \rangle_{\Gamma} = \lambda,
\end{equation*}
\end{enumerate}

因此上式变为
\begin{equation*}
  \begin{split}
    \langle V w, w \rangle_{\Gamma}
    &= \mathcal{A} + 2 \alpha \mathcal{B} + \alpha^{2} \mathcal{C} \\
    & \ge c_{1}^{\text{V}} \, \left\| \widetilde{w} \right\|_{H^{-\frac{1}{2}}(\Gamma)}^{2} + \alpha^{2} \lambda \\
    & \ge \min \{ c_{1}^{\text{V}}, \lambda \}
    \left[
    \left\|
    \widetilde{w} \right\|_{H^{-\frac{1}{2}}(\Gamma)}^{2} + \alpha^{2}
    \right]
  \end{split}
\end{equation*}

\item 联立LHS和RHS,可证。
\end{enumerate}
\end{proof}

第二种思路是从从自然密度方程$w_{\text{eq}} \in H^{-\frac{1}{2}}(\Gamma)$
\label{sec:bvp-bie-single-ellipticity-natural-density-approach}入手。
已知$w_{\text{eq}}$是以下带有约束条件的算子方程的解
\begin{equation*}
  \begin{split}
    \left(
    V w_{\text{eq}}
    \right) (x) &= \lambda, \quad x \in \Gamma, \\
    \langle w_{\text{eq}}, 1 \rangle_{\Gamma} &= 1.
  \end{split}
\end{equation*}

我们首先对$w_{\text{eq}}$作规模调节:
\begin{equation*}
  w_{\text{eq}} \coloneqq \lambda \, \widetilde{w}_{\text{eq}},
\end{equation*}

新的算子方程变为
\begin{equation}
  \label{eq:bvp-bie-single-scaling-equation}
  \begin{split}
    \left(
    V \widetilde{w}_{\text{eq}}
    \right) (x) &= 1, \quad x \in \Gamma, \\
    \langle \widetilde{w}_{\text{eq}}, 1 \rangle_{\Gamma} &= \frac{1}{\lambda}.
  \end{split}
\end{equation}

求解思路略。大致说来,第一步求解边界积分方程,对应\eqref{eq:bvp-bie-single-scaling-equation}第一行,以求得作为自然密度的解$w_{\text{eq}}$。第二步,将规模调整得到的$\widetilde{w}_{\text{eq}}$引入第二行,计算容度。

\subsubsection{单层位势算子的可逆性}
\label{sec:bvp-bie-single-invertibility}
已知单层位势边界积分算子$V:H^{-\frac{1}{2}}(\Gamma) \mapsto H^{\frac{1}{2}}(\Gamma)$第一有界\eqref{eq:bvp-single-layer-operator-v-norm},第二$H^{-\frac{1}{2}}(\Gamma)$-椭圆($d=2$的证明见Theorem \ref{theorem:bvp-bie-single-ellipticity-d2},$d=3$的证明见Theorem \ref{theorem:bvp-bie-vw-w-gamma})。那么可应用拉克斯一密格拉蒙定理(Theorem \ref{theorem:lax-milgram-lemma})证得单层位势$V$可逆,即$V^{-1}:H^{\frac{1}{2}}(\Gamma) \mapsto H^{-\frac{1}{2}}(\Gamma)$有界,并且满足
\begin{equation*}
  \left\| V^{-1} \nu \right\|_{H^{-\frac{1}{2}}(\Gamma)}
  \le \frac{1}{c_{1}^{\text{V}}} \, \left\| \nu \right\|_{H^{\frac{1}{2}}(\Gamma)}, \quad \forall \, \nu \in H^{\frac{1}{2}}(\Gamma).
\end{equation*}

在此基础上,双线性内积形式的单层位势$V$
\begin{equation*}
\begin{split}
  \langle V w, w_{\text{eq}} \rangle_{\Gamma}
  & = \langle w, V w_{\text{eq}} \rangle_{\Gamma} \\
  & = \lambda \langle w, 1 \rangle_{\Gamma} = 0, \quad \forall \, w \in H_{*}^{\frac{1}{2}}(\Gamma).
\end{split}
\end{equation*}

\begin{equation*}
  \begin{split}
    \therefore & V w \in H_{*}^{\frac{1}{2}}(\Gamma), \\
    & H_{*}^{\frac{1}{2}}(\Gamma) \coloneqq
    \left\{
    \nu \in H^{\frac{1}{2}}(\Gamma): \langle \nu, w_{\text{eq}} \rangle_{\Gamma} = 0
    \right\}.
  \end{split}
\end{equation*}

因此可见,单层位势$V:H_{*}^{-\frac{1}{2}}(\Gamma) \mapsto H_{*}^{\frac{1}{2}}(\Gamma)$是一个同构(isomorphism)\footnote{同构的概念,可参考\cite[pp.34, Definition 3.11]{Muscat:2014cc}。}。
