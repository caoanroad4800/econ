%!TEX root = ../DSGEnotes.tex
\chapter{准最大似然估计}
\label{sec:qmle-model}

对于一组随机变量,
不再假设一个关于这组随机变量均值的方程,而是采取更通用的形式,假定它们的联合分布(密度方程)是正确设定的,在此基础上可以进行最大似然估计(第\ref{sec:mle-model}章)。传统最大似然估计法假设,事先设定的密度方程就等于真实的密度方程,从而根据这样的假设,设定误差是不存在的。但是从实际观测的数据来看,``正确设定”的情况很少见——如何在可能存在模型误设定的情况下,更有效地进行统计推断?将MLE作一个一般化扩展,就有了准最大似然估计(quasi-maximum likelihood estimation, QMLE)\index{quasi-maximum likelihood estimation (QMLE) \dotfill 准最大似然估计},将假定条件放宽到,认为事先设定的密度方程在多数情况下只是对真实值的一种近似。这里以\cite{White:1994vu}为例介绍QMLE,更多相关讨论可参考如\cite{White:1982bm, Amemiya:1985uq, Gourieroux:1993bj, Chen:2013bk}。

\section{Kullback-Leibler信息准则}
\label{sec:qmle-klic-intro}

\subsection{信息}
\label{sec:qmle-klic-information}

假设在一项随机实验中,$A$事件发生的概率是$0 \le p \le 1$。我们关注于``事件A会发生"这一信号,其价值或重要性,对于不同$p$值来说是不同的:当$p$值较小时,信号的价值高;反之当$p$值很大时,信号的价值低。那么,信号``事件A会发生"所含有的信息量(information content) ,应当理解为一个随$p$而递减的方程$\tau \left(p \right)$,其形式可以设定如下
\begin{equation*}
    \tau \left( p \right) = \log \left( \frac{1}{p} \right),
\end{equation*}
需要注意的是,``事件$A$不会发生"的信息量$\tau \left( 1 - p \right)$,将不同于$\tau \left( p \right)$,除了当$p=0.5$时
\begin{equation}
    \tau(1-p)
    \begin{cases}
        \neq \tau(p) & p \neq 0.5 \\
        = \tau \left( p \right)  & p = 0.5 .
    \end{cases}
\end{equation}

从这个角度,我们可以将$\tau(p)$理解为,在知道了``事件$A$会以概率$P(A)=p$发生"时的``惊讶"程度。将全部$p \in [0,1]$下的惊讶作加权平均,可得期望信息(information)\index{infor
mation! \dotfill 信息} $I$。$I$可理解为事件$A$的熵(entropy)\index{entropy \dotfill 熵}。

现在来看信号``事件$A$发生的概率从$p$变为$q$''的价值:$\left| q-p \right|$大时价值高,小时价值低,其信息含量之差
\begin{equation*}
    \tau (p) - \tau(q) = \log \left( \frac{q}{p} \right)
    \begin{cases}
        > 0 & q > p \\
        <0, & q < p.
    \end{cases}
\end{equation*}
现在将$A$扩展到一组$n$个事件$A_{1},\ldots,A_{n}$的情况,其中$A_{i}, \, i=1,\ldots,n$的信息量为$\log \left( q_{i}/p_{i} \right)$,那么全部信息的加权期望值为
\begin{equation*}
    I = \sum_{i=1}^{n} q_{i} \log \left( \frac{q_{i}}{p_{i}} \right).
\end{equation*}

这样我们可以进一步讨论密度方程的信息含量了。

\subsection{Kullback-Leibler信息准则}
\label{sec:qmle-klic}
假定对于一组随机变量,存在一个未知的真实密度方程$g(\xi)$,和一个猜测的密度方程$f(\xi)$。由于$g$未知,我们将$g$相对于$f$的Kullback-Leibler信息准则(Kullback-Leibler information criterion, KLIC)\index{Kullback-Leibler information criterion (KLIC) \dotfill Kullback-Leibler信息准则} $KL \left( g:f \right)$定义为
\begin{equation}
    \label{eq:qmle-klic-def}
    KL \left( g:f \right)
    \coloneqq \int_{\mathbb{R}} \log
    \left(
    \frac{g \left( \xi \right)}{g \left( \xi \right)}
    \right)
    g \left( \xi \right)
    \, \mathrm{d} \xi,
\end{equation}
用于描述当知道当$g \left( \xi \right)$是随机变量$z$的真实密度之后的期望``惊讶"程度。

\begin{theorem}[KLIC非负]
    \label{theorem:qmle-klic-nonnegative}
    KLIC \eqref{eq:qmle-klic-def}满足
    \begin{equation}
        \label{eq:qmle-klic-nonnegative}
        KL \left( g:f \right) \ge 0
    \end{equation}
    并且当且仅当$g \overset{\text{a.e.}}{=} f$时,等号成立
    \footnote{
    \text{a.e.}是几乎处处(almost everywhere)的缩写,这里指的$g=f$除了一个勒贝格测度为$0$的集合之外,都成立。
    }。
\end{theorem}
\begin{proof}
    已知
    \begin{equation*}
    \begin{split}
    & \log \left( 1 + x \right) < x \quad \forall \, x > -1, \\
    & \hookrightarrow \log \left( \frac{g}{f} \right) = - \log \left( 1 + \frac{f-g}{g} \right) > 1 - \frac{f}{g}, %\\
%    & \hookrightarrow
    \end{split}
    \end{equation*}

    进而
    \begin{equation*}
        \begin{split}
            \int_{\mathbb{R}} \log \left( \frac{g}{f} \right) g \, \mathrm{d} \xi \,
            & > \int_{\mathbb{R}} \log \left( 1 - \frac{f}{g} \right) g \, \mathrm{d} \xi = \int_{\mathbb{R}} \left( g - f \right) \, \mathrm{d} \xi = 0.
        \end{split}
    \end{equation*}

    显然,$g \overset{\text{a.e.}}{=}$时,$KL \left(g:f \right)=0$。反之,如果已知$KL \left(g:f \right)=0$,那么假设条件$g \overset{\text{a.e.}}{=}$成立。
\end{proof}

然而值得注意的是,KLIC第一不适合用作无方向的公制测量单位,这是因为通常来说$KL(g:f) \neq KL(f:g)$,第二不满足三角不等式关系。因此只适合将KLIC看做是一个测度密度方程$f$和$g$的接近程度的简便工具。

设$\left\{ z_{t} \right\} \in \mathbb{R}^{\nu}$为一个定义在概率空间$\left( \Omega, \mathbb{F}, \mathbb{P}_{0} \right)$中的实值随机变量数列\footnote{
$\Omega$是样本空间,指所有可能结果(输出)的集合;$\mathbb{F}$是事件集合,单独某一事件或一组事件组成,含有0个或更多输出;$\mathbb{P}_{0}$是将事件和事件发生概率联系在一起的方程集合。}。在不产生混淆的情况下,我们将随机变量向量,及其实现(realization)表示为相同的符号$z$,用
\begin{equation*}
    z^{T} = \left( z_{1}, z_{2}, \ldots, z_{T} \right)
\end{equation*}
表示$\left\{ z_{t} \right\}$生成的全部信息集合。向量$z_{t} \in \mathbb{R}^{\nu \times 1}$包括两部分,分别是标量$y_{t}$和向量$\omega_{t} \in \mathbb{R}^{\left( \nu-1 \right) \times 1}$。那么对应于$\mathbb{P}_{0}$,存在一个关于$z^{T}$的联合概率密度方程(joint probability density function, JPDF)\index{probability density function (PDF)!joint \dotfill 联合概率密度方程} $g^{T}\left( z^{T} \right) $,满足
\begin{equation}
    \label{eq:qmle-klic-jpdf-true-g}
    g^{T} \left( z^{T} \right) = g\left( z_{1} \right)
    \, \prod_{t=2}^{T} \frac{g^{t} \left( z^{t} \right)}{g^{t-1} \left( z^{t-1} \right)}
    = g \left( z_{1} \right) \prod_{t=2}^{T} g_{t} \left(z_{t} | z^{t-1} \right),
\end{equation}
其中
\begin{itemize}
    \item $g^{t}$是JPDF表示$t$个随机变量$z_{1}, z_{2}, \ldots, z_{t}$的JPDF,
    \item $g_{t}$是PDF,表示基于全部过去信息$z_{1},\ldots,z_{t-1}$基础上,现有随机事件$z_{t}$的密度方程,
    \item $g \left( z_{1} \right)$是初始无条件PDF,
    \item 因而我们可以将$g^{T}$理解为一个描述随机行为$z^{T}$的机制,又称$Z^{T}$的数据生成过程(data generating process, DGP)\index{data genrating process (DGP) \dotfill 数据生成过程}。
\end{itemize}

由于真实的DGP $g^{T}$未知,我们要猜测条件PDF $f_{t} \left( z_{t} | z^{t-1} ; \theta \right)$用作$g^{T}$的近似,其中参数向量$\theta \in \Theta \subseteq \mathbb{R}^{k}$。在此基础上,猜测的JPDF $f^{T}$为
\begin{equation}
    \label{eq:qmle-klic-jpdf-quasi-f}
    f^{T} \left( z^{T}; \theta \right) = f \left( z_{1} \right)
    \prod_{t=2}^{T} f_{t} \left( z_{t} | z^{t-1} ; \theta \right),
\end{equation}
类似地,$f \left( z_{1} \right)$也是初始无条件PDF。上式
称为关于$g^{T}$的准似然方程(quasi-likelihood function)\index{quasi-likelihood function (QLHF)\dotfill 准似然方程},``准"是因为,基于这样的设定,$f^{T}$ \eqref{eq:qmle-klic-jpdf-quasi-f}可以与$g^{T} \left( z^{T}; \theta \right)$ \eqref{eq:qmle-klic-jpdf-true-g}不一致。

为了表述方便,将非条件密度写成条件密度的形式$f\left( z_{1} \right), \, g \left(z_{1} \right)$;进而可以将$f^{T} \left( z^{T}; \theta \right), \, g^{T} \left( z^{T} \right)$表示为全部有条件PDF的乘积。显然,猜测JPDF $f^{T}$越是接近真实DGP $y^{T}$,信号的价值越高。$g^{T}$相对于$f^{T}$的KLIC写为
\begin{equation}
    \label{eq:qmle-klic-gf}
    KL \left(g^{T}:f^{T}; \theta \right)
    = \int_{\mathbb{R}^{T}} \log
    \left(
    \frac{
    g^{T} \left(\xi^{T}\right)
    }{
    f^{T} \left(\xi^{T}; \theta \right)
    }
    \right) g^{T} \left(\xi^{T}\right) \,
    \mathrm{d} \left( \xi^{T} \right).
\end{equation}
那么可以考虑如下优化方案,对于DGP $g^{T}$,选择近似方程$f^{T}$使$KL \left(g^{T}:f^{T}; \theta \right)$最小化,即生成的``惊讶"水平最低。
由于$g^{T}$与系数$\theta$无关,最小化\eqref{eq:qmle-klic-gf}的问题等价于最大化如下问题
\begin{equation}
    \label{eq:qmle-klic-gf-expectation}
    \int_{\mathbb{R}^{T}}
    \left[
    \log f^{T} \left( \xi^{T}; \theta \right)
    \right]
    g^{T} \left( \xi^{T} \right) \,
    \mathrm{d} \left( \xi^{T} \right) =
    E \log f^{T} \left( z^{T}; \theta \right),
\end{equation}
$E$表示期望值。最大化\eqref{eq:qmle-klic-gf-expectation}的问题又进一步等价于最大化条件JPDF的平均期望值:汇总$f_{t} \left(z_{t} | z^{t-1}; \theta \right) \, \forall \, t=1,\ldots,T$,有
\begin{equation}
    \label{eq:qmle-klic-gf-expectation-avg}
    \overline{L}_{T} \left( \theta \right) =
    \frac{1}{T} E \left[
    \log f^{T} \left(z^{T}; \theta \right)
    \right]
    = \frac{1}{T} E
    \left[ \sum_{t=1}^{T} \log f_{t} \left( z_{t} | z^{t-1}; \theta \right)\right].
\end{equation}

对\eqref{eq:qmle-klic-gf-expectation-avg}作最大似然处理,算得最优参数$\theta^{*}$
\begin{equation}
    \label{eq:qmle-klic-gf-expectation-avg-estimation}
    \theta^{*} = \underset{\theta}{\argmax} \, \overline{L}_{T} \left( \theta \right),
\end{equation}
同时也是\eqref{eq:qmle-klic-gf-expectation-avg}、\eqref{eq:qmle-klic-gf-expectation}的最优参数。

\begin{definition}[正确设定的识别条件]
    \label{definition:qmle-correct-specification-identification}
如果$\exists! \theta_{0} \in \Theta$,\footnote{$\exists!$表示存在且存在唯一一个,有时也表示为$\exists_{=1}$。}满足
\begin{equation*}
    f^{T} \left(z^{T}; \theta_{0} \right) = g^{T} \left(z^{T} \right)\, \forall t \in T,
\end{equation*}
那么我们说$\left\{ f_{t} \left(z^{t} | z^{t-1}; \theta \right) \right\}_{t=1}^{T}$对$\left\{ z_{t} \right\}_{t=1}^{T}$整体上是正确设定的(correct specification in its entirety for $\left\{ z_{t} \right\}_{t=1}^{T}$),此时最优系数被识别为$\theta_{0}$,满足KLIC的最小化
\begin{equation*}
    KL \left(g^{T}:f^{T}; \theta_{0} \right) = 0.
\end{equation*}
\end{definition}

根据这样的识别条件可得,$\theta^{*} = \theta_{0}$。

尽管道理上来说是这样,可实际操作过程中\eqref{eq:qmle-klic-gf-expectation-avg-estimation}不易估计:这是由于$\overline{L}_{T} \left( \theta \right)$ \eqref{eq:qmle-klic-gf-expectation-avg}的计算涉及到未知DGP $g^{T}$,和取期望值的计算。对此,我们常常采取近似的替代方案,将$\overline{L}_{T} \left( \theta \right)$替换为一个样本$\overline{L}_{T} \left( \theta \right)$下的
QLHF $L_{T} \left( z^{T}; \theta \right)$
\begin{equation}
    \label{eq:qmle-klic-qlhf-lt}
    L_{t} \left(z^{T}; \theta \right) \coloneqq \frac{1}{T} \sum_{t=1}^{T}
    \log f_{t} \left(z_{t} | z^{t-1}; \theta \right),
\end{equation}
对应地,最优估计值
\begin{equation*}
    \tilde{\theta}_{T} = \underset{\theta}{\argmax} \, L_{T}(z^{T}; \theta),
\end{equation*}
称为$\theta$的准最大似然估计(quasi maximum likelihood estimator, QMLE)\index{maximum likelihood estimator (MLE)!quasi \dotfill 准最大似然估计}。加入前缀``准"(quasi)是为了表明,估计的解$\tilde{\theta}$来自于一个可能存在误设定问题的对数似然方程$\log f_{t}$。$\left\{ f_{t} \right\}$可能并非关于$\left\{ z_{t} \right\}$整体设定正确,此时$f^{T} \neq g^{T}, \, \tilde{\theta}_{T} \neq \theta, \, KL >0$。
反之若$\left\{ f_{t} \right\}$设定正确,那么$f^{T} = g^{T}, \, \tilde{\theta}_{T} = \theta, \, KL =0$,此时准最大似然估计等价于前述的表准最大似然估计。

在现实应用中,为$z^{T}$设置完整的概率模型可能是一项艰巨的任务,涉及到太多随机变量:例如$T$个随机变量的向量$z_{t}$,每个$z_{t}$中都含有$\nu$个随机变量。这就需要作以适当简化。对于经济学来说,往往只关心其中的一部分核心内容,例如对随机变量$y_{t}$建模,假定$y_{t}$与前定变量(preditermined variable, 如\cite{Klein:2000bc}的定义)的向量$x_{t}$有关,$x_{t}$中包含元素$\left( \omega_{t}, z^{t-1} \right)$。这样一来,模型中只需要考虑$y_{t}$的条件行为:
除非$\omega_{t}$还有其他明确描述,条件PDF $g_{t} \left(y_{t} | x_{t} \right)$提供了关于$\left\{ z_{t} \right\}$的一些关键(而非全部)信息。在此基础上,可以构建QLHF $f_{t} \left(y_{t} | x_{t}; \theta \right) $来近似$g_{t} \left( y_{t} | x_{t} \right)$,KLIC如\eqref{eq:qmle-klic-gf}所示。
对于全部$t=1,\ldots,T$,可定义平均KILC为 $\overline{KL} _{T} \left(
\left\{g_{t} : f_{t} \right\}; \theta \right)$
\begin{equation}
    \label{eq:qmle-kilc-average}
    \overline{KL} _{T} \left(
    \left\{g_{t} : f_{t} \right\}; \theta \right)
    \coloneqq \frac{1}{T} \sum_{t=1}^{T} KL \left(g_{t}:f_{t}; \theta \right),
\end{equation}
对应平均条件JPDF的平均期望
\begin{equation}
    \label{eq:qmle-conditional-pdf-avg-expectation}
    \overline{L}_{T} \left( \theta \right) = \frac{1}{T}
    \sum_{t=1}^{T} E \left[
    \log f_{t} \left(y_{t}| x_{t} ; \theta \right)
    \right].
\end{equation}

为了简化表述,设$y^{T} = \left(y_{1}, \ldots, y_{t} \right), x^{T} = \left(x_{1}, \ldots, x_{t} \right)$。则优化条件为,利用最大似然估计法求解QLME $\theta^{*}$
\begin{equation*}
\theta^{*} = \underset{\theta}{\argmax} \, \overline{L}_{T} \left( \theta \right),
\end{equation*}
同时也是$\underset{\theta}{\argmin} \,  \overline{KL}_{T} \left( \left\{g_{t} : f_{t} \right\}; \theta  \right)$的值。根据Definition \eqref{definition:qmle-correct-specification-identification},识别条件为:如果$\exists! \theta_{0} \in \Theta$,满足
\begin{equation*}
    f_{t} \left(y_{t} | x_{t} ; \theta_{0} \right) = g \left( y_{t} | x_{t} \right) \, \forall t,
\end{equation*}
那么$\left\{ f_{t} \right\}$对$\left\{ y_{t} | x_{t} \right\}$整体设定正确。进而有$\overline{KL}_{T} \left( \left\{ g_{t}: f_{t} \right\}; \theta_{0} \right) = 0$,以及$\theta^{*} = \theta_{0}$。

和前述情况一样,$\overline{L}_{T} \left( \theta \right)$ \eqref{eq:qmle-conditional-pdf-avg-expectation}无法直接计算,同样地,采用类似于\eqref{eq:qmle-klic-qlhf-lt}的方式,用样本QLHF $L_{T}\left( y^{T}, x^{T}; \theta \right)$作近似
\begin{equation}
    \label{eq:qmle-conditional-pdf-avg-expectation-approx}
    L_{T} \left(y^{T}, x^{T}; \theta \right) \coloneqq
    \frac{1}{T} \sum_{t=1}^{T} \log f_{t} \left(y_{t} | x_{t}; \theta \right).
\end{equation}

基于\eqref{eq:qmle-conditional-pdf-avg-expectation-approx}作准最大似然估计求得QMLE $\tilde{\theta}_{T}$
\begin{equation*}
    \tilde{\theta}_{t} = \underset{\theta}{\argmax} \, L_{t} \left( y^{T}, x^{T}; \theta \right).
\end{equation*}
若$\left\{f_{t} \right\}$对于$\left\{y_{t} | x_{t} \right\}$完整正确设定,则QMLE $\tilde{\theta}_{T}$可视为等价于标准MLE。

现在来考虑另一个情况,$y_{t}$具有某些特殊性质,我们猜测该特性表现为条件正态分布,均值和方差分别为$\mu_{t} \left(x_{t}; \theta \right)$和$\sigma^{2}$,满足
\begin{equation*}
    y_{t} | x_{t} \sim \mathcal{N} \left( \mu_{t} \left(x_{t}; \beta \right) , \sigma^{2} \right).
\end{equation*}
根据高斯分布的性质可以算得
\begin{equation*}
    \theta = \left( \beta^{\top} \sigma^{2} \right)^{\top}.
\end{equation*}

首先列出QLHF的计算式及其样本近似
\begin{equation*}
    \overline{L} \left( \theta \right)
    \approx L_{T} \left( y^{T}, x^{T} ; \theta \right)
    = \frac{1}{T} \sum_{t=1}^{T} \log f \left(y_{t} | x_{t}; \theta \right),
\end{equation*}
进而QLME $\tilde{\beta}_{T}$
\begin{equation*}
    \tilde{\beta}_{T} = \underset{\beta}{\argmax} \, L_{T} \left( y^{T}, x^{T} ; \theta \right),
\end{equation*}
同时也有
\begin{equation*}
    \tilde{\beta}_{T} = \underset{\beta}{\argmin} \,
    \frac{1}{T} \sum_{t=1}^{T}
    \left[
    y_{t} - \mu_{t} \left( x_{t}; \beta \right)
    \right]^{\top} \,
    \left[
    y_{t} - \mu_{t} \left(x_{t}; \beta \right)
    \right]
\end{equation*}
关系成立,即它同时也是非线性最小二乘法(nonlinear regression, NLS)\index{nonlinear regression (NLS) \dotfill 非线性最小二乘法}的估计。由此可见,NLS可以看做是QMLE在某些假定条件下的特例,这个假定条件是:观测数据表现出有条件的正态分布,以及有条件的同方差。识别条件为,设$\exists! \theta_{0}$,满足
\begin{equation*}
    \mu_{t} \left( x_{t}; \theta_{0} \right) = E \left(y_{t}|x_{t} \right)
\end{equation*}
时,我们称$\left\{ \mu_{t} \right\}$对条件均值$\left\{ E \left(y_{t} | x_{t} \right) \right\}$整体正确设定。

除此而外,也可以考虑一个更加灵活的设定方式,如
\begin{equation*}
    y_{t} | x_{t} \sim \mathcal{N} \left( \mu_{t} \left(x_{t} ; \beta \right), h\left(x_{t}; \alpha \right) \right),
\end{equation*}
可以帮助我们进一步分析、估计条件方差。

\subsection{QMLE的渐进性质}
\label{sec:qmle-asymptotic}
通常来说,QLHF $L_{T} \left( \cdot ; \theta \right)$是一个关于系数向量$\theta$的非线性方程系统。非线性意味着我们往往无法直接求得解析解QMLE $\tilde{\theta}_{T}$,而是要靠非线性优化算法作数值近似。那么在数值近似之前,就需要探讨QMLE的金斯特征。本节中,我们假定QLHF $L_{T} \left( \cdot ; \theta \right)$在紧凑参数空间$\Theta$中的概率为$1$,1次、2次连续可导,积分和微分运算可交换顺序,并且识别条件成立:$\exists! \theta^{*} \in \Theta$,使得KLIC \eqref{eq:qmle-klic-def}或\eqref{eq:qmle-klic-gf}最大化。要求$\theta^{*} \in \Theta$是为了确保平均值扩展\eqref{eq:qmle-mean-value-expansion}和随后QLHF $L_{T}$的渐进分析是有效的。

以关于$z^{T}$的QLHF $L_{T} \left( z^{T}; \tilde{\theta}_{T} \right)$为例,由于可能存在误设定问题,QMLE $\tilde{\theta}_{T}$可能不同于$\theta^{*}$。
将一阶矩条件$\triangledown L_{T} \left( z^{T}; \tilde{\theta}_{T} \right)$针对$\theta^{*}$作平均值扩展
\begin{equation}
    \label{eq:qmle-mean-value-expansion}
    \triangledown L_{T} \left( z^{T}; \tilde{\theta}_{T} \right) \approx
    \triangledown L_{T} \left(z^{T}; \theta^{*} \right)
    + \triangledown^{2} L_{t} \left(z^{T}; \theta_{T}^{\dagger} \right) \left(\tilde{\theta}_{T} - \theta^{*} \right),
\end{equation}
其中$\theta_{T}^{\dagger} \in \Theta$是个在$\tilde{\theta}_{T}$和$\theta^{*}$之间的参数向量。通过选取合适的$\tilde{\theta}_{T}$值,使得\eqref{eq:qmle-mean-value-expansion} LHS =0,进而我们有一阶矩条件和二阶矩条件的关系
\begin{equation*}
    \triangledown L_{T} \left(z^{T}; \theta^{*} \right)
    = -\triangledown^{2} L_{t} \left(z^{T}; \theta_{T}^{\dagger} \right) \left(\tilde{\theta}_{T} - \theta^{*} \right).
\end{equation*}

假定$\triangledown^{2} L_{t} \left(z^{T}; \theta_{T}^{\dagger} \right)$可逆,即要求QLHF $L_{T}$在给定的$\theta^{*}$系数下为局部二次形式。集合全部$\theta_{T}^{\dagger} \in \Theta$的取值,将二阶矩的期望值设为期望海森矩阵(expected Hessian matrix)\index{Hessian matrix \dotfill 海森矩阵}
\begin{equation}
    \label{eq:qmle-hessian-matrix-def}
    H_{T} \left( \theta \right) \coloneqq E
    \left[
    \triangledown^{2} L_{T} \left( z^{T}; \theta \right)
    \right].
\end{equation}

如果$\triangledown^{2} L_{T} \left(z^{T}; \theta_{T}^{\dagger} \right)$遵循弱均匀大数法则(WULLN, 第\ref{sec:lln}节),那么如下条件在参数空间$\Theta$中均匀成立
\begin{equation*}
    \triangledown^{2} L_{T} \left( z^{T}; \theta \right)
    -\underbrace{
    E
    \left[
    \triangledown^{2} L_{T} \left( z^{T}; \theta \right)
    \right]
    } _{=H_{T} \left( \theta \right)}
    \overset{\mathbb{P}}{\longrightarrow} 0.
\end{equation*}

因此可将\eqref{eq:qmle-mean-value-expansion}改写为
\begin{equation}
    \label{eq:qmle-mean-value-expansion-hessian}
    \sqrt{T} \left( \tilde{\theta}_{T} - \theta^{*} \right)
    = - \left( H_{T} \left(\theta^{*} \right) \right)^{-1} \,
    \sqrt{T} \,
    \triangledown L_{T} \left( z^{T}; \theta^{*} \right)
    + \mathcal{O}_{p} \left( 1 \right),
\end{equation}
就是说,QMLE系数距其真实系数的标准化距离 $\sqrt{T} \left( \tilde{\theta}_{T} - \theta^{*} \right)$ 的渐进分布,就事实上取决于标准化core方程 $\sqrt{T} \,
\triangledown L_{T} \left( z^{T}; \theta^{*} \right)$ 的渐进分布。

将标准化socre方程的方差协方差矩阵设为$B_{T} \left( \theta \right)$
\begin{equation}
    \label{eq:qmle-score-information-matrix-def}
    B_{T} \left( \theta \right) = \var
    \left(
    \sqrt{T} \,
    \triangledown L_{T} \left( z^{T}; \theta^{*} \right)
    \right),
\end{equation}
又称信息矩阵(information matrix)\index{information matrix \dotfill 信息矩阵}。

若假定$\log f_{t} \left( z_{t} | z^{t-1} \right)$满足中心极限定理(CLT, 第\ref{sec:kernel-gaussian-central-limit-theorem}),可得
\begin{equation}
    \label{eq:qmle-info-matrix-distribution}
    \left( B_{T} \left(\theta^{*} \right) \right)^{-\frac{1}{2}}
    \sqrt{T}
    \triangledown L_{T} \left(z^{T}; \theta^{*} \right)
    - \underbrace{
    E \left[
    \triangledown L_{t} \left( z^{T}; \theta^{*} \right)
    \right]
    }_{\eqqcolon \mathcal{A}}
    \overset{D}{\longrightarrow} \mathcal{N} \left(0, I_{k} \right).
\end{equation}

\begin{proposition}
    \eqref{eq:qmle-info-matrix-distribution}中的$\mathcal{A}=0$。
\end{proposition}
\begin{proof}
根据微分和积分顺序可以互换的假定,有
\begin{equation}
    \label{eq:qmle-klic-gf-expectation-avg-1derivative}
    E \left[
    \triangledown L_{t} \left( z^{T}; \theta \right)
    \right]
    =
    \triangledown E \left[
    L_{T} \left( z^{T}; \theta \right)
    \right]
    \triangledown \overline{L}_{T} \left( \theta \right),
\end{equation}
其中RHS是\eqref{eq:qmle-klic-gf-expectation-avg}的一阶导数。

根据QMLE $\theta^{*}$的性质,以下关系依次成立
\begin{equation*}
    \begin{split}
        & \theta^{*} = \underset{\theta}{\argmin} \, KL \left(z^{T}; \theta \right), \\
        \hookrightarrow & KL \left(z^{T}; \theta^{*} \right) = 0, \\
        \hookrightarrow & \triangledown \overline{L}_{T} \left( \theta^{*} \right) = 0,
    \end{split}
\end{equation*}
进一步代入关于$\theta^{*}$的关系式\eqref{eq:qmle-klic-gf-expectation-avg-1derivative}可得
\begin{equation*}
    \mathcal{A} =
    E \left[
    \triangledown L_{t} \left( z^{T}; \theta^{*} \right)
    \right]
    = 0.
\end{equation*}
\end{proof}

\eqref{eq:qmle-info-matrix-distribution}改写为
\begin{equation*}
    %\label{eq:qmle-info-matrix-distribution-2}
    \left( B_{T} \left(\theta^{*} \right) \right)^{-\frac{1}{2}}
    \sqrt{T}
    \triangledown L_{T} \left(z^{T}; \theta^{*} \right)
    \overset{D}{\longrightarrow} \mathcal{N} \left(0, I_{k} \right),
\end{equation*}

进一步代回\eqref{eq:qmle-mean-value-expansion-hessian},改写成
\begin{equation*}
%    \label{eq:qmle-mean-value-expansion-hessian-2}
    \sqrt{T} \left( \tilde{\theta}_{T} - \theta^{*} \right)
    = - \left( H_{T} \left(\theta^{*} \right) \right)^{-1} \,
    B_{T} \left(\theta^{*} \right)^{\frac{1}{2}}
    \underbrace{
    \left[
    B_{T} \left(\theta^{*} \right)^{- \frac{1}{2}}
    \sqrt{T}
    \triangledown L_{T} \left( z^{T}; \theta^{*} \right)
    \right]
    }_{\overset{D}{\longrightarrow} \mathcal{N} \left(0, I_{k} \right)}
    + \mathcal{O}_{p} \left( 1 \right),
\end{equation*}
表现为渐进的正态分布特征。由此可得以下结论
\begin{theorem}[QMLE的渐进正态分布特征]
    \label{theorem:qmle-asymptotic-normal-distribution}
    若\eqref{eq:qmle-mean-value-expansion},
    \eqref{eq:qmle-mean-value-expansion-hessian},
    \eqref{eq:qmle-info-matrix-distribution}均成立,那么有
    QMLE系数距其真实系数的标准化距离 $\sqrt{T} \left( \tilde{\theta}_{T} - \theta^{*} \right)$表现为渐进正态分布,满足
    \begin{equation}
        \label{eq:qmle-asymptotic-normal-distribution}
        C_{T} \left( \theta^{*} \right)^{-\frac{1}{2}}
        \sqrt{T} \left( \tilde{\theta}_{T} - \theta^{*} \right)
        \overset{D}{\longrightarrow} \mathcal{N} \left(0, I_{k} \right),
    \end{equation}
    其中方差协方差矩阵$C_{T} \left( \theta^{*} \right)$满足关系
    \begin{equation}
        \label{eq:qmle-asymptotic-normal-distribution-c}
            C_{T} \left( \theta^{*} \right) =
            \left[
            H_{T} \left( \theta^{*} \right)
            \right]^{-1}
            B_{T} \left( \theta^{*} \right)
            \left[
            H_{T} \left( \theta^{*} \right)
            \right]^{-1},
    \end{equation}
    $B_{T} \left( \theta^{*} \right), \, H_{T} \left( \theta^{*} \right)$分别表示标准化score方程的方差协方差矩阵(信息矩阵)和期望海森矩阵,由  \eqref{eq:qmle-score-information-matrix-def}和\eqref{eq:qmle-hessian-matrix-def}可得
    \begin{equation*}
        \begin{split}
            & B_{T} \left( \theta^{*} \right) = \var
            \left(
            \sqrt{T} \triangledown L_{T} \left( z^{T}; \theta^{*} \right)
            \right), \\
            & H_{T} \left( \theta^{*} \right) = E \triangledown^{2}
            L_{T} \left( z^{T}; \theta^{*} \right).
        \end{split}
    \end{equation*}
\end{theorem}

再来看关于$\left\{ y_{t}|x_{t} \right\}_{t=1}^{T}$的QLHF $L_{T} \left(y^{T} , x^{T}; \theta \right)$的情况。类似于,我们可得QMLE$\tilde{\theta}_{T}$的渐进正态分布特征
\begin{corollary}[QMLE的渐进正态分布特征]
    \label{corollary:qmle-asymptotic-normal-distribution}
    QMLE系数距其真实系数的标准化距离 $\sqrt{T} \left( \tilde{\theta}_{T} - \theta^{*} \right)$表现为渐进正态分布,满足
    \begin{equation*}
        C_{T} \left( \theta^{*} \right)^{-\frac{1}{2}}
        \sqrt{T} \left( \tilde{\theta}_{T} - \theta^{*} \right)
        \overset{D}{\longrightarrow} \mathcal{N} \left(0, I_{k} \right),
    \end{equation*}
    其中
    \begin{equation*}
        \begin{split}
            &C_{T} \left( \theta^{*} \right) =
            \left[
            H_{T} \left( \theta^{*} \right)
            \right]^{-1}
            B_{T} \left( \theta^{*} \right)
            \left[
            H_{T} \left( \theta^{*} \right)
            \right]^{-1},\\
            & B_{T} \left( \theta^{*} \right) = \var
            \left(
            \sqrt{T} \triangledown L_{T} \left( y^{T}, x^{T}; \theta^{*} \right)
            \right), \\
            & H_{T} \left( \theta^{*} \right) = E \triangledown^{2}
            L_{T} \left( y^{T}, x^{T}; \theta^{*} \right).
        \end{split}
    \end{equation*}
\end{corollary}

\section{信息矩阵等式}
\label{eq:qlme-ime}
准最大似然估计的一个重要结论是信息矩阵等式(information matrix equality, IME)\index{information matrix equality (IME) \dotfill 信息矩阵等式},是说在特定条件下,信息矩阵$B_{T} \left( \theta \right)$ \eqref{eq:qmle-score-information-matrix-def}等于海森矩阵期望值\eqref{eq:qmle-hessian-matrix-def}的负数$- E H_{T} \left( \theta \right)$。
IME可以简化方差协方差矩阵\eqref{eq:qmle-asymptotic-normal-distribution-c} $C_{T} \left( \theta \right)$的计算。

\subsection{一种设定形式下的IME}
\label{eq:qlme-ime-z}
先来看设定形式$\left\{z^{T} \right\}_{t=1}^{T}$。
将score方程定义为$s_{t} \left( z^{t}; \theta \right)$,反映对数PDF的一阶变化
\begin{equation*}
    %\label{eq:qlme-ime-def}
    s_{t} \left( z^{t}; \theta \right) = \triangledown \log f_{t}  \left( z_{t} | z^{t-1} ; \theta \right),
\end{equation*}
从而
\begin{equation*}
    \triangledown f_{t} \left( z_{t} | z^{t-1} ; \theta \right)
    = s_{t} \left(z^{t}; \theta \right)
    f_{t} \left( z_{t} | z^{t-1} ; \theta \right),
\end{equation*}

汇总全部$t=1,\ldots,T$的score方程,有加权平均
\begin{equation}
    \label{eq:qlme-ime-def}
    s^{T} \left( z^{T}; \theta \right) = \frac{1}{T} \sum_{t=1}^{T} s_{t}  \left(z^{t}; \theta \right).
\end{equation}

进而有
\begin{equation}
    \label{eq:qmle-ime-mid-equality}
    \begin{split}
        s^{T} \left(z^{T}; \theta \right) \, f^{T} \left(z^{T}; \theta \right) & =
        \left[
        \frac{1}{T} \sum_{t=1}^{T} s_{t} \left(z^{t}; \theta \right)
        \right] \,
        \left[
        \prod_{\tau=1}^{T} f^{\tau} \left(z_{t} | z^{\tau-1} ; \theta \right)
        \right] \\
        & = \frac{1}{T} \sum_{t=1}^{T}
        \left[
        \triangledown \log f_{t} \left(z_{t} | z^{t-1}; \theta \right) \,
        \prod_{\tau=1,\tau \neq t}^{T} f^{\tau} \left(z_{\tau} | z^{\tau - 1}; \theta \right)
        \right] \\
        & = \frac{1}{T} \triangledown f^{T} \left(z^{T}; \theta \right)
    \end{split}
\end{equation}

根据PDF的定义
\begin{equation*}
    \int_{\mathbb{R}} f^{T} \left(z^{T}; \theta \right) \, \mathrm{d} z^{T} = 1,
\end{equation*}
可得导数的积分必为$0$,
\begin{align*}
    & \int_{\mathbb{R}} \triangledown f^{T} \left(z^{T}; \theta \right) \, \mathrm{d} z^{T} = 0, \\
    & \int_{\mathbb{R}} \triangledown^{2} f^{T} \left(z^{T}; \theta \right) \, \mathrm{d} z^{T} = 0.
\end{align*}

设积分微分顺序可以互换,则
\begin{align}
    \label{eq:qmle-ime-mid-int1}
    & \frac{1}{T} \int_{\mathbb{R}} \triangledown f^{T} \left( z^{T}; \theta \right) \, \mathrm{d} z^{T} = \int_{\mathbb{R}} s^{T} \left(z^{T}; \theta \right) \, \mathrm{d} z^{T} =0, \\
    \label{eq:qmle-ime-mid-int2}
    & \frac{1}{T} \int_{\mathbb{R}} \triangledown^{2} f^{T} \left(z^{T}; \theta \right) \, \mathrm{d} z^{T} = \int_{\mathbb{R}}
    \left\{
    \triangledown s^{T} \left( z^{T}; \theta \right)
    % f^{T} \left(z^{T}; \theta \right)
    + T \left[
    s^{T} \left(z^{T} ; \theta \right) \, s^{T} \left(z^{T} ; \theta \right)^{\top}
    \right]
    \right\} f^{T} \left(z^{T}; \theta \right)
    \, \mathrm{d} z^{T} =0.
\end{align}

进一步的分析需要分两种情况来讨论。第一种情况,设$\left\{ f_{t} \left(z^{t}|z^{t-1} ; \theta \right) \right\}_{t=1}^{T}$对$\left\{ z_{t} \right\}_{t=1}^{T}$整体正确设定。那么
在真实(但未知的)PDF $g^{T} \left(z^{T} \right)$对应的系数$\theta_{0}$下,对近似PDF $f^{T} \left( z^{T} ; \theta_{0} \right)$取期望值,由\eqref{eq:qmle-ime-mid-int1}-\eqref{eq:qmle-ime-mid-int2}可得
\begin{align}
    \label{eq:qmle-ime-mid-correct-int1}
    & E \left[ s^{T} \left( z^{T} ; \theta_{0} \right) \right]=0, \\
    \label{eq:qmle-ime-mid-correct-int2}
    & \underbrace{
    E \left[
    \triangledown s^{T} \left( z^{T} ; \theta_{0} \right)
    \right]
    }_{\eqqcolon \mathcal{A}}
     + \underbrace{
     T \, E
    \left[
    s^{T} \left( z^{T}; \theta_{0} \right) \,
    s^{T} \left( z^{T}; \theta_{0} \right)^{\top}
    \right]
    }_{\eqqcolon \mathcal{B}}
     =0,
\end{align}
这便成为信息矩阵等式。
\begin{theorem}[信息矩阵等式]
    \label{theorem:qmle-ime-def}
    假定 $\exists \, \theta_{0} \in \Theta$满足
    \begin{equation*}
        f_{t} \left(z_{t} | z^{t-1} ; \theta_{0} \right) = g_{t}
        \left(z_{t} | z^{t-1} ; \theta_{0} \right),
    \end{equation*}
    那么可得
    \begin{equation}
        \label{eq:qmle-ime-def}
        \begin{split}
            & H_{T} \left( \theta_{0} \right) = - B_{T} \left( \theta_{0} \right), \\
            & H_{t} \left( \theta_{0} \right) = E \triangledown^{2} L_{T} \left(z^{T}; \theta_{0} \right), \\
            & B_{T} \left( \theta_{0} \right) = \var \left( \sqrt{T} \triangledown L_{T} \left(z^{T}; \theta_{0} \right) \right).
        \end{split}
    \end{equation}
\end{theorem}

\begin{corollary}[QLME的克拉美罗下边界(CRLB)]
    \label{corollary:qmle-ime-lower-bound}

    假定Theorem \ref{theorem:qmle-ime-def}成立,那么计算标准化距离$\sqrt{T} \left( \tilde{\theta}_{T} - \theta_{0} \right)$的所需的方差协方差矩阵 $c_{T} \left( \theta_{0} \right)$ \eqref{eq:mle-scalar-crlb-unbiased} ,可简化为
    \begin{equation}
    \label{eq:qmle-ime-lower-bound}
    C_{T} \left( \theta_{0} \right) \ge B_{T} \left( \theta_{0} \right)^{-1} = - H_{T} \left( \theta_{0} \right)^{-1},
    \end{equation}
    QMLE $\theta_{0}$ 逐渐逼近克拉美罗下界(Cramér-Rao lower bound, CRLB)\index{Cramér-Rao lower Bound \dotfill 克拉美罗下界}。
\end{corollary}

第二种情况,如果$\left\{ f_{t} \left(z^{t}|z^{t-1} ; \theta \right) \right\}_{t=1}^{T}$对$\left\{ z_{t} \right\}_{t=1}^{T}$整体并非正确设定,那么score方程很可能与真实PDF $g^{T}$无关,换句话说,哪怕系数取值的确是$\theta^{*}$,我们也无法保证score方程的期望均值是$0$,\eqref{eq:qmle-ime-mid-correct-int1}-\eqref{eq:qmle-ime-mid-correct-int2}改写为
\begin{align}
    \label{eq:qmle-ime-mid-incorrect-int1}
    & E \left[ s^{T} \left( z^{T} ; \theta_{0} \right) \right] \neq 0, \\
    \label{eq:qmle-ime-mid-incorrect-int2}
    & \underbrace{
    E \left[
    \triangledown s^{T} \left( z^{T} ; \theta_{0} \right)
    \right]
    }_{\eqqcolon \mathcal{A}}
     + \underbrace{
     T \, E
    \left[
    s^{T} \left( z^{T}; \theta_{0} \right) \,
    s^{T} \left( z^{T}; \theta_{0} \right)^{\top}
    \right]
    }_{\eqqcolon \mathcal{B}}
     \neq 0,
\end{align}
在这种情况下,信息矩阵等式\eqref{eq:qmle-ime-def}不成立。

\subsection{另一种score方程定义下的IME}
\label{eq:qlme-ime-yx}
来看另一种设定形式$\left\{ y_{t} | x_{t} \right\}_{t=1}^{T}$。类似于\eqref{eq:qmle-ime-mid-equality},有
\begin{align}
    \label{eq:qmle-ime-mid-equality-yx-single}
    & \triangledown f_{t} \left( y_{t} | x_{t}; \theta \right) = s_{t} \left(y_{t} | x_{t}; \theta \right) \, f_{t} \left( y_{t} | x_{t}; \theta \right), \\
    \label{eq:qmle-ime-mid-equality-yx-sum}
    & \triangledown f^{T} \left( y^{T} | x^{T}; \theta \right) = s^{T} \left(y^{T} , x^{T}; \theta \right) \, f^{T} \left( y^{T} | x^{T}; \theta \right),
\end{align}
并且有
\begin{equation*}
  \int_{\mathbb{R}} f_{t} \left(y_{t} | x_{t} ; \theta \right) \, \mathrm{d} y_{t} =1.
\end{equation*}

那么类似地,
\begin{equation*}
  \begin{split}
    & \int_{\mathbb{R}} s_{t} \left(y_{t}, x_{t}, \theta_{t} \right) \,
    f_{t} \left( y{t} | x_{t} ; \theta \right) \, \mathrm{d} y_{t} =0, \\
    & \int_{\mathbb{R}}
    \left[
    \triangledown s_{t} \left( y_{t}, x_{t}; \theta \right)
    + s_{t} \left( y_{t}, x_{t}; \theta \right) \,
    \left( y_{t}, x_{t}; \theta \right)^{\top}
    \right]
    \, f_{t} \left( y_{t} | x_{t}; \theta \right)
    \, \mathrm{d} y_{t} =0.
  \end{split}
\end{equation*}

如果$\left\{ f_{t} \right\}_{t=1}^{T} $关于$\left\{ y_{t} | x_{t} \right\}_{t=1}^{T}$整体正确设定,那么
\begin{equation*}
    E \left[ s_{t} \left( y_{t}, x_{t}; \theta \right) | x_{t} \right] =0,
\end{equation*}
进而根据迭代期望法则(law of iterated expectations, LIE, 第\ref{sec:qmle-lie-intro}节)可得
\begin{equation}
    \label{eq:qmle-ime-yx-s0}
    E \left[ s_{t} \left( y_{t}, x_{t}; \theta_{0} \right) \right] =0,
\end{equation}
因此可见score方程的平均值是正确设定的。

根据条件方差的分解(Lemma \ref{lemma:conditional-variance-decomposition}),\eqref{eq:qmle-ime-yx-s0}改写为
\begin{equation*}
    E \left[
    \triangledown s_{t} \left( y_{t}, s_{t} ; \theta_{0} \right) | x_{t}
    \right]
    + E \left[
    s_{t} \left( y_{t}, s_{t} ; \theta_{0} \right) \,
    s_{t} \left( y_{t}, s_{t} ; \theta_{0} \right)^{\top}
    | x_{t}
    \right] =0,
\end{equation*}
并且这个等式对$t=1,\ldots,T$均成立。加总可得
\begin{equation}
    \label{eq:qmle-ime-yx-s0-def}
\begin{split}
    & \frac{1}{T} \sum_{t=1}^{T} E \left[ \triangledown s_{t} \left( y_{t}, s_{t} ; \theta_{0} \right) \right]
    +
    \frac{1}{T} \sum_{t=1}^{T} E \left[
    s_{t} \left( y_{t}, s_{t} ; \theta_{0} \right) \,
    s_{t} \left( y_{t}, s_{t} ; \theta_{0} \right)^{\top}
    \right] \\
    & =H_{t} \left( \theta_{0} \right)
    + \frac{1}{T} \sum_{t=1}^{T} E
    \left[
    s_{t} \left( y_{t}, s_{t} ; \theta_{0} \right) \,
    s_{t} \left( y_{t}, s_{t} ; \theta_{0} \right)^{\top}
    \right]
    =0,
\end{split}
\end{equation}
在满足特定前提假设的条件下,上式等价于IME。

\subsection{动态误设定}
\label{sec:qmle-ime-dynamic-misspecification}
    需要指出的是,\eqref{eq:qmle-ime-yx-s0-def}并不必然总是等价于IME。例如,考虑一组对$\left\{ y_{t} | x_{t} \right\}_{t=1}^{T}$整体正确设定了的$\left\{ f_{t} \left( y_{t} | x_{t} ; \theta \right) \right\}_{t=1}^{T}$。如果改设定对$\left\{ y_{t} | \omega_{t}, z^{t-1} \right\}_{t=1}^{T}$而言并非正确设定,
    那么我们称之为动态误设定(dynamic misspecification):$\nexists \theta_{0} \in \Theta$,
    使得满足$f_{t} \left( y_{t} | x_{t}; \theta_{0} \right) = g_{t} \left( y_{t} | \omega_{t}, z^{t-1} \right)$,
    换句话说,$\left\{\omega_{t}, z^{t-1} \right\}$所包含的信息,无法由$\left\{ x_{t} \right\}$所完全表现出来。

在不存在动态误设定的情况下,根据定义我们有
\begin{equation*}
    E \left[
    s_{t} \left( y_{t}, x_{t} ; \theta_{0} \right) | x_{t}
    \right]
    = E \left[
    s_{t} \left( y_{t}, x_{t} ; \theta_{0} \right) | \omega_{t}, z^{t-1} \right],
\end{equation*}

进而
\begin{equation}
    \label{eq:qmle-ime-dynamic-misspecification}
    \begin{split}
        B_{T} \left( \theta_{0} \right)
        = & \frac{1}{T} E
        \left[
        \left(
        \sum_{t=1}^{T} s_{t} \left( y_{t}, x_{t} ; \theta_{0} \right)
        \right) \,
        \left(
        \sum_{t=1}^{T} s_{t} \left( y_{t}, x_{t} ; \theta_{0} \right)
        \right)^{\top}
        \right] \\
        = & \frac{1}{T} \sum_{t=1}^{T} E
        \left[
        \left(
        s_{t} \left( y_{t}, x_{t} ; \theta_{0} \right)
        \right) \,
        \left(
        s_{t} \left( y_{t}, x_{t} ; \theta_{0} \right)
        \right)^{\top}
        \right] \\
        & + \underbrace{
        \frac{1}{T} \sum_{\tau = 1}^{T-1} \sum_{t=\tau+1}^{T} E
        \left[
        s_{t-\tau} \left( y_{t-\tau}, x_{t-\tau}; \theta_{0} \right) \,
        s_{t} \left(y_{t}, x_{t} ; \theta_{0} \right)^{\top}
        \right]
        }_{\eqqcolon \mathcal{A}_{1}} \\
        & + \underbrace{
        \frac{1}{T} \sum_{\tau = 1}^{T-1} \sum_{t=\tau+1}^{T} E
        \left[
        s_{t} \left(y_{t}, x_{t}; \theta_{0} \right) \,
        s_{t+\tau} \left( y_{t + \tau}, x_{t + \tau}; \theta_{0} \right)^{\top}
        \right]
        }_{\eqqcolon \mathcal{A}_{2}} \\
        & = \frac{1}{T} \sum_{t=1}^{T} E \left[
        s_{t} \left( y_{t}, x_{t}; \theta_{0} \right) \,
        s_{t} \left( y_{t}, x_{t}; \theta_{0} \right)^{\top}
        \right],
    \end{split}
\end{equation}
根据LIE \eqref{eq:qmle-lie-vector}和条件方差分解\eqref{eq:conditional-variance-decomposition}可得
\begin{equation*}
    \mathcal{A}_{1} = \mathcal{A}_{2} =0.
\end{equation*}

另一方面,在可能存在动态误设定的情况下,\eqref{eq:qmle-ime-dynamic-misspecification}的等号不成立:score方程的协方差可能不为$0$。这表明,个别信息矩阵的平均值$\var \left( s_{t} \left(y_{t}, x_{t}; \theta_{0} \right) \right)$可以不是信息矩阵。

\begin{theorem}[信息矩阵等式]
    \label{theorem:qmle-ime-yx-def}
    假设$\left\{ y_{t} | x_{t} \right\}_{t=1}^{T}$的设定形式,并且$\exists \theta_{0} \in \Theta$满足$f_{t} \left( y_{t} | x_{t} ; \theta \right) = g_{t} \left(y_{t} | x_{t} \right)$,并且其中不存在动态误设定的情况。那么我们有以下IME成立
    \begin{equation}
        \label{eq:qmle-ime-yx-def}
        \begin{split}
            & H_{T} \left( \theta_{0} \right) = - B_{T} \left( \theta_{0} \right), \\
            & H_{T} \left( \theta_{0} \right) = E \left[
            \triangledown^{2} L_{T} \left( z^{T}; \theta_{0} \right)
            \right]
            = \frac{1}{T} \sum_{t=1}^{T}
            E \left[
            \triangledown s_{t} \left( y_{t}, x_{t} ; \theta_{0} \right)
            \right], \\
            & B_{T} \left( \theta_{0} \right) = \var
            \left(
            \sqrt{T} \triangledown L_{T} \left( z^{T}; \theta \right)
            \right)
            = \frac{1}{T} \sum_{t=1}^{T} E
            \left[
            s_{t} \left( y_{t}, x_{t}; \theta_{0} \right) \,
            s_{t} \left( y_{t}, x_{t}; \theta_{0} \right)^{\top}
            \right].
        \end{split}
    \end{equation}
\end{theorem}

\begin{corollary}[QLME的克拉美罗下界(CRLB)]
    \label{corollary:qmle-ime-yx-lower-bound}
    假定Theorem \eqref{theorem:qmle-ime-yx-def}成立,那么计算标准化距离$\sqrt{T} \left( \tilde{\theta}_{T} - \theta_{0} \right)$的所需的方差协方差矩阵 $c_{T} \left( \theta_{0} \right)$ \eqref{eq:mle-scalar-crlb-unbiased} ,可简化为
\begin{equation}
    \label{eq:qmle-ime-yx-lower-bound}
    C_{T} \left( \theta_{0} \right) \ge B_{T} \left( \theta_{0} \right)^{-1} = - H_{T} \left( \theta_{0} \right)^{-1},
\end{equation}
QMLE $\theta_{0}$ 逐渐逼近克拉美罗下界(Cramér-Rao lower bound, CRLB)\index{Cramér-Rao lower Bound \dotfill 克拉美罗下界}。
\end{corollary}

\subsection{迭代期望法则和条件方差分解}
\label{sec:qmle-lie-intro}
迭代期望法则(law of iterated expectations, LIE)\index{law of iterated expectations (LIE) \dotfill 迭代期望法则},可简单表示为\citep[pp. 29]{Wooldridge:2002vr}
\begin{equation}
    \label{eq:qmle-lie-simple}
    E \left[ y \right] = E_{x} \left[ E_{y} \left( y | x \right)\right].
\end{equation}

现在将$x \in \mathbb{R}^{M}$理解为一个离散向量,向量中包含随机变量$c_{1}, \ldots, c_{M}$,对应概率$p_{1}, \ldots, p_{M}$,则LIE是指
\begin{equation}
    \label{eq:qmle-lie-vector}
\begin{split}
    E \left[ y \right] & = p_{1} E \left[ y | x=c_{1} \right]
    +p_{2} E \left[ y | x=c_{2} \right] + \ldots
    + p_{M} E \left[ y | x=c_{M} \right] \\
    & = \sum_{j=1}^{M} p_{j} E \left[ y | x=c_{j} \right],
\end{split}
\end{equation}
可见$E_{y}[y]$就是一组$E_{y}[y|x]$的加权平均,权重$p_{j}$是$x$值取$c_{j}$的概率,可以表示为
\begin{equation}
    E \left[ y \right] = E_{x} \left[ E_{y} \left( y | x \right) \right],
\end{equation}
即加权平均的加权平均。

\begin{lemma}[条件方差的分解]
    \label{lemma:conditional-variance-decomposition}
    给定$x$后$y$的条件方差$\var \left(y|x \right)$,可分解为两部分之和,一是条件均值的方差,二是围绕条件均值的期望方差
    \begin{equation}
        \label{eq:conditional-variance-decomposition}
        \var \left( y | x \right) = \var \left( E \left( y | x \right) \right)
        + E_{x} \left[ \var \left( y | x \right) \right].
    \end{equation}
\end{lemma}
\begin{proof}
    \begin{equation*}
        \begin{split}
            \var \left( y | x \right) & = E \left[ \left( y - E(y) \right)^{2} \right]
            = E \left[
            \left(
            y
            \underbrace{
            - E \left( y|x \right) + E \left( y | x \right)
            }_{=0}
            - E(y)
            \right)^{2}
            \right] \\
            & = E
            \left[
            \left( y - E \left( y | x \right) \right)^{2}
            \right]
            + E
            \left[
            \left( E \left( y | x \right) - E \left( y \right) \right)^{2}
            \right]
            + 2 \underbrace{
            E \left[
            \left( y - E \left( y | x \right) \right) \,
            \left( E \left( y | x \right) - E \left( y \right) \right)
            \right]
            }_{\eqqcolon \mathcal{C}}
            .
        \end{split}
    \end{equation*}

    已知
\begin{equation*}
    \begin{split}
        & E \left[ y | x \right] = E \left[ \left( y - E \left( y | x \right) \right) | x \right] = E \left( y | x \right) - E \left( y | x \right) =0, \\
        \hookrightarrow & \mathcal{C} = 0,
    \end{split}
\end{equation*}
进一步引入LIE关系\eqref{eq:qmle-lie-vector},上式改写为

\begin{equation*}
\begin{split}
    \var \left( y | x \right) & =
    E_{x}
    \left\{
    E
    \left[
    \left( y - E \left( y | x \right) \right)^{2}
    \right]
    \right\}
    + E
    \left\{
    E
    \left[
    \left( E \left( y | x \right) - E \left( y \right) \right)^{2}
    \right]
    \right\} \\
    & = E_{x} \left[ \var \left( y | x \right)\right]
    + \var_{x} \left[ E \left( y | x \right) \right],
\end{split}
\end{equation*}
证毕。
\end{proof}

\section{一个例子}
\label{sec:qmle-example}

考虑这样的设定
\begin{equation}
    \label{eq:qmle-example-setup}
    y_{t} | x_{t} \sim \mathcal{N} \left( x_{t}^{\top} \beta, \sigma^{2} \right), \quad \forall \, t = 1,\ldots, T.
\end{equation}

根据高斯正态分布的性质,设系数矩阵$\theta$满足
\begin{equation*}
    \theta = \left( \beta^{\top}, \sigma^{2} \right)^{\top}.
\end{equation*}

对数LHF $\log f_{t} \left( y_{t} | x_{t} ; \theta \right)$满足
\begin{equation*}
    \log f_{t} \left( y_{t} | x_{t} ; \theta \right)
    = -\frac{1}{2} \log \left( 2 \pi \right)
    - \frac{1}{2} \log \left( \sigma^{2} \right)
    - \frac{
    \left( y_{t} - x_{t}^{\top} \beta \right)^{2}
    }{
    2 \sigma^{2}
    }
\end{equation*}

将全部$t=1,\ldots,T$个对数LHF加权求和,设为样本的平均LHF $L_{T} \left(y^{T}, x^{T}; \theta \right)$
\begin{equation*}
    L_{T} \left(y^{T}, x^{T}; \theta \right)
    = -\frac{1}{2} \log \left( 2 \pi \right)
    - \frac{1}{2} \log \left( \sigma^{2} \right)
    - \frac{1}{T} \sum_{t=1}^{T}
    \frac{
    \left( y_{t} - x_{t}^{\top} \beta \right)^{2}
    }{
    2 \sigma^{2}
    }.
\end{equation*}

进而可计算1、2次导数
\begin{align}
    \label{eq:qmle-example-lhf-derivative-1}
    \triangledown L_{T} \left(y^{T}, x^{T}; \theta \right)
    & = \frac{1}{T} \sum_{t=1}^{T}
    \begin{pmatrix}
        \frac{
        x_{t} \left( y_{t} - x_{t}^{\top} \beta \right)
        }{
        \sigma^{2}
        } \\
        -\frac{1}{2 \sigma^{2}} + \frac{
        \left( y_{t} - x_{t}^{\top} \beta \right)^{2}
        }{
        2 \left( \sigma^{2} \right)^{2}
        }
    \end{pmatrix}, \\
    \label{eq:qmle-example-lhf-derivative-2}
    \triangledown^{2} L_{T} \left(y^{T}, x^{T}; \theta \right)
    & = \frac{1}{T} \sum_{t=1}^{T}
    \begin{pmatrix}
        - \frac{x_{t} \, x_{t}^{\top}}{\sigma^{2}} &
        - \frac{x_{t} \left( y_{t} - x_{t}^{\top} \beta \right)}{\left( \sigma^{2} \right)^{2}} \\
        - \frac{
        \left( y_{t} - x_{t}^{\top} \beta \right) \, x_{t}^{\top}
        }{
        \left( \sigma^{2} \right)^{2}
        } &
        \frac{
        1
        }{
        2 \left( \sigma^{2} \right)^{2}
        }
        - \frac{
        \left( y_{t} - x_{t}^{\top} \beta \right)^{2}
        }{
        \left( \sigma^{2} \right)^{3}
        }
    \end{pmatrix},
\end{align}

\eqref{eq:qmle-example-lhf-derivative-1}是socre方程矩阵,据此可进行准最大似然估计:
\begin{itemize}
    \item $\beta$的QMLE $\hat{\beta}_{T}$:
    \begin{equation*}
        \hat{\beta}_{T} = \underset{\beta}{\argmax} \, L_{T} \left( y^{T}, x^{T} ; \theta \right) = \underset{\beta}{\arg} \left\{ \triangledown L_{T} \left( y^{T}, x^{T}, \theta \right) =0 \right\},
    \end{equation*}
    \item $\sigma^{2}$的QMLE $\overline{\sigma}^{2}$就是OLE残差的平均值:
    \begin{equation*}
        \hat{\sigma}^{2} = \frac{1}{T} \sum_{t=1}^{T} \left( y_{t} - x_{t}^{\top} \beta \right)^{2}.
    \end{equation*}
\end{itemize}

假设上述设定对$\left\{ y_{t} | x_{t} \right\}_{t=1}^{T}$整体设定正确,那么$\exists \, \theta_{0} = \left( \beta_{0}^{\top} \sigma_{0}^{2} \right)^{\top}$,使得在给定$x_{t}$的情况下,$y_{t}$的真实分布$g^{T} \left(y^{T}, x^{T}; \theta_{0} \right) $满足
\begin{equation*}
g^{T} \left(y^{T}, x^{T} \right) \sim \mathcal{N} \left( x_{t}^{\top} \beta_{0}, \sigma_{0}^{2} \right),
\end{equation*}
据此可计算海森矩阵$H_{T} \left( \theta \right)$:
\begin{itemize}
\item 对均值取期望,得
\begin{equation}
    \label{eq:qmle-example-expectation-mean}
    \begin{split}
        E \left[ x_{t} \left( y_{t} - x_{t}^{\top} \beta \right) \right]
        & = E \left( x_{t} \, x_{t}^{\top} \right) \, \left( \beta_{0} - \beta \right) \\
        & = 0 \qquad \text{如果} \beta = \beta_{0}.
    \end{split}
\end{equation}
\item 对方差取期望,得
\begin{equation}
    \label{eq:qmle-example-expectation-variance}
    \begin{split}
        E \left[ \left( y_{t} - x_{t}^{\top} \beta \right)^{2} \right]
        & = \left[
        \left(
        y_{t} - x_{t}^{\top} \beta_{0} + x_{t}^{\top} \beta_{0} - x_{t}^{\top} \beta
        \right)^{2}
        \right] \\
        & = E \left[ \left( y_{t} - x_{t}^{\top} \beta_{0} \right)^{2} \right]
        + E \left[
        \left(
        x_{t}^{\top} \beta_{0} - x_{t}^{\top} \beta
        \right)^{2}
        \right]
        +
        \underbrace{
        2 E \left[
        \left( y_{t} - x_{t}^{\top} \beta_{0} \right) \,
        \left(x_{t}^{\top} \beta_{0} - x_{t}^{\top} \beta \right)
        \right]
        }_{=0} \\
        & = \sigma_{0}^{2}
        + E \left[
        \left(
        x_{t}^{\top} \beta_{0} - x_{t}^{\top} \beta
        \right)^{2}
        \right] \\
        & = \sigma_{0}^{2} \qquad \text{如果 } \beta=\beta_{0}.
    \end{split}
\end{equation}
\item 将\eqref{eq:qmle-example-expectation-mean}-\eqref{eq:qmle-example-expectation-variance}代回\eqref{eq:qmle-example-lhf-derivative-2},
算得海森矩阵$H_{T} \left( \theta \right)$
\begin{equation}
    \label{eq:qmle-example-hessian}
    H_{T} \left( \theta \right) = E \left[ \triangledown^{2} L_{T} \left( \theta \right) \right]
    = \frac{1}{T} \sum_{t=1}^{T}
    \begin{pmatrix}
        - \frac{
        E \left[ x_{t} \, x_{t}^{\top} \right]
        }{
        \sigma^{2}
        } &
        - \frac{
        E \left(x_{t} x_{t}^{\top} \right) \left( \beta_{0} - \beta \right)
        }{
        \left( \sigma^{2} \right)^{2}
        } \\
        - \frac{
        \left( \beta_{0} - \beta \right)^{\top} E \left(x_{t} x_{t}^{\top} \right)
        }{
        \left( \sigma^{2} \right)^{2}
        } &
        \frac{
        1
        }{
        2 \left( \sigma^{2} \right)^{2}
        }
        - \frac{
        \sigma_{0}^{2} + E \left[
        \left( x_{t}^{\top} \beta_{0} - x_{t}^{\top} \beta \right)^{2}
        \right]
        }{
        \left( \sigma^{2} \right)^{3}
        }
    \end{pmatrix},
\end{equation}
\end{itemize}

对于$\theta = \theta_{0} \Longleftrightarrow \left( \beta^{\top} , \sigma^{2} \right)^{\top} = \left(\beta_{0}^{\top} ,\sigma_{0}^{2} \right)^{\top}$,代入上式,有
\begin{equation}
    \label{eq:qmle-example-hessian-theta0}
    H_{T} \left( \theta_{0} \right)
    = \frac{1}{T} \sum_{t=1}^{T}
    \begin{pmatrix}
        - \frac{E \left( x_{t} x_{t}^{\top} \right)}{\sigma_{0}^{2}}
        & 0 \\
        0 & - \frac{1}{2 \left( \sigma_{0}^{2} \right)^{2}}
    \end{pmatrix}.
\end{equation}

再来看信息矩阵$B_{T} \left( \theta \right)$的计算。若假设不存在动态误设定,那么
\begin{equation}
    \label{eq:qmle-example-information}
    B_{T} \left( \theta \right)
    =\frac{1}{T} \sum_{t=1}^{T} E
    \begin{pmatrix}
        \frac{
        \left( y_{t} - x_{t}^{\top} \beta \right)^{2} x_{t} x_{t}^{\top}
        }{
        \left( \sigma^{2} \right)^{2}
        } &
        -\frac{
        x_{t} \left( y_{t} - x_{t}^{\top} \beta \right)
        }{
        2 \left( \sigma^{2} \right)^{2}
        }
        + \frac{
        x_{t} \left( y_{t} - x_{t}^{\top} \beta \right)^{3}
        }{
        2 \left( \sigma^{2} \right)^{3}
        } \\
        \frac{
        \left( y_{t} - x_{t}^{\top} \beta \right) x_{t}^{\top}
        }{
        2 \left( \sigma^{2} \right)^{2}
        }
        +
        \frac{
        \left( y_{t} - x_{t}^{\top} \beta \right)^{3} x_{t}^{\top}
        }{
        2 \left( \sigma^{2} \right)^{3}
        } &
        \frac{
        1
        }{
        4 \left( \sigma^{2} \right)^{2}
        }
        - \frac{
        \left( y_{t} - x_{t}^{\top} \beta \right)^{2}
        }{
        2 \left( \sigma^{2} \right)^{3}
        }
        + \frac{
        \left( y_{t} - x_{t}^{\top} \beta \right)^{4}
        }{
        4 \left( \sigma^{2} \right)^{4}
        }
    \end{pmatrix}.
\end{equation}

根据假定,$y_{t}$满足条件正态分布,那么它的3、4阶条件中心距(central moment formula)\todo{central moment formula}分别为
%$0$和$3 \left( \sigma_{0}^{21} \right)^{2}$:这是由于
\begin{itemize}
\item
\begin{equation*}
    \begin{split}
        E \left[ \left( y_{t} - x_{t}^{\top} \beta \right)^{3} \right]
        & = - 3 \sigma_{0}^{2} \,
        E \left[
        \left(
        x_{t}^{\top} \beta_{0} - x_{t}^{\top} \beta
        \right)
        \right]
        - E
        \left[
        \left(
        x_{t}^{\top} \beta_{0} - x_{t}^{\top} \beta
        \right)^{3}
        \right] \\
        & = 0 \qquad \text{如果} \beta_{0} = \beta,
    \end{split}
\end{equation*}
\item
\begin{equation*}
\begin{split}
    E \left[ \left( y_{t} - x_{t}^{\top} \beta \right)^{4} \right]
    & = 3 \left( \sigma_{0}^{2} \right)^{2}
    + 6 \left( \sigma_{0}^{2} \right)
    E \left[
    \left(
    x_{t}^{\top} \beta_{0} - x_{t}^{\top} \beta
    \right)^{2}
    \right]
    + E \left[
    \left(
    x_{t}^{\top} \beta_{0} - x_{t}^{\top} \beta
    \right)^{3}
    \right] \\
    & = 3 \left( \sigma_{0}^{2} \right)^{2} \qquad \text{如果 } \beta_{0} = \beta.
\end{split}
\end{equation*}
\end{itemize}

对于$\theta = \theta_{0} \Longleftrightarrow \left( \beta^{\top} , \sigma^{2} \right)^{\top} = \left( \beta_{0}^{\top} , \sigma_{0}^{2} \right)^{\top}$,代回\eqref{eq:qmle-example-information},有$B_{T} \left( \theta_{0} \right)$
\begin{equation}
    \label{eq:qmle-example-information-beta0}
    B_{T} \left( \theta_{0} \right)
    = \frac{1}{T} \sum_{t=1}^{T}
    \begin{pmatrix}
        \frac{
        E \left[ x_{t} x_{t}^{\top} \right]
        }{
        \sigma_{0}^{2}
        } & 0 \\
        0 & \frac{1}{2 \left( \sigma_{0}^{2} \right)^{2}}
    \end{pmatrix}
\end{equation}

结合\eqref{eq:qmle-example-hessian-theta0},
\eqref{eq:qmle-example-information-beta0}可见
\begin{equation*}
    B_{T} \left( \theta_{0} \right) = - H_{T} \left( \theta_{0} \right),
\end{equation*}
即IME成立。

设$H_{T} \left( \tilde{\theta}_{T} \right)$是一个关于$H_{T} \left( \theta_{0} \right)$的一致估计,
$\tilde{\theta}_{T} = \left( \tilde{\beta}_{T}^{\top} \tilde{\sigma}_{T}^{2} \right)^{\top}$,进而

\begin{equation}
    \label{eq:qmle-example-hessian-theta-consistent}
    H_{T} \left( \tilde{\theta}_{T} \right)
    = -
    \begin{pmatrix}
        - \frac{
        \sum_{t=1}^{T} \left( x_{t} x_{t}^{\top} \right)
        }{
        T \, \tilde{\sigma}_{T}^{2}
        }
        & 0 \\
        0 & - \frac{1}{2 \left( \tilde{\sigma}_{T}^{2} \right)^{2}}
    \end{pmatrix}.
\end{equation}

在此基础上由IME条件可得,$B_{T} \left( \theta_{0} \right)$的一致估计$B_{T} \left( \tilde{\theta}_{T} \right)$可由下式算得
\begin{equation*}
    B_{T} \left( \tilde{\theta}_{T} \right) = - H_{T} \left( \tilde{\theta}_{T} \right).
\end{equation*}

值得注意的是,$H_{T} \left( \tilde{\theta}_{T} \right)$的左上分块矩阵是对$\tilde{\beta}_{T}$系数的方差协方差矩阵的标准估计:若存在动态误设定的情况,则$B_{T} \left( \theta \right)$的估计值不同于上式,这导致IME不成立。

\section{异方差问题}
\label{sec:qmle-ime-heteroskedasticity}

哪怕模型对$E \left( y_{t} | x_{t} \right)$设定正确,不存在动态误设定的情况,仍然可能由于其他一些条件矩的误设定问题,如条件异方差(heteroskedasticity)\index{heteroskedasticity \dotfill},而使得IME不成立。举例来说,在第\ref{sec:qmle-example}节,设定\eqref{eq:qmle-example-setup}
的基础上,额外假定分布方差服从DGP以下过程
\begin{equation}
    \label{eq:eq:qmle-example-variance-setup}
    y_{t} | x_{t} \sim \mathcal{N} \left( x_{t}^{\top} \beta, h \left(x_{t}^{\top} \gamma_{0} \right) \right),
\end{equation}
即条件条件方差$h \left(x_{t}^{\top} \gamma_{0} \right)$随着$x_{t}$而变化,存在条件异方差问题。

这个设定,尽管就其条件均值$\left( x_{t}^{\top} \beta \right)$而言是设定正确的,但却并不是对$\left\{ y_{t} | x_{t} \right\}_{t=1}^{T}$整体的正确设定,因为它存在条件异方差。这样一来,根据上一节例子的思路,海森矩阵$H_{T} \left( \theta \right)$的左上分块值由\eqref{eq:qmle-example-hessian}变为
\begin{equation}
    \label{eq:qmle-example-hessian-heteroskedasticity}
    - \frac{1}{T \, \sigma^{2}} \sum_{t=1}^{T} E \left( x_{t} x_{t}^{\top} \right), %\sigma^{2}
\end{equation}

而信息矩阵$B_{T} \left( \theta \right)$的左上分块值由 \eqref{eq:qmle-example-information}仍然是
\begin{equation}
    \label{eq:qmle-example-information-heteroskedasticity}
    \frac{1}{T} \sum_{t=1}^{T} \frac{
    E \left[
    \left( y_{t} - x_{t}^{\top} \beta \right)^{2}
    \right]
    }{
    \sigma^{2}
    }
    \frac{
    E \left[ x_{t} x_{t}^{\top} \right]
    }{
    \sigma^{2}
    }
\end{equation}

由于当前设定对条件均值是正确的,对条件方差是不正确的,可以将对应的KLIC最小估计设为$\theta^{*} = \left(\beta_{0}^{\top}, \sigma^{*,2} \right)^{\top}$。在$\theta = \theta^{*}$处的海森矩阵分块\eqref{eq:qmle-example-hessian-heteroskedasticity}和信息矩阵分块\eqref{eq:qmle-example-information-heteroskedasticity}分别为
\begin{align*}
    & - \frac{1}{T \, \sigma^{*,2}} \sum_{t=1}^{T} E \left( x_{t} x_{t}^{\top} \right), \\
    & \frac{1}{T \, \left( \sigma^{*,2} \right)^{2}} \sum_{t=1}^{T} E
    \left( h \left( x_{t}^{\top} \gamma_{0} \right) x_{t} x_{t}^{\top} \right),
\end{align*}
显然二者不相等:尽管条件方差设定“正确”,但整体上的IME不成立。

对$H_{T} \left( \theta^{*} \right), B_{T} \left(\theta^{*} \right)$ 的一致估计$H_{T}\left( \tilde{\theta}_{T} \right), B_{T} \left(\tilde{\theta}_{T} \right)$分别为
\begin{equation*}
    \begin{split}
        H_{T}\left( \tilde{\theta}_{T} \right)
        & = \begin{pmatrix}
        - \frac{
        \sum_{t=1}^{T} x_{t} x_{t}^{\top}
        }{
        T \, \tilde{\sigma}_{T}^{2}
        } & 0 \\
        0 & \frac{1}{2 \left( \tilde{\sigma}_{T}^{2} \right)^{2}}
        \end{pmatrix}, \\
        B_{T} \left(\tilde{\theta}_{T} \right)
        & = \begin{pmatrix}
        \frac{
        \sum_{t=1}^{T} \hat{e}_{t}^{2} x_{t} x_{t}^{\top}
        }{
        T \, \left( \tilde{\sigma}_{T}^{2} \right)^{2}
        } & 0 \\
        0 &
        - \frac{1}{4 \left( \tilde{\sigma}_{T}^{2} \right)^{2}}
        + \frac{
        \sum_{t=1}^{T} \hat{e}_{t}^{4}
        }{
        T \, 4 \, \left( \tilde{\sigma}_{T}^{2} \right)^{4}
        }
        \end{pmatrix},
    \end{split}
\end{equation*}
其中$\hat{e}_{t}$是OLS估计的残差。$H_{T}\left( \tilde{\theta}_{T} \right), B_{T} \left(\tilde{\theta}_{T} \right)$都是块对角矩阵,因此$H_{T}\left( \tilde{\theta}_{T} \right)^{-1}, B_{T} \left(\tilde{\theta}_{T} \right) H_{T}\left( \tilde{\theta}_{T} \right)^{-1}$矩阵的左上分块是
\begin{equation*}
    \left( \frac{1}{T} \sum_{t=1}^{T} x_{t} x_{t}^{\top} \right)^{-1}
    \left( \frac{1}{T} \sum_{t=1}^{T} \hat{e}_{t} x_{t}x_{t}^{\top} \right)
    \left( \frac{1}{T} \sum_{t=1}^{T} x_{t} x_{t}^{\top} \right)^{-1},
\end{equation*}
时OLS估计$\hat{\beta}_{T}$的方差协方差矩阵的Eicker-White estimator\todo{考虑一下,要不要做个section?}。
